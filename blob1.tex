\documentclass[11pt,leqno]{article}

\newcommand{\pathtotrunk}{./}
%auto-ignore
%this ensures the arxiv doesn't try to start TeXing here.

%!TEX root = ../blob1.tex

\ifpdf
	\usepackage[pdftex,plainpages=false,hypertexnames=false,pdfpagelabels]{hyperref}
	\usepackage[pdftex]{graphicx}
\else
	\usepackage[plainpages=false,hypertexnames=false,pdfpagelabels]{hyperref}
	\usepackage{graphicx}
\fi

%must load tikz after graphicx
\usepackage{tikz}
\usetikzlibrary{shapes}
\usetikzlibrary{backgrounds}
\usetikzlibrary{decorations,decorations.pathreplacing}
\usetikzlibrary{fit,calc,through}

%\pgfrealjobname{blob1}

\makeatletter
\@ifclassloaded{beamer}{}{%
	\newtheorem{example}[prop]{Example}	% for compatibility with beamer
  }%
\@ifclassloaded{pnastwo}
  {}
  {
	\newtheorem{claim}[prop]{Claim}
	\newtheorem{remark}[prop]{Remark}
	\newtheorem{lemma}[prop]{Lemma}
  }%
\makeatother



\usepackage{color}

% idea from tex-overflow
\usepackage{xcolor}
\definecolor{dark-red}{rgb}{0.7,0.25,0.25}
\definecolor{dark-blue}{rgb}{0.15,0.15,0.55}
\definecolor{medium-blue}{rgb}{0,0,0.65}
\hypersetup{
    colorlinks, linkcolor={dark-red},
    citecolor={dark-blue}, urlcolor={medium-blue}
}


% margin stuff
\setlength{\textwidth}{6.5in}
\setlength{\oddsidemargin}{0in}
\setlength{\evensidemargin}{0in}
\setlength{\textheight}{8.5in}
\setlength{\topmargin}{-.25in}

%!TEX root = ../blob1.tex

\title{Blob Homology}

\author{Scott~Morrison}
\address{
}%
\email{scott@tqft.net} \urladdr{http://tqft.net/}

\author{Kevin~Walker}
\address{
}%
\email{kevin@canyon23.net} \urladdr{http://canyon23.net/}


\date{
  First edition: the mysterious future
  This edition: \today.
}

%\primaryclass{57M25} \secondaryclass{57M27; 57Q45}
%\keywords{

%}

%!TEX root = ../blob1.tex

%%%%% excerpts from KW's include file of standard macros
%%% (with various new ones added)

\def\z{\mathbb{Z}}
\def\r{\mathbb{R}}
\def\c{\mathbb{C}}
\def\t{\mathbb{T}}
\def\ebb{\mathbb{E}}

\def\du{\sqcup}
\def\bd{\partial}
\def\sub{\subset}
\def\subeq{\subseteq}
\def\sup{\supset}
%\def\setmin{\smallsetminus}
\def\setmin{\setminus}
\def\ep{\epsilon}
\def\sgl{_\mathrm{gl}}
\def\op{^\mathrm{op}}
\def\deq{\stackrel{\mathrm{def}}{=}}
\def\pd#1#2{\frac{\partial #1}{\partial #2}}
\def\lf{\overline{\cC}}
\def\ot{\otimes}
\def\inv{^{-1}}

\def\spl{_\pitchfork}

%\def\nn#1{{{\it \small [#1]}}}
\def\nn#1{{{\color[rgb]{.2,.5,.6} \small [#1]}}}
\long\def\noop#1{}

% equations
\newcommand{\eq}[1]{\begin{displaymath}#1\end{displaymath}}
\newcommand{\eqar}[1]{\begin{eqnarray*}#1\end{eqnarray*}}
\newcommand{\eqspl}[1]{\begin{displaymath}\begin{split}#1\end{split}\end{displaymath}}

% tricky way to iterate macros over a list
\def\semicolon{;}
\def\applytolist#1{
    \expandafter\def\csname multi#1\endcsname##1{
        \def\multiack{##1}\ifx\multiack\semicolon
            \def\next{\relax}
        \else
            \csname #1\endcsname{##1}
            \def\next{\csname multi#1\endcsname}
        \fi
        \next}
    \csname multi#1\endcsname}

% \def\cA{{\cal A}} for A..Z
\def\calc#1{\expandafter\def\csname c#1\endcsname{{\mathcal #1}}}
\applytolist{calc}QWERTYUIOPLKJHGFDSAZXCVBNM;

% \DeclareMathOperator{\pr}{pr} etc.
\def\declaremathop#1{\expandafter\DeclareMathOperator\csname #1\endcsname{#1}}
\applytolist{declaremathop}{pr}{im}{gl}{ev}{coinv}{tr}{rot}{Eq}{obj}{mor}{ob}{Rep}{Tet}{cat}{Maps}{Diff}{Homeo}{sign}{supp}{Nbd}{res};


%%%%%% end excerpt




\title{The Blob Complex}

\begin{document}

\makeatletter
\@addtoreset{equation}{section}
\gdef\theequation{\thesection.\arabic{equation}}
\makeatother

\maketitle

%[revision $\ge$ 527;  $\ge$ 30 August 2010]
%
%{\color[rgb]{.9,.5,.2} \large \textbf{Draft version, read with caution.}}
%We're in the midst of revising this, and hope to have a version on the arXiv soon.

\begin{abstract}
Given an $n$-manifold $M$ and an $n$-category $\cC$, we define a chain complex
(the ``blob complex") $\bc_*(M; \cC)$.
The blob complex can be thought of as a derived category analogue of the Hilbert space of a TQFT, 
and also as a generalization of Hochschild homology to $n$-categories and $n$-manifolds.
It enjoys a number of nice formal properties, including a higher dimensional
generalization of Deligne's conjecture about the action of the little disks operad on Hochschild cochains.
Along the way, we give a definition of a weak $n$-category with strong duality which
is particularly well suited for work with TQFTs.
\end{abstract}

\hypersetup{
    colorlinks, linkcolor={black},
    citecolor={dark-blue}, urlcolor={medium-blue}
}

\tableofcontents

\hypersetup{
    colorlinks, linkcolor={dark-red},
    citecolor={dark-blue}, urlcolor={medium-blue}
}


%\let\stdsection\section
%\renewcommand\section{\newpage\stdsection}

%!TEX root = ../blob1.tex

\section{Introduction}

We construct a chain complex $\bc_*(M; \cC)$ --- the ``blob complex'' --- 
associated to an $n$-manifold $M$ and a linear $n$-category $\cC$ with strong duality.
This blob complex provides a simultaneous generalization of several well known constructions:
\begin{itemize}
\item The 0-th homology $H_0(\bc_*(M; \cC))$ is isomorphic to the usual 
topological quantum field theory invariant of $M$ associated to $\cC$.
(See Proposition \ref{thm:skein-modules} later in the introduction and \S \ref{sec:constructing-a-tqft}.)
\item When $n=1$ and $\cC$ is just a 1-category (e.g.\ an associative algebra), 
the blob complex $\bc_*(S^1; \cC)$ is quasi-isomorphic to the Hochschild complex $\HC_*(\cC)$.
(See Theorem \ref{thm:hochschild} and \S \ref{sec:hochschild}.)
%\item When $\cC$ is the polynomial algebra $k[t]$, thought of as an n-category, we have 
%that $\bc_*(M; k[t])$ is homotopy equivalent to $C_*(\Sigma^\infty(M), k)$, the singular chains
%on the configuration space of unlabeled points in $M$. (See \S \ref{sec:comm_alg}.)
\item When $\cC$ is $\pi^\infty_{\leq n}(T)$, the $A_\infty$ version of the fundamental $n$-groupoid of
the space $T$ (Example \ref{ex:chains-of-maps-to-a-space}), 
$\bc_*(M; \cC)$ is homotopy equivalent to $C_*(\Maps(M\to T))$,
the singular chains on the space of maps from $M$ to $T$.
(See Theorem \ref{thm:map-recon}.)
\end{itemize}

The blob complex definition is motivated by the desire for a derived analogue of the usual TQFT Hilbert space 
(replacing the quotient of fields by local relations with some sort of resolution), 
and for a generalization of Hochschild homology to higher $n$-categories.
One can think of it as the push-out of these two familiar constructions.
More detailed motivations are described in \S \ref{sec:motivations}.

The blob complex has good formal properties, summarized in \S \ref{sec:properties}.
These include an action of $\CH{M}$, 
extending the usual $\Homeo(M)$ action on the TQFT space $H_0$ (Theorem \ref{thm:evaluation}) and a gluing 
formula allowing calculations by cutting manifolds into smaller parts (Theorem \ref{thm:gluing}).

We expect applications of the blob complex to contact topology and Khovanov homology 
but do not address these in this paper.
%See \S \ref{sec:future} for slightly more detail.

Throughout, we have resisted the temptation to work in the greatest possible generality.
(Don't worry, it wasn't that hard.)
In most of the places where we say ``set" or ``vector space", any symmetric monoidal category 
with sufficient limits and colimits would do.
We could also replace many of our chain complexes with topological spaces (or indeed, work at the generality of model categories).


\subsection{Structure of the paper}
The subsections of the introduction explain our motivations in defining the blob complex (see \S \ref{sec:motivations}), 
summarize the formal properties of the blob complex (see \S \ref{sec:properties}), describe known specializations (see \S \ref{sec:specializations}), and outline the major results of the paper (see \S \ref{sec:structure} and \S \ref{sec:applications}).
%and outline anticipated future directions (see \S \ref{sec:future}).
%\nn{recheck this list after done editing intro}

The first part of the paper (sections \S \ref{sec:fields}--\S \ref{sec:evaluation}) gives the definition of the blob complex, 
and establishes some of its properties.
There are many alternative definitions of $n$-categories, and part of the challenge of defining the blob complex is 
simply explaining what we mean by an ``$n$-category with strong duality'' as one of the inputs.
At first we entirely avoid this problem by introducing the notion of a ``system of fields", and define the blob complex 
associated to an $n$-manifold and an $n$-dimensional system of fields.
We sketch the construction of a system of fields from a *-$1$-category and from a pivotal $2$-category.

Nevertheless, when we attempt to establish all of the observed properties of the blob complex, 
we find this situation unsatisfactory.
Thus, in the second part of the paper (\S\S \ref{sec:ncats}-\ref{sec:ainfblob}) we give yet another 
definition of an $n$-category, or rather a definition of an $n$-category with strong duality.
(Removing the duality conditions from our definition would make it more complicated rather than less.) 
We call these ``disk-like $n$-categories'', to differentiate them from previous versions.
Moreover, we find that we need analogous $A_\infty$ $n$-categories, and we define these as well following very similar axioms.
(See \S \ref{n-cat-names} below for a discussion of $n$-category terminology.)

The basic idea is that each potential definition of an $n$-category makes a choice about the ``shape" of morphisms.
We try to be as lax as possible: a disk-like $n$-category associates a 
vector space to every $B$ homeomorphic to the $n$-ball.
These vector spaces glue together associatively, and we require that there is an action of the homeomorphism groupoid.
For an $A_\infty$ $n$-category, we associate a chain complex instead of a vector space to 
each such $B$ and ask that the action of 
homeomorphisms extends to a suitably defined action of the complex of singular chains of homeomorphisms.
The axioms for an $A_\infty$ $n$-category are designed to capture two main examples: 
the blob complexes of $n$-balls labelled by a 
disk-like $n$-category, and the complex $\CM{-}{T}$ of maps to a fixed target space $T$.

In \S \ref{ssec:spherecat} we explain how $n$-categories can be viewed as objects in an $n{+}1$-category 
of sphere modules.
When $n=1$ this just the familiar 2-category of 1-categories, bimodules and intertwiners.

In \S \ref{ss:ncat_fields}  we explain how to construct a system of fields from a disk-like $n$-category 
(using a colimit along certain decompositions of a manifold into balls). 
With this in hand, we write $\bc_*(M; \cC)$ to indicate the blob complex of a manifold $M$ 
with the system of fields constructed from the $n$-category $\cC$. 
%\nn{KW: I don't think we use this notational convention any more, right?}
In \S \ref{sec:ainfblob} we give an alternative definition 
of the blob complex for an $A_\infty$ $n$-category on an $n$-manifold (analogously, using a homotopy colimit).
Using these definitions, we show how to use the blob complex to ``resolve" any ordinary $n$-category as an 
$A_\infty$ $n$-category, and relate the first and second definitions of the blob complex.
We use the blob complex for $A_\infty$ $n$-categories to establish important properties of the blob complex (in both variants), 
in particular the ``gluing formula" of Theorem \ref{thm:gluing} below.

The relationship between all these ideas is sketched in Figure \ref{fig:outline}.

% NB: the following tikz requires a *more recent* version of PGF than is distributed with MacTex 2010.
% grab the latest build from http://www.texample.net/tikz/builds/
% unzip it in your personal tex tree, and run "mktexlsr ." there
\tikzstyle{box} = [rectangle, rounded corners, draw,outer sep = 5pt, inner sep = 5pt, line width=0.5pt]

\begin{figure}[t]
{\center
\beginpgfgraphicnamed{gadgets-external}%
\begin{tikzpicture}[align=center,line width = 1.5pt]
\newcommand{\xxa}{2}
\newcommand{\xxb}{8}
\newcommand{\yya}{14}
\newcommand{\yyb}{10}
\newcommand{\yyc}{6}

\node[box] at (-4,\yyb) (tC) {$C$ \\ a `traditional' \\ weak $n$-category};
\node[box] at (\xxa,\yya) (C) {$\cC$ \\ a disk-like \\ $n$-category};
\node[box] at (\xxb,\yya) (A) {$\underrightarrow{\cC}(M)$ \\ the (dual) TQFT \\ Hilbert space};
\node[box] at (\xxa,\yyb) (FU) {$(\cF, U)$ \\ fields and\\ local relations};
\node[box] at (\xxb,\yyb) (BC) {$\bc_*(M; \cF)$ \\ the blob complex};
\node[box] at (\xxa,\yyc) (Cs) {$\cC_*$ \\ an $A_\infty$ \\$n$-category};
\node[box] at (\xxb,\yyc) (BCs) {$\underrightarrow{\cC_*}(M)$};



\draw[->] (C) -- node[above] {$\displaystyle \colim_{\cell(M)} \cC$} node[below] {\S\S \ref{sec:constructing-a-tqft} \& \ref{ss:ncat_fields}} (A);
\draw[->] (FU) -- node[below] {blob complex \\ for $M$} (BC);
\draw[->] (Cs) -- node[above] {$\displaystyle \hocolim_{\cell(M)} \cC_*$} node[below] {\S \ref{ss:ncat_fields}} (BCs);

\draw[->] (FU) -- node[right=10pt] {$\cF(M)/U$} (A);

\draw[->] (tC) -- node[above] {Example \ref{ex:traditional-n-categories(fields)}} (FU);

\draw[->] (C.-100) -- node[left] {
	\S \ref{ss:ncat_fields}
	%$\displaystyle \cF(M) = \DirectSum_{c \in\cell(M)} \cC(c)$ \\ $\displaystyle U(B) = \DirectSum_{c \in \cell(B)} \ker \ev: \cC(c) \to \cC(B)$
   } (FU.100);
\draw[->] (C) -- node[above left=3pt] {restrict to \\ standard balls} (tC);
\draw[->] (FU.80) -- node[right] {restrict \\ to balls} (C.-80);
\draw[->] (BC) -- node[right] {$H_0$ \\ c.f. Proposition \ref{thm:skein-modules}} (A);

\draw[->] (FU) -- node[left] {blob complex \\ for balls} (Cs);
\draw[<->] (BC) -- node[right] {$\iso$ by \\ Corollary \ref{cor:new-old}} (BCs);
\end{tikzpicture}
\endpgfgraphicnamed%
\mbox{} % <-- weird, doesn't compile unless I put something here after the \endpgfgraphicnamed...? -S
}
\caption{The main gadgets and constructions of the paper.}
\label{fig:outline}
\end{figure}

Later sections address other topics.
Section \S \ref{sec:deligne} gives
a higher dimensional generalization of the Deligne conjecture 
(that the little discs operad acts on Hochschild cochains) in terms of the blob complex.
The appendices prove technical results about $\CH{M}$ and
make connections between our definitions of $n$-categories and familiar definitions for $n=1$ and $n=2$, 
as well as relating the $n=1$ case of our $A_\infty$ $n$-categories with usual $A_\infty$ algebras. 
%Appendix \ref{sec:comm_alg} describes the blob complex when $\cC$ is a commutative algebra, 
%thought of as a disk-like $n$-category, in terms of the topology of $M$.



\subsection{Motivation}
\label{sec:motivations}

We will briefly sketch our original motivation for defining the blob complex.

As a starting point, consider TQFTs constructed via fields and local relations.
(See \S\ref{sec:tqftsviafields} or \cite{kw:tqft}.)
This gives a satisfactory treatment for semisimple TQFTs
(i.e.\ TQFTs for which the cylinder 1-category associated to an
$n{-}1$-manifold $Y$ is semisimple for all $Y$).

For non-semi-simple TQFTs, this approach is less satisfactory.
Our main motivating example (though we will not develop it in this paper)
is the (decapitated) $4{+}1$-dimensional TQFT associated to Khovanov homology.
It associates a bigraded vector space $A_{Kh}(W^4, L)$ to a 4-manifold $W$ together
with a link $L \subset \bd W$.
The original Khovanov homology of a link in $S^3$ is recovered as $A_{Kh}(B^4, L)$.
%\todo{I'm tempted to replace $A_{Kh}$ with $\cl{Kh}$ throughout this page -S}

How would we go about computing $A_{Kh}(W^4, L)$?
For the Khovanov homology of a link in $S^3$ the main tool is the exact triangle (long exact sequence)
relating resolutions of a crossing.
Unfortunately, the exactness breaks if we glue $B^4$ to itself and attempt
to compute $A_{Kh}(S^1\times B^3, L)$.
According to the gluing theorem for TQFTs, gluing along $B^3 \subset \bd B^4$
corresponds to taking a coend (self tensor product) over the cylinder category
associated to $B^3$ (with appropriate boundary conditions).
The coend is not an exact functor, so the exactness of the triangle breaks.

The obvious solution to this problem is to replace the coend with its derived counterpart, 
Hochschild homology.
This presumably works fine for $S^1\times B^3$ (the answer being the Hochschild homology
of an appropriate bimodule), but for more complicated 4-manifolds this leaves much to be desired.
If we build our manifold up via a handle decomposition, the computation
would be a sequence of derived coends.
A different handle decomposition of the same manifold would yield a different
sequence of derived coends.
To show that our definition in terms of derived coends is well-defined, we
would need to show that the above two sequences of derived coends yield 
isomorphic answers, and that the isomorphism does not depend on any
choices we made along the way.
This is probably not easy to do.

Instead, we would prefer a definition for a derived version of $A_{Kh}(W^4, L)$
which is manifestly invariant.
(That is, a definition that does not
involve choosing a decomposition of $W$.
After all, one of the virtues of our starting point --- TQFTs via field and local relations ---
is that it has just this sort of manifest invariance.)

The solution is to replace $A_{Kh}(W^4, L)$, which is a quotient
\[
 \text{linear combinations of fields} \;\big/\; \text{local relations} ,
\]
with an appropriately free resolution (the blob complex)
\[
	\cdots\to \bc_2(W, L) \to \bc_1(W, L) \to \bc_0(W, L) .
\]
Here $\bc_0$ is linear combinations of fields on $W$,
$\bc_1$ is linear combinations of local relations on $W$,
$\bc_2$ is linear combinations of relations amongst relations on $W$,
and so on. We now have a short exact sequence of chain complexes relating resolutions of the link $L$ (c.f. Lemma \ref{lem:hochschild-exact} which shows exactness with respect to boundary conditions in the context of Hochschild homology).


\subsection{Formal properties}
\label{sec:properties}
The blob complex enjoys the following list of formal properties.

\begin{property}[Functoriality]
\label{property:functoriality}%
The blob complex is functorial with respect to homeomorphisms.
That is, 
for a fixed $n$-dimensional system of fields $\cF$, the association
\begin{equation*}
X \mapsto \bc_*(X; \cF)
\end{equation*}
is a functor from $n$-manifolds and homeomorphisms between them to chain 
complexes and isomorphisms between them.
\end{property}
As a consequence, there is an action of $\Homeo(X)$ on the chain complex $\bc_*(X; \cF)$; 
this action is extended to all of $C_*(\Homeo(X))$ in Theorem \ref{thm:evaluation} below.

The blob complex is also functorial with respect to $\cF$, 
although we will not address this in detail here.

\begin{property}[Disjoint union]
\label{property:disjoint-union}
The blob complex of a disjoint union is naturally isomorphic to the tensor product of the blob complexes.
\begin{equation*}
\bc_*(X_1 \du X_2) \iso \bc_*(X_1) \tensor \bc_*(X_2)
\end{equation*}
\end{property}

If an $n$-manifold $X$ contains $Y \sqcup Y^\text{op}$ as a codimension $0$ submanifold of its boundary, 
write $X_\text{gl} = X \bigcup_{Y}\selfarrow$ for the manifold obtained by gluing together $Y$ and $Y^\text{op}$.
Note that this includes the case of gluing two disjoint manifolds together.
\begin{property}[Gluing map]
\label{property:gluing-map}%
Given a gluing $X \to X_\mathrm{gl}$, there is
a natural map
\[
	\bc_*(X) \to \bc_*(X_\mathrm{gl}) 
\]
(natural with respect to homeomorphisms, and also associative with respect to iterated gluings).
\end{property}

\begin{property}[Contractibility]
\label{property:contractibility}%
With field coefficients, the blob complex on an $n$-ball is contractible in the sense 
that it is homotopic to its $0$-th homology.
Moreover, the $0$-th homology of balls can be canonically identified with the vector spaces 
associated by the system of fields $\cF$ to balls.
\begin{equation*}
\xymatrix{\bc_*(B^n;\cF) \ar[r]^(0.4){\iso}_(0.4){\text{qi}} & H_0(\bc_*(B^n;\cF)) \ar[r]^(0.6)\iso & A_\cF(B^n)}
\end{equation*}
\end{property}

Property \ref{property:functoriality} will be immediate from the definition given in
\S \ref{sec:blob-definition}, and we'll recall it at the appropriate point there.
Properties \ref{property:disjoint-union}, \ref{property:gluing-map} and 
\ref{property:contractibility} are established in \S \ref{sec:basic-properties}.

\subsection{Specializations}
\label{sec:specializations}

The blob complex is a simultaneous generalization of the TQFT skein module construction and of Hochschild homology.

\newtheorem*{thm:skein-modules}{Proposition \ref{thm:skein-modules}}

\begin{thm:skein-modules}[Skein modules]
The $0$-th blob homology of $X$ is the usual 
(dual) TQFT Hilbert space (a.k.a.\ skein module) associated to $X$
by $\cF$.
(See \S \ref{sec:local-relations}.)
\begin{equation*}
H_0(\bc_*(X;\cF)) \iso A_{\cF}(X)
\end{equation*}
\end{thm:skein-modules}

\newtheorem*{thm:hochschild}{Theorem \ref{thm:hochschild}}

\begin{thm:hochschild}[Hochschild homology when $X=S^1$]
The blob complex for a $1$-category $\cC$ on the circle is
quasi-isomorphic to the Hochschild complex.
\begin{equation*}
\xymatrix{\bc_*(S^1;\cC) \ar[r]^(0.47){\iso}_(0.47){\text{qi}} & \HC_*(\cC).}
\end{equation*}
\end{thm:hochschild}

Proposition \ref{thm:skein-modules} is immediate from the definition, and
Theorem \ref{thm:hochschild} is established in \S \ref{sec:hochschild}.
%We also note \S \ref{sec:comm_alg} which describes the blob complex when $\cC$ is a one of 
%certain commutative algebras thought of as $n$-categories.


\subsection{Structure of the blob complex}
\label{sec:structure}

In the following $\CH{X}$ is the singular chain complex of the space of homeomorphisms of $X$, fixed on $\bdy X$.

\newtheorem*{thm:CH}{Theorem \ref{thm:CH}}

\begin{thm:CH}[$C_*(\Homeo(-))$ action]
There is a chain map
\begin{equation*}
e_X: \CH{X} \tensor \bc_*(X) \to \bc_*(X).
\end{equation*}
such that
\begin{enumerate}
\item Restricted to $C_0(\Homeo(X))$ this is the action of homeomorphisms described in Property \ref{property:functoriality}. 

\item For
any codimension $0$-submanifold $Y \sqcup Y^\text{op} \subset \bdy X$ the following diagram
(using the gluing maps described in Property \ref{property:gluing-map}) commutes (up to homotopy).
\begin{equation*}
\xymatrix@C+2cm{
     \CH{X} \otimes \bc_*(X)
        \ar[r]_{e_{X}}  \ar[d]^{\gl^{\Homeo}_Y \otimes \gl_Y}  &
            \bc_*(X) \ar[d]_{\gl_Y} \\
     \CH{X \bigcup_Y \selfarrow} \otimes \bc_*(X \bigcup_Y \selfarrow) \ar[r]^<<<<<<<<<<<<{e_{(X \bigcup_Y \scalebox{0.5}{\selfarrow})}}    & \bc_*(X \bigcup_Y \selfarrow)
}
\end{equation*}
\end{enumerate}
%Moreover any such chain map is unique, up to an iterated homotopy.
%(That is, any pair of homotopies have a homotopy between them, and so on.)
%\nn{revisit this after proof below has stabilized}
\end{thm:CH}

\newtheorem*{thm:CH-associativity}{Theorem \ref{thm:CH-associativity}}


Further,
\begin{thm:CH-associativity}
The chain map of Theorem \ref{thm:CH} is associative, in the sense that the following diagram commutes (up to homotopy).
\begin{equation*}
\xymatrix{
\CH{X} \tensor \CH{X} \tensor \bc_*(X) \ar[r]^<<<<<{\id \tensor e_X} \ar[d]^{\compose \tensor \id} & \CH{X} \tensor \bc_*(X) \ar[d]^{e_X} \\
\CH{X} \tensor \bc_*(X) \ar[r]^{e_X} & \bc_*(X)
}
\end{equation*}
\end{thm:CH-associativity}

Since the blob complex is functorial in the manifold $X$, this is equivalent to having chain maps
$$ev_{X \to Y} : \CH{X \to Y} \tensor \bc_*(X) \to \bc_*(Y)$$
for any homeomorphic pair $X$ and $Y$, 
satisfying corresponding conditions.

In \S \ref{sec:ncats} we introduce the notion of disk-like $n$-categories, 
from which we can construct systems of fields.
Below, when we talk about the blob complex for a disk-like $n$-category, 
we are implicitly passing first to this associated system of fields.
Further, in \S \ref{sec:ncats} we also have the notion of an $A_\infty$ $n$-category. 
In that section we describe how to use the blob complex to 
construct $A_\infty$ $n$-categories from ordinary $n$-categories:

\newtheorem*{ex:blob-complexes-of-balls}{Example \ref{ex:blob-complexes-of-balls}}

\begin{ex:blob-complexes-of-balls}[Blob complexes of products with balls form an $A_\infty$ $n$-category]
%\label{thm:blobs-ainfty}
Let $\cC$ be  an ordinary $n$-category.
Let $Y$ be an $n{-}k$-manifold. 
There is an $A_\infty$ $k$-category $\bc_*(Y;\cC)$, defined on each $m$-ball $D$, for $0 \leq m < k$, 
to be the set $$\bc_*(Y;\cC)(D) = \cC(Y \times D)$$ and on $k$-balls $D$ to be the set 
$$\bc_*(Y;\cC)(D) = \bc_*(Y \times D; \cC).$$ 
(When $m=k$ the subsets with fixed boundary conditions form a chain complex.) 
These sets have the structure of an $A_\infty$ $k$-category, with compositions coming from the gluing map in 
Property \ref{property:gluing-map} and with the action of families of homeomorphisms given in Theorem \ref{thm:evaluation}.
\end{ex:blob-complexes-of-balls}
\begin{rem}
Perhaps the most interesting case is when $Y$ is just a point; 
then we have a way of building an $A_\infty$ $n$-category from an ordinary $n$-category.
We think of this $A_\infty$ $n$-category as a free resolution.
\end{rem}

There is a version of the blob complex for $\cC$ an $A_\infty$ $n$-category
instead of an ordinary $n$-category; this is described in \S \ref{sec:ainfblob}.
The definition is in fact simpler, almost tautological, and we use a different notation, $\cl{\cC}(M)$. 
The next theorem describes the blob complex for product manifolds, 
in terms of the $A_\infty$ blob complex of the $A_\infty$ $n$-categories constructed as in the previous example.
%The notation is intended to reflect the close parallel with the definition of the TQFT skein module via a colimit.

\newtheorem*{thm:product}{Theorem \ref{thm:product}}

\begin{thm:product}[Product formula]
Let $W$ be a $k$-manifold and $Y$ be an $n-k$ manifold.
Let $\cC$ be an $n$-category.
Let $\bc_*(Y;\cC)$ be the $A_\infty$ $k$-category associated to $Y$ via blob homology 
(see Example \ref{ex:blob-complexes-of-balls}).
Then
\[
	\bc_*(Y\times W; \cC) \simeq \cl{\bc_*(Y;\cC)}(W).
\]
\end{thm:product}
The statement can be generalized to arbitrary fibre bundles, and indeed to arbitrary maps
(see \S \ref{ss:product-formula}).

Fix a disk-like $n$-category $\cC$, which we'll omit from the notation.
Recall that for any $(n-1)$-manifold $Y$, the blob complex $\bc_*(Y)$ is naturally an $A_\infty$ category.
(See Appendix \ref{sec:comparing-A-infty} for the translation between disk-like $A_\infty$ $1$-categories and the usual algebraic notion of an $A_\infty$ category.)

\newtheorem*{thm:gluing}{Theorem \ref{thm:gluing}}

\begin{thm:gluing}[Gluing formula]
\mbox{}% <-- gets the indenting right
\begin{itemize}
\item For any $n$-manifold $X$, with $Y$ a codimension $0$-submanifold of its boundary, the blob complex of $X$ is naturally an
$A_\infty$ module for $\bc_*(Y)$.

\item For any $n$-manifold $X_\text{gl} = X\bigcup_Y \selfarrow$, the blob complex $\bc_*(X_\text{gl})$ is the $A_\infty$ self-tensor product of
$\bc_*(X)$ as an $\bc_*(Y)$-bimodule:
\begin{equation*}
\bc_*(X_\text{gl}) \simeq \bc_*(X) \Tensor^{A_\infty}_{\mathclap{\bc_*(Y)}} \selfarrow
\end{equation*}
\end{itemize}
\end{thm:gluing}

Theorem \ref{thm:product} is proved in \S \ref{ss:product-formula}, and Theorem \ref{thm:gluing} in \S \ref{sec:gluing}.

\subsection{Applications}
\label{sec:applications}
Finally, we give two applications of the above machinery.

\newtheorem*{thm:map-recon}{Theorem \ref{thm:map-recon}}

\begin{thm:map-recon}[Mapping spaces]
Let $\pi^\infty_{\le n}(T)$ denote the $A_\infty$ $n$-category based on maps 
$B^n \to T$.
(The case $n=1$ is the usual $A_\infty$-category of paths in $T$.)
Then 
$$\bc_*(X; \pi^\infty_{\le n}(T)) \simeq \CM{X}{T}.$$
\end{thm:map-recon}

This says that we can recover (up to homotopy) the space of maps to $T$ via blob homology from local data. 
Note that there is no restriction on the connectivity of $T$.
The proof appears in \S \ref{sec:map-recon}.

\newtheorem*{thm:deligne}{Theorem \ref{thm:deligne}}

\begin{thm:deligne}[Higher dimensional Deligne conjecture]
The singular chains of the $n$-dimensional surgery cylinder operad act on blob cochains.
Since the little $n{+}1$-balls operad is a suboperad of the $n$-dimensional surgery cylinder operad,
this implies that the little $n{+}1$-balls operad acts on blob cochains of the $n$-ball.
\end{thm:deligne}
See \S \ref{sec:deligne} for a full explanation of the statement, and the proof.



\noop{ %%%%%%%%%%%%%%%%%%%%%%%%%%%%%%
\subsection{Future directions}
\label{sec:future}
\nn{KW: Perhaps we should delete this subsection and salvage only the first few sentences.}
Throughout, we have resisted the temptation to work in the greatest generality possible.
(Don't worry, it wasn't that hard.)
In most of the places where we say ``set" or ``vector space", any symmetric monoidal category would do.
We could also replace many of our chain complexes with topological spaces (or indeed, work at the generality of model categories).
%%%%%%
And likely it will prove useful to think about the connections between what we do here and $(\infty,k)$-categories.
More could be said about finite characteristic 
(there appears in be $2$-torsion in $\bc_1(S^2; \cC)$ for any spherical $2$-category $\cC$, for example).
Much more could be said about other types of manifolds, in particular oriented, 
$\operatorname{Spin}$ and $\operatorname{Pin}^{\pm}$ manifolds, where boundary issues become more complicated.
(We'd recommend thinking about boundaries as germs, rather than just codimension $1$ manifolds.) 
We've also take the path of least resistance by considering $\operatorname{PL}$ manifolds; 
there may be some differences for topological manifolds and smooth manifolds.

The paper ``Skein homology'' \cite{MR1624157} has similar motivations, and it may be 
interesting to investigate if there is a connection with the material here.

Many results in Hochschild homology can be understood ``topologically" via the blob complex.
For example, we expect that the shuffle product on the Hochschild homology of a commutative algebra $A$ 
(see \cite[\S 4.2]{MR1600246}) simply corresponds to the gluing operation on $\bc_*(S^1 \times [0,1]; A)$, 
but haven't investigated the details.

Most importantly, however, \nn{applications!} \nn{cyclic homology, $n=2$ cases, contact, Kh} \nn{stabilization} \nn{stable categories, generalized cohomology theories}
} %%% end \noop %%%%%%%%%%%%%%%%%%%%%

\subsection{\texorpdfstring{$n$}{n}-category terminology}
\label{n-cat-names}

Section \S \ref{sec:ncats} adds to the zoo of $n$-category definitions, and the new creatures need names.
Unfortunately, we have found it difficult to come up with terminology which satisfies all
of the colleagues whom we have consulted, or even satisfies just ourselves.

One distinction we need to make is between $n$-categories which are associative in dimension $n$ and those
that are associative only up to higher homotopies.
The latter are closely related to $(\infty, n)$-categories (i.e.\ $\infty$-categories where all morphisms
of dimension greater than $n$ are invertible), but we don't want to use that name
since we think of the higher homotopies not as morphisms of the $n$-category but
rather as belonging to some auxiliary category (like chain complexes)
that we are enriching in.
We have decided to call them ``$A_\infty$ $n$-categories", since they are a natural generalization 
of the familiar $A_\infty$ 1-categories.
We also considered the names ``homotopy $n$-categories" and ``infinity $n$-categories".
When we need to emphasize that we are talking about an $n$-category which is not $A_\infty$ in this sense
we will say ``ordinary $n$-category".
% small problem: our n-cats are of course strictly associative, since we have more morphisms.
% when we say ``associative only up to homotopy" above we are thinking about
% what would happen we we tried to convert to a more traditional n-cat with fewer morphisms

Another distinction we need to make is between our style of definition of $n$-categories and
more traditional and combinatorial definitions.
We will call instances of our definition ``disk-like $n$-categories", since $n$-dimensional disks
play a prominent role in the definition.
(In general we prefer ``$k$-ball" to ``$k$-disk", but ``ball-like" doesn't roll off 
the tongue as well as ``disk-like''.)

Another thing we need a name for is the ability to rotate morphisms around in various ways.
For 2-categories, ``pivotal" is a standard term for what we mean.
A more general term is ``duality", but duality comes in various flavors and degrees.
We are mainly interested in a very strong version of duality, where the available ways of
rotating $k$-morphisms correspond to all the ways of rotating $k$-balls.
We sometimes refer to this as ``strong duality", and sometimes we consider it to be implied
by ``disk-like".
(But beware: disks can come in various flavors, and some of them, such as framed disks,
don't actually imply much duality.)
Another possibility considered here was ``pivotal $n$-category", but we prefer to preserve pivotal for its usual sense. 
It will thus be a theorem that our disk-like 2-categories 
are equivalent to pivotal 2-categories, c.f. \S \ref{ssec:2-cats}.

Finally, we need a general name for isomorphisms between balls, where the balls could be
piecewise linear or smooth or topological or Spin or framed or etc., or some combination thereof.
We have chosen to use ``homeomorphism" for the appropriate sort of isomorphism, so the reader should
keep in mind that ``homeomorphism" could mean PL homeomorphism or diffeomorphism (and so on)
depending on context.

\subsection{Thanks and acknowledgements}
% attempting to make this chronological rather than alphabetical
We'd like to thank 
Justin Roberts (for helpful discussions in the very early stages of this work), 
Michael Freedman, 
Peter Teichner (for helping us improve an earlier version of the $n$-category definition), 
David Ben-Zvi, 
Vaughan Jones, 
Chris Schommer-Pries, 
Thomas Tradler,
Kevin Costello, 
Chris Douglas,
Alexander Kirillov,
and
Michael Shulman
for many interesting and useful conversations. 
Peter Teichner ran a reading course based on an earlier draft of this paper, and the detailed feedback
we got from the student lecturers lead to very many improvements in later drafts.
So big thanks to
Aaron Mazel-Gee,
Nate Watson,
Alan Wilder,
Dmitri Pavlov,
Ansgar Schneider,
and
Dan Berwick-Evans.
During this work, Kevin Walker has been at Microsoft Station Q, and Scott Morrison has been at 
Microsoft Station Q and the Miller Institute for Basic Research at UC Berkeley. 
We'd like to thank the Aspen Center for Physics for the pleasant and productive 
environment provided there during the final preparation of this manuscript.



%!TEX root = ../blob1.tex

\section{TQFTs via fields}
\label{sec:fields}
\label{sec:tqftsviafields}

In this section we review the notion of a ``system of fields and local relations".
For more details see \cite{kw:tqft}. From a system of fields and local relations we can readily construct TQFT invariants of manifolds. This is described in \S \ref{sec:constructing-a-TQFT}. A system of fields is very closely related to an $n$-category. In Example \ref{ex:traditional-n-categories(fields)}, which runs throughout this section, we sketch the construction of a system of fields from an $n$-category. We make this more precise for $n=1$ or $2$ in \S \ref{sec:example:traditional-n-categories(fields)}, and much later, after we've have given our own definition of a `topological $n$-category' in \S \ref{sec:ncats}, we explain precisely how to go back and forth between a topological $n$-category and a system of fields and local relations.

We only consider compact manifolds, so if $Y \sub X$ is a closed codimension 0
submanifold of $X$, then $X \setmin Y$ implicitly means the closure
$\overline{X \setmin Y}$.


\subsection{Systems of fields}

Let $\cM_k$ denote the category with objects 
unoriented PL manifolds of dimension
$k$ and morphisms homeomorphisms.
(We could equally well work with a different category of manifolds ---
oriented, topological, smooth, spin, etc. --- but for definiteness we
will stick with unoriented PL.)

Fix a symmetric monoidal category $\cS$ whose objects are sets. While reading the definition, you should just think about the cases $\cS = \Set$ or $\cS = \Vect$.

A $n$-dimensional {\it system of fields} in $\cS$
is a collection of functors $\cC_k : \cM_k \to \Set$ for $0 \leq k \leq n$
together with some additional data and satisfying some additional conditions, all specified below.

Before finishing the definition of fields, we give two motivating examples of systems of fields.

\begin{example}
\label{ex:maps-to-a-space(fields)}
Fix a target space $T$, and let $\cC(X)$ be the set of continuous maps
from $X$ to $T$.
\end{example}

\begin{example}
\label{ex:traditional-n-categories(fields)}
Fix an $n$-category $C$, and let $\cC(X)$ be 
the set of sub-cell-complexes of $X$ with codimension-$j$ cells labeled by
$j$-morphisms of $C$.
One can think of such sub-cell-complexes as dual to pasting diagrams for $C$.
This is described in more detail in \S \ref{sec:example:traditional-n-categories(fields)}.
\end{example}

Now for the rest of the definition of system of fields.
\begin{enumerate}
\item There are boundary restriction maps $\cC_k(X) \to \cC_{k-1}(\bd X)$, 
and these maps are a natural
transformation between the functors $\cC_k$ and $\cC_{k-1}\circ\bd$.
For $c \in \cC_{k-1}(\bd X)$, we will denote by $\cC_k(X; c)$ the subset of 
$\cC(X)$ which restricts to $c$.
In this context, we will call $c$ a boundary condition.
\item The subset $\cC_n(X;c)$ of top fields with a given boundary condition is an object in our symmetric monoidal category $\cS$. (This condition is of course trivial when $\cS = \Set$.) If the objects are sets with extra structure (e.g. $\cS = \Vect$ or $\Kom$), then this extra structure is considered part of the definition of $\cC_n$. Any maps mentioned below between top level fields must be morphisms in $\cS$.
\item $\cC_k$ is compatible with the symmetric monoidal
structures on $\cM_k$, $\Set$ and $\cS$: $\cC_k(X \du W) \cong \cC_k(X)\times \cC_k(W)$,
compatibly with homeomorphisms, restriction to boundary, and orientation reversal.
We will call the projections $\cC(X_1 \du X_2) \to \cC(X_i)$
restriction maps.
\item Gluing without corners.
Let $\bd X = Y \du -Y \du W$, where $Y$ and $W$ are closed $k{-}1$-manifolds.
Let $X\sgl$ denote $X$ glued to itself along $\pm Y$.
Using the boundary restriction, disjoint union, and (in one case) orientation reversal
maps, we get two maps $\cC_k(X) \to \cC(Y)$, corresponding to the two
copies of $Y$ in $\bd X$.
Let $\Eq_Y(\cC_k(X))$ denote the equalizer of these two maps.
Then (here's the axiom/definition part) there is an injective ``gluing" map
\[
	\Eq_Y(\cC_k(X)) \hookrightarrow \cC_k(X\sgl) ,
\]
and this gluing map is compatible with all of the above structure (actions
of homeomorphisms, boundary restrictions, orientation reversal, disjoint union).
Furthermore, up to homeomorphisms of $X\sgl$ isotopic to the identity,
the gluing map is surjective.
From the point of view of $X\sgl$ and the image $Y \subset X\sgl$ of the 
gluing surface, we say that fields in the image of the gluing map
are transverse to $Y$ or splittable along $Y$.
\item Gluing with corners.
Let $\bd X = Y \cup -Y \cup W$, where $\pm Y$ and $W$ might intersect along their boundaries.
Let $X\sgl$ denote $X$ glued to itself along $\pm Y$.
Note that $\bd X\sgl = W\sgl$, where $W\sgl$ denotes $W$ glued to itself
(without corners) along two copies of $\bd Y$.
Let $c\sgl \in \cC_{k-1}(W\sgl)$ be a be a splittable field on $W\sgl$ and let
$c \in \cC_{k-1}(W)$ be the cut open version of $c\sgl$.
Let $\cC^c_k(X)$ denote the subset of $\cC(X)$ which restricts to $c$ on $W$.
(This restriction map uses the gluing without corners map above.)
Using the boundary restriction, gluing without corners, and (in one case) orientation reversal
maps, we get two maps $\cC^c_k(X) \to \cC(Y)$, corresponding to the two
copies of $Y$ in $\bd X$.
Let $\Eq^c_Y(\cC_k(X))$ denote the equalizer of these two maps.
Then (here's the axiom/definition part) there is an injective ``gluing" map
\[
	\Eq^c_Y(\cC_k(X)) \hookrightarrow \cC_k(X\sgl, c\sgl) ,
\]
and this gluing map is compatible with all of the above structure (actions
of homeomorphisms, boundary restrictions, orientation reversal, disjoint union).
Furthermore, up to homeomorphisms of $X\sgl$ isotopic to the identity,
the gluing map is surjective.
From the point of view of $X\sgl$ and the image $Y \subset X\sgl$ of the 
gluing surface, we say that fields in the image of the gluing map
are transverse to $Y$ or splittable along $Y$.
\item There are maps $\cC_{k-1}(Y) \to \cC_k(Y \times I)$, denoted
$c \mapsto c\times I$.
These maps comprise a natural transformation of functors, and commute appropriately
with all the structure maps above (disjoint union, boundary restriction, etc.).
Furthermore, if $f: Y\times I \to Y\times I$ is a fiber-preserving homeomorphism
covering $\bar{f}:Y\to Y$, then $f(c\times I) = \bar{f}(c)\times I$.
\end{enumerate}

There are two notations we commonly use for gluing.
One is 
\[
	x\sgl \deq \gl(x) \in \cC(X\sgl) , 
\]
for $x\in\cC(X)$.
The other is
\[
	x_1\bullet x_2 \deq \gl(x_1\otimes x_2) \in \cC(X\sgl) , 
\]
in the case that $X = X_1 \du X_2$, with $x_i \in \cC(X_i)$.

\medskip

Using the functoriality and $\cdot\times I$ properties above, together
with boundary collar homeomorphisms of manifolds, we can define the notion of 
{\it extended isotopy}.
Let $M$ be an $n$-manifold and $Y \subset \bd M$ be a codimension zero submanifold
of $\bd M$.
Let $x \in \cC(M)$ be a field on $M$ and such that $\bd x$ is splittable along $\bd Y$.
Let $c$ be $x$ restricted to $Y$.
Let $M \cup (Y\times I)$ denote $M$ glued to $Y\times I$ along $Y$.
Then we have the glued field $x \bullet (c\times I)$ on $M \cup (Y\times I)$.
Let $f: M \cup (Y\times I) \to M$ be a collaring homeomorphism.
Then we say that $x$ is {\it extended isotopic} to $f(x \bullet (c\times I))$.
More generally, we define extended isotopy to be the equivalence relation on fields
on $M$ generated by isotopy plus all instance of the above construction
(for all appropriate $Y$ and $x$).

\nn{should also say something about pseudo-isotopy}


\nn{remark that if top dimensional fields are not already linear
then we will soon linearize them(?)}

For top dimensional ($n$-dimensional) manifolds, we're actually interested
in the linearized space of fields.
By default, define $\lf(X) = \c[\cC(X)]$; that is, $\lf(X)$ is
the vector space of finite
linear combinations of fields on $X$.
If $X$ has boundary, we of course fix a boundary condition: $\lf(X; a) = \c[\cC(X; a)]$.
Thus the restriction (to boundary) maps are well defined because we never
take linear combinations of fields with differing boundary conditions.

In some cases we don't linearize the default way; instead we take the
spaces $\lf(X; a)$ to be part of the data for the system of fields.
In particular, for fields based on linear $n$-category pictures we linearize as follows.
Define $\lf(X; a) = \c[\cC(X; a)]/K$, where $K$ is the space generated by
obvious relations on 0-cell labels.
More specifically, let $L$ be a cell decomposition of $X$
and let $p$ be a 0-cell of $L$.
Let $\alpha_c$ and $\alpha_d$ be two labelings of $L$ which are identical except that
$\alpha_c$ labels $p$ by $c$ and $\alpha_d$ labels $p$ by $d$.
Then the subspace $K$ is generated by things of the form
$\lambda \alpha_c + \alpha_d - \alpha_{\lambda c + d}$, where we leave it to the reader
to infer the meaning of $\alpha_{\lambda c + d}$.
Note that we are still assuming that $n$-categories have linear spaces of $n$-morphisms.

\nn{Maybe comment further: if there's a natural basis of morphisms, then no need;
will do something similar below; in general, whenever a label lives in a linear
space we do something like this; ? say something about tensor
product of all the linear label spaces?  Yes:}

For top dimensional ($n$-dimensional) manifolds, we linearize as follows.
Define an ``almost-field" to be a field without labels on the 0-cells.
(Recall that 0-cells are labeled by $n$-morphisms.)
To each unlabeled 0-cell in an almost field there corresponds a (linear) $n$-morphism
space determined by the labeling of the link of the 0-cell.
(If the 0-cell were labeled, the label would live in this space.)
We associate to each almost-labeling the tensor product of these spaces (one for each 0-cell).
We now define $\lf(X; a)$ to be the direct sum over all almost labelings of the
above tensor products.


\subsection{Systems of fields from $n$-categories}
\label{sec:example:traditional-n-categories(fields)}
We now describe in more detail Example \ref{ex:traditional-n-categories(fields)}, systems of fields coming from sub-cell-complexes labeled
by $n$-category morphisms.

Given an $n$-category $C$ with the right sort of duality
(e.g. a pivotal 2-category, 1-category with duals, star 1-category, disklike $n$-category),
we can construct a system of fields as follows.
Roughly speaking, $\cC(X)$ will the set of all embedded cell complexes in $X$
with codimension $i$ cells labeled by $i$-morphisms of $C$.
We'll spell this out for $n=1,2$ and then describe the general case.

If $X$ has boundary, we require that the cell decompositions are in general
position with respect to the boundary --- the boundary intersects each cell
transversely, so cells meeting the boundary are mere half-cells. Put another way, the cell decompositions we consider are dual to standard cell
decompositions of $X$.

We will always assume that our $n$-categories have linear $n$-morphisms.

For $n=1$, a field on a 0-manifold $P$ is a labeling of each point of $P$ with
an object (0-morphism) of the 1-category $C$.
A field on a 1-manifold $S$ consists of
\begin{itemize}
    \item a cell decomposition of $S$ (equivalently, a finite collection
of points in the interior of $S$);
    \item a labeling of each 1-cell (and each half 1-cell adjacent to $\bd S$)
by an object (0-morphism) of $C$;
    \item a transverse orientation of each 0-cell, thought of as a choice of
``domain" and ``range" for the two adjacent 1-cells; and
    \item a labeling of each 0-cell by a morphism (1-morphism) of $C$, with
domain and range determined by the transverse orientation and the labelings of the 1-cells.
\end{itemize}

If $C$ is an algebra (i.e. if $C$ has only one 0-morphism) we can ignore the labels
of 1-cells, so a field on a 1-manifold $S$ is a finite collection of points in the
interior of $S$, each transversely oriented and each labeled by an element (1-morphism)
of the algebra.

\medskip

For $n=2$, fields are just the sort of pictures based on 2-categories (e.g.\ tensor categories)
that are common in the literature.
We describe these carefully here.

A field on a 0-manifold $P$ is a labeling of each point of $P$ with
an object of the 2-category $C$.
A field of a 1-manifold is defined as in the $n=1$ case, using the 0- and 1-morphisms of $C$.
A field on a 2-manifold $Y$ consists of
\begin{itemize}
    \item a cell decomposition of $Y$ (equivalently, a graph embedded in $Y$ such
that each component of the complement is homeomorphic to a disk);
    \item a labeling of each 2-cell (and each partial 2-cell adjacent to $\bd Y$)
by a 0-morphism of $C$;
    \item a transverse orientation of each 1-cell, thought of as a choice of
``domain" and ``range" for the two adjacent 2-cells;
    \item a labeling of each 1-cell by a 1-morphism of $C$, with
domain and range determined by the transverse orientation of the 1-cell
and the labelings of the 2-cells;
    \item for each 0-cell, a homeomorphism of the boundary $R$ of a small neighborhood
of the 0-cell to $S^1$ such that the intersections of the 1-cells with $R$ are not mapped
to $\pm 1 \in S^1$; and
    \item a labeling of each 0-cell by a 2-morphism of $C$, with domain and range
determined by the labelings of the 1-cells and the parameterizations of the previous
bullet.
\end{itemize}
\nn{need to say this better; don't try to fit everything into the bulleted list}

For general $n$, a field on a $k$-manifold $X^k$ consists of
\begin{itemize}
    \item A cell decomposition of $X$;
    \item an explicit general position homeomorphism from the link of each $j$-cell
to the boundary of the standard $(k-j)$-dimensional bihedron; and
    \item a labeling of each $j$-cell by a $(k-j)$-dimensional morphism of $C$, with
domain and range determined by the labelings of the link of $j$-cell.
\end{itemize}

%\nn{next definition might need some work; I think linearity relations should
%be treated differently (segregated) from other local relations, but I'm not sure
%the next definition is the best way to do it}

\medskip




\subsection{Local relations}
\label{sec:local-relations}
Local relations are certain subspaces of the fields on balls, which form an ideal under gluing. Again, we give the examples first.

\addtocounter{prop}{-2}
\begin{example}[contd.]
For maps into spaces, $U(B; c)$ is generated by fields of the form $a-b \in \lf(B; c)$,
where $a$ and $b$ are maps (fields) which are homotopic rel boundary.
\end{example}

\begin{example}[contd.]
For $n$-category pictures, $U(B; c)$ is equal to the kernel of the evaluation map
$\lf(B; c) \to \mor(c', c'')$, where $(c', c'')$ is some (any) division of $c$ into
domain and range.
\end{example}

These motivate the following definition.

\begin{defn}
A {\it local relation} is a collection subspaces $U(B; c) \sub \lf(B; c)$,
for all $n$-manifolds $B$ which are
homeomorphic to the standard $n$-ball and all $c \in \cC(\bd B)$, 
satisfying the following properties.
\begin{enumerate}
\item functoriality: 
$f(U(B; c)) = U(B', f(c))$ for all homeomorphisms $f: B \to B'$
\item local relations imply extended isotopy: 
if $x, y \in \cC(B; c)$ and $x$ is extended isotopic 
to $y$, then $x-y \in U(B; c)$.
\item ideal with respect to gluing:
if $B = B' \cup B''$, $x\in U(B')$, and $c\in \cC(B'')$, then $x\bullet r \in U(B)$
\end{enumerate}
\end{defn}
See \cite{kw:tqft} for further details.


\subsection{Constructing a TQFT}
\label{sec:constructing-a-tqft}

In this subsection we briefly review the construction of a TQFT from a system of fields and local relations.
(For more details, see \cite{kw:tqft}.)

Let $W$ be an $n{+}1$-manifold.
We can think of the path integral $Z(W)$ as assigning to each
boundary condition $x\in \cC(\bd W)$ a complex number $Z(W)(x)$.
In other words, $Z(W)$ lies in $\c^{\lf(\bd W)}$, the vector space of linear
maps $\lf(\bd W)\to \c$.

The locality of the TQFT implies that $Z(W)$ in fact lies in a subspace
$Z(\bd W) \sub \c^{\lf(\bd W)}$ defined by local projections.
The linear dual to this subspace, $A(\bd W) = Z(\bd W)^*$,
can be thought of as finite linear combinations of fields modulo local relations.
(In other words, $A(\bd W)$ is a sort of generalized skein module.)
This is the motivation behind the definition of fields and local relations above.

In more detail, let $X$ be an $n$-manifold.
%To harmonize notation with the next section, 
%let $\bc_0(X)$ be the vector space of finite linear combinations of fields on $X$, so
%$\bc_0(X) = \lf(X)$.
Define $U(X) \sub \lf(X)$ to be the space of local relations in $\lf(X)$;
$U(X)$ is generated by things of the form $u\bullet r$, where
$u\in U(B)$ for some embedded $n$-ball $B\sub X$ and $r\in \cC(X\setmin B)$.
Define
\[
	A(X) \deq \lf(X) / U(X) .
\]
(The blob complex, defined in the next section, 
is in some sense the derived version of $A(X)$.)
If $X$ has boundary we can similarly define $A(X; c)$ for each 
boundary condition $c\in\cC(\bd X)$.

The above construction can be extended to higher codimensions, assigning
a $k$-category $A(Y)$ to an $n{-}k$-manifold $Y$, for $0 \le k \le n$.
These invariants fit together via actions and gluing formulas.
We describe only the case $k=1$ below.
(The construction of the $n{+}1$-dimensional part of the theory (the path integral) 
requires that the starting data (fields and local relations) satisfy additional
conditions.
We do not assume these conditions here, so when we say ``TQFT" we mean a decapitated TQFT
that lacks its $n{+}1$-dimensional part.)

Let $Y$ be an $n{-}1$-manifold.
Define a (linear) 1-category $A(Y)$ as follows.
The objects of $A(Y)$ are $\cC(Y)$.
The morphisms from $a$ to $b$ are $A(Y\times I; a, b)$, where $a$ and $b$ label the two boundary components of the cylinder $Y\times I$.
Composition is given by gluing of cylinders.

Let $X$ be an $n$-manifold with boundary and consider the collection of vector spaces
$A(X; \cdot) \deq \{A(X; c)\}$ where $c$ ranges through $\cC(\bd X)$.
This collection of vector spaces affords a representation of the category $A(\bd X)$, where
the action is given by gluing a collar $\bd X\times I$ to $X$.

Given a splitting $X = X_1 \cup_Y X_2$ of a closed $n$-manifold $X$ along an $n{-}1$-manifold $Y$,
we have left and right actions of $A(Y)$ on $A(X_1; \cdot)$ and $A(X_2; \cdot)$.
The gluing theorem for $n$-manifolds states that there is a natural isomorphism
\[
	A(X) \cong A(X_1; \cdot) \otimes_{A(Y)} A(X_2; \cdot) .
\]



%!TEX root = ../blob1.tex

\section{The blob complex}
\subsection{Definitions}
\label{sec:blob-definition}
Let $X$ be an $n$-manifold.
Let $(\cF,U)$ be a fixed system of fields and local relations.
We'll assume it is enriched over \textbf{Vect}; 
if it is not we can make it so by allowing finite
linear combinations of elements of $\cF(X; c)$, for fixed $c\in \cF(\bd X)$.

%In this section we will usually suppress boundary conditions on $X$ from the notation, e.g. by writing $\cF(X)$ instead of $\cF(X; c)$.

We want to replace the quotient
\[
	A(X) \deq \cF(X) / U(X)
\]
of Definition \ref{defn:TQFT-invariant} with a resolution
\[
	\cdots \to \bc_2(X) \to \bc_1(X) \to \bc_0(X) .
\]

We will define $\bc_0(X)$, $\bc_1(X)$ and $\bc_2(X)$, then give the general case $\bc_k(X)$. 
In fact, on the first pass we will intentionally describe the definition in a misleadingly simple way, 
then explain the technical difficulties, and finally give a cumbersome but complete definition in 
Definition \ref{defn:blobs}. 
If (we don't recommend it) you want to keep track of the ways in which 
this initial description is misleading, or you're reading through a second time to understand the 
technical difficulties, keep note that later we will give precise meanings to ``a ball in $X$'', 
``nested'' and ``disjoint'', that are not quite the intuitive ones. 
Moreover some of the pieces 
into which we cut manifolds below are not themselves manifolds, and it requires special attention 
to define fields on these pieces.

We of course define $\bc_0(X) = \cF(X)$.
In other words, $\bc_0(X)$ is just the vector space of all fields on $X$.

(If $X$ has nonempty boundary, instead define $\bc_0(X; c) = \cF(X; c)$ for $c \in \cF(\bdy X)$.
The blob complex $\bc_*(X; c)$ will depend on a fixed boundary condition $c\in \cF(\bdy X)$.
We'll omit such boundary conditions from the notation in the rest of this section.)

We want the vector space $\bc_1(X)$ to capture 
``the space of all local relations that can be imposed on $\bc_0(X)$".
Thus we say  a $1$-blob diagram consists of:
\begin{itemize}
\item An closed ball in $X$ (``blob") $B \sub X$.
\item A boundary condition $c \in \cF(\bdy B) = \cF(\bd(X \setmin B))$.
\item A field $r \in \cF(X \setmin B; c)$.
\item A local relation field $u \in U(B; c)$.
\end{itemize}
(See Figure \ref{blob1diagram}.) Since $c$ is implicitly determined by $u$ or $r$, we usually omit it from the notation.
\begin{figure}[t]\begin{equation*}
\mathfig{.6}{definition/single-blob}
\end{equation*}\caption{A 1-blob diagram.}\label{blob1diagram}\end{figure}
In order to get the linear structure correct, we define
\[
	\bc_1(X) \deq \bigoplus_B \bigoplus_c U(B; c) \otimes \cF(X \setmin B; c) .
\]
The first direct sum is indexed by all blobs $B\subset X$, and the second
by all boundary conditions $c \in \cF(\bd B)$.
Note that $\bc_1(X)$ is spanned by 1-blob diagrams $(B, u, r)$.

Define the boundary map $\bd : \bc_1(X) \to \bc_0(X)$ by 
\[ 
	(B, u, r) \mapsto u\bullet r, 
\]
where $u\bullet r$ denotes the field on $X$ obtained by gluing $u$ to $r$.
In other words $\bd : \bc_1(X) \to \bc_0(X)$ is given by
just erasing the blob from the picture
(but keeping the blob label $u$).

Note that directly from the definition we have
\begin{prop}
\label{thm:skein-modules}
The skein module $A(X)$ is naturally isomorphic to $\bc_0(X)/\bd(\bc_1(X))) = H_0(\bc_*(X))$.
\end{prop}
This also establishes the second 
half of Property \ref{property:contractibility}.

Next, we want the vector space $\bc_2(X)$ to capture ``the space of all relations 
(redundancies, syzygies) among the 
local relations encoded in $\bc_1(X)$''.
A $2$-blob diagram, comes in one of two types, disjoint and nested.
A disjoint 2-blob diagram consists of
\begin{itemize}
\item A pair of closed balls (blobs) $B_1, B_2 \sub X$ with disjoint interiors.
\item A field $r \in \cF(X \setmin (B_1 \cup B_2); c_1, c_2)$
(where $c_i \in \cF(\bd B_i)$).
\item Local relation fields $u_i \in U(B_i; c_i)$, $i=1,2$.
\end{itemize}
(See Figure \ref{blob2ddiagram}.)
\begin{figure}[t]\begin{equation*}
\mathfig{.6}{definition/disjoint-blobs}
\end{equation*}\caption{A disjoint 2-blob diagram.}\label{blob2ddiagram}\end{figure}
We also identify $(B_1, B_2, u_1, u_2, r)$ with $-(B_2, B_1, u_2, u_1, r)$;
reversing the order of the blobs changes the sign.
Define $\bd(B_1, B_2, u_1, u_2, r) = 
(B_2, u_2, u_1\bullet r) - (B_1, u_1, u_2\bullet r) \in \bc_1(X)$.
In other words, the boundary of a disjoint 2-blob diagram
is the sum (with alternating signs)
of the two ways of erasing one of the blobs.
It's easy to check that $\bd^2 = 0$.

A nested 2-blob diagram consists of
\begin{itemize}
\item A pair of nested balls (blobs) $B_1 \subseteq B_2 \subseteq X$.
\item A field $r' \in \cF(B_2 \setminus B_1; c_1, c_2)$ 
(for some $c_1 \in \cF(\bdy B_1)$ and $c_2 \in \cF(\bdy B_2)$).
\item A field $r \in \cF(X \setminus B_2; c_2)$.
\item A local relation field $u \in U(B_1; c_1)$.
\end{itemize}
(See Figure \ref{blob2ndiagram}.)
\begin{figure}[t]\begin{equation*}
\mathfig{.6}{definition/nested-blobs}
\end{equation*}\caption{A nested 2-blob diagram.}\label{blob2ndiagram}\end{figure}
Define $\bd(B_1, B_2, u, r', r) = (B_2, u\bullet r', r) - (B_1, u, r' \bullet r)$.
As in the disjoint 2-blob case, the boundary of a nested 2-blob is the alternating
sum of the two ways of erasing one of the blobs.
When  we erase the inner blob, the outer blob inherits the label $u\bullet r'$.
It is again easy to check that $\bd^2 = 0$. Note that the requirement that
local relations are an ideal with respect to gluing guarantees that $u\bullet r' \in U(B_2)$.

As with the $1$-blob diagrams, in order to get the linear structure correct the actual definition is 
\begin{eqnarray*}
	\bc_2(X) & \deq &
	\left( 
		\bigoplus_{B_1, B_2\; \text{disjoint}} \bigoplus_{c_1, c_2}
			U(B_1; c_1) \otimes U(B_2; c_2) \otimes \cF(X\setmin (B_1\cup B_2); c_1, c_2)
	\right)  \bigoplus \\
	&& \quad\quad  \left( 
		\bigoplus_{B_1 \subset B_2} \bigoplus_{c_1, c_2}
			U(B_1; c_1) \otimes \cF(B_2 \setmin B_1; c_1, c_2) \tensor \cF(X \setminus B_2; c_2)
	\right) .
\end{eqnarray*}
% __ (already said this above)
%For the disjoint blobs, reversing the ordering of $B_1$ and $B_2$ introduces a minus sign
%(rather than a new, linearly independent, 2-blob diagram). 

\medskip

Roughly, $\bc_k(X)$ is generated by configurations of $k$ blobs, pairwise disjoint or nested, 
along with fields on all the components that the blobs divide $X$ into. 
Blobs which have no other blobs inside are called `twig blobs', 
and the fields on the twig blobs must be local relations.
The boundary is the alternating sum of erasing one of the blobs.
In order to describe this general case in full detail, we must give a more precise description of
which configurations of balls inside $X$ we permit.
These configurations are generated by two operations:
\begin{itemize}
\item For any (possibly empty) configuration of blobs on an $n$-ball $D$, we can add
$D$ itself as an outermost blob.
(This is used in the proof of Proposition \ref{bcontract}.)
\item If $X\sgl$ is obtained from $X$ by gluing, then any permissible configuration of blobs
on $X$ gives rise to a permissible configuration on $X\sgl$.
(This is necessary for Proposition \ref{blob-gluing}.)
\end{itemize}
Combining these two operations can give rise to configurations of blobs whose complement in $X$ is not
a manifold.
Thus will need to be more careful when speaking of a field $r$ on the complement of the blobs.

\begin{example}
Consider the four subsets of $\Real^3$,
\begin{align*}
A & = [0,1] \times [0,1] \times [0,1] \\
B & = [0,1] \times [-1,0] \times [0,1] \\
C & = [-1,0] \times \setc{(y,z)}{z \sin(1/z) \leq y \leq 1, z \in [0,1]} \\
D & = [-1,0] \times \setc{(y,z)}{-1 \leq y \leq z \sin(1/z), z \in [0,1]}.
\end{align*}
Here $A \cup B = [0,1] \times [-1,1] \times [0,1]$ and $C \cup D = [-1,0] \times [-1,1] \times [0,1]$. 
Now, $\{A\}$ is a valid configuration of blobs in $A \cup B$, 
and $\{D\}$ is a valid configuration of blobs in $C \cup D$, 
so we must allow $\{A, D\}$ as a configuration of blobs in $[-1,1]^2 \times [0,1]$. 
Note however that the complement is not a manifold.
\end{example}

\begin{defn}
\label{defn:gluing-decomposition}
A \emph{gluing decomposition} of an $n$-manifold $X$ is a sequence of manifolds 
$M_0 \to M_1 \to \cdots \to M_m = X$ such that each $M_k$ is obtained from $M_{k-1}$ 
by gluing together some disjoint pair of homeomorphic $n{-}1$-manifolds in the boundary of $M_{k-1}$.
If, in addition, $M_0$ is a disjoint union of balls, we call it a \emph{ball decomposition}.
\end{defn}
Given a gluing decomposition $M_0 \to M_1 \to \cdots \to M_m = X$, we say that a field is 
splittable along it if it is the image of a field on $M_0$.

In the example above, note that
\[
	A \sqcup B \sqcup C \sqcup D \to (A \cup B) \sqcup (C \cup D) \to A \cup B \cup C \cup D
\]
is a  ball decomposition, but other sequences of gluings starting from $A \sqcup B \sqcup C \sqcup D$
have intermediate steps which are not manifolds.

We'll now slightly restrict the possible configurations of blobs.
%%%%% oops -- I missed the similar discussion after the definition
%The basic idea is that each blob in a configuration 
%is the image a ball, with embedded interior and possibly glued-up boundary;
%distinct blobs should either have disjoint interiors or be nested;
%and the entire configuration should be compatible with some gluing decomposition of $X$.
\begin{defn}
\label{defn:configuration}
A configuration of $k$ blobs in $X$ is an ordered collection of $k$ subsets $\{B_1, \ldots B_k\}$ 
of $X$ such that there exists a gluing decomposition $M_0  \to \cdots \to M_m = X$ of $X$ and 
for each subset $B_i$ there is some $0 \leq r \leq m$ and some connected component $M_r'$ of 
$M_r$ which is a ball, so $B_i$ is the image of $M_r'$ in $X$. 
We say that such a gluing decomposition 
is \emph{compatible} with the configuration. 
A blob $B_i$ is a twig blob if no other blob $B_j$ is a strict subset of it. 
\end{defn}
In particular, this implies what we said about blobs above: 
that for any two blobs in a configuration of blobs in $X$, 
they either have disjoint interiors, or one blob is contained in the other. 
We describe these as disjoint blobs and nested blobs. 
Note that nested blobs may have boundaries that overlap, or indeed coincide. 
Blobs may meet the boundary of $X$.
Further, note that blobs need not actually be embedded balls in $X$, since parts of the 
boundary of the ball $M_r'$ may have been glued together.

Note that often the gluing decomposition for a configuration of blobs may just be the trivial one: 
if the boundaries of all the blobs cut $X$ into pieces which are all manifolds, 
we can just take $M_0$ to be these pieces, and $M_1 = X$.

In the informal description above, in the definition of a $k$-blob diagram we asked for any 
collection of $k$ balls which were pairwise disjoint or nested. 
We now further insist that the balls are a configuration in the sense of Definition \ref{defn:configuration}. 
Also, we asked for a local relation on each twig blob, and a field on the complement of the twig blobs; 
this is unsatisfactory because that complement need not be a manifold. Thus, the official definitions are
\begin{defn}
\label{defn:blob-diagram}
A $k$-blob diagram on $X$ consists of
\begin{itemize}
\item a configuration $\{B_1, \ldots B_k\}$ of $k$ blobs in $X$,
\item and a field $r \in \cF(X)$ which is splittable along some gluing decomposition compatible with that configuration,
\end{itemize}
such that
the restriction $u_i$ of $r$ to each twig blob $B_i$ lies in the subspace 
$U(B_i) \subset \cF(B_i)$. 
(See Figure \ref{blobkdiagram}.) 
More precisely, each twig blob $B_i$ is the image of some ball $M_r'$ as above, 
and it is really the restriction to $M_r'$ that must lie in the subspace $U(M_r')$.
\end{defn}
\begin{figure}[t]\begin{equation*}
\mathfig{.7}{definition/k-blobs}
\end{equation*}\caption{A $k$-blob diagram.}\label{blobkdiagram}\end{figure}
and
\begin{defn}
\label{defn:blobs}
The $k$-th vector space $\bc_k(X)$ of the \emph{blob complex} of $X$ is the direct sum over all 
configurations of $k$ blobs in $X$ of the vector space of $k$-blob diagrams with that configuration, 
modulo identifying the vector spaces for configurations that only differ by a permutation of the blobs 
by the sign of that permutation. 
The differential $\bc_k(X) \to \bc_{k-1}(X)$ is, as above, the signed sum of ways of 
forgetting one blob from the configuration, preserving the field $r$:
\begin{equation*}
\bdy(\{B_1, \ldots B_k\}, r) = \sum_{i=1}^{k} (-1)^{i+1} (\{B_1, \ldots, \widehat{B_i}, \ldots, B_k\}, r)
\end{equation*}
\end{defn}
We readily see that if a gluing decomposition is compatible with some configuration of blobs, 
then it is also compatible with any configuration obtained by forgetting some blobs, 
ensuring that the differential in fact lands in the space of $k{-}1$-blob diagrams.
A slight compensation to the complication of the official definition arising from attention 
to splitting is that the differential now just preserves the entire field $r$ without 
having to say anything about gluing together fields on smaller components.

Note that Property \ref{property:functoriality}, that the blob complex is functorial with respect to homeomorphisms, 
is immediately obvious from the definition.
A homeomorphism acts in an obvious way on blobs and on fields.

\begin{remark} \label{blobsset-remark} \rm
We note that blob diagrams in $X$ have a structure similar to that of a simplicial set,
but with simplices replaced by a more general class of combinatorial shapes.
Let $P$ be the minimal set of (isomorphisms classes of) polyhedra which is closed under products
and cones, and which contains the point.
We can associate an element $p(b)$ of $P$ to each blob diagram $b$ 
(equivalently, to each rooted tree) according to the following rules:
\begin{itemize}
\item $p(\emptyset) = pt$, where $\emptyset$ denotes a 0-blob diagram or empty tree;
\item $p(a \du b) = p(a) \times p(b)$, where $a \du b$ denotes the distant (non-overlapping) union 
of two blob diagrams (equivalently, join two trees at the roots); and
\item $p(\bar{b}) = \kone(p(b))$, where $\bar{b}$ is obtained from $b$ by adding an outer blob which 
encloses all the others (equivalently, add a new edge to the root, with the new vertex becoming the root).
\end{itemize}
For example, a diagram of $k$ strictly nested blobs corresponds to a $k$-simplex, while
a diagram of $k$ disjoint blobs corresponds to a $k$-cube.
(When the fields come from an $n$-category, this correspondence works best if we think of each 
twig label $u_i$ as having the form
$x - s(e(x))$, where $x$ is an arbitrary field on $B_i$, $e: \cF(B_i) \to C$ is the evaluation map, 
and $s:C \to \cF(B_i)$ is some fixed section of $e$.)

For lack of a better name, 
we'll call elements of $P$ cone-product polyhedra, 
and say that blob diagrams have the structure of a cone-product set (analogous to simplicial set).
\end{remark}



%!TEX root = ../blob1.tex

\section{Basic properties of the blob complex}
\label{sec:basic-properties}

In this section we complete the proofs of Properties 2-4.
Throughout the paper, where possible, we prove results using Properties 1-4, 
rather than the actual definition of blob homology.
This allows the possibility of future improvements on or alternatives to our definition.
In fact, we hope that there may be a characterization of the blob complex in 
terms of Properties 1-4, but at this point we are unaware of one.

Recall Property \ref{property:disjoint-union}, 
that there is a natural isomorphism $\bc_*(X \du Y) \cong \bc_*(X) \otimes \bc_*(Y)$.

\begin{proof}[Proof of Property \ref{property:disjoint-union}]
Given blob diagrams $b_1$ on $X$ and $b_2$ on $Y$, we can combine them
(putting the $b_1$ blobs before the $b_2$ blobs in the ordering) to get a
blob diagram $(b_1, b_2)$ on $X \du Y$.
Because of the blob reordering relations, all blob diagrams on $X \du Y$ arise this way.
In the other direction, any blob diagram on $X\du Y$ is equal (up to sign)
to one that puts $X$ blobs before $Y$ blobs in the ordering, and so determines
a pair of blob diagrams on $X$ and $Y$.
These two maps are compatible with our sign conventions.
(We follow the usual convention for tensors products of complexes, 
as in e.g. \cite{MR1438306}: $d(a \tensor b) = da \tensor b + (-1)^{\deg(a)} a \tensor db$.)
The two maps are inverses of each other.
\end{proof}

For the next proposition we will temporarily restore $n$-manifold boundary
conditions to the notation.

Suppose that for all $c \in \cC(\bd B^n)$
we have a splitting $s: H_0(\bc_*(B^n, c)) \to \bc_0(B^n; c)$
of the quotient map
$p: \bc_0(B^n; c) \to H_0(\bc_*(B^n, c))$.
For example, this is always the case if the coefficient ring is a field.
Then
\begin{prop} \label{bcontract}
For all $c \in \cC(\bd B^n)$ the natural map $p: \bc_*(B^n, c) \to H_0(\bc_*(B^n, c))$
is a chain homotopy equivalence
with inverse $s: H_0(\bc_*(B^n, c)) \to \bc_*(B^n; c)$.
Here we think of $H_0(\bc_*(B^n, c))$ as a 1-step complex concentrated in degree 0.
\end{prop}
\begin{proof}
By assumption $p\circ s = \id$, so all that remains is to find a degree 1 map
$h : \bc_*(B^n; c) \to \bc_*(B^n; c)$ such that $\bd h + h\bd = \id - s \circ p$.
For $i \ge 1$, define $h_i : \bc_i(B^n; c) \to \bc_{i+1}(B^n; c)$ by adding
an $(i{+}1)$-st blob equal to all of $B^n$.
In other words, add a new outermost blob which encloses all of the others.
Define $h_0 : \bc_0(B^n; c) \to \bc_1(B^n; c)$ by setting $h_0(x)$ equal to
the 1-blob with blob $B^n$ and label $x - s(p(x)) \in U(B^n; c)$.
\end{proof}
This proves Property \ref{property:contractibility} (the second half of the 
statement of this Property was immediate from the definitions).
Note that even when there is no splitting $s$, we can let $h_0 = 0$ and get a homotopy
equivalence to the 2-step complex $U(B^n; c) \to \cC(B^n; c)$.

For fields based on $n$-categories, $H_0(\bc_*(B^n; c)) \cong \mor(c', c'')$,
where $(c', c'')$ is some (any) splitting of $c$ into domain and range.

\begin{cor} \label{disj-union-contract}
If $X$ is a disjoint union of $n$-balls, then $\bc_*(X; c)$ is contractible.
\end{cor}

\begin{proof}
This follows from Properties \ref{property:disjoint-union} and \ref{property:contractibility}.
\end{proof}

Recall the definition of the support of a blob diagram as the union of all the 
blobs of the diagram.
For future use we prove the following lemma.

\begin{lemma} \label{support-shrink}
Let $L_* \sub \bc_*(X)$ be a subcomplex generated by some
subset of the blob diagrams on $X$, and let $f: L_* \to L_*$
be a chain map which does not increase supports and which induces an isomorphism on
$H_0(L_*)$.
Then $f$ is homotopic (in $\bc_*(X)$) to the identity $L_*\to L_*$.
\end{lemma}

\begin{proof}
We will use the method of acyclic models.
Let $b$ be a blob diagram of $L_*$, let $S\sub X$ be the support of $b$, and let
$r$ be the restriction of $b$ to $X\setminus S$.
Note that $S$ is a disjoint union of balls.
Assign to $b$ the acyclic (in positive degrees) subcomplex $T(b) \deq r\bullet\bc_*(S)$.
Note that if a diagram $b'$ is part of $\bd b$ then $T(B') \sub T(b)$.
Both $f$ and the identity are compatible with $T$ (in the sense of acyclic models), 
so $f$ and the identity map are homotopic. \nn{We should actually have a section with a definition of ``compatible" and this statement as a lemma}
\end{proof}

For the next proposition we will temporarily restore $n$-manifold boundary
conditions to the notation. Let $X$ be an $n$-manifold, with $\bd X = Y \cup Y \cup Z$.
Gluing the two copies of $Y$ together yields an $n$-manifold $X\sgl$
with boundary $Z\sgl$.
Given compatible fields (boundary conditions) $a$, $b$ and $c$ on $Y$, $Y$ and $Z$,
we have the blob complex $\bc_*(X; a, b, c)$.
If $b = a$, then we can glue up blob diagrams on
$X$ to get blob diagrams on $X\sgl$.
This proves Property \ref{property:gluing-map}, which we restate here in more detail.
\todo{This needs more detail, because this is false without careful attention to non-manifold components, etc.}

\textbf{Property \ref{property:gluing-map}.}\emph{
There is a natural chain map
\eq{
    \gl: \bigoplus_a \bc_*(X; a, a, c) \to \bc_*(X\sgl; c\sgl).
}
The sum is over all fields $a$ on $Y$ compatible at their
($n{-}2$-dimensional) boundaries with $c$.
``Natural" means natural with respect to the actions of diffeomorphisms.
}

This map is very far from being an isomorphism, even on homology.
We fix this deficit in \S\ref{sec:gluing} below.


%!TEX root = ../blob1.tex

\section{Hochschild homology when $n=1$}
\label{sec:hochschild}

So far we have provided no evidence that blob homology is interesting in degrees 
greater than zero.
In this section we analyze the blob complex in dimension $n=1$.
We find that $\bc_*(S^1, \cC)$ is homotopy equivalent to the 
Hochschild complex of the 1-category $\cC$.
\nn{cat vs fields --- need to make sure this is clear}
Thus the blob complex is a natural generalization of something already
known to be interesting in higher homological degrees.

It is also worth noting that the original idea for the blob complex came from trying
to find a more ``local" description of the Hochschild complex.

Let $C$ be a *-1-category.
Then specializing the definitions from above to the case $n=1$ we have:
\begin{itemize}
\item $\cC(pt) = \ob(C)$ .
\item Let $R$ be a 1-manifold and $c \in \cC(\bd R)$.
Then an element of $\cC(R; c)$ is a collection of (transversely oriented)
points in the interior
of $R$, each labeled by a morphism of $C$.
The intervals between the points are labeled by objects of $C$, consistent with
the boundary condition $c$ and the domains and ranges of the point labels.
\item There is an evaluation map $e: \cC(I; a, b) \to \mor(a, b)$ given by
composing the morphism labels of the points.
Note that we also need the * of *-1-category here in order to make all the morphisms point
the same way.
\item For $x \in \mor(a, b)$ let $\chi(x) \in \cC(I; a, b)$ be the field with a single
point (at some standard location) labeled by $x$.
Then the kernel of the evaluation map $U(I; a, b)$ is generated by things of the
form $y - \chi(e(y))$.
Thus we can, if we choose, restrict the blob twig labels to things of this form.
\end{itemize}

We want to show that $\bc_*(S^1)$ is homotopy equivalent to the
Hochschild complex of $C$.
In order to prove this we will need to extend the 
definition of the blob complex to allow points to also
be labeled by elements of $C$-$C$-bimodules.
(See Subsections \ref{moddecss} and \ref{ssec:spherecat} for a more general (i.e.\ $n>1$)
version of this construction.)

Fix points $p_1, \ldots, p_k \in S^1$ and $C$-$C$-bimodules $M_1, \ldots M_k$.
We define a blob-like complex $K_*(S^1, (p_i), (M_i))$.
The fields have elements of $M_i$ labeling 
the fixed points $p_i$ and elements of $C$ labeling other (variable) points.
As before, the regions between the marked points are labeled by
objects of $\cC$.
The blob twig labels lie in kernels of evaluation maps.
(The range of these evaluation maps is a tensor product (over $C$) of $M_i$'s,
corresponding to the $p_i$'s that lie within the twig blob.)
Let $K_*(M) = K_*(S^1, (*), (M))$, where $* \in S^1$ is some standard base point.
In other words, fields for $K_*(M)$ have an element of $M$ at the fixed point $*$
and elements of $C$ at variable other points.

In the theorems, propositions and lemmas below we make various claims
about complexes being homotopy equivalent.
In all cases the complexes in question are free (and hence projective), 
so it suffices to show that they are quasi-isomorphic.

We claim that
\begin{thm} \label{hochthm}
The blob complex $\bc_*(S^1; C)$ on the circle is homotopy equivalent to the
usual Hochschild complex for $C$.
\end{thm}

This follows from two results. First, we see that
\begin{lem}
\label{lem:module-blob}%
The complex $K_*(C)$ (here $C$ is being thought of as a
$C$-$C$-bimodule, not a category) is homotopy equivalent to the blob complex
$\bc_*(S^1; C)$. (Proof later.)
\end{lem}

Next, we show that for any $C$-$C$-bimodule $M$,
\begin{prop} \label{prop:hoch}
The complex $K_*(M)$ is homotopy equivalent to $\HC_*(M)$, the usual
Hochschild complex of $M$.
\end{prop}
\begin{proof}
Recall that the usual Hochschild complex of $M$ is uniquely determined,
up to quasi-isomorphism, by the following properties:
\begin{enumerate}
\item \label{item:hochschild-additive}%
$\HC_*(M_1 \oplus M_2) \cong \HC_*(M_1) \oplus \HC_*(M_2)$.
\item \label{item:hochschild-exact}%
An exact sequence $0 \to M_1 \into M_2 \onto M_3 \to 0$ gives rise to an
exact sequence $0 \to \HC_*(M_1) \into \HC_*(M_2) \onto \HC_*(M_3) \to 0$.
\item \label{item:hochschild-coinvariants}%
$\HH_0(M)$ is isomorphic to the coinvariants of $M$, $\coinv(M) =
M/\langle cm-mc \rangle$.
\item \label{item:hochschild-free}%
$\HC_*(C\otimes C)$ is contractible.
(Here $C\otimes C$ denotes
the free $C$-$C$-bimodule with one generator.)
That is, $\HC_*(C\otimes C)$ is
quasi-isomorphic to its $0$-th homology (which in turn, by \ref{item:hochschild-coinvariants}
above, is just $C$) via the quotient map $\HC_0 \onto \HH_0$.
\end{enumerate}
(Together, these just say that Hochschild homology is `the derived functor of coinvariants'.)
We'll first recall why these properties are characteristic.

Take some $C$-$C$ bimodule $M$, and choose a free resolution
\begin{equation*}
\cdots \to F_2 \xrightarrow{f_2} F_1 \xrightarrow{f_1} F_0.
\end{equation*}
We will show that for any functor $\cP$ satisfying properties
\ref{item:hochschild-additive}, \ref{item:hochschild-exact},
\ref{item:hochschild-coinvariants} and \ref{item:hochschild-free}, there
is a quasi-isomorphism
$$\cP_*(M) \iso \coinv(F_*).$$
%
Observe that there's a quotient map $\pi: F_0 \onto M$, and by
construction the cone of the chain map $\pi: F_* \to M$ is acyclic. Now
construct the total complex $\cP_i(F_j)$, with $i,j \geq 0$, graded by
$i+j$. We have two chain maps
\begin{align*}
\cP_i(F_*) & \xrightarrow{\cP_i(\pi)} \cP_i(M) \\
\intertext{and}
\cP_*(F_j) & \xrightarrow{\cP_0(F_j) \onto H_0(\cP_*(F_j))} \coinv(F_j).
\end{align*}
The cone of each chain map is acyclic. In the first case, this is because the `rows' indexed by $i$ are acyclic since $\HC_i$ is exact.
In the second case, this is because the `columns' indexed by $j$ are acyclic, since $F_j$ is free.
Because the cones are acyclic, the chain maps are quasi-isomorphisms. Composing one with the inverse of the other, we obtain the desired quasi-isomorphism
$$\cP_*(M) \quismto \coinv(F_*).$$

%If $M$ is free, that is, a direct sum of copies of
%$C \tensor C$, then properties \ref{item:hochschild-additive} and
%\ref{item:hochschild-free} determine $\HC_*(M)$. Otherwise, choose some
%free cover $F \onto M$, and define $K$ to be this map's kernel. Thus we
%have a short exact sequence $0 \to K \into F \onto M \to 0$, and hence a
%short exact sequence of complexes $0 \to \HC_*(K) \into \HC_*(F) \onto \HC_*(M)
%\to 0$. Such a sequence gives a long exact sequence on homology
%\begin{equation*}
%%\begin{split}
%\cdots \to \HH_{i+1}(F) \to \HH_{i+1}(M) \to \HH_i(K) \to \HH_i(F) \to \cdots % \\
%%\cdots \to \HH_1(F) \to \HH_1(M) \to \HH_0(K) \to \HH_0(F) \to \HH_0(M).
%%\end{split}
%\end{equation*}
%For any $i \geq 1$, $\HH_{i+1}(F) = \HH_i(F) = 0$, by properties
%\ref{item:hochschild-additive} and \ref{item:hochschild-free}, and so
%$\HH_{i+1}(M) \iso \HH_i(F)$. For $i=0$, \todo{}.
%
%This tells us how to
%compute every homology group of $\HC_*(M)$; we already know $\HH_0(M)$
%(it's just coinvariants, by property \ref{item:hochschild-coinvariants}),
%and higher homology groups are determined by lower ones in $\HC_*(K)$, and
%hence recursively as coinvariants of some other bimodule.

Proposition \ref{prop:hoch} then follows from the following lemmas, establishing that $K_*$ has precisely these required properties.
\begin{lem}
\label{lem:hochschild-additive}%
Directly from the definition, $K_*(M_1 \oplus M_2) \cong K_*(M_1) \oplus K_*(M_2)$.
\end{lem}
\begin{lem}
\label{lem:hochschild-exact}%
An exact sequence $0 \to M_1 \into M_2 \onto M_3 \to 0$ gives rise to an
exact sequence $0 \to K_*(M_1) \into K_*(M_2) \onto K_*(M_3) \to 0$.
\end{lem}
\begin{lem}
\label{lem:hochschild-coinvariants}%
$H_0(K_*(M))$ is isomorphic to the coinvariants of $M$.
\end{lem}
\begin{lem}
\label{lem:hochschild-free}%
$K_*(C\otimes C)$ is quasi-isomorphic to $H_0(K_*(C \otimes C)) \iso C$.
\end{lem}

The remainder of this section is devoted to proving Lemmas
\ref{lem:module-blob},
\ref{lem:hochschild-exact}, \ref{lem:hochschild-coinvariants} and
\ref{lem:hochschild-free}.
\end{proof}

\begin{proof}[Proof of Lemma \ref{lem:module-blob}]
We show that $K_*(C)$ is quasi-isomorphic to $\bc_*(S^1)$.
$K_*(C)$ differs from $\bc_*(S^1)$ only in that the base point *
is always a labeled point in $K_*(C)$, while in $\bc_*(S^1)$ it may or may not be.
In particular, there is an inclusion map $i: K_*(C) \to \bc_*(S^1)$.

We want to define a homotopy inverse to the above inclusion, but before doing so
we must replace $\bc_*(S^1)$ with a homotopy equivalent subcomplex.
Let $J_* \sub \bc_*(S^1)$ be the subcomplex where * does not lie on the boundary
of any blob.  Note that the image of $i$ is contained in $J_*$.
Note also that in $\bc_*(S^1)$ (away from $J_*$) 
a blob diagram could have multiple (nested) blobs whose
boundaries contain *, on both the right and left of *.

We claim that $J_*$ is homotopy equivalent to $\bc_*(S^1)$.
Let $F_*^\ep \sub \bc_*(S^1)$ be the subcomplex where either
(a) the point * is not on the boundary of any blob or
(b) there are no labeled points or blob boundaries within distance $\ep$ of *.
Note that all blob diagrams are in $F_*^\ep$ for $\ep$ sufficiently small.
Let $b$ be a blob diagram in $F_*^\ep$.
Define $f(b)$ to be the result of moving any blob boundary points which lie on *
to distance $\ep$ from *.
(Move right or left so as to shrink the blob.)
Extend to get a chain map $f: F_*^\ep \to F_*^\ep$.
By Lemma \ref{support-shrink}, $f$ is homotopic to the identity.
Since the image of $f$ is in $J_*$, and since any blob chain is in $F_*^\ep$
for $\ep$ sufficiently small, we have that $J_*$ is homotopic to all of $\bc_*(S^1)$.

We now define a homotopy inverse $s: J_* \to K_*(C)$ to the inclusion $i$.
If $y$ is a field defined on a neighborhood of *, define $s(y) = y$ if
* is a labeled point in $y$.
Otherwise, define $s(y)$ to be the result of adding a label 1 (identity morphism) at *.
Extending linearly, we get the desired map $s: \bc_*(S^1) \to K_*(C)$.
It is easy to check that $s$ is a chain map and $s \circ i = \id$.

Let $N_\ep$ denote the ball of radius $\ep$ around *.
Let $L_*^\ep \sub \bc_*(S^1)$ be the subcomplex 
spanned by blob diagrams
where there are no labeled points
in $N_\ep$, except perhaps $*$, and $N_\ep$ is either disjoint from or contained in 
every blob in the diagram.
Note that for any chain $x \in \bc_*(S^1)$, $x \in L_*^\ep$ for sufficiently small $\ep$.

We define a degree $1$ map $j_\ep: L_*^\ep \to L_*^\ep$ as follows. Let $x \in L_*^\ep$ be a blob diagram.
\nn{maybe add figures illustrating $j_\ep$?}
If $*$ is not contained in any twig blob, we define $j_\ep(x)$ by adding $N_\ep$ as a new twig blob, with label $y - s(y)$ where $y$ is the restriction
of $x$ to $N_\ep$. If $*$ is contained in a twig blob $B$ with label $u=\sum z_i$,
write $y_i$ for the restriction of $z_i$ to $N_\ep$, and let
$x_i$ be equal to $x$ on $S^1 \setmin B$, equal to $z_i$ on $B \setmin N_\ep$,
and have an additional blob $N_\ep$ with label $y_i - s(y_i)$.
Define $j_\ep(x) = \sum x_i$.

It is not hard to show that on $L_*^\ep$
\[
	\bd j_\ep  + j_\ep \bd = \id - i \circ s .
\]
\nn{need to check signs coming from blob complex differential}
Since for $\ep$ small enough $L_*^\ep$ captures all of the
homology of $\bc_*(S^1)$, 
it follows that the mapping cone of $i \circ s$ is acyclic and therefore (using the fact that
these complexes are free) $i \circ s$ is homotopic to the identity.
\end{proof}

\begin{proof}[Proof of Lemma \ref{lem:hochschild-exact}]
We now prove that $K_*$ is an exact functor.

As a warm-up, we prove
that the functor on $C$-$C$ bimodules
\begin{equation*}
M \mapsto \ker(C \tensor M \tensor C \xrightarrow{c_1 \tensor m \tensor c_2 \mapsto c_1 m c_2} M)
\end{equation*}
is exact.
Suppose we have a short exact sequence of $C$-$C$ bimodules $$\xymatrix{0 \ar[r] & K \ar@{^{(}->}[r]^f & E \ar@{->>}[r]^g & Q \ar[r] & 0}.$$
We'll write $\hat{f}$ and $\hat{g}$ for the image of $f$ and $g$ under the functor, so 
\[
	\hat{f}(\textstyle\sum_i a_i \tensor k_i \tensor b_i) = 
						\textstyle\sum_i a_i \tensor f(k_i) \tensor b_i ,
\]
and similarly for $\hat{g}$.
Most of what we need to check is easy.
Suppose we have $\sum_i (a_i \tensor k_i \tensor b_i) \in \ker(C \tensor K \tensor C \to K)$, assuming without loss of generality that $\{a_i \tensor b_i\}_i$ is linearly independent in $C \tensor C$, and $\hat{f}(a \tensor k \tensor b) = 0 \in \ker(C \tensor E \tensor C \to E)$. We must then have $f(k_i) = 0 \in E$ for each $i$, which implies $k_i=0$ itself. 
If $\sum_i (a_i \tensor e_i \tensor b_i) \in \ker(C \tensor E \tensor C \to E)$ is in the image of $\ker(C \tensor K \tensor C \to K)$ under $\hat{f}$, again by assuming the set  $\{a_i \tensor b_i\}_i$ is linearly independent we can deduce that each
$e_i$ is in the image of the original $f$, and so is in the kernel of the original $g$, and so $\hat{g}(\sum_i a_i \tensor e_i \tensor b_i) = 0$.
If $\hat{g}(\sum_i a_i \tensor e_i \tensor b_i) = 0$, then each $g(e_i) = 0$, so $e_i = f(\widetilde{e_i})$ for some $\widetilde{e_i} \in K$, and $\sum_i a_i \tensor e_i \tensor b_i = \hat{f}(\sum_i a_i \tensor \widetilde{e_i} \tensor b_i)$.
Finally, the interesting step is in checking that any $q = \sum_i a_i \tensor q_i \tensor b_i$ such that $\sum_i a_i q_i b_i = 0$ is in the image of $\ker(C \tensor E \tensor C \to C)$ under $\hat{g}$.
For each $i$, we can find $\widetilde{q_i}$ so $g(\widetilde{q_i}) = q_i$. However $\sum_i a_i \widetilde{q_i} b_i$ need not be zero.
Consider then $$\widetilde{q} = \sum_i (a_i \tensor \widetilde{q_i} \tensor b_i) - 1 \tensor (\sum_i a_i \widetilde{q_i} b_i) \tensor 1.$$ Certainly
$\widetilde{q} \in \ker(C \tensor E \tensor C \to E)$. Further,
\begin{align*}
\hat{g}(\widetilde{q}) & = \sum_i (a_i \tensor g(\widetilde{q_i}) \tensor b_i) - 1 \tensor (\sum_i a_i g(\widetilde{q_i}) b_i) \tensor 1 \\
                       & = q - 0
\end{align*}
(here we used that $g$ is a map of $C$-$C$ bimodules, and that $\sum_i a_i q_i b_i = 0$).

Similar arguments show that the functors
\begin{equation}
\label{eq:ker-functor}%
M \mapsto \ker(C^{\tensor k} \tensor M \tensor C^{\tensor l} \to M)
\end{equation}
are all exact too. Moreover, tensor products of such functors with each
other and with $C$ or $\ker(C^{\tensor k} \to C)$ (e.g., producing the functor $M \mapsto \ker(M \tensor C \to M)
\tensor C \tensor \ker(C \tensor C \to M)$) are all still exact.

Finally, then we see that the functor $K_*$ is simply an (infinite)
direct sum of copies of this sort of functor. The direct sum is indexed by
configurations of nested blobs and of labels; for each such configuration, we have one of the above tensor product functors,
with the labels of twig blobs corresponding to tensor factors as in \eqref{eq:ker-functor} or $\ker(C^{\tensor k} \to C)$ (depending on whether they contain a marked point $p_i$), and all other labelled points corresponding
to tensor factors of $C$ and $M$.
\end{proof}
\begin{proof}[Proof of Lemma \ref{lem:hochschild-coinvariants}]
We show that $H_0(K_*(M))$ is isomorphic to the coinvariants of $M$.

We define a map $\ev: K_0(M) \to M$. If $x \in K_0(M)$ has the label $m \in M$ at $*$, and labels $c_i \in C$ at the other labeled points of $S^1$, reading clockwise from $*$,
we set $\ev(x) = m c_1 \cdots c_k$. We can think of this as $\ev : M \tensor C^{\tensor k} \to M$, for each direct summand of $K_0(M)$ indexed by a configuration of labeled points.

There is a quotient map $\pi: M \to \coinv{M}$.
We claim that the composition $\pi \compose \ev$ is well-defined on the quotient $H_0(K_*(M))$; 
i.e.\ that $\pi(\ev(\bd y)) = 0$ for all $y \in K_1(M)$.
There are two cases, depending on whether the blob of $y$ contains the point *.
If it doesn't, then
suppose $y$ has label $m$ at $*$, labels $c_i$ at other labeled points outside the blob, and the field inside the blob is a sum, with the $j$-th term having
labeled points $d_{j,i}$. Then $\sum_j d_{j,1} \tensor \cdots \tensor d_{j,k_j} \in \ker(\DirectSum_k C^{\tensor k} \to C)$, and so
$\ev(\bdy y) = 0$, because $$C^{\tensor \ell_1} \tensor \ker(\DirectSum_k C^{\tensor k} \to C) \tensor C^{\tensor \ell_2} \subset \ker(\DirectSum_k C^{\tensor k} \to C).$$
Similarly, if $*$ is contained in the blob, then the blob label is a sum, with the $j$-th term have labelled points $d_{j,i}$ to the left of $*$, $m_j$ at $*$, and $d_{j,i}'$ to the right of $*$,
and there are labels $c_i$ at the labeled points outside the blob. We know that
$$\sum_j d_{j,1} \tensor \cdots \tensor d_{j,k_j} \tensor m_j \tensor d_{j,1}' \tensor \cdots \tensor d_{j,k'_j}' \in \ker(\DirectSum_{k,k'} C^{\tensor k} \tensor M \tensor C^{\tensor k'} \tensor \to M),$$
and so
\begin{align*}
\ev(\bdy y) & = \sum_j m_j d_{j,1}' \cdots d_{j,k'_j}' c_1 \cdots c_k d_{j,1} \cdots d_{j,k_j} \\
            & = \sum_j d_{j,1} \cdots d_{j,k_j} m_j d_{j,1}' \cdots d_{j,k'_j}' c_1 \cdots c_k \\
            & = 0
\end{align*}
where this time we use the fact that we're mapping to $\coinv{M}$, not just $M$.

The map $\pi \compose \ev: H_0(K_*(M)) \to \coinv{M}$ is clearly surjective ($\ev$ surjects onto $M$); we now show that it's injective. \todo{}
\end{proof}
\begin{proof}[Proof of Lemma \ref{lem:hochschild-free}]
We show that $K_*(C\otimes C)$ is
quasi-isomorphic to the 0-step complex $C$. We'll do this in steps, establishing quasi-isomorphisms and homotopy equivalences
$$K_*(C \tensor C) \quismto K'_* \htpyto K''_* \quismto C.$$

Let $K'_* \sub K_*(C\otimes C)$ be the subcomplex where the label of
the point $*$ is $1 \otimes 1 \in C\otimes C$.
We will show that the inclusion $i: K'_* \to K_*(C\otimes C)$ is a quasi-isomorphism.

Fix a small $\ep > 0$.
Let $N_\ep$ be the ball of radius $\ep$ around $* \in S^1$.
Let $K_*^\ep \sub K_*(C\otimes C)$ be the subcomplex
generated by blob diagrams $b$ such that $N_\ep$ is either disjoint from
or contained in each blob of $b$, and the only labeled point inside $N_\ep$ is $*$.
%and the two boundary points of $N_\ep$ are not labeled points of $b$.
For a field $y$ on $N_\ep$, let $s_\ep(y)$ be the equivalent picture with~$*$
labeled by $1\otimes 1$ and the only other labeled points at distance $\pm\ep/2$ from $*$.
(See Figure \ref{fig:sy}.) Note that $y - s_\ep(y) \in U(N_\ep)$. We can think of
$\sigma_\ep$ as a chain map $K_*^\ep \to K_*^\ep$ given by replacing the restriction $y$ to $N_\ep$ of each field
appearing in an element of  $K_*^\ep$ with $s_\ep(y)$.
Note that $\sigma_\ep(x) \in K'_*$.
\begin{figure}[!ht]
\begin{align*}
y & = \mathfig{0.2}{hochschild/y} &
s_\ep(y) & = \mathfig{0.2}{hochschild/sy}
\end{align*}
\caption{Defining $s_\ep$.}
\label{fig:sy}
\end{figure}

Define a degree 1 map $j_\ep : K_*^\ep \to K_*^\ep$ as follows.
Let $x \in K_*^\ep$ be a blob diagram.
If $*$ is not contained in any twig blob, $j_\ep(x)$ is obtained by adding $N_\ep$ to
$x$ as a new twig blob, with label $y - s_\ep(y)$, where $y$ is the restriction of $x$ to $N_\ep$.
If $*$ is contained in a twig blob $B$ with label $u = \sum z_i$, $j_\ep(x)$ is obtained as follows.
Let $y_i$ be the restriction of $z_i$ to $N_\ep$.
Let $x_i$ be equal to $x$ outside of $B$, equal to $z_i$ on $B \setmin N_\ep$,
and have an additional blob $N_\ep$ with label $y_i - s_\ep(y_i)$.
Define $j_\ep(x) = \sum x_i$.
\nn{need to check signs coming from blob complex differential}
Note that if $x \in K'_* \cap K_*^\ep$ then $j_\ep(x) \in K'_*$ also.

The key property of $j_\ep$ is
\eq{
    \bd j_\ep + j_\ep \bd = \id - \sigma_\ep.
}
If $j_\ep$ were defined on all of $K_*(C\otimes C)$, this would show that $\sigma_\ep$
is a homotopy inverse to the inclusion $K'_* \to K_*(C\otimes C)$.
One strategy would be to try to stitch together various $j_\ep$ for progressively smaller
$\ep$ and show that $K'_*$ is homotopy equivalent to $K_*(C\otimes C)$.
Instead, we'll be less ambitious and just show that
$K'_*$ is quasi-isomorphic to $K_*(C\otimes C)$.

If $x$ is a cycle in $K_*(C\otimes C)$, then for sufficiently small $\ep$ we have
$x \in K_*^\ep$.
(This is true for any chain in $K_*(C\otimes C)$, since chains are sums of
finitely many blob diagrams.)
Then $x$ is homologous to $s_\ep(x)$, which is in $K'_*$, so the inclusion map
$K'_* \sub K_*(C\otimes C)$ is surjective on homology.
If $y \in K_*(C\otimes C)$ and $\bd y = x \in K'_*$, then $y \in K_*^\ep$ for some $\ep$
and
\eq{
    \bd y = \bd (\sigma_\ep(y) + j_\ep(x)) .
}
Since $\sigma_\ep(y) + j_\ep(x) \in F'$, it follows that the inclusion map is injective on homology.
This completes the proof that $K'_*$ is quasi-isomorphic to $K_*(C\otimes C)$.

Let $K''_* \sub K'_*$ be the subcomplex of $K'_*$ where $*$ is not contained in any blob.
We will show that the inclusion $i: K''_* \to K'_*$ is a homotopy equivalence.

First, a lemma:  Let $G''_*$ and $G'_*$ be defined the same as $K''_*$ and $K'_*$, except with
$S^1$ replaced some (any) neighborhood of $* \in S^1$.
Then $G''_*$ and $G'_*$ are both contractible
and the inclusion $G''_* \sub G'_*$ is a homotopy equivalence.
For $G'_*$ the proof is the same as in (\ref{bcontract}), except that the splitting
$G'_0 \to H_0(G'_*)$ concentrates the point labels at two points to the right and left of $*$.
For $G''_*$ we note that any cycle is supported \nn{need to establish terminology for this; maybe
in ``basic properties" section above} away from $*$.
Thus any cycle lies in the image of the normal blob complex of a disjoint union
of two intervals, which is contractible by (\ref{bcontract}) and (\ref{disjunion}).
Actually, we need the further (easy) result that the inclusion
$G''_* \to G'_*$ induces an isomorphism on $H_0$.

Next we construct a degree 1 map (homotopy) $h: K'_* \to K'_*$ such that
for all $x \in K'_*$ we have
\eq{
    x - \bd h(x) - h(\bd x) \in K''_* .
}
Since $K'_0 = K''_0$, we can take $h_0 = 0$.
Let $x \in K'_1$, with single blob $B \sub S^1$.
If $* \notin B$, then $x \in K''_1$ and we define $h_1(x) = 0$.
If $* \in B$, then we work in the image of $G'_*$ and $G''_*$ (with respect to $B$).
Choose $x'' \in G''_1$ such that $\bd x'' = \bd x$.
Since $G'_*$ is contractible, there exists $y \in G'_2$ such that $\bd y = x - x''$.
Define $h_1(x) = y$.
The general case is similar, except that we have to take lower order homotopies into account.
Let $x \in K'_k$.
If $*$ is not contained in any of the blobs of $x$, then define $h_k(x) = 0$.
Otherwise, let $B$ be the outermost blob of $x$ containing $*$.
By xxxx above, $x = x' \bullet p$, where $x'$ is supported on $B$ and $p$ is supported away from $B$.
So $x' \in G'_l$ for some $l \le k$.
Choose $x'' \in G''_l$ such that $\bd x'' = \bd (x' - h_{l-1}\bd x')$.
Choose $y \in G'_{l+1}$ such that $\bd y = x' - x'' - h_{l-1}\bd x'$.
Define $h_k(x) = y \bullet p$.
This completes the proof that $i: K''_* \to K'_*$ is a homotopy equivalence.
\nn{need to say above more clearly and settle on notation/terminology}

Finally, we show that $K''_*$ is contractible.
\nn{need to also show that $H_0$ is the right thing; easy, but I won't do it now}
Let $x$ be a cycle in $K''_*$.
The union of the supports of the diagrams in $x$ does not contain $*$, so there exists a
ball $B \subset S^1$ containing the union of the supports and not containing $*$.
Adding $B$ as a blob to $x$ gives a contraction.
\nn{need to say something else in degree zero}
\end{proof}

\medskip

For purposes of illustration, we describe an explicit chain map
$\HC_*(M) \to K_*(M)$
between the Hochschild complex and the blob complex (with bimodule point)
for degree $\le 2$.
This map can be completed to a homotopy equivalence, though we will not prove that here.
There are of course many such maps; what we describe here is one of the simpler possibilities.
Describing the extension to higher degrees is straightforward but tedious.
\nn{but probably we should include the general case in a future version of this paper}

Recall that in low degrees $\HC_*(M)$ is
\[
	\cdots \stackrel{\bd}{\to} M \otimes C\otimes C \stackrel{\bd}{\to} 
			M \otimes C \stackrel{\bd}{\to} M
\]
with
\eqar{
	\bd(m\otimes a)  & = & ma - am \\
	\bd(m\otimes a \otimes b) & = & ma\otimes b - m\otimes ab + bm \otimes a .
}
In degree 0, we send $m\in M$ to the 0-blob diagram $\mathfig{0.05}{hochschild/0-chains}$; the base point
in $S^1$ is labeled by $m$ and there are no other labeled points.
In degree 1, we send $m\ot a$ to the sum of two 1-blob diagrams
as shown in Figure \ref{fig:hochschild-1-chains}.

\begin{figure}[!ht]
\begin{equation*}
\mathfig{0.4}{hochschild/1-chains}
\end{equation*}
\begin{align*}
u_1 & = \mathfig{0.05}{hochschild/u_1-1} - \mathfig{0.05}{hochschild/u_1-2} & u_2 & = \mathfig{0.05}{hochschild/u_2-1} - \mathfig{0.05}{hochschild/u_2-2} 
\end{align*}
\caption{The image of $m \tensor a$ in the blob complex.}
\label{fig:hochschild-1-chains}
\end{figure}

In degree 2, we send $m\ot a \ot b$ to the sum of 24 (=6*4) 2-blob diagrams as shown in
Figure \ref{fig:hochschild-2-chains}. In Figure \ref{fig:hochschild-2-chains} the 1- and 2-blob diagrams are indicated only by their support.
We leave it to the reader to determine the labels of the 1-blob diagrams.
\begin{figure}[!ht]
\begin{equation*}
\mathfig{0.6}{hochschild/2-chains-0}
\end{equation*}
\begin{equation*}
\mathfig{0.4}{hochschild/2-chains-1} \qquad \mathfig{0.4}{hochschild/2-chains-2}
\end{equation*}
\caption{The 0-, 1- and 2-chains in the image of $m \tensor a \tensor b$. Only the supports of the 1- and 2-blobs are shown.}
\label{fig:hochschild-2-chains}
\end{figure}
Each 2-cell in the figure is labeled by a ball $V$ in $S^1$ which contains the support of all
1-blob diagrams in its boundary.
Such a 2-cell corresponds to a sum of the 2-blob diagrams obtained by adding $V$
as an outer (non-twig) blob to each of the 1-blob diagrams in the boundary of the 2-cell.
Figure \ref{fig:hochschild-example-2-cell} shows this explicitly for one of the 2-cells.
Note that the (blob complex) boundary of this sum of 2-blob diagrams is
precisely the sum of the 1-blob diagrams corresponding to the boundary of the 2-cell.
(Compare with the proof of \ref{bcontract}.)

\begin{figure}[!ht]
\begin{equation*}
A = \mathfig{0.1}{hochschild/v_1} + \mathfig{0.1}{hochschild/v_2} + \mathfig{0.1}{hochschild/v_3} + \mathfig{0.1}{hochschild/v_4}
\end{equation*}
\begin{align*}
v_1 & = \mathfig{0.05}{hochschild/v_1-1} -  \mathfig{0.05}{hochschild/v_1-2} &  v_2 & = \mathfig{0.05}{hochschild/v_2-1} -  \mathfig{0.05}{hochschild/v_2-2} \\ 
v_3 & = \mathfig{0.05}{hochschild/v_3-1} -  \mathfig{0.05}{hochschild/v_3-2} &  v_4 & = \mathfig{0.05}{hochschild/v_4-1} -  \mathfig{0.05}{hochschild/v_4-2}
\end{align*}
\caption{One of the 2-cells from Figure \ref{fig:hochschild-2-chains}.}
\label{fig:hochschild-example-2-cell}
\end{figure}


%!TEX root = ../blob1.tex

\section{Action of \texorpdfstring{$\CH{X}$}{C*(Homeo(M))}}
\label{sec:evaluation}

In this section we extend the action of homeomorphisms on $\bc_*(X)$
to an action of {\it families} of homeomorphisms.
That is, for each pair of homeomorphic manifolds $X$ and $Y$
we define a chain map
\[
    e_{XY} : CH_*(X, Y) \otimes \bc_*(X) \to \bc_*(Y) ,
\]
where $CH_*(X, Y) = C_*(\Homeo(X, Y))$, the singular chains on the space
of homeomorphisms from $X$ to $Y$.
(If $X$ and $Y$ have non-empty boundary, these families of homeomorphisms
are required to be fixed on the boundaries.)
See \S \ref{ss:emap-def} for a more precise statement.

The most convenient way to prove that maps $e_{XY}$ with the desired properties exist is to 
introduce a homotopy equivalent alternate version of the blob complex $\btc_*(X)$
which is more amenable to this sort of action.
Recall from Remark \ref{blobsset-remark} that blob diagrams
have the structure of a sort-of-simplicial set.
Blob diagrams can also be equipped with a natural topology, which converts this
sort-of-simplicial set into a sort-of-simplicial space.
Taking singular chains of this space we get $\btc_*(X)$.
The details are in \S \ref{ss:alt-def}.
For technical reasons we also show that requiring the blobs to be
embedded yields a homotopy equivalent complex.

%Since $\bc_*(X)$ and $\btc_*(X)$ are homotopy equivalent one could try to construct
%the $CH_*$ actions directly in terms of $\bc_*(X)$.
%This was our original approach, but working out the details created a nearly unreadable mess.
%We have salvaged a sketch of that approach in \S \ref{ss:old-evmap-remnants}.
%
%\nn{should revisit above intro after this section is done}


\subsection{Alternative definitions of the blob complex}
\label{ss:alt-def}

\newcommand\sbc{\bc^{\cU}}

In this subsection we define a subcomplex (small blobs) and supercomplex (families of blobs)
of the blob complex, and show that they are both homotopy equivalent to $\bc_*(X)$.

\medskip

If $b$ is a blob diagram in $\bc_*(X)$, define the {\it support} of $b$, denoted
$\supp(b)$ or $|b|$, to be the union of the blobs of $b$.
For a general $k$-chain $a\in \bc_k(X)$, define the support of $a$ to be the union
of the supports of the blob diagrams which appear in it.

If $f: P\times X\to X$ is a family of homeomorphisms and $Y\sub X$, we say that $f$ is 
{\it supported on $Y$} if $f(p, x) = f(p', x)$ for all $x\in X\setmin Y$ and all $p,p'\in P$.
We will sometimes abuse language and talk about ``the" support of $f$,
again denoted $\supp(f)$ or $|f|$, to mean some particular choice of $Y$ such that
$f$ is supported on $Y$.

If $f: M \cup (Y\times I) \to M$ is a collaring homeomorphism
(cf. end of \S \ref{ss:syst-o-fields}),
we say that $f$ is supported on $S\sub M$ if $f(x) = x$ for all $x\in M\setmin S$.

Fix $\cU$, an open cover of $X$.
Define the ``small blob complex" $\bc^{\cU}_*(M)$ to be the subcomplex of $\bc_*(X)$ 
of all blob diagrams in which every blob is contained in some open set of $\cU$, 
and moreover each field labeling a region cut out by the blobs is splittable 
into fields on smaller regions, each of which is contained in some open set of $\cU$.

\begin{lemma}[Small blobs] \label{small-blobs-b}  \label{thm:small-blobs}
The inclusion $i: \bc^{\cU}_*(M) \into \bc_*(M)$ is a homotopy equivalence.
\end{lemma}

\begin{proof}
It suffices to show that for any finitely generated pair of subcomplexes 
\[
	(C_*, D_*) \sub (\bc_*(X), \sbc_*(X))
\]
we can find a homotopy $h:C_*\to \bc_*(X)$ such that $h(D_*) \sub \sbc_*(X)$
and
\[
	x + h\bd(x) + \bd h(X) \in \sbc_*(X)
\]
for all $x\in C_*$.

For simplicity we will assume that all fields are splittable into small pieces, so that
$\sbc_0(X) = \bc_0$.
(This is true for all of the examples presented in this paper.)
Accordingly, we define $h_0 = 0$.

Next we define $h_1$.
Let $b\in C_1$ be a 1-blob diagram.
Let $B$ be the blob of $b$.
We will construct a 1-chain $s(b)\in \sbc_1$ such that $\bd(s(b)) = \bd b$
and the support of $s(b)$ is contained in $B$.
(If $B$ is not embedded in $X$, then we implicitly work in some term of a decomposition
of $X$ where $B$ is embedded.
See \ref{defn:configuration} and preceding discussion.)
It then follows from \ref{disj-union-contract} that we can choose
$h_1(b) \in \bc_1(X)$ such that $\bd(h_1(b)) = s(b) - b$.

Roughly speaking, $s(b)$ consists of a series of 1-blob diagrams implementing a series
of small collar maps, plus a shrunken version of $b$.
The composition of all the collar maps shrinks $B$ to a sufficiently small ball.

Let $\cV_1$ be an auxiliary open cover of $X$, subordinate to $\cU$ and 
also satisfying conditions specified below.
Let $b = (B, u, r)$, $u = \sum a_i$ be the label of $B$, $a_i\in \bc_0(B)$.
Choose a sequence of collar maps $f_j:\bc_0(B)\to\bc_0(B)$ such that each has support
contained in an open set of $\cV_1$ and the composition of the corresponding collar homeomorphisms
yields an embedding $g:B\to B$ such that $g(B)$ is contained in an open set of $\cV_1$.
\nn{need to say this better; maybe give fig}
Let $g_j:B\to B$ be the embedding at the $j$-th stage.
There are 1-blob diagrams $c_{ij} \in \bc_1(B)$ such that $c_{ij}$ is compatible with $\cV_1$
and $\bd c_{ij} = g_j(a_i) = g_{j-1}(a_i)$.
Define
\[
	s(b) = \sum_{i,j} c_{ij} + g(b)
\]
and choose $h_1(b) \in \bc_1(X)$ such that 
\[
	\bd(h_1(b)) = s(b) - b .
\]

Next we define $h_2$.
Let $b\in C_2$ be a 2-blob diagram.
Let $B = |b|$, either a ball or a union of two balls.
By possibly working in a decomposition of $X$, we may assume that the ball(s)
of $B$ are disjointly embedded.
We will construct a 2-chain $s(b)\in \sbc_2$ such that
\[
	\bd(s(b)) = \bd(h_1(\bd b) + b) = s(\bd b)
\]
and the support of $s(b)$ is contained in $B$.
It then follows from \ref{disj-union-contract} that we can choose
$h_2(b) \in \bc_2(X)$ such that $\bd(h_2(b)) = s(b) - b - h_1(\bd b)$.

Similarly to the construction of $h_1$ above, 
$s(b)$ consists of a series of 2-blob diagrams implementing a series
of small collar maps, plus a shrunken version of $b$.
The composition of all the collar maps shrinks $B$ to a sufficiently small 
disjoint union of balls.

Let $\cV_2$ be an auxiliary open cover of $X$, subordinate to $\cU$ and
also satisfying conditions specified below.
As before, choose a sequence of collar maps $f_j$ 
such that each has support
contained in an open set of $\cV_1$ and the composition of the corresponding collar homeomorphisms
yields an embedding $g:B\to B$ such that $g(B)$ is contained in an open set of $\cV_1$.
Let $g_j:B\to B$ be the embedding at the $j$-th stage.

Fix $j$.
We will construct a 2-chain $d_j$ such that $\bd(d_j) = g_j(s(\bd b)) - g_{j-1}(s(\bd b))$.
Let $g_{j-1}(s(\bd b)) = \sum e_k$, and let $\{p_m\}$ be the 0-blob diagrams
appearing in the boundaries of the $e_k$.
As in the construction of $h_1$, we can choose 1-blob diagrams $q_m$ such that
$\bd q_m = f_j(p_m) - p_m$ and $\supp(q_m)$ is contained in an open set of $\cV_1$.
%%% \nn{better not to do this, to make things more parallel with general case (?)}
%Furthermore, we can arrange that all of the $q_m$ have the same support, and that this support
%is contained in a open set of $\cV_1$.
%(This is possible since there are only finitely many $p_m$.)
If $x$ is a sum of $p_m$'s, we denote the corresponding sum of $q_m$'s by $q(x)$.

Now consider, for each $k$, $e_k + q(\bd e_k)$.
This is a 1-chain whose boundary is $f_j(\bd e_k)$.
The support of $e_k$ is $g_{j-1}(V)$ for some $V\in \cV_1$, and
the support of $q(\bd e_k)$ is contained in a union $V'$ of finitely many open sets
of $\cV_1$, all of which contain the support of $f_j$.
%the support of $q(\bd e_k)$ is contained in $V'$ for some $V'\in \cV_1$.
We now reveal the mysterious condition (mentioned above) which $\cV_1$ satisfies:
the union of $g_{j-1}(V)$ and $V'$, for all of the finitely many instances
arising in the construction of $h_2$, lies inside a disjoint union of balls $U$
such that each individual ball lies in an open set of $\cV_2$.
(In this case there are either one or two balls in the disjoint union.)
For any fixed open cover $\cV_2$ this condition can be satisfied by choosing $\cV_1$ 
to be a sufficiently fine cover.
It follows from \ref{disj-union-contract}
that we can choose $x_k \in \bc_2(X)$ with $\bd x_k = f_j(e_k) - e_k - q(\bd e_k)$
and with $\supp(x_k) = U$.
We can now take $d_j \deq \sum x_k$.
It is clear that $\bd d_j = \sum (f_j(e_k) - e_k) = g_j(s(\bd b)) - g_{j-1}(s(\bd b))$, as desired.
\nn{should maybe have figure}

We now define 
\[
	s(b) = \sum d_j + g(b),
\]
where $g$ is the composition of all the $f_j$'s.
It is easy to verify that $s(b) \in \sbc_2$, $\supp(s(b)) = \supp(b)$, and 
$\bd(s(b)) = s(\bd b)$.
If follows that we can choose $h_2(b)\in \bc_2(X)$ such that $\bd(h_2(b)) = s(b) - b - h_1(\bd b)$.
This completes the definition of $h_2$.

The general case $h_l$ is similar.
When constructing the analogue of $x_k$ above, we will need to find a disjoint union of balls $U$
which contains finitely many open sets from $\cV_{l-1}$
such that each ball is contained in some open set of $\cV_l$.
For sufficiently fine $\cV_{l-1}$ this will be possible.
Since $C_*$ is finite, the process terminates after finitely many, say $r$, steps.
We take $\cV_r = \cU$.

\nn{should probably be more specific at the end}
\end{proof}


\medskip

Next we define the sort-of-simplicial space version of the blob complex, $\btc_*(X)$.
First we must specify a topology on the set of $k$-blob diagrams, $\BD_k$.
We give $\BD_k$ the finest topology such that
\begin{itemize}
\item For any $b\in \BD_k$ the action map $\Homeo(X) \to \BD_k$, $f \mapsto f(b)$ is continuous.
\item \nn{something about blob labels and vector space structure}
\item \nn{maybe also something about gluing}
\end{itemize}

Next we define $\btc_*(X)$ to be the total complex of the double complex (denoted $\btc_{**}$) 
whose $(i,j)$ entry is $C_j(\BD_i)$, the singular $j$-chains on the space of $i$-blob diagrams.
The vertical boundary of the double complex,
denoted $\bd_t$, is the singular boundary, and the horizontal boundary, denoted $\bd_b$, is
the blob boundary.

We will regard $\bc_*(X)$ as the subcomplex $\btc_{*0}(X) \sub \btc_{**}(X)$.
The main result of this subsection is

\begin{lemma} \label{lem:bc-btc}
The inclusion $\bc_*(X) \sub \btc_*(X)$ is a homotopy equivalence
\end{lemma}

Before giving the proof we need a few preliminary results.

\begin{lemma} \label{bt-contract}
$\btc_*(B^n)$ is contractible (acyclic in positive degrees).
\end{lemma}
\begin{proof}
We will construct a contracting homotopy $h: \btc_*(B^n)\to \btc_*(B^n)$.

We will assume a splitting $s:H_0(\btc_*(B^n))\to \btc_0(B^n)$
of the quotient map $q:\btc_0(B^n)\to H_0(\btc_*(B^n))$.
Let $r = s\circ q$.

For $x\in \btc_{ij}$ with $i\ge 1$ define
\[
	h(x) = e(x) ,
\]
where
\[
	e: \btc_{ij}\to\btc_{i+1,j}
\]
adds an outermost blob, equal to all of $B^n$, to the $j$-parameter family of blob diagrams.

A generator $y\in \btc_{0j}$ is a map $y:P\to \BD_0$, where $P$ is some $j$-dimensional polyhedron.
We define $r(y)\in \btc_{0j}$ to be the constant function $r\circ y : P\to \BD_0$.
Let $c(r(y))\in \btc_{0,j+1}$ be the constant map from the cone of $P$ to $\BD_0$ taking
the same value (i.e.\ $r(y(p))$ for any $p\in P$).
Let $e(y - r(y)) \in \btc_{1j}$ denote the $j$-parameter family of 1-blob diagrams
whose value at $p\in P$ is the blob $B^n$ with label $y(p) - r(y(p))$.
Now define, for $y\in \btc_{0j}$,
\[
	h(y) = e(y - r(y)) + c(r(y)) .
\]
\nn{up to sign, at least}

We must now verify that $h$ does the job it was intended to do.
For $x\in \btc_{ij}$ with $i\ge 2$ we have
\nn{ignoring signs}
\begin{align*}
	\bd h(x) + h(\bd x) &= \bd(e(x)) + e(\bd x) \\
			&= \bd_b(e(x)) + \bd_t(e(x)) + e(\bd_b x) + e(\bd_t x) \\
			&= \bd_b(e(x)) + e(\bd_b x) \quad\quad\text{(since $\bd_t(e(x)) = e(\bd_t x)$)} \\
			&= x .
\end{align*}
For $x\in \btc_{1j}$ we have
\nn{ignoring signs}
\begin{align*}
	\bd h(x) + h(\bd x) &= \bd_b(e(x)) + \bd_t(e(x)) + e(\bd_b x - r(\bd_b x)) + c(r(\bd_b x)) + e(\bd_t x) \\
			&= \bd_b(e(x)) + e(\bd_b x) \quad\quad\text{(since $r(\bd_b x) = 0$)} \\
			&= x .
\end{align*}
For $x\in \btc_{0j}$ with $j\ge 1$ we have
\nn{ignoring signs}
\begin{align*}
	\bd h(x) + h(\bd x) &= \bd_b(e(x - r(x))) + \bd_t(e(x - r(x))) + \bd_t(c(r(x))) + 
											e(\bd_t x - r(\bd_t x)) + c(r(\bd_t x)) \\
			&= x - r(x) + \bd_t(c(r(x))) + c(r(\bd_t x)) \\
			&= x - r(x) + r(x) \\
			&= x.
\end{align*}
For $x\in \btc_{00}$ we have
\nn{ignoring signs}
\begin{align*}
	\bd h(x) + h(\bd x) &= \bd_b(e(x - r(x))) + \bd_t(c(r(x))) \\
			&= x - r(x) + r(x) - r(x)\\
			&= x - r(x).
\end{align*}
\end{proof}

\begin{lemma} \label{btc-prod}
For manifolds $X$ and $Y$, we have $\btc_*(X\du Y) \simeq \btc_*(X)\otimes\btc_*(Y)$.
\end{lemma}
\begin{proof}
This follows from the Eilenber-Zilber theorem and the fact that
\[
	\BD_k(X\du Y) \cong \coprod_{i+j=k} \BD_i(X)\times\BD_j(Y) .
\]
\end{proof}

For $S\sub X$, we say that $a\in \btc_k(X)$ is {\it supported on $S$}
if there exists $S' \subeq S$, $a'\in \btc_k(S')$
and $r\in \btc_0(X\setmin S')$ such that $a = a'\bullet r$.

\newcommand\sbtc{\btc^{\cU}}
Let $\cU$ be an open cover of $X$.
Let $\sbtc_*(X)\sub\btc_*(X)$ be the subcomplex generated by
$a\in \btc_*(X)$ such that there is a decomposition $X = \cup_i D_i$
such that each $D_i$ is a ball contained in some open set of $\cU$ and
$a$ is splittable along this decomposition.
In other words, $a$ can be obtained by gluing together pieces, each of which
is small with respect to $\cU$.

\begin{lemma} \label{small-top-blobs}
For any open cover $\cU$ of $X$, the inclusion $\sbtc_*(X)\sub\btc_*(X)$
is a homotopy equivalence.
\end{lemma}
\begin{proof}
This follows from a combination of Lemma \ref{extension_lemma_c} and the techniques of
the proof of Lemma \ref{small-blobs-b}.

It suffices to show that we can deform a finite subcomplex $C_*$ of $\btc_*(X)$ into $\sbtc_*(X)$
(relative to any designated subcomplex of $C_*$ already in $\sbtc_*(X)$).
The first step is to replace families of general blob diagrams with families that are 
small with respect to $\cU$.
This is done as in the proof of Lemma \ref{small-blobs-b}; the technique of the proof works in families.
Each such family is homotopic to a sum families which can be a ``lifted" to $\Homeo(X)$.
That is, $f:P \to \BD_k$ has the form $f(p) = g(p)(b)$ for some $g:P\to \Homeo(X)$ and $b\in \BD_k$.
(We are ignoring a complication related to twig blob labels, which might vary
independently of $g$, but this complication does not affect the conclusion we draw here.)
We now apply Lemma \ref{extension_lemma_c} to get families which are supported 
on balls $D_i$ contained in open sets of $\cU$.
\end{proof}


\begin{proof}[Proof of \ref{lem:bc-btc}]
Armed with the above lemmas, we can now proceed similarly to the proof of \ref{small-blobs-b}.

It suffices to show that for any finitely generated pair of subcomplexes 
$(C_*, D_*) \sub (\btc_*(X), \bc_*(X))$
we can find a homotopy $h:C_*\to \btc_*(X)$ such that $h(D_*) \sub \bc_*(X)$
and $x + h\bd(x) + \bd h(X) \in \bc_*(X)$ for all $x\in C_*$.

By Lemma \ref{small-top-blobs}, we may assume that $C_* \sub \btc_*^\cU(X)$ for some
cover $\cU$ of our choosing.
We choose $\cU$ fine enough so that each generator of $C_*$ is supported on a disjoint union of balls.
(This is possible since the original $C_*$ was finite and therefore had bounded dimension.)

Since $\bc_0(X) = \btc_0(X)$, we can take $h_0 = 0$.

Let $b \in C_1$ be a generator.
Since $b$ is supported in a disjoint union of balls,
we can find $s(b)\in \bc_1$ with $\bd (s(b)) = \bd b$
(by \ref{disj-union-contract}), and also $h_1(b) \in \btc_2$
such that $\bd (h_1(b)) = s(b) - b$
(by \ref{bt-contract} and \ref{btc-prod}).

Now let $b$ be a generator of $C_2$.
If $\cU$ is fine enough, there is a disjoint union of balls $V$
on which $b + h_1(\bd b)$ is supported.
Since $\bd(b + h_1(\bd b)) = s(\bd b) \in \bc_2$, we can find
$s(b)\in \bc_2$ with $\bd(s(b)) = \bd(b + h_1(\bd b))$ (by \ref{disj-union-contract}).
By \ref{bt-contract} and \ref{btc-prod}, we can now find
$h_2(b) \in \btc_3$, also supported on $V$, such that $\bd(h_2(b)) = s(b) - b - h_1(\bd b)$

The general case, $h_k$, is similar.
\end{proof}

The proof of \ref{lem:bc-btc} constructs a homotopy inverse to the inclusion
$\bc_*(X)\sub \btc_*(X)$.
One might ask for more: a contractible set of possible homotopy inverses, or at least an
$m$-connected set for arbitrarily large $m$.
The latter can be achieved with finer control over the various
choices of disjoint unions of balls in the above proofs, but we will not pursue this here.




\subsection{Action of \texorpdfstring{$\CH{X}$}{C*(Homeo(M))}}
\label{ss:emap-def}

Let $CH_*(X, Y)$ denote $C_*(\Homeo(X \to Y))$, the singular chain complex of
the space of homeomorphisms
between the $n$-manifolds $X$ and $Y$ 
(any given singular chain extends a fixed homeomorphism $\bd X \to \bd Y$).
We also will use the abbreviated notation $CH_*(X) \deq CH_*(X, X)$.
(For convenience, we will permit the singular cells generating $CH_*(X, Y)$ to be more general
than simplices --- they can be based on any linear polyhedron.
\nn{be more restrictive here?  does more need to be said?})

\begin{thm}  \label{thm:CH}
For $n$-manifolds $X$ and $Y$ there is a chain map
\eq{
    e_{XY} : CH_*(X, Y) \otimes \bc_*(X) \to \bc_*(Y) ,
}
well-defined up to homotopy,
such that
\begin{enumerate}
\item on $CH_0(X, Y) \otimes \bc_*(X)$ it agrees with the obvious action of 
$\Homeo(X, Y)$ on $\bc_*(X)$  described in Property (\ref{property:functoriality}), and
\item for any compatible splittings $X\to X\sgl$ and $Y\to Y\sgl$, 
the following diagram commutes up to homotopy
\begin{equation*}
\xymatrix@C+2cm{
      CH_*(X, Y) \otimes \bc_*(X)
        \ar[r]_(.6){e_{XY}}  \ar[d]^{\gl \otimes \gl}   &
            \bc_*(Y)\ar[d]^{\gl} \\
     CH_*(X\sgl, Y\sgl) \otimes \bc_*(X\sgl) \ar[r]_(.6){e_{X\sgl Y\sgl}}   & 	\bc_*(Y\sgl)  
}
\end{equation*}
\end{enumerate}
\end{thm}

\begin{proof}
In light of Lemma \ref{lem:bc-btc}, it suffices to prove the theorem with 
$\bc_*$ replaced by $\btc_*$.
And in fact for $\btc_*$ we get a sharper result: we can omit
the ``up to homotopy" qualifiers.

Let $f\in CH_k(X, Y)$, $f:P^k\to \Homeo(X \to Y)$ and $a\in \btc_{ij}(X)$, 
$a:Q^j \to \BD_i(X)$.
Define $e_{XY}(f\ot a)\in \btc_{i,j+k}(Y)$ by
\begin{align*}
	e_{XY}(f\ot a) : P\times Q &\to \BD_i(Y) \\
	(p,q) &\mapsto f(p)(a(q))  .
\end{align*}
It is clear that this agrees with the previously defined $CH_0(X, Y)$ action on $\btc_*$,
and it is also easy to see that the diagram in item 2 of the statement of the theorem
commutes on the nose.
\end{proof}


\begin{thm}
\label{thm:CH-associativity}
The $CH_*(X, Y)$ actions defined above are associative.
That is, the following diagram commutes up to homotopy:
\[ \xymatrix{
& CH_*(Y, Z) \ot \bc_*(Y) \ar[dr]^{e_{YZ}} & \\
CH_*(X, Y) \ot CH_*(Y, Z) \ot \bc_*(X) \ar[ur]^{e_{XY}\ot\id} \ar[dr]_{\mu\ot\id} & & \bc_*(Z) \\
& CH_*(X, Z) \ot \bc_*(X) \ar[ur]_{e_{XZ}} &
} \]
Here $\mu:CH_*(X, Y) \ot CH_*(Y, Z)\to CH_*(X, Z)$ is the map induced by composition
of homeomorphisms.
\end{thm}
\begin{proof}
The corresponding diagram for $\btc_*$ commutes on the nose.
\end{proof}







\noop{


\subsection{[older version still hanging around]}
\label{ss:old-evmap-remnants}

\nn{should comment at the start about any assumptions about smooth, PL etc.}

\nn{should maybe mention alternate def of blob complex (sort-of-simplicial space instead of
sort-of-simplicial set) where this action would be easy}

Let $CH_*(X, Y)$ denote $C_*(\Homeo(X \to Y))$, the singular chain complex of
the space of homeomorphisms
between the $n$-manifolds $X$ and $Y$ (any given singular chain extends a fixed homeomorphism $\bd X \to \bd Y$).
We also will use the abbreviated notation $CH_*(X) \deq CH_*(X, X)$.
(For convenience, we will permit the singular cells generating $CH_*(X, Y)$ to be more general
than simplices --- they can be based on any linear polyhedron.
\nn{be more restrictive here?  does more need to be said?})

\begin{thm}  \label{thm:CH}
For $n$-manifolds $X$ and $Y$ there is a chain map
\eq{
    e_{XY} : CH_*(X, Y) \otimes \bc_*(X) \to \bc_*(Y)
}
such that
\begin{enumerate}
\item on $CH_0(X, Y) \otimes \bc_*(X)$ it agrees with the obvious action of 
$\Homeo(X, Y)$ on $\bc_*(X)$  described in Property (\ref{property:functoriality}), and
\item for any compatible splittings $X\to X\sgl$ and $Y\to Y\sgl$, 
the following diagram commutes up to homotopy
\begin{equation*}
\xymatrix@C+2cm{
      CH_*(X, Y) \otimes \bc_*(X)
        \ar[r]_(.6){e_{XY}}  \ar[d]^{\gl \otimes \gl}   &
            \bc_*(Y)\ar[d]^{\gl} \\
     CH_*(X\sgl, Y\sgl) \otimes \bc_*(X\sgl) \ar[r]_(.6){e_{X\sgl Y\sgl}}   & 	\bc_*(Y\sgl)  
}
\end{equation*}
\end{enumerate}
Moreover, for any $m \geq 0$, we can find a family of chain maps $\{e_{XY}\}$ 
satisfying the above two conditions which is $m$-connected. In particular, this means that the choice of chain map above is unique up to homotopy.
\end{thm}
\begin{rem}
Note that the statement doesn't quite give uniqueness up to iterated homotopy. We fully expect that this should actually be the case, but haven't been able to prove this.
\end{rem}


Before giving the proof, we state the essential technical tool of Lemma \ref{extension_lemma}, 
and then give an outline of the method of proof.

Without loss of generality, we will assume $X = Y$.

\medskip

Let $f: P \times X \to X$ be a family of homeomorphisms (e.g. a generator of $CH_*(X)$)
and let $S \sub X$.
We say that {\it $f$ is supported on $S$} if $f(p, x) = f(q, x)$ for all
$x \notin S$ and $p, q \in P$. Equivalently, $f$ is supported on $S$ if 
there is a family of homeomorphisms $f' : P \times S \to S$ and a ``background"
homeomorphism $f_0 : X \to X$ so that
\begin{align*}
	f(p,s) & = f_0(f'(p,s)) \;\;\;\; \mbox{for}\; (p, s) \in P\times S \\
\intertext{and}
	f(p,x) & = f_0(x) \;\;\;\; \mbox{for}\; (p, x) \in {P \times (X \setmin S)}.
\end{align*}
Note that if $f$ is supported on $S$ then it is also supported on any $R \sup S$.
(So when we talk about ``the" support of a family, there is some ambiguity,
but this ambiguity will not matter to us.)

Let $\cU = \{U_\alpha\}$ be an open cover of $X$.
A $k$-parameter family of homeomorphisms $f: P \times X \to X$ is
{\it adapted to $\cU$} 
if the support of $f$ is contained in the union
of at most $k$ of the $U_\alpha$'s.

\begin{lemma}  \label{extension_lemma}
Let $x \in CH_k(X)$ be a singular chain such that $\bd x$ is adapted to $\cU$.
Then $x$ is homotopic (rel boundary) to some $x' \in CH_k(X)$ which is adapted to $\cU$.
Furthermore, one can choose the homotopy so that its support is equal to the support of $x$.
\end{lemma}

The proof will be given in \S\ref{sec:localising}.

\medskip

Before diving into the details, we outline our strategy for the proof of Theorem \ref{thm:CH}.
Let $p$ be a singular cell in $CH_k(X)$ and $b$ be a blob diagram in $\bc_*(X)$.
We say that $p\ot b$ is {\it localizable} if there exists $V \sub X$ such that
\begin{itemize}
\item $V$ is homeomorphic to a disjoint union of balls, and
\item $\supp(p) \cup \supp(b) \sub V$.
\end{itemize}
(Recall that $\supp(b)$ is defined to be the union of the blobs of the diagram $b$.)

Assuming that $p\ot b$ is localizable as above, 
let $W = X \setmin V$, $W' = p(W)$ and $V' = X\setmin W'$.
We then have a factorization 
\[
	p = \gl(q, r),
\]
where $q \in CH_k(V, V')$ and $r \in CH_0(W, W')$.
We can also factorize $b = \gl(b_V, b_W)$, where $b_V\in \bc_*(V)$ and $b_W\in\bc_0(W)$.
According to the commutative diagram of the proposition, we must have
\[
	e_X(p\otimes b) = e_X(\gl(q\otimes b_V, r\otimes b_W)) = 
				gl(e_{VV'}(q\otimes b_V), e_{WW'}(r\otimes b_W)) .
\]
Since $r$ is a  0-parameter family of homeomorphisms, we must have
\[
	e_{WW'}(r\otimes b_W) = r(b_W),
\]
where $r(b_W)$ denotes the obvious action of homeomorphisms on blob diagrams (in
this case a 0-blob diagram).
Since $V'$ is a disjoint union of balls, $\bc_*(V')$ is acyclic in degrees $>0$ 
(by Properties \ref{property:disjoint-union} and \ref{property:contractibility}).
Assuming inductively that we have already defined $e_{VV'}(\bd(q\otimes b_V))$,
there is, up to (iterated) homotopy, a unique choice for $e_{VV'}(q\otimes b_V)$
such that 
\[
	\bd(e_{VV'}(q\otimes b_V)) = e_{VV'}(\bd(q\otimes b_V)) .
\]

Thus the conditions of the proposition determine (up to homotopy) the evaluation
map for localizable generators $p\otimes b$.
On the other hand, Lemma \ref{extension_lemma} allows us to homotope 
arbitrary generators to sums of localizable generators.
This (roughly) establishes the uniqueness part of the proposition.
To show existence, we must show that the various choices involved in constructing
evaluation maps in this way affect the final answer only by a homotopy.

Now for a little more detail.
(But we're still just motivating the full, gory details, which will follow.)
Choose a metric on $X$, and let $\cU_\gamma$ be the open cover of $X$ by balls of radius $\gamma$.
By Lemma \ref{extension_lemma} we can restrict our attention to $k$-parameter families 
$p$ of homeomorphisms such that $\supp(p)$ is contained in the union of $k$ $\gamma$-balls.
For fixed blob diagram $b$ and fixed $k$, it's not hard to show that for $\gamma$ small enough
$p\ot b$ must be localizable.
On the other hand, for fixed $k$ and $\gamma$ there exist $p$ and $b$ such that $p\ot b$ is not localizable,
and for fixed $\gamma$ and $b$ there exist non-localizable $p\ot b$ for sufficiently large $k$.
Thus we will need to take an appropriate limit as $\gamma$ approaches zero.

The construction of $e_X$, as outlined above, depends on various choices, one of which 
is the choice, for each localizable generator $p\ot b$, 
of disjoint balls $V$ containing $\supp(p)\cup\supp(b)$.
Let $V'$ be another disjoint union of balls containing $\supp(p)\cup\supp(b)$,
and assume that there exists yet another disjoint union of balls $W$ containing 
$V\cup V'$.
Then we can use $W$ to construct a homotopy between the two versions of $e_X$ 
associated to $V$ and $V'$.
If we impose no constraints on $V$ and $V'$ then such a $W$ need not exist.
Thus we will insist below that $V$ (and $V'$) be contained in small metric neighborhoods
of $\supp(p)\cup\supp(b)$.
Because we want not mere homotopy uniqueness but iterated homotopy uniqueness,
we will similarly require that $W$ be contained in a slightly larger metric neighborhood of 
$\supp(p)\cup\supp(b)$, and so on.


\begin{proof}[Proof of Theorem \ref{thm:CH}.]
We'll use the notation $|b| = \supp(b)$ and $|p| = \supp(p)$.

Choose a metric on $X$.
Choose a monotone decreasing sequence of positive real numbers $\ep_i$ converging to zero
(e.g.\ $\ep_i = 2^{-i}$).
Choose another sequence of positive real numbers $\delta_i$ such that $\delta_i/\ep_i$
converges monotonically to zero (e.g.\ $\delta_i = \ep_i^2$).
Let $\phi_l$ be an increasing sequence of positive numbers
satisfying the inequalities of Lemma \ref{xx2phi} below.
Given a generator $p\otimes b$ of $CH_*(X)\otimes \bc_*(X)$ and non-negative integers $i$ and $l$
define
\[
	N_{i,l}(p\ot b) \deq \Nbd_{l\ep_i}(|b|) \cup \Nbd_{\phi_l\delta_i}(|p|).
\]
In other words, for each $i$
we use the metric to choose nested neighborhoods of $|b|\cup |p|$ (parameterized
by $l$), with $\ep_i$ controlling the size of the buffers around $|b|$ and $\delta_i$ controlling
the size of the buffers around $|p|$.

Next we define subcomplexes $G_*^{i,m} \sub CH_*(X)\otimes \bc_*(X)$.
Let $p\ot b$ be a generator of $CH_*(X)\otimes \bc_*(X)$ and let $k = \deg(p\ot b)
= \deg(p) + \deg(b)$.
We say $p\ot b$ is in $G_*^{i,m}$ exactly when either (a) $\deg(p) = 0$ or (b)
there exist codimension-zero submanifolds $V_0,\ldots,V_m \sub X$ such that each $V_j$
is homeomorphic to a disjoint union of balls and
\[
	N_{i,k}(p\ot b) \subeq V_0 \subeq N_{i,k+1}(p\ot b)
			\subeq V_1 \subeq \cdots \subeq V_m \subeq N_{i,k+m+1}(p\ot b) ,
\]
and further $\bd(p\ot b) \in G_*^{i,m}$.
We also require that $b$ is splitable (transverse) along the boundary of each $V_l$.

Note that $G_*^{i,m+1} \subeq G_*^{i,m}$.

As sketched above and explained in detail below, 
$G_*^{i,m}$ is a subcomplex where it is easy to define
the evaluation map.
The parameter $m$ controls the number of iterated homotopies we are able to construct
(see Lemma \ref{m_order_hty}).
The larger $i$ is (i.e.\ the smaller $\ep_i$ is), the better $G_*^{i,m}$ approximates all of
$CH_*(X)\ot \bc_*(X)$ (see Lemma \ref{Gim_approx}).

Next we define a chain map (dependent on some choices) $e_{i,m}: G_*^{i,m} \to \bc_*(X)$.
(When the domain is clear from context we will drop the subscripts and write
simply  $e: G_*^{i,m} \to \bc_*(X)$).
Let $p\ot b \in G_*^{i,m}$.
If $\deg(p) = 0$, define
\[
	e(p\ot b) = p(b) ,
\]
where $p(b)$ denotes the obvious action of the homeomorphism(s) $p$ on the blob diagram $b$.
For general $p\ot b$ ($\deg(p) \ge 1$) assume inductively that we have already defined
$e(p'\ot b')$ when $\deg(p') + \deg(b') < k = \deg(p) + \deg(b)$.
Choose $V = V_0$ as above so that 
\[
	N_{i,k}(p\ot b) \subeq V \subeq N_{i,k+1}(p\ot b) .
\]
Let $\bd(p\ot b) = \sum_j p_j\ot b_j$, and let $V^j$ be the choice of neighborhood
of $|p_j|\cup |b_j|$ made at the preceding stage of the induction.
For all $j$, 
\[
	V^j \subeq N_{i,k}(p_j\ot b_j) \subeq N_{i,k}(p\ot b) \subeq V .
\]
(The second inclusion uses the facts that $|p_j| \subeq |p|$ and $|b_j| \subeq |b|$.)
We therefore have splittings
\[
	p = p'\bullet p'' , \;\; b = b'\bullet b'' , \;\; e(\bd(p\ot b)) = f'\bullet f'' ,
\]
where $p' \in CH_*(V)$, $p'' \in CH_*(X\setmin V)$, 
$b' \in \bc_*(V)$, $b'' \in \bc_*(X\setmin V)$, 
$f' \in \bc_*(p(V))$, and $f'' \in \bc_*(p(X\setmin V))$.
(Note that since the family of homeomorphisms $p$ is constant (independent of parameters)
near $\bd V$, the expressions $p(V) \sub X$ and $p(X\setmin V) \sub X$ are
unambiguous.)
We have $\deg(p'') = 0$ and, inductively, $f'' = p''(b'')$.
%We also have that $\deg(b'') = 0 = \deg(p'')$.
Choose $x' \in \bc_*(p(V))$ such that $\bd x' = f'$.
This is possible by Properties \ref{property:disjoint-union} and \ref{property:contractibility}  and the fact that isotopic fields
differ by a local relation.
Finally, define
\[
	e(p\ot b) \deq x' \bullet p''(b'') .
\]

Note that above we are essentially using the method of acyclic models \nn{\S \ref{sec:moam}}.
For each generator $p\ot b$ we specify the acyclic (in positive degrees) 
target complex $\bc_*(p(V)) \bullet p''(b'')$.

The definition of $e: G_*^{i,m} \to \bc_*(X)$ depends on two sets of choices:
The choice of neighborhoods $V$ and the choice of inverse boundaries $x'$.
The next lemma shows that up to (iterated) homotopy $e$ is independent
of these choices.
(Note that independence of choices of $x'$ (for fixed choices of $V$)
is a standard result in the method of acyclic models.)

%\begin{lemma}
%Let $\tilde{e} :  G_*^{i,m} \to \bc_*(X)$ be a chain map constructed like $e$ above, but with
%different choices of $x'$ at each step.
%(Same choice of $V$ at each step.)
%Then $e$ and $\tilde{e}$ are homotopic via a homotopy in $\bc_*(p(V)) \bullet p''(b'')$.
%Any two choices of such a first-order homotopy are second-order homotopic, and so on, 
%to arbitrary order.
%\end{lemma}

%\begin{proof}
%This is a standard result in the method of acyclic models.
%\nn{should we say more here?}
%\nn{maybe this lemma should be subsumed into the next lemma.  probably it should.}
%\end{proof}

\begin{lemma} \label{m_order_hty}
Let $\tilde{e} :  G_*^{i,m} \to \bc_*(X)$ be a chain map constructed like $e$ above, but with
different choices of $V$ (and hence also different choices of $x'$) at each step.
If $m \ge 1$ then $e$ and $\tilde{e}$ are homotopic.
If $m \ge 2$ then any two choices of this first-order homotopy are second-order homotopic.
Continuing, $e :  G_*^{i,m} \to \bc_*(X)$ is well-defined up to $m$-th order homotopy.
\end{lemma}

\begin{proof}
We construct $h: G_*^{i,m} \to \bc_*(X)$ such that $\bd h + h\bd = e - \tilde{e}$.
The chain maps $e$ and $\tilde{e}$ coincide on bidegrees $(0, j)$, so define $h$
to be zero there.
Assume inductively that $h$ has been defined for degrees less than $k$.
Let $p\ot b$ be a generator of degree $k$.
Choose $V_1$ as in the definition of $G_*^{i,m}$ so that
\[
	N_{i,k+1}(p\ot b) \subeq V_1 \subeq N_{i,k+2}(p\ot b) .
\]
There are splittings
\[
	p = p'_1\bullet p''_1 , \;\; b = b'_1\bullet b''_1 , 
			\;\; e(p\ot b) - \tilde{e}(p\ot b) - h(\bd(p\ot b)) = f'_1\bullet f''_1 ,
\]
where $p'_1 \in CH_*(V_1)$, $p''_1 \in CH_*(X\setmin V_1)$, 
$b'_1 \in \bc_*(V_1)$, $b''_1 \in \bc_*(X\setmin V_1)$, 
$f'_1 \in \bc_*(p(V_1))$, and $f''_1 \in \bc_*(p(X\setmin V_1))$.
Inductively, $\bd f'_1 = 0$ and $f_1'' = p_1''(b_1'')$.
Choose $x'_1 \in \bc_*(p(V_1))$ so that $\bd x'_1 = f'_1$.
Define 
\[
	h(p\ot b) \deq x'_1 \bullet p''_1(b''_1) .
\]
This completes the construction of the first-order homotopy when $m \ge 1$.

The $j$-th order homotopy is constructed similarly, with $V_j$ replacing $V_1$ above.
\end{proof}

Note that on $G_*^{i,m+1} \subeq G_*^{i,m}$, we have defined two maps,
$e_{i,m}$ and $e_{i,m+1}$.
An easy variation on the above lemma shows that 
the restrictions of $e_{i,m}$ and $e_{i,m+1}$ to $G_*^{i,m+1}$ are $m$-th 
order homotopic.

Next we show how to homotope chains in $CH_*(X)\ot \bc_*(X)$ to one of the 
$G_*^{i,m}$.
Choose a monotone decreasing sequence of real numbers $\gamma_j$ converging to zero.
Let $\cU_j$ denote the open cover of $X$ by balls of radius $\gamma_j$.
Let $h_j: CH_*(X)\to CH_*(X)$ be a chain map homotopic to the identity whose image is 
spanned by families of homeomorphisms with support compatible with $\cU_j$, 
as described in Lemma \ref{extension_lemma}.
Recall that $h_j$ and also the homotopy connecting it to the identity do not increase
supports.
Define
\[
	g_j \deq h_j\circ h_{j-1} \circ \cdots \circ h_1 .
\]
The next lemma says that for all generators $p\ot b$ we can choose $j$ large enough so that
$g_j(p)\ot b$ lies in $G_*^{i,m}$, for arbitrary $m$ and sufficiently large $i$ 
(depending on $b$, $\deg(p)$ and $m$).
%(Note: Don't confuse this $n$ with the top dimension $n$ used elsewhere in this paper.)

\begin{lemma} \label{Gim_approx}
Fix a blob diagram $b$, a homotopy order $m$ and a degree $n$ for $CH_*(X)$.
Then there exists a constant $k_{bmn}$ such that for all $i \ge k_{bmn}$
there exists another constant $j_{ibmn}$ such that for all $j \ge j_{ibmn}$ and all $p\in CH_n(X)$ 
we have $g_j(p)\ot b \in G_*^{i,m}$.
\end{lemma}

For convenience we also define $k_{bmp} = k_{bmn}$
and $j_{ibmp} = j_{ibmn}$ where $n=\deg(p)$.
Note that we may assume that
\[
	k_{bmp} \ge k_{alq}
\]
for all $l\ge m$ and all $q\ot a$ which appear in the boundary of $p\ot b$.
Additionally, we may assume that
\[
	j_{ibmp} \ge j_{ialq}
\]
for all $l\ge m$ and all $q\ot a$ which appear in the boundary of $p\ot b$.


\begin{proof}

There exists $\lambda > 0$ such that for every  subset $c$ of the blobs of $b$ the set $\Nbd_u(c)$ is homeomorphic to $|c|$ for all $u < \lambda$ .
(Here we are using the fact that the blobs are 
piecewise smooth or piecewise-linear and that $\bd c$ is collared.)
We need to consider all such $c$ because all generators appearing in
iterated boundaries of $p\ot b$ must be in $G_*^{i,m}$.)

Let $r = \deg(b)$ and 
\[
	t = r+n+m+1 = \deg(p\ot b) + m + 1.
\]

Choose $k = k_{bmn}$ such that
\[
	t\ep_k < \lambda
\]
and
\[
	n\cdot (2 (\phi_t + 1) \delta_k) < \ep_k .
\]
Let $i \ge k_{bmn}$.
Choose $j = j_i$ so that
\[
	\gamma_j < \delta_i
\]
and also so that $\phi_t \gamma_j$ is less than the constant $\rho(M)$ of Lemma \ref{xxzz11}.

Let $j \ge j_i$ and $p\in CH_n(X)$.
Let $q$ be a generator appearing in $g_j(p)$.
Note that $|q|$ is contained in a union of $n$ elements of the cover $\cU_j$,
which implies that $|q|$ is contained in a union of $n$ metric balls of radius $\delta_i$.
We must show that $q\ot b \in G_*^{i,m}$, which means finding neighborhoods
$V_0,\ldots,V_m \sub X$ of $|q|\cup |b|$ such that each $V_j$
is homeomorphic to a disjoint union of balls and
\[
	N_{i,n}(q\ot b) \subeq V_0 \subeq N_{i,n+1}(q\ot b)
			\subeq V_1 \subeq \cdots \subeq V_m \subeq N_{i,t}(q\ot b) .
\]
Recall that
\[
	N_{i,a}(q\ot b) \deq \Nbd_{a\ep_i}(|b|) \cup \Nbd_{\phi_a\delta_i}(|q|).
\]
By repeated applications of Lemma \ref{xx2phi} we can find neighborhoods $U_0,\ldots,U_m$
of $|q|$, each homeomorphic to a disjoint union of balls, with
\[
	\Nbd_{\phi_{n+l} \delta_i}(|q|) \subeq U_l \subeq \Nbd_{\phi_{n+l+1} \delta_i}(|q|) .
\]
The inequalities above guarantee that 
for each $0\le l\le m$ we can find $u_l$ with 
\[
	(n+l)\ep_i \le u_l \le (n+l+1)\ep_i
\]
such that each component of $U_l$ is either disjoint from $\Nbd_{u_l}(|b|)$ or contained in 
$\Nbd_{u_l}(|b|)$.
This is because there are at most $n$ components of $U_l$, and each component
has radius $\le (\phi_t + 1) \delta_i$.
It follows that
\[
	V_l \deq \Nbd_{u_l}(|b|) \cup U_l
\]
is homeomorphic to a disjoint union of balls and satisfies
\[
	N_{i,n+l}(q\ot b) \subeq V_l \subeq N_{i,n+l+1}(q\ot b) .
\]

The same argument shows that each generator involved in iterated boundaries of $q\ot b$
is in $G_*^{i,m}$.
\end{proof}

In the next three lemmas, which provide the estimates needed above, we have made no effort to optimize the various bounds.
(The bounds are, however, optimal in the sense of minimizing the amount of work
we do.  Equivalently, they are the first bounds we thought of.)

We say that a subset $S$ of a metric space has radius $\le r$ if $S$ is contained in
some metric ball of radius $r$.

\begin{lemma}
Let $S \sub \ebb^n$ (Euclidean $n$-space) have radius $\le r$.  
Then $\Nbd_a(S)$ is homeomorphic to a ball for $a \ge 2r$.
\end{lemma}

\begin{proof} \label{xxyy2}
Let $S$ be contained in $B_r(y)$, $y \in \ebb^n$.
Note that if $a \ge 2r$ then $\Nbd_a(S) \sup B_r(y)$.
Let $z\in \Nbd_a(S) \setmin B_r(y)$.
Consider the triangle
with vertices $z$, $y$ and $s$ with $s\in S$ such that $z \in B_a(s)$.
The length of the edge $yz$ is greater than $r$ which is greater
than the length of the edge $ys$.
It follows that the angle at $z$ is less than $\pi/2$ (less than $\pi/3$, in fact),
which means that points on the edge $yz$ near $z$ are closer to $s$ than $z$ is,
which implies that these points are also in $\Nbd_a(S)$.
Hence $\Nbd_a(S)$ is star-shaped with respect to $y$.
\end{proof}

If we replace $\ebb^n$ above with an arbitrary compact Riemannian manifold $M$,
the same result holds, so long as $a$ is not too large:
\nn{replace this with a PL version}

\begin{lemma} \label{xxzz11}
Let $M$ be a compact Riemannian manifold.
Then there is a constant $\rho(M)$ such that for all
subsets $S\sub M$ of radius $\le r$ and all $a$ such that $2r \le a \le \rho(M)$,
$\Nbd_a(S)$ is homeomorphic to a ball.
\end{lemma}

\begin{proof}
Choose $\rho = \rho(M)$ such that $3\rho/2$ is less than the radius of injectivity of $M$,
and also so that for any point $y\in M$ the geodesic coordinates of radius $3\rho/2$ around
$y$ distort angles by only a small amount.
Now the argument of the previous lemma works.
\end{proof}



\begin{lemma} \label{xx2phi}
Let $S \sub M$ be contained in a union (not necessarily disjoint)
of $k$ metric balls of radius $r$.
Let $\phi_1, \phi_2, \ldots$ be an increasing sequence of real numbers satisfying
$\phi_1 \ge 2$ and $\phi_{i+1} \ge \phi_i(2\phi_i + 2) + \phi_i$.
For convenience, let $\phi_0 = 0$.
Assume also that $\phi_k r \le \rho(M)$,
where $\rho(M)$ is as in Lemma \ref{xxzz11}.
Then there exists a neighborhood $U$ of $S$,
homeomorphic to a disjoint union of balls, such that
\[
	\Nbd_{\phi_{k-1} r}(S) \subeq U \subeq \Nbd_{\phi_k r}(S) .
\]
\end{lemma}

\begin{proof}
For $k=1$ this follows from Lemma \ref{xxzz11}.
Assume inductively that it holds for $k-1$.
Partition $S$ into $k$ disjoint subsets $S_1,\ldots,S_k$, each of radius $\le r$.
By Lemma \ref{xxzz11}, each $\Nbd_{\phi_{k-1} r}(S_i)$ is homeomorphic to a ball.
If these balls are disjoint, let $U$ be their union.
Otherwise, assume WLOG that $S_{k-1}$ and $S_k$ are distance less than $2\phi_{k-1}r$ apart.
Let $R_i = \Nbd_{\phi_{k-1} r}(S_i)$ for $i = 1,\ldots,k-2$ 
and $R_{k-1} = \Nbd_{\phi_{k-1} r}(S_{k-1})\cup \Nbd_{\phi_{k-1} r}(S_k)$.
Each $R_i$ is contained in a metric ball of radius $r' \deq (2\phi_{k-1}+2)r$.
Note that the defining inequality of the $\phi_i$ guarantees that
\[
	\phi_{k-1}r' = \phi_{k-1}(2\phi_{k-1}+2)r \le \phi_k r \le \rho(M) .
\]
By induction, there is a neighborhood $U$ of $R \deq \bigcup_i R_i$, 
homeomorphic to a disjoint union
of balls, and such that
\[
	U \subeq \Nbd_{\phi_{k-1}r'}(R) = \Nbd_{t}(S) \subeq \Nbd_{\phi_k r}(S) ,
\]
where $t = \phi_{k-1}(2\phi_{k-1}+2)r + \phi_{k-1} r$.
\end{proof}


We now return to defining the chain maps $e_X$.


Let $R_*$ be the chain complex with a generating 0-chain for each non-negative
integer and a generating 1-chain connecting each adjacent pair $(j, j+1)$.
(So $R_*$ is a simplicial version of the non-negative reals.)
Denote the 0-chains by $j$ (for $j$ a non-negative integer) and the 1-chain connecting $j$ and $j+1$
by $\iota_j$.
Define a map (homotopy equivalence)
\[
	\sigma: R_*\ot CH_*(X, X) \otimes \bc_*(X) \to CH_*(X, X)\ot \bc_*(X)
\]
as follows.
On $R_0\ot CH_*(X, X) \otimes \bc_*(X)$ we define
\[
	\sigma(j\ot p\ot b) = g_j(p)\ot b .
\]
On $R_1\ot CH_*(X, X) \otimes \bc_*(X)$ we define
\[
	\sigma(\iota_j\ot p\ot b) = f_j(p)\ot b ,
\]
where $f_j$ is the homotopy from $g_j$ to $g_{j+1}$.

Next we specify subcomplexes $G^m_* \sub R_*\ot CH_*(X, X) \otimes \bc_*(X)$ on which we will eventually
define a version of the action map $e_X$.
A generator $j\ot p\ot b$ is defined to be in $G^m_*$ if $j\ge j_{kbmp}$, where
$k = k_{bmp}$ is the constant from Lemma \ref{Gim_approx}.
Similarly $\iota_j\ot p\ot b$ is in $G^m_*$ if $j\ge j_{kbmp}$.
The inequality following Lemma \ref{Gim_approx} guarantees that $G^m_*$ is indeed a subcomplex
and that $G^m_* \sup G^{m+1}_*$.

It is easy to see that each $G^m_*$ is homotopy equivalent (via the inclusion map) 
to $R_*\ot CH_*(X, X) \otimes \bc_*(X)$
and hence to $CH_*(X, X) \otimes \bc_*(X)$, and furthermore that the homotopies are well-defined
up to a contractible set of choices.

Next we define a map
\[
	e_m : G^m_* \to \bc_*(X) .
\]
Let $p\ot b$ be a generator of $G^m_*$.
Each $g_j(p)\ot b$ or $f_j(p)\ot b$ is a linear combination of generators $q\ot c$,
where $\supp(q)\cup\supp(c)$ is contained in a disjoint union of balls satisfying 
various conditions specified above.
As in the construction of the maps $e_{i,m}$ above,
it suffices to specify for each such $q\ot c$ a disjoint union of balls
$V_{qc} \sup \supp(q)\cup\supp(c)$, such that $V_{qc} \sup V_{q'c'}$
whenever $q'\ot c'$ appears in the boundary of $q\ot c$.

Let $q\ot c$ be a summand of $g_j(p)\ot b$, as above.
Let $i$ be maximal such that $j\ge j_{ibmp}$
(notation as in Lemma \ref{Gim_approx}).
Then $q\ot c \in G^{i,m}_*$ and we choose $V_{qc} \sup \supp(q)\cup\supp(c)$
such that 
\[
	N_{i,d}(q\ot c) \subeq V_{qc} \subeq N_{i,d+1}(q\ot c) ,
\]
where $d = \deg(q\ot c)$.
Let $\tilde q = f_j(q)$.
The summands of $f_j(p)\ot b$ have the form $\tilde q \ot c$, 
where $q\ot c$ is a summand of $g_j(p)\ot b$.
Since the homotopy $f_j$ does not increase supports, we also have that
\[
	V_{qc} \sup \supp(\tilde q) \cup \supp(c) .
\]
So we define $V_{\tilde qc} = V_{qc}$.

It is now easy to check that we have $V_{qc} \sup V_{q'c'}$
whenever $q'\ot c'$ appears in the boundary of $q\ot c$.
As in the construction of the maps $e_{i,m}$ above,
this allows us to construct a map
\[
	e_m : G^m_* \to \bc_*(X) 
\]
which is well-defined up to homotopy.
As in the proof of Lemma \ref{m_order_hty}, we can show that the map is well-defined up
to $m$-th order homotopy.
Put another way, we have specified an $m$-connected subcomplex of the complex of
all maps $G^m_* \to \bc_*(X)$.
On $G^{m+1}_* \sub G^m_*$ we have defined two maps, $e_m$ and $e_{m+1}$.
One can similarly (to the proof of Lemma \ref{m_order_hty}) show that 
these two maps agree up to $m$-th order homotopy.
More precisely, one can show that the subcomplex of maps containing the various
$e_{m+1}$ candidates is contained in the corresponding subcomplex for $e_m$.

\medskip

Next we show that the action maps are compatible with gluing.
Let $G^m_*$ and $\ol{G}^m_*$ be the complexes, as above, used for defining
the action maps $e_{X\sgl}$ and $e_X$.
The gluing map $X\sgl\to X$ induces a map
\[
	\gl:  R_*\ot CH_*(X, X) \otimes \bc_*(X)  \to R_*\ot CH_*(X\sgl, X \sgl) \otimes \bc_*(X \sgl) ,
\]
and it is easy to see that $\gl(G^m_*)\sub \ol{G}^m_*$.
From this it follows that the diagram in the statement of Theorem \ref{thm:CH} commutes.

\todo{this paragraph isn't very convincing, or at least I don't see what's going on}
Finally we show that the action maps defined above are independent of
the choice of metric (up to iterated homotopy).
The arguments are very similar to ones given above, so we only sketch them.
Let $g$ and $g'$ be two metrics on $X$, and let $e$ and $e'$ be the corresponding
actions $CH_*(X, X) \ot \bc_*(X)\to\bc_*(X)$.
We must show that $e$ and $e'$ are homotopic.
As outlined in the discussion preceding this proof,
this follows from the facts that both $e$ and $e'$ are compatible
with gluing and that $\bc_*(B^n)$ is contractible.
As above, we define a subcomplex $F_*\sub  CH_*(X, X) \ot \bc_*(X)$ generated
by $p\ot b$ such that $|p|\cup|b|$ is contained in a disjoint union of balls.
Using acyclic models, we can construct a homotopy from $e$ to $e'$ on $F_*$.
We now observe that $CH_*(X, X) \ot \bc_*(X)$ retracts to $F_*$.
Similar arguments show that this homotopy from $e$ to $e'$ is well-defined
up to second order homotopy, and so on.

This completes the proof of Theorem \ref{thm:CH}.
\end{proof}


\begin{rem*}
\label{rem:for-small-blobs}
For the proof of Lemma \ref{lem:CH-small-blobs} below we will need the following observation on the action constructed above.
Let $b$ be a blob diagram and $p:P\times X\to X$ be a family of homeomorphisms.
Then we may choose $e$ such that $e(p\ot b)$ is a sum of generators, each
of which has support close to $p(t,|b|)$ for some $t\in P$.
More precisely, the support of the generators is contained in the union of a small neighborhood
of $p(t,|b|)$ with some small balls.
(Here ``small" is in terms of the metric on $X$ that we chose to construct $e$.)
\end{rem*}


\begin{thm}
\label{thm:CH-associativity}
The $CH_*(X, Y)$ actions defined above are associative.
That is, the following diagram commutes up to homotopy:
\[ \xymatrix{
& CH_*(Y, Z) \ot \bc_*(Y) \ar[dr]^{e_{YZ}} & \\
CH_*(X, Y) \ot CH_*(Y, Z) \ot \bc_*(X) \ar[ur]^{e_{XY}\ot\id} \ar[dr]_{\mu\ot\id} & & \bc_*(Z) \\
& CH_*(X, Z) \ot \bc_*(X) \ar[ur]_{e_{XZ}} &
} \]
Here $\mu:CH_*(X, Y) \ot CH_*(Y, Z)\to CH_*(X, Z)$ is the map induced by composition
of homeomorphisms.
\end{thm}

\begin{proof}
The strategy of the proof is similar to that of Theorem \ref{thm:CH}.
We will identify a subcomplex 
\[
	G_* \sub CH_*(X, Y) \ot CH_*(Y, Z) \ot \bc_*(X)
\]
where it is easy to see that the two sides of the diagram are homotopic, then 
show that there is a deformation retraction of $CH_*(X, Y) \ot CH_*(Y, Z) \ot \bc_*(X)$ into $G_*$.

Let $p\ot q\ot b$ be a generator of $CH_*(X, Y) \ot CH_*(Y, Z) \ot \bc_*(X)$.
By definition, $p\ot q\ot b\in G_*$ if there is a disjoint union of balls in $X$ which
contains $|p| \cup p\inv(|q|) \cup |b|$.
(If $p:P\times X\to Y$, then $p\inv(|q|)$ means the union over all $x\in P$ of 
$p(x, \cdot)\inv(|q|)$.)

As in the proof of Theorem \ref{thm:CH}, we can construct a homotopy 
between the upper and lower maps restricted to $G_*$.
This uses the facts that the maps agree on $CH_0(X, Y) \ot CH_0(Y, Z) \ot \bc_*(X)$,
that they are compatible with gluing, and the contractibility of $\bc_*(X)$.

We can now apply Lemma \ref{extension_lemma_c}, using a series of increasingly fine covers, 
to construct a deformation retraction of $CH_*(X, Y) \ot CH_*(Y, Z) \ot \bc_*(X)$ into $G_*$.
\end{proof}

} % end \noop


%!TEX root = ../blob1.tex

\def\xxpar#1#2{\smallskip\noindent{\bf #1} {\it #2} \smallskip}
\def\mmpar#1#2#3{\smallskip\noindent{\bf #1} (#2). {\it #3} \smallskip}

\section{\texorpdfstring{$n$}{n}-categories and their modules}
\label{sec:ncats}

\subsection{Definition of \texorpdfstring{$n$}{n}-categories}
\label{ss:n-cat-def}

Before proceeding, we need more appropriate definitions of $n$-categories, 
$A_\infty$ $n$-categories, as well as modules for these, and tensor products of these modules.
(As is the case throughout this paper, by ``$n$-category" we mean some notion of
a ``weak" $n$-category with ``strong duality".)

Compared to other definitions in the literature,
the definitions presented below tie the categories more closely to the topology
and avoid combinatorial questions about, for example, finding a minimal sufficient
collection of generalized associativity axioms; we prefer maximal sets of axioms to minimal sets.
It is easy to show that examples of topological origin
(e.g.\ categories whose morphisms are maps into spaces or decorated balls, or bordism categories), 
satisfy our axioms.
To show that examples of a more purely algebraic origin satisfy our axioms, 
one would typically need the combinatorial
results that we have avoided here.

See \S\ref{n-cat-names} for a discussion of $n$-category terminology.

%\nn{Say something explicit about Lurie's work here? 
%It seems like this was something that Dan Freed wanted explaining when we talked to him in Aspen}

\medskip

The axioms for an $n$-category are spread throughout this section.
Collecting these together, an $n$-category is a gadget satisfying Axioms \ref{axiom:morphisms}, \ref{nca-boundary}, \ref{axiom:composition},  \ref{nca-assoc}, \ref{axiom:product} and \ref{axiom:extended-isotopies}; for an $A_\infty$ $n$-category, we replace Axiom \ref{axiom:extended-isotopies} with Axiom \ref{axiom:families}.

Strictly speaking, before we can state the axioms for $k$-morphisms we need all the axioms 
for $k{-}1$-morphisms.
So readers who prefer things to be presented in a strictly logical order should read this subsection $n$ times, first imagining that $k=0$, then that $k=1$, and so on until they reach $k=n$.

\medskip

There are many existing definitions of $n$-categories, with various intended uses.
In any such definition, there are sets of $k$-morphisms for each $0 \leq k \leq n$.
Generally, these sets are indexed by instances of a certain typical shape. 
Some $n$-category definitions model $k$-morphisms on the standard bihedron (interval, bigon, and so on).
Other definitions have a separate set of 1-morphisms for each interval $[0,l] \sub \r$, 
a separate set of 2-morphisms for each rectangle $[0,l_1]\times [0,l_2] \sub \r^2$,
and so on.
(This allows for strict associativity; see \cite{ulrike-tillmann-2008,0909.2212}.)
Still other definitions (see, for example, \cite{MR2094071})
model the $k$-morphisms on more complicated combinatorial polyhedra.

For our definition, we will allow our $k$-morphisms to have any shape, so long as it is homeomorphic to the standard $k$-ball.
Thus we associate a set of $k$-morphisms $\cC_k(X)$ to any $k$-manifold $X$ homeomorphic 
to the standard $k$-ball.
By ``a $k$-ball" we mean any $k$-manifold which is homeomorphic to the 
standard $k$-ball.
We {\it do not} assume that it is equipped with a 
preferred homeomorphism to the standard $k$-ball, and the same applies to ``a $k$-sphere" below.

Given a homeomorphism $f:X\to Y$ between $k$-balls (not necessarily fixed on 
the boundary), we want a corresponding
bijection of sets $f:\cC_k(X)\to \cC_k(Y)$.
(This will imply ``strong duality", among other things.) Putting these together, we have

\begin{axiom}[Morphisms]
\label{axiom:morphisms}
For each $0 \le k \le n$, we have a functor $\cC_k$ from 
the category of $k$-balls and 
homeomorphisms to the category of sets and bijections.
\end{axiom}


(Note: We often omit the subscript $k$.)

We are being deliberately vague about what flavor of $k$-balls
we are considering.
They could be unoriented or oriented or Spin or $\mbox{Pin}_\pm$.
They could be topological or PL or smooth.
%\nn{need to check whether this makes much difference}
(If smooth, ``homeomorphism" should be read ``diffeomorphism", and we would need
to be fussier about corners and boundaries.)
For each flavor of manifold there is a corresponding flavor of $n$-category.
For simplicity, we will concentrate on the case of PL unoriented manifolds.

An ambitious reader may want to keep in mind two other classes of balls.
The first is balls equipped with a map to some other space $Y$ (c.f. \cite{MR2079378}). 
This will be used below (see the end of \S \ref{ss:product-formula}) to describe the blob complex of a fiber bundle with
base space $Y$.
The second is balls equipped with a section of the tangent bundle, or the frame
bundle (i.e.\ framed balls), or more generally some partial flag bundle associated to the tangent bundle.
These can be used to define categories with less than the ``strong" duality we assume here,
though we will not develop that idea fully in this paper.

Next we consider domains and ranges of morphisms (or, as we prefer to say, boundaries
of morphisms).
The 0-sphere is unusual among spheres in that it is disconnected.
Correspondingly, for 1-morphisms it makes sense to distinguish between domain and range.
(Actually, this is only true in the oriented case, with 1-morphisms parameterized
by {\it oriented} 1-balls.)
For $k>1$ and in the presence of strong duality the division into domain and range makes less sense.
For example, in a pivotal tensor category, there are natural isomorphisms $\Hom{}{A}{B \tensor C} \isoto \Hom{}{B^* \tensor A}{C}$, etc. 
(sometimes called ``Frobenius reciprocity''), which canonically identify all the morphism spaces which have the same boundary.
We prefer not to make the distinction in the first place.

Instead, we will combine the domain and range into a single entity which we call the 
boundary of a morphism.
Morphisms are modeled on balls, so their boundaries are modeled on spheres.
In other words, we need to extend the functors $\cC_{k-1}$ from balls to spheres, for 
$1\le k \le n$.
At first it might seem that we need another axiom 
(more specifically, additional data) for this, but in fact once we have
all the axioms in this subsection for $0$ through $k-1$ we can use a colimit
construction, as described in \S\ref{ss:ncat-coend} below, to extend $\cC_{k-1}$
to spheres (and any other manifolds):

\begin{lem}
\label{lem:spheres}
For each $1 \le k \le n$, we have a functor $\cl{\cC}_{k-1}$ from 
the category of $k{-}1$-spheres and 
homeomorphisms to the category of sets and bijections.
\end{lem}

We postpone the proof of this result until after we've actually given all the axioms.
Note that defining this functor for some $k$ only requires the data described in Axiom \ref{axiom:morphisms} at level $k$, 
along with the data described in the other axioms at lower levels. 

%In fact, the functors for spheres are entirely determined by the functors for balls and the subsequent axioms. (In particular, $\cC(S^k)$ is the colimit of $\cC$ applied to decompositions of $S^k$ into balls.) However, it is easiest to think of it as additional data at this point.

\begin{axiom}[Boundaries]\label{nca-boundary}
For each $k$-ball $X$, we have a map of sets $\bd: \cC_k(X)\to \cl{\cC}_{k-1}(\bd X)$.
These maps, for various $X$, comprise a natural transformation of functors.
\end{axiom}

Note that the first ``$\bd$" above is part of the data for the category, 
while the second is the ordinary boundary of manifolds.
Given $c\in\cl{\cC}(\bd(X))$, we will write $\cC(X; c)$ for $\bd^{-1}(c)$, those morphisms with specified boundary $c$.

Most of the examples of $n$-categories we are interested in are enriched in the following sense.
The various sets of $n$-morphisms $\cC(X; c)$, for all $n$-balls $X$ and
all $c\in \cl{\cC}(\bd X)$, have the structure of an object in some auxiliary symmetric monoidal category
with sufficient limits and colimits
(e.g.\ vector spaces, or modules over some ring, or chain complexes),
%\nn{actually, need both disj-union/sum and product/tensor-product; what's the name for this sort of cat?}
and all the structure maps of the $n$-category should be compatible with the auxiliary
category structure.
Note that this auxiliary structure is only in dimension $n$; if $\dim(Y) < n$ then 
$\cC(Y; c)$ is just a plain set.

\medskip

In order to simplify the exposition we have concentrated on the case of 
unoriented PL manifolds and avoided the question of what exactly we mean by 
the boundary of a manifold with extra structure, such as an oriented manifold.
In general, all manifolds of dimension less than $n$ should be equipped with the germ
of a thickening to dimension $n$, and this germ should carry whatever structure we have 
on $n$-manifolds.
In addition, lower dimensional manifolds should be equipped with a framing
of their normal bundle in the thickening; the framing keeps track of which
side (iterated) bounded manifolds lie on.
For example, the boundary of an oriented $n$-ball
should be an $n{-}1$-sphere equipped with an orientation of its once stabilized tangent
bundle and a choice of direction in this bundle indicating
which side the $n$-ball lies on.

\medskip

We have just argued that the boundary of a morphism has no preferred splitting into
domain and range, but the converse meets with our approval.
That is, given compatible domain and range, we should be able to combine them into
the full boundary of a morphism.
The following lemma will follow from the colimit construction used to define $\cl{\cC}_{k-1}$
on spheres.

\begin{lem}[Boundary from domain and range]
\label{lem:domain-and-range}
Let $S = B_1 \cup_E B_2$, where $S$ is a $k{-}1$-sphere $(1\le k\le n)$,
$B_i$ is a $k{-}1$-ball, and $E = B_1\cap B_2$ is a $k{-}2$-sphere (Figure \ref{blah3}).
Let $\cC(B_1) \times_{\cl{\cC}(E)} \cC(B_2)$ denote the fibered product of the 
two maps $\bd: \cC(B_i)\to \cl{\cC}(E)$.
Then we have an injective map
\[
	\gl_E : \cC(B_1) \times_{\cl{\cC}(E)} \cC(B_2) \into \cl{\cC}(S)
\]
which is natural with respect to the actions of homeomorphisms.
(When $k=1$ we stipulate that $\cl{\cC}(E)$ is a point, so that the above fibered product
becomes a normal product.)
\end{lem}

\begin{figure}[!ht] \centering
\begin{tikzpicture}[%every label/.style={green}
]
\node[fill=black, circle, label=below:$E$, inner sep=1.5pt](S) at (0,0) {};
\node[fill=black, circle, label=above:$E$, inner sep=1.5pt](N) at (0,2) {};
\draw (S) arc  (-90:90:1);
\draw (N) arc  (90:270:1);
\node[left] at (-1,1) {$B_1$};
\node[right] at (1,1) {$B_2$};
\end{tikzpicture}
\caption{Combining two balls to get a full boundary.}\label{blah3}\end{figure}

Note that we insist on injectivity above. 
The lemma follows from Definition \ref{def:colim-fields} and Lemma \ref{lem:colim-injective}.
%\nn{we might want a more official looking proof...}

We do not insist on surjectivity of the gluing map, since this is not satisfied by all of the examples
we are trying to axiomatize.
If our $k$-morphisms $\cC(X)$ are labeled cell complexes embedded in $X$, then a $k$-morphism is
in the image of the gluing map precisely which the cell complex is in general position
with respect to $E$.

If $S$ is a 0-sphere (the case $k=1$ above), then $S$ can be identified with the {\it disjoint} union
of two 0-balls $B_1$ and $B_2$ and the colimit construction $\cl{\cC}(S)$ can be identified
with the (ordinary, not fibered) product $\cC(B_1) \times \cC(B_2)$.

Let $\cl{\cC}(S)\trans E$ denote the image of $\gl_E$.
We will refer to elements of $\cl{\cC}(S)\trans E$ as ``splittable along $E$" or ``transverse to $E$". 

If $X$ is a $k$-ball and $E \sub \bd X$ splits $\bd X$ into two $k{-}1$-balls $B_1$ and $B_2$
as above, then we define $\cC(X)\trans E = \bd^{-1}(\cl{\cC}(\bd X)\trans E)$.

We will call the projection $\cl{\cC}(S)\trans E \to \cC(B_i)$
a {\it restriction} map and write $\res_{B_i}(a)$
(or simply $\res(a)$ when there is no ambiguity), for $a\in \cl{\cC}(S)\trans E$.
More generally, we also include under the rubric ``restriction map"
the boundary maps of Axiom \ref{nca-boundary} above,
another class of maps introduced after Axiom \ref{nca-assoc} below, as well as any composition
of restriction maps.
In particular, we have restriction maps $\cC(X)\trans E \to \cC(B_i)$
($i = 1, 2$, notation from previous paragraph).
These restriction maps can be thought of as 
domain and range maps, relative to the choice of splitting $\bd X = B_1 \cup_E B_2$.


Next we consider composition of morphisms.
For $n$-categories which lack strong duality, one usually considers
$k$ different types of composition of $k$-morphisms, each associated to a different direction.
(For example, vertical and horizontal composition of 2-morphisms.)
In the presence of strong duality, these $k$ distinct compositions are subsumed into 
one general type of composition which can be in any ``direction".

\begin{axiom}[Composition]
\label{axiom:composition}
Let $B = B_1 \cup_Y B_2$, where $B$, $B_1$ and $B_2$ are $k$-balls ($0\le k\le n$)
and $Y = B_1\cap B_2$ is a $k{-}1$-ball (Figure \ref{blah5}).
Let $E = \bd Y$, which is a $k{-}2$-sphere.
Note that each of $B$, $B_1$ and $B_2$ has its boundary split into two $k{-}1$-balls by $E$.
We have restriction (domain or range) maps $\cC(B_i)\trans E \to \cC(Y)$.
Let $\cC(B_1)\trans E \times_{\cC(Y)} \cC(B_2)\trans E$ denote the fibered product of these two maps. 
We have a map
\[
	\gl_Y : \cC(B_1)\trans E \times_{\cC(Y)} \cC(B_2)\trans E \to \cC(B)\trans E
\]
which is natural with respect to the actions of homeomorphisms, and also compatible with restrictions
to the intersection of the boundaries of $B$ and $B_i$.
If $k < n$,
or if $k=n$ and we are in the $A_\infty$ case, 
we require that $\gl_Y$ is injective.
(For $k=n$ in the ordinary (non-$A_\infty$) case, see below.)
\end{axiom}

\begin{figure}[!ht] \centering
\begin{tikzpicture}[%every label/.style={green},
				x=1.5cm,y=1.5cm]
\node[fill=black, circle, label=below:$E$, inner sep=2pt](S) at (0,0) {};
\node[fill=black, circle, label=above:$E$, inner sep=2pt](N) at (0,2) {};
\draw (S) arc  (-90:90:1);
\draw (N) arc  (90:270:1);
\draw (N) -- (S);
\node[left] at (-1/4,1) {$B_1$};
\node[right] at (1/4,1) {$B_2$};
\node at (1/6,3/2)  {$Y$};
\end{tikzpicture}
\caption{From two balls to one ball.}\label{blah5}\end{figure}

\begin{axiom}[Strict associativity] \label{nca-assoc}
The composition (gluing) maps above are strictly associative.
Given any splitting of a ball $B$ into smaller balls
$$\bigsqcup B_i \to B,$$ 
any sequence of gluings (in the sense of Definition \ref{defn:gluing-decomposition}, where all the intermediate steps are also disjoint unions of balls) yields the same result.
\end{axiom}

\begin{figure}[!ht]
$$\mathfig{.65}{ncat/strict-associativity}$$
\caption{An example of strict associativity.}\label{blah6}\end{figure}

We'll use the notation  $a\bullet b$ for the glued together field $\gl_Y(a, b)$.
In the other direction, we will call the projection from $\cC(B)\trans E$ to $\cC(B_i)\trans E$ 
a restriction map (one of many types of map so called) and write $\res_{B_i}(a)$ for $a\in \cC(B)\trans E$.
%Compositions of boundary and restriction maps will also be called restriction maps.
%For example, if $B$ is a $k$-ball and $Y\sub \bd B$ is a $k{-}1$-ball, there is a
%restriction map from $\cC(B)_{\bd Y}$ to $\cC(Y)$.

We will write $\cC(B)\trans Y$ for the image of $\gl_Y$ in $\cC(B)$.
We will call elements of $\cC(B)\trans Y$ morphisms which are 
``splittable along $Y$'' or ``transverse to $Y$''.
We have $\cC(B)\trans Y \sub \cC(B)\trans E \sub \cC(B)$.

More generally, let $\alpha$ be a splitting of $X$ into smaller balls.
Let $\cC(X)_\alpha \sub \cC(X)$ denote the image of the iterated gluing maps from 
the smaller balls to $X$.
We  say that elements of $\cC(X)_\alpha$ are morphisms which are ``splittable along $\alpha$".
In situations where the splitting is notationally anonymous, we will write
$\cC(X)\spl$ for the morphisms which are splittable along (a.k.a.\ transverse to)
the unnamed splitting.
If $\beta$ is a ball decomposition of $\bd X$, we define $\cC(X)_\beta \deq \bd\inv(\cl{\cC}(\bd X)_\beta)$;
this can also be denoted $\cC(X)\spl$ if the context contains an anonymous
decomposition of $\bd X$ and no competing splitting of $X$.

The above two composition axioms are equivalent to the following one,
which we state in slightly vague form.

\xxpar{Multi-composition:}
{Given any splitting $B_1 \sqcup \cdots \sqcup B_m \to B$ of a $k$-ball
into small $k$-balls, there is a 
map from an appropriate subset (like a fibered product) 
of $\cC(B_1)\spl\times\cdots\times\cC(B_m)\spl$ to $\cC(B)\spl$,
and these various $m$-fold composition maps satisfy an
operad-type strict associativity condition (Figure \ref{fig:operad-composition}).}

\begin{figure}[!ht]
$$\mathfig{.8}{ncat/operad-composition}$$
\caption{Operad composition and associativity}\label{fig:operad-composition}\end{figure}

The next axiom is related to identity morphisms, though that might not be immediately obvious.

\begin{axiom}[Product (identity) morphisms, preliminary version]
For each $k$-ball $X$ and $m$-ball $D$, with $k+m \le n$, there is a map $\cC(X)\to \cC(X\times D)$, 
usually denoted $a\mapsto a\times D$ for $a\in \cC(X)$.
These maps must satisfy the following conditions.
\begin{enumerate}
\item
If $f:X\to X'$ and $\tilde{f}:X\times D \to X'\times D'$ are homeomorphisms such that the diagram
\[ \xymatrix{
	X\times D \ar[r]^{\tilde{f}} \ar[d]_{\pi} & X'\times D' \ar[d]^{\pi} \\
	X \ar[r]^{f} & X'
} \]
commutes, then we have 
\[
	\tilde{f}(a\times D) = f(a)\times D' .
\]
\item
Product morphisms are compatible with gluing (composition) in both factors:
\[
	(a'\times D)\bullet(a''\times D) = (a'\bullet a'')\times D
\]
and
\[
	(a\times D')\bullet(a\times D'') = a\times (D'\bullet D'') .
\]
\item
Product morphisms are associative:
\[
	(a\times D)\times D' = a\times (D\times D') .
\]
(Here we are implicitly using functoriality and the obvious homeomorphism
$(X\times D)\times D' \to X\times(D\times D')$.)
\item
Product morphisms are compatible with restriction:
\[
	\res_{X\times E}(a\times D) = a\times E
\]
for $E\sub \bd D$ and $a\in \cC(X)$.
\end{enumerate}
\end{axiom}

We will need to strengthen the above preliminary version of the axiom to allow
for products which are ``pinched" in various ways along their boundary.
(See Figure \ref{pinched_prods}.)
\begin{figure}[t]
$$
\begin{tikzpicture}[baseline=0]
\begin{scope}
\path[clip] (0,0) arc (135:45:4) arc (-45:-135:4);
\draw[blue,line width=2pt] (0,0) arc (135:45:4) arc (-45:-135:4);
\foreach \x in {0, 0.5, ..., 6} {
	\draw[green!50!brown] (\x,-2) -- (\x,2);
}
\end{scope}
\draw[blue,line width=1.5pt] (0,-3) -- (5.66,-3);
\draw[->,red,line width=2pt] (2.83,-1.5) -- (2.83,-2.5);
\end{tikzpicture}
\qquad \qquad
\begin{tikzpicture}[baseline=-0.15cm]
\begin{scope}
\path[clip] (0,1) arc (90:135:8 and 4)  arc (-135:-90:8 and 4) -- cycle;
\draw[blue,line width=2pt] (0,1) arc (90:135:8 and 4)  arc (-135:-90:8 and 4) -- cycle;
\foreach \x in {-6, -5.5, ..., 0} {
	\draw[green!50!brown] (\x,-2) -- (\x,2);
}
\end{scope}
\draw[blue,line width=1.5pt] (-5.66,-3.15) -- (0,-3.15);
\draw[->,red,line width=2pt] (-2.83,-1.5) -- (-2.83,-2.5);
\end{tikzpicture}
$$
\caption{Examples of pinched products}\label{pinched_prods}
\end{figure}
(The need for a strengthened version will become apparent in Appendix \ref{sec:comparing-defs}
where we construct a traditional category from a disk-like category.)
Define a {\it pinched product} to be a map
\[
	\pi: E\to X
\]
such that $E$ is a $k{+}m$-ball, $X$ is a $k$-ball ($m\ge 1$), and $\pi$ is locally modeled
on a standard iterated degeneracy map
\[
	d: \Delta^{k+m}\to\Delta^k .
\]
(We thank Kevin Costello for suggesting this approach.)

Note that for each interior point $x\in X$, $\pi\inv(x)$ is an $m$-ball,
and for each boundary point $x\in\bd X$, $\pi\inv(x)$ is a ball of dimension
$l \le m$, with $l$ depending on $x$.
It is easy to see that a composition of pinched products is again a pinched product.
A {\it sub pinched product} is a sub-$m$-ball $E'\sub E$ such that the restriction
$\pi:E'\to \pi(E')$ is again a pinched product.
A {union} of pinched products is a decomposition $E = \cup_i E_i$
such that each $E_i\sub E$ is a sub pinched product.
(See Figure \ref{pinched_prod_unions}.)
\begin{figure}[t]
$$
\begin{tikzpicture}[baseline=0]
\begin{scope}
\path[clip] (0,0) arc (135:45:4) arc (-45:-135:4);
\draw[blue,line width=2pt] (0,0) arc (135:45:4) arc (-45:-135:4);
\draw[blue] (0,0) -- (5.66,0);
\foreach \x in {0, 0.5, ..., 6} {
	\draw[green!50!brown] (\x,-2) -- (\x,2);
}
\end{scope}
\end{tikzpicture}
\qquad
\begin{tikzpicture}[baseline=0]
\begin{scope}
\path[clip] (0,-1) rectangle (4,1);
\draw[blue,line width=2pt] (0,-1) rectangle (4,1);
\draw[blue] (0,0) -- (5,0);
\foreach \x in {0, 0.5, ..., 6} {
	\draw[green!50!brown] (\x,-2) -- (\x,2);
}
\end{scope}
\end{tikzpicture}
\qquad
\begin{tikzpicture}[baseline=0]
\begin{scope}
\path[clip] (0,0) arc (135:45:4) arc (-45:-135:4);
\draw[blue,line width=2pt] (0,0) arc (135:45:4) arc (-45:-135:4);
\draw[blue] (2.83,3) circle (3);
\foreach \x in {0, 0.5, ..., 6} {
	\draw[green!50!brown] (\x,-2) -- (\x,2);
}
\end{scope}
\end{tikzpicture}
$$
$$
\begin{tikzpicture}[baseline=0]
\begin{scope}
\path[clip] (0,-1) rectangle (4,1);
\draw[blue,line width=2pt] (0,-1) rectangle (4,1);
\draw[blue] (0,-1) -- (4,1);
\foreach \x in {0, 0.5, ..., 6} {
	\draw[green!50!brown] (\x,-2) -- (\x,2);
}
\end{scope}
\end{tikzpicture}
\qquad
\begin{tikzpicture}[baseline=0]
\begin{scope}
\path[clip] (0,-1) rectangle (5,1);
\draw[blue,line width=2pt] (0,-1) rectangle (5,1);
\draw[blue] (1,-1) .. controls  (2,-1) and (3,1) .. (4,1);
\foreach \x in {0, 0.5, ..., 6} {
	\draw[green!50!brown] (\x,-2) -- (\x,2);
}
\end{scope}
\end{tikzpicture}
$$
\caption{Five examples of unions of pinched products}\label{pinched_prod_unions}
\end{figure}

The product axiom will give a map $\pi^*:\cC(X)\to \cC(E)$ for each pinched product
$\pi:E\to X$.
Morphisms in the image of $\pi^*$ will be called product morphisms.
Before stating the axiom, we illustrate it in our two motivating examples of $n$-categories.
In the case where $\cC(X) = \{f: X\to T\}$, we define $\pi^*(f) = f\circ\pi$.
In the case where $\cC(X)$ is the set of all labeled embedded cell complexes $K$ in $X$, 
define $\pi^*(K) = \pi\inv(K)$, with each codimension $i$ cell $\pi\inv(c)$ labeled by the
same (traditional) $i$-morphism as the corresponding codimension $i$ cell $c$.


%\addtocounter{axiom}{-1}
\begin{axiom}[Product (identity) morphisms]
\label{axiom:product}
For each pinched product $\pi:E\to X$, with $X$ a $k$-ball and $E$ a $k{+}m$-ball ($m\ge 1$),
there is a map $\pi^*:\cC(X)\to \cC(E)$.
These maps must satisfy the following conditions.
\begin{enumerate}
\item
If $\pi:E\to X$ and $\pi':E'\to X'$ are pinched products, and
if $f:X\to X'$ and $\tilde{f}:E \to E'$ are maps such that the diagram
\[ \xymatrix{
	E \ar[r]^{\tilde{f}} \ar[d]_{\pi} & E' \ar[d]^{\pi'} \\
	X \ar[r]^{f} & X'
} \]
commutes, then we have 
\[
	\pi'^*\circ f = \tilde{f}\circ \pi^*.
\]
\item
Product morphisms are compatible with gluing (composition).
Let $\pi:E\to X$, $\pi_1:E_1\to X_1$, and $\pi_2:E_2\to X_2$ 
be pinched products with $E = E_1\cup E_2$.
Let $a\in \cC(X)$, and let $a_i$ denote the restriction of $a$ to $X_i\sub X$.
Then 
\[
	\pi^*(a) = \pi_1^*(a_1)\bullet \pi_2^*(a_2) .
\]
\item
Product morphisms are associative.
If $\pi:E\to X$ and $\rho:D\to E$ are pinched products then
\[
	\rho^*\circ\pi^* = (\pi\circ\rho)^* .
\]
\item
Product morphisms are compatible with restriction.
If we have a commutative diagram
\[ \xymatrix{
	D \ar@{^(->}[r] \ar[d]_{\rho} & E \ar[d]^{\pi} \\
	Y \ar@{^(->}[r] & X
} \]
such that $\rho$ and $\pi$ are pinched products, then
\[
	\res_D\circ\pi^* = \rho^*\circ\res_Y .
\]
\end{enumerate}
\end{axiom}


\medskip

All of the axioms listed above hold for both ordinary $n$-categories and $A_\infty$ $n$-categories.
The last axiom (below), concerning actions of 
homeomorphisms in the top dimension $n$, distinguishes the two cases.

We start with the ordinary $n$-category case.

\begin{axiom}[\textup{\textbf{[preliminary]}} Isotopy invariance in dimension $n$]
Let $X$ be an $n$-ball and $f: X\to X$ be a homeomorphism which restricts
to the identity on $\bd X$ and is isotopic (rel boundary) to the identity.
Then $f$ acts trivially on $\cC(X)$; that is $f(a) = a$ for all $a\in \cC(X)$.
\end{axiom}

This axiom needs to be strengthened to force product morphisms to act as the identity.
Let $X$ be an $n$-ball and $Y\sub\bd X$ be an $n{-}1$-ball.
Let $J$ be a 1-ball (interval).
Let $s_{Y,J}: X\cup_Y (Y\times J) \to X$ be a collaring homeomorphism
(see the end of \S\ref{ss:syst-o-fields}).
Here we use $Y\times J$ with boundary entirely pinched.
We define a map
\begin{eqnarray*}
	\psi_{Y,J}: \cC(X) &\to& \cC(X) \\
	a & \mapsto & s_{Y,J}(a \cup ((a|_Y)\times J)) .
\end{eqnarray*}
(See Figure \ref{glue-collar}.)
\begin{figure}[!ht]
\begin{equation*}
\begin{tikzpicture}
\def\rad{1}
\def\srad{0.75}
\def\gap{4.5}
\foreach \i in {0, 1, 2} {
	\node(\i) at ($\i*(\gap,0)$) [draw, circle through = {($\i*(\gap,0)+(\rad,0)$)}] {};
	\node(\i-small) at (\i.east) [circle through={($(\i.east)+(\srad,0)$)}] {};
	\foreach \n in {1,2} {
		\fill (intersection \n of \i-small and \i) node(\i-intersection-\n) {} circle (2pt);
	}
}

\begin{scope}[decoration={brace,amplitude=10,aspect=0.5}]
	\draw[decorate] (0-intersection-1.east) -- (0-intersection-2.east);
\end{scope}
\node[right=1mm] at (0.east) {$a$};
\draw[->] ($(0.east)+(0.75,0)$) -- ($(1.west)+(-0.2,0)$);

\draw (1-small)  circle (\srad);
\foreach \theta in {90, 72, ..., -90} {
	\draw[blue] (1) -- ($(1)+(\rad,0)+(\theta:\srad)$);
}
\filldraw[fill=white] (1) circle (\rad);
\foreach \n in {1,2} {
	\fill (intersection \n of 1-small and 1) circle (2pt);
}
\node[below] at (1-small.south) {$a \times J$};
\draw[->] ($(1.east)+(1,0)$) -- ($(2.west)+(-0.2,0)$);

\begin{scope}
\path[clip] (2) circle (\rad);
\draw[clip] (2.east) circle (\srad);
\foreach \y in {1, 0.86, ..., -1} {
	\draw[blue] ($(2)+(-1,\y) $)-- ($(2)+(1,\y)$);
}
\end{scope}
\end{tikzpicture}
\end{equation*}
\begin{equation*}
\xymatrix@C+2cm{\cC(X) \ar[r]^(0.45){\text{glue}} & \cC(X \cup \text{collar}) \ar[r]^(0.55){\text{homeo}} & \cC(X)}
\end{equation*}

\caption{Extended homeomorphism.}\label{glue-collar}\end{figure}
We call a map of this form a {\it collar map}.
It can be thought of as the action of the inverse of
a map which projects a collar neighborhood of $Y$ onto $Y$,
or as the limit of homeomorphisms $X\to X$ which expand a very thin collar of $Y$
to a larger collar.
We call the equivalence relation generated by collar maps and homeomorphisms
isotopic (rel boundary) to the identity {\it extended isotopy}.

The revised axiom is

%\addtocounter{axiom}{-1}
\begin{axiom}[\textup{\textbf{[ordinary  version]}} Extended isotopy invariance in dimension $n$.]
\label{axiom:extended-isotopies}
Let $X$ be an $n$-ball and $f: X\to X$ be a homeomorphism which restricts
to the identity on $\bd X$ and isotopic (rel boundary) to the identity.
Then $f$ acts trivially on $\cC(X)$.
In addition, collar maps act trivially on $\cC(X)$.
\end{axiom}

\smallskip

For $A_\infty$ $n$-categories, we replace
isotopy invariance with the requirement that families of homeomorphisms act.
For the moment, assume that our $n$-morphisms are enriched over chain complexes.
Let $\Homeo_\bd(X)$ denote homeomorphisms of $X$ which fix $\bd X$ and
$C_*(\Homeo_\bd(X))$ denote the singular chains on this space.


%\addtocounter{axiom}{-1}
\begin{axiom}[\textup{\textbf{[$A_\infty$ version]}} Families of homeomorphisms act in dimension $n$.]
\label{axiom:families}
For each $n$-ball $X$ and each $c\in \cl{\cC}(\bd X)$ we have a map of chain complexes
\[
	C_*(\Homeo_\bd(X))\ot \cC(X; c) \to \cC(X; c) .
\]
These action maps are required to be associative up to homotopy,
%\nn{iterated homotopy?}
and also compatible with composition (gluing) in the sense that
a diagram like the one in Theorem \ref{thm:CH} commutes.
%\nn{repeat diagram here?}
%\nn{restate this with $\Homeo(X\to X')$?  what about boundary fixing property?}
On $C_0(\Homeo_\bd(X))\ot \cC(X; c)$ the action should coincide 
with the one coming from Axiom \ref{axiom:morphisms}.
\end{axiom}

We should strengthen the above $A_\infty$ axiom to apply to families of collar maps.
To do this we need to explain how collar maps form a topological space.
Roughly, the set of collared $n{-}1$-balls in the boundary of an $n$-ball has a natural topology,
and we can replace the class of all intervals $J$ with intervals contained in $\r$.
Having chains on the space of collar maps act gives rise to coherence maps involving
weak identities.
We will not pursue this in detail here.

A potential variant on the above axiom would be to drop the ``up to homotopy" and require a strictly associative action. (In fact, the alternative construction of the blob complex described in \S \ref{ss:alt-def} gives $n$-categories as in Example \ref{ex:blob-complexes-of-balls} which satisfy this stronger axiom; since that construction is only homotopy equivalent to the usual one, only the weaker axiom carries across.)

Note that if we take homology of chain complexes, we turn an $A_\infty$ $n$-category
into a ordinary $n$-category (enriched over graded groups).
In a different direction, if we enrich over topological spaces instead of chain complexes,
we get a space version of an $A_\infty$ $n$-category, with $\Homeo_\bd(X)$ acting 
instead of  $C_*(\Homeo_\bd(X))$.
Taking singular chains converts such a space type $A_\infty$ $n$-category into a chain complex
type $A_\infty$ $n$-category.

\medskip

The alert reader will have already noticed that our definition of a (ordinary) $n$-category
is extremely similar to our definition of a system of fields.
There are two differences.
First, for the $n$-category definition we restrict our attention to balls
(and their boundaries), while for fields we consider all manifolds.
Second,  in category definition we directly impose isotopy
invariance in dimension $n$, while in the fields definition we 
instead remember a subspace of local relations which contain differences of isotopic fields. 
(Recall that the compensation for this complication is that we can demand that the gluing map for fields is injective.)
Thus a system of fields and local relations $(\cF,U)$ determines an $n$-category $\cC_ {\cF,U}$ simply by restricting our attention to
balls and, at level $n$, quotienting out by the local relations:
\begin{align*}
\cC_{\cF,U}(B^k) & = \begin{cases}\cF(B) & \text{when $k<n$,} \\ \cF(B) / U(B) & \text{when $k=n$.}\end{cases}
\end{align*}
This $n$-category can be thought of as the local part of the fields.
Conversely, given a disk-like $n$-category we can construct a system of fields via 
a colimit construction; see \S \ref{ss:ncat_fields} below.

In the $n$-category axioms above we have intermingled data and properties for expository reasons.
Here's a summary of the definition which segregates the data from the properties.

An $n$-category consists of the following data:
\begin{itemize}
\item functors $\cC_k$ from $k$-balls to sets, $0\le k\le n$ (Axiom \ref{axiom:morphisms});
\item boundary natural transformations $\cC_k \to \cl{\cC}_{k-1} \circ \bd$ (Axiom \ref{nca-boundary});
\item ``composition'' or ``gluing'' maps $\gl_Y : \cC(B_1)\trans E \times_{\cC(Y)} \cC(B_2)\trans E \to \cC(B_1\cup_Y B_2)\trans E$ (Axiom \ref{axiom:composition});
\item ``product'' or ``identity'' maps $\pi^*:\cC(X)\to \cC(E)$ for each pinched product $\pi:E\to X$ (Axiom \ref{axiom:product});
\item if enriching in an auxiliary category, additional structure on $\cC_n(X; c)$;
\item in the $A_\infty$ case, an action of $C_*(\Homeo_\bd(X))$, and similarly for families of collar maps (Axiom \ref{axiom:families}).
\end{itemize}
The above data must satisfy the following conditions:
\begin{itemize}
\item The gluing maps are compatible with actions of homeomorphisms and boundary 
restrictions (Axiom \ref{axiom:composition}).
\item For $k<n$ the gluing maps are injective (Axiom \ref{axiom:composition}).
\item The gluing maps are strictly associative (Axiom \ref{nca-assoc}).
\item The product maps are associative and also compatible with homeomorphism actions, gluing and restriction (Axiom \ref{axiom:product}).
\item If enriching in an auxiliary category, all of the data should be compatible 
with the auxiliary category structure on $\cC_n(X; c)$.
\item For ordinary categories, invariance of $n$-morphisms under extended isotopies (Axiom \ref{axiom:extended-isotopies}).
\end{itemize}


\subsection{Examples of \texorpdfstring{$n$}{n}-categories}
\label{ss:ncat-examples}


We now describe several classes of examples of $n$-categories satisfying our axioms.
We typically specify only the morphisms; the rest of the data for the category
(restriction maps, gluing, product morphisms, action of homeomorphisms) is usually obvious.

\begin{example}[Maps to a space]
\rm
\label{ex:maps-to-a-space}%
Let $T$ be a topological space.
We define $\pi_{\leq n}(T)$, the fundamental $n$-category of $T$, as follows.
For $X$ a $k$-ball with $k < n$, define $\pi_{\leq n}(T)(X)$ to be the set of 
all continuous maps from $X$ to $T$.
For $X$ an $n$-ball define $\pi_{\leq n}(T)(X)$ to be continuous maps from $X$ to $T$ modulo
homotopies fixed on $\bd X$.
(Note that homotopy invariance implies isotopy invariance.)
For $a\in \cC(X)$ define the product morphism $a\times D \in \cC(X\times D)$ to
be $a\circ\pi_X$, where $\pi_X : X\times D \to X$ is the projection.
\end{example}

\noop{
Recall we described a system of fields and local relations based on maps to $T$ in Example \ref{ex:maps-to-a-space(fields)} above.
Constructing a system of fields from $\pi_{\leq n}(T)$ recovers that example.
\nn{shouldn't this go elsewhere?  we haven't yet discussed constructing a system of fields from
an n-cat}
}

\begin{example}[Maps to a space, with a fiber] \label{ex:maps-with-fiber}
\rm
\label{ex:maps-to-a-space-with-a-fiber}%
We can modify the example above, by fixing a
closed $m$-manifold $F$, and defining $\pi^{\times F}_{\leq n}(T)(X) = \Maps(X \times F \to T)$, 
otherwise leaving the definition in Example \ref{ex:maps-to-a-space} unchanged.
Taking $F$ to be a point recovers the previous case.
\end{example}

\begin{example}[Linearized, twisted, maps to a space]
\rm
\label{ex:linearized-maps-to-a-space}%
We can linearize Examples \ref{ex:maps-to-a-space} and \ref{ex:maps-to-a-space-with-a-fiber} as follows.
Let $\alpha$ be an $(n{+}m{+}1)$-cocycle on $T$ with values in a ring $R$
(have in mind the trivial cocycle).
For $X$ of dimension less than $n$ define $\pi^{\alpha, \times F}_{\leq n}(T)(X)$ as before, ignoring $\alpha$.
For $X$ an $n$-ball and $c\in \Maps(\bdy X \times F \to T)$ define $\pi^{\alpha, \times F}_{\leq n}(T)(X; c)$ to be
the $R$-module of finite linear combinations of continuous maps from $X\times F$ to $T$,
modulo the relation that if $a$ is homotopic to $b$ (rel boundary) via a homotopy
$h: X\times F\times I \to T$, then $a = \alpha(h)b$.
(In order for this to be well-defined we must choose $\alpha$ to be zero on degenerate simplices.
Alternatively, we could equip the balls with fundamental classes.)
\end{example}

\begin{example}[$n$-categories from TQFTs]
\rm
\label{ex:ncats-from-tqfts}%
Let $\cF$ be a TQFT in the sense of \S\ref{sec:fields}: an $n$-dimensional 
system of fields (also denoted $\cF$) and local relations.
Let $W$ be an $n{-}j$-manifold.
Define the $j$-category $\cF(W)$ as follows.
If $X$ is a $k$-ball with $k<j$, let $\cF(W)(X) \deq \cF(W\times X)$.
If $X$ is a $j$-ball and $c\in \cl{\cF(W)}(\bd X)$, 
let $\cF(W)(X; c) \deq A_\cF(W\times X; c)$.
\end{example}

The next example is only intended to be illustrative, as we don't specify 
which definition of a ``traditional $n$-category" we intend.
Further, most of these definitions don't even have an agreed-upon notion of 
``strong duality", which we assume here.
\begin{example}[Traditional $n$-categories]
\rm
\label{ex:traditional-n-categories}
Given a ``traditional $n$-category with strong duality" $C$
define $\cC(X)$, for $X$ a $k$-ball with $k < n$,
to be the set of all $C$-labeled embedded cell complexes of $X$ (c.f. \S \ref{sec:fields}).
For $X$ an $n$-ball and $c\in \cl{\cC}(\bd X)$, define $\cC(X; c)$ to be finite linear
combinations of $C$-labeled embedded cell complexes of $X$
modulo the kernel of the evaluation map.
Define a product morphism $a\times D$, for $D$ an $m$-ball, to be the product of the cell complex of $a$ with $D$,
with each cell labelled according to the corresponding cell for $a$.
(These two cells have the same codimension.)
More generally, start with an $n{+}m$-category $C$ and a closed $m$-manifold $F$.
Define $\cC(X)$, for $\dim(X) < n$,
to be the set of all $C$-labeled embedded cell complexes of $X\times F$.
Define $\cC(X; c)$, for $X$ an $n$-ball,
to be the dual Hilbert space $A(X\times F; c)$.
(See \S\ref{sec:constructing-a-tqft}.)
\end{example}

\noop{
\nn{shouldn't this go elsewhere?  we haven't yet discussed constructing a system of fields from
an n-cat}
Recall we described a system of fields and local relations based on a ``traditional $n$-category" 
$C$ in Example \ref{ex:traditional-n-categories(fields)} above.
\nn{KW: We already refer to \S \ref{sec:fields} above}
Constructing a system of fields from $\cC$ recovers that example. 
\todo{Except that it doesn't: pasting diagrams v.s. string diagrams.}
\nn{KW: but the above example is all about string diagrams.  the only difference is at the top level,
where the quotient is built in.
but (string diagrams)/(relations) is isomorphic to 
(pasting diagrams composed of smaller string diagrams)/(relations)}
}


\begin{example}[The bordism $n$-category, ordinary version]
\label{ex:bord-cat}
\rm
\label{ex:bordism-category}
For a $k$-ball $X$, $k<n$, define $\Bord^n(X)$ to be the set of all $k$-dimensional PL
submanifolds $W$ of $X\times \Real^\infty$ such that $\bd W$ is 
contained in $\bd X \times \Real^\infty$.
For an $n$-ball $X$ define $\Bord^n(X)$ to be homeomorphism classes (rel boundary) of such $n$-dimensional submanifolds;
we identify $W$ and $W'$ if $\bd W = \bd W'$ and there is a homeomorphism
$W \to W'$ which restricts to the identity on the boundary.
\end{example}

%\nn{the next example might be an unnecessary distraction.  consider deleting it.}

%\begin{example}[Variation on the above examples]
%We could allow $F$ to have boundary and specify boundary conditions on $X\times \bd F$,
%for example product boundary conditions or take the union over all boundary conditions.
%%\nn{maybe should not emphasize this case, since it's ``better" in some sense
%%to think of these guys as affording a representation
%%of the $n{+}1$-category associated to $\bd F$.}
%\end{example}


%We have two main examples of $A_\infty$ $n$-categories, coming from maps to a target space and from the blob complex.

\begin{example}[Chains (or space) of maps to a space]
\rm
\label{ex:chains-of-maps-to-a-space}
We can modify Example \ref{ex:maps-to-a-space} above to define the fundamental $A_\infty$ $n$-category $\pi^\infty_{\le n}(T)$ of a topological space $T$.
For a $k$-ball $X$, with $k < n$, the set $\pi^\infty_{\leq n}(T)(X)$ is just $\Maps(X \to T)$.
Define $\pi^\infty_{\leq n}(T)(X; c)$ for an $n$-ball $X$ and $c \in \pi^\infty_{\leq n}(T)(\bdy X)$ to be the chain complex
\[
	C_*(\Maps_c(X\times F \to T)),
\]
where $\Maps_c$ denotes continuous maps restricting to $c$ on the boundary,
and $C_*$ denotes singular chains.
Alternatively, if we take the $n$-morphisms to be simply $\Maps_c(X\times F \to T)$, 
we get an $A_\infty$ $n$-category enriched over spaces.
\end{example}

See also Theorem \ref{thm:map-recon} below, recovering $C_*(\Maps(M \to T))$ up to 
homotopy as the blob complex of $M$ with coefficients in $\pi^\infty_{\le n}(T)$.

\begin{example}[Blob complexes of balls (with a fiber)]
\rm
\label{ex:blob-complexes-of-balls}
Fix an $n{-}k$-dimensional manifold $F$ and an $n$-dimensional system of fields $\cE$.
We will define an $A_\infty$ $k$-category $\cC$.
When $X$ is a $m$-ball, with $m<k$, define $\cC(X) = \cE(X\times F)$.
When $X$ is an $k$-ball,
define $\cC(X; c) = \bc^\cE_*(X\times F; c)$
where $\bc^\cE_*$ denotes the blob complex based on $\cE$.
\end{example}

This example will be used in Theorem \ref{thm:product} below, which allows us to compute the blob complex of a product.
Notice that with $F$ a point, the above example is a construction turning an ordinary 
$n$-category $\cC$ into an $A_\infty$ $n$-category.
We think of this as providing a ``free resolution" 
of the ordinary $n$-category. 
%\nn{say something about cofibrant replacements?}
In fact, there is also a trivial, but mostly uninteresting, way to do this: 
we can think of each vector space associated to an $n$-ball as a chain complex concentrated in degree $0$, 
and take $\CD{B}$ to act trivially. 

Beware that the ``free resolution" of the ordinary $n$-category $\pi_{\leq n}(T)$ 
is not the $A_\infty$ $n$-category $\pi^\infty_{\leq n}(T)$.
It's easy to see that with $n=0$, the corresponding system of fields is just 
linear combinations of connected components of $T$, and the local relations are trivial.
There's no way for the blob complex to magically recover all the data of $\pi^\infty_{\leq 0}(T) \iso C_* T$.

\begin{example}[The bordism $n$-category, $A_\infty$ version]
\rm
\label{ex:bordism-category-ainf}
As in Example \ref{ex:bord-cat}, for $X$ a $k$-ball, $k<n$, we define $\Bord^{n,\infty}(X)$
to be the set of all $k$-dimensional
submanifolds $W$ of $X\times \Real^\infty$ such that $\bd W$ is 
contained in $\bd X \times \Real^\infty$.
For an $n$-ball $X$ with boundary condition $c$ 
define $\Bord^{n,\infty}(X; c)$ to be the space of all $k$-dimensional
submanifolds $W$ of $X\times \Real^\infty$ such that 
$W$ coincides with $c$ at $\bd X \times \Real^\infty$.
(The topology on this space is induced by ambient isotopy rel boundary.
This is homotopy equivalent to a disjoint union of copies $\mathrm{B}\!\Homeo(W')$, where
$W'$ runs though representatives of homeomorphism types of such manifolds.)
\end{example}



Let $\cE\cB_n$ be the operad of smooth embeddings of $k$ (little)
copies of the standard $n$-ball $B^n$ into another (big) copy of $B^n$.
(We require that the interiors of the little balls be disjoint, but their 
boundaries are allowed to meet.
Note in particular that the space for $k=1$ contains a copy of $\Diff(B^n)$, namely
the embeddings of a ``little" ball with image all of the big ball $B^n$.
(But note also that this inclusion is not
necessarily a homotopy equivalence.)
The operad $\cE\cB_n$ is homotopy equivalent to the standard framed little $n$-ball operad:
by shrinking the little balls (precomposing them with dilations), 
we see that both operads are homotopic to the space of $k$ framed points
in $B^n$.
It is easy to see that $n$-fold loop spaces $\Omega^n(T)$  have
an action of $\cE\cB_n$.
%\nn{add citation for this operad if we can find one}

\begin{example}[$E_n$ algebras]
\rm
\label{ex:e-n-alg}
Let $A$ be an $\cE\cB_n$-algebra.
Note that this implies a $\Diff(B^n)$ action on $A$, 
since $\cE\cB_n$ contains a copy of $\Diff(B^n)$.
We will define an $A_\infty$ $n$-category $\cC^A$.
If $X$ is a ball of dimension $k<n$, define $\cC^A(X)$ to be a point.
In other words, the $k$-morphisms are trivial for $k<n$.
If $X$ is an $n$-ball, we define $\cC^A(X)$ via a colimit construction.
(Plain colimit, not homotopy colimit.)
Let $J$ be the category whose objects are embeddings of a disjoint union of copies of 
the standard ball $B^n$ into $X$, and who morphisms are given by engulfing some of the 
embedded balls into a single larger embedded ball.
To each object of $J$ we associate $A^{\times m}$ (where $m$ is the number of balls), and
to each morphism of $J$ we associate a morphism coming from the $\cE\cB_n$ action on $A$.
Alternatively and more simply, we could define $\cC^A(X)$ to be 
$\Diff(B^n\to X)\times A$ modulo the diagonal action of $\Diff(B^n)$.
The remaining data for the $A_\infty$ $n$-category 
--- composition and $\Diff(X\to X')$ action ---
also comes from the $\cE\cB_n$ action on $A$.
%\nn{should we spell this out?}

Conversely, one can show that a disk-like $A_\infty$ $n$-category $\cC$, where the $k$-morphisms
$\cC(X)$ are trivial (single point) for $k<n$, gives rise to 
an $\cE\cB_n$-algebra.
%\nn{The paper is already long; is it worth giving details here?}

If we apply the homotopy colimit construction of the next subsection to this example, 
we get an instance of Lurie's topological chiral homology construction.
\end{example}


\subsection{From balls to manifolds}
\label{ss:ncat_fields} \label{ss:ncat-coend}
In this section we show how to extend an $n$-category $\cC$ as described above 
(of either the ordinary or $A_\infty$ variety) to an invariant of manifolds, which we denote by $\cl{\cC}$.
This extension is a certain colimit, and the arrow in the notation is intended as a reminder of this.

In the case of ordinary $n$-categories, this construction factors into a construction of a 
system of fields and local relations, followed by the usual TQFT definition of a 
vector space invariant of manifolds given as Definition \ref{defn:TQFT-invariant}.
For an $A_\infty$ $n$-category, $\cl{\cC}$ is defined using a homotopy colimit instead.
Recall that we can take a ordinary $n$-category $\cC$ and pass to the ``free resolution", 
an $A_\infty$ $n$-category $\bc_*(\cC)$, by computing the blob complex of balls 
(recall Example \ref{ex:blob-complexes-of-balls} above).
We will show in Corollary \ref{cor:new-old} below that the homotopy colimit invariant 
for a manifold $M$ associated to this $A_\infty$ $n$-category is actually the 
same as the original blob complex for $M$ with coefficients in $\cC$.

Recall that we've already anticipated this construction in the previous section, 
inductively defining $\cl{\cC}$ on $k$-spheres in terms of $\cC$ on $k$-balls, 
so that we can state the boundary axiom for $\cC$ on $k+1$-balls.

\medskip

We will first define the ``decomposition" poset $\cell(W)$ for any $k$-manifold $W$, for $1 \leq k \leq n$. 
An $n$-category $\cC$ provides a functor from this poset to the category of sets, 
and we  will define $\cl{\cC}(W)$ as a suitable colimit 
(or homotopy colimit in the $A_\infty$ case) of this functor. 
We'll later give a more explicit description of this colimit.
In the case that the $n$-category $\cC$ is enriched (e.g. associates vector spaces or chain complexes to $n$-balls with boundary data), 
then the resulting colimit is also enriched, that is, the set associated to $W$ splits into subsets according to boundary data, and each of these subsets has the appropriate structure (e.g. a vector space or chain complex).

Recall (Definition \ref{defn:gluing-decomposition}) that a {\it ball decomposition} of $W$ is a 
sequence of gluings $M_0\to M_1\to\cdots\to M_m = W$ such that $M_0$ is a disjoint union of balls
$\du_a X_a$.
Abusing notation, we let $X_a$ denote both the ball (component of $M_0$) and
its image in $W$ (which is not necessarily a ball --- parts of $\bd X_a$ may have been glued together).
Define a {\it permissible decomposition} of $W$ to be a map
\[
	\coprod_a X_a \to W,
\]
which can be completed to a ball decomposition $\du_a X_a = M_0\to\cdots\to M_m = W$.
Roughly, a permissible decomposition is like a ball decomposition where we don't care in which order the balls
are glued up to yield $W$, so long as there is some (non-pathological) way to glue them.

Given permissible decompositions $x = \{X_a\}$ and $y = \{Y_b\}$ of $W$, we say that $x$ is a refinement
of $y$, or write $x \le y$, if there is a ball decomposition $\du_a X_a = M_0\to\cdots\to M_m = W$
with $\du_b Y_b = M_i$ for some $i$.

\begin{defn}
The poset $\cell(W)$ has objects the permissible decompositions of $W$, 
and a unique morphism from $x$ to $y$ if and only if $x$ is a refinement of $y$.
See Figure \ref{partofJfig} for an example.
\end{defn}

\begin{figure}[!ht]
\begin{equation*}
\mathfig{.63}{ncat/zz2}
\end{equation*}
\caption{A small part of $\cell(W)$}
\label{partofJfig}
\end{figure}

An $n$-category $\cC$ determines 
a functor $\psi_{\cC;W}$ from $\cell(W)$ to the category of sets 
(possibly with additional structure if $k=n$).
Each $k$-ball $X$ of a decomposition $y$ of $W$ has its boundary decomposed into $k{-}1$-balls,
and, as described above, we have a subset $\cC(X)\spl \sub \cC(X)$ of morphisms whose boundaries
are splittable along this decomposition.

\begin{defn}
Define the functor $\psi_{\cC;W} : \cell(W) \to \Set$ as follows.
For a decomposition $x = \bigsqcup_a X_a$ in $\cell(W)$, $\psi_{\cC;W}(x)$ is the subset
\begin{equation}
\label{eq:psi-C}
	\psi_{\cC;W}(x) \sub \prod_a \cC(X_a)\spl
\end{equation}
where the restrictions to the various pieces of shared boundaries amongst the cells
$X_a$ all agree (this is a fibered product of all the labels of $n$-cells over the labels of $n-1$-cells).
If $x$ is a refinement of $y$, the map $\psi_{\cC;W}(x) \to \psi_{\cC;W}(y)$ is given by the composition maps of $\cC$.
\end{defn}

If $k=n$ in the above definition and we are enriching in some auxiliary category, 
we need to say a bit more.
We can rewrite Equation \ref{eq:psi-C} as
\begin{equation} \label{eq:psi-CC}
	\psi_{\cC;W}(x) \deq \coprod_\beta \prod_a \cC(X_a; \beta) ,
\end{equation}
where $\beta$ runs through labelings of the $k{-}1$-skeleton of the decomposition
(which are compatible when restricted to the $k{-}2$-skeleton), and $\cC(X_a; \beta)$
means the subset of $\cC(X_a)$ whose restriction to $\bd X_a$ agress with $\beta$.
If we are enriching over $\cS$ and $k=n$, then $\cC(X_a; \beta)$ is an object in 
$\cS$ and the coproduct and product in Equation \ref{eq:psi-CC} should be replaced by the approriate
operations in $\cS$ (e.g. direct sum and tensor product if $\cS$ is Vect).

Finally, we construct $\cl{\cC}(W)$ as the appropriate colimit of $\psi_{\cC;W}$:

\begin{defn}[System of fields functor]
\label{def:colim-fields}
If $\cC$ is an $n$-category enriched in sets or vector spaces, $\cl{\cC}(W)$ is the usual colimit of the functor $\psi_{\cC;W}$.
That is, for each decomposition $x$ there is a map
$\psi_{\cC;W}(x)\to \cl{\cC}(W)$, these maps are compatible with the refinement maps
above, and $\cl{\cC}(W)$ is universal with respect to these properties.
\end{defn}

\begin{defn}[System of fields functor, $A_\infty$ case]
When $\cC$ is an $A_\infty$ $n$-category, $\cl{\cC}(W)$ for $W$ a $k$-manifold with $k < n$ 
is defined as above, as the colimit of $\psi_{\cC;W}$.
When $W$ is an $n$-manifold, the chain complex $\cl{\cC}(W)$ is the homotopy colimit of the functor $\psi_{\cC;W}$.
\end{defn}

We can specify boundary data $c \in \cl{\cC}(\bdy W)$, and define functors $\psi_{\cC;W,c}$ 
with values the subsets of those of $\psi_{\cC;W}$ which agree with $c$ on the boundary of $W$.

We now give more concrete descriptions of the above colimits.

In the non-enriched case (e.g.\ $k<n$), where each $\cC(X_a; \beta)$ is just a set,
the colimit is
\[
	\cl{\cC}(W,c) = \left( \coprod_x \coprod_\beta \prod_a \cC(X_a; \beta) \right) \Bigg/ \sim ,
\]
where $x$ runs through decomposition of $W$, and $\sim$ is the obvious equivalence relation 
induced by refinement and gluing.
If $\cC$ is enriched over vector spaces and $W$ is an $n$-manifold, 
we can take
\begin{equation*}
	\cl{\cC}(W,c) = \left( \bigoplus_x \bigoplus_\beta \bigotimes_a \cC(X_a; \beta) \right) \Bigg/ K,
\end{equation*}
where $K$ is the vector space spanned by elements $a - g(a)$, with
$a\in \psi_{\cC;W,c}(x)$ for some decomposition $x$, and $g: \psi_{\cC;W,c}(x)
\to \psi_{\cC;W,c}(y)$ is value of $\psi_{\cC;W,c}$ on some antirefinement $x \leq y$.

In the $A_\infty$ case, enriched over chain complexes, the concrete description of the homotopy colimit
is more involved.
We will describe two different (but homotopy equivalent) versions of the homotopy colimit of $\psi_{\cC;W}$.
The first is the usual one, which works for any indexing category.
The second construction, which we call the {\it local} homotopy colimit,
is more closely related to the blob complex
construction of \S \ref{sec:blob-definition} and takes advantage of local (gluing) properties
of the indexing category $\cell(W)$.

Define an $m$-sequence in $W$ to be a sequence $x_0 \le x_1 \le \dots \le x_m$ of permissible decompositions of $W$.
Such sequences (for all $m$) form a simplicial set in $\cell(W)$.
Define $\cl{\cC}(W)$ as a vector space via
\[
	\cl{\cC}(W) = \bigoplus_{(x_i)} \psi_{\cC;W}(x_0)[m] ,
\]
where the sum is over all $m$ and all $m$-sequences $(x_i)$, and each summand is degree shifted by $m$. 
Elements of a summand indexed by an $m$-sequence will be call $m$-simplices.
We endow $\cl{\cC}(W)$ with a differential which is the sum of the differential of the $\psi_{\cC;W}(x_0)$
summands plus another term using the differential of the simplicial set of $m$-sequences.
More specifically, if $(a, \bar{x})$ denotes an element in the $\bar{x}$
summand of $\cl{\cC}(W)$ (with $\bar{x} = (x_0,\dots,x_k)$), define
\[
	\bd (a, \bar{x}) = (\bd a, \bar{x}) + (-1)^{\deg{a}} (g(a), d_0(\bar{x})) + (-1)^{\deg{a}} \sum_{j=1}^k (-1)^{j} (a, d_j(\bar{x})) ,
\]
where $d_j(\bar{x}) = (x_0,\dots,x_{j-1},x_{j+1},\dots,x_k)$ and $g: \psi_\cC(x_0)\to \psi_\cC(x_1)$
is the usual gluing map coming from the antirefinement $x_0 \le x_1$.
%\nn{maybe mention that there is a version that emphasizes minimal gluings (antirefinements) which
%combine only two balls at a time; for $n=1$ this version will lead to usual definition
%of $A_\infty$ category}

We can think of this construction as starting with a disjoint copy of a complex for each
permissible decomposition (the 0-simplices).
Then we glue these together with mapping cylinders coming from gluing maps
(the 1-simplices).
Then we kill the extra homology we just introduced with mapping 
cylinders between the mapping cylinders (the 2-simplices), and so on.

Next we describe the local homotopy colimit.
This is similar to the usual homotopy colimit, but using
a cone-product set (Remark \ref{blobsset-remark}) in place of a simplicial set.
The cone-product $m$-polyhedra for the set are pairs $(x, E)$, where $x$ is a decomposition of $W$
and $E$ is an $m$-blob diagram such that each blob is a union of balls of $x$.
(Recall that this means that the interiors of
each pair of blobs (i.e.\ balls) of $E$ are either disjoint or nested.)
To each $(x, E)$ we associate the chain complex $\psi_{\cC;W}(x)$, shifted in degree by $m$.
The boundary has a term for omitting each blob of $E$.
If we omit an innermost blob then we replace $x$ by the formal difference $x - \gl(x)$, where
$\gl(x)$ is obtained from $x$ by gluing together the balls of $x$ contained in the blob we are omitting.
The gluing maps of $\cC$ give us a maps from $\psi_{\cC;W}(x)$ to $\psi_{\cC;W}(\gl(x))$.

One can show that the usual hocolimit and the local hocolimit are homotopy equivalent using an 
Eilenberg-Zilber type subdivision argument.

\medskip

$\cl{\cC}(W)$ is functorial with respect to homeomorphisms of $k$-manifolds. 
Restricting the $k$-spheres, we have now proved Lemma \ref{lem:spheres}.

It is easy to see that
there are well-defined maps $\cl{\cC}(W)\to\cl{\cC}(\bd W)$, and that these maps
comprise a natural transformation of functors.

\begin{lem}
\label{lem:colim-injective}
Let $W$ be a manifold of dimension less than $n$.  Then for each
decomposition $x$ of $W$ the natural map $\psi_{\cC;W}(x)\to \cl{\cC}(W)$ is injective.
\end{lem}
\begin{proof}
$\cl{\cC}(W)$ is a colimit of a diagram of sets, and each of the arrows in the diagram is
injective.
Concretely, the colimit is the disjoint union of the sets (one for each decomposition of $W$),
modulo the relation which identifies the domain of each of the injective maps
with it's image.

To save ink and electrons we will simplify notation and write $\psi(x)$ for $\psi_{\cC;W}(x)$.

Suppose $a, \hat{a}\in \psi(x)$ have the same image in $\cl{\cC}(W)$ but $a\ne \hat{a}$.
Then there exist
\begin{itemize}
\item decompositions $x = x_0, x_1, \ldots , x_{k-1}, x_k = x$ and $v_1,\ldots, v_k$ of $W$;
\item anti-refinements $v_i\to x_i$ and $v_i\to x_{i-1}$; and
\item elements $a_i\in \psi(x_i)$ and $b_i\in \psi(v_i)$, with $a_0 = a$ and $a_k = \hat{a}$, 
such that $b_i$ and $b_{i+1}$both map to (glue up to) $a_i$.
\end{itemize}
In other words, we have a zig-zag of equivalences starting at $a$ and ending at $\hat{a}$.
The idea of the proof is to produce a similar zig-zag where everything antirefines to the same
disjoint union of balls, and then invoke Axiom \ref{nca-assoc} which ensures associativity.

Let $z$ be a decomposition of $W$ which is in general position with respect to all of the 
$x_i$'s and $v_i$'s.
There there decompositions $x'_i$ and $v'_i$ (for all $i$) such that
\begin{itemize}
\item $x'_i$ antirefines to $x_i$ and $z$;
\item $v'_i$ antirefines to $x'_i$, $x'_{i-1}$ and $v_i$;
\item $b_i$ is the image of some $b'_i\in \psi(v'_i)$; and
\item $a_i$ is the image of some $a'_i\in \psi(x'_i)$, which in turn is the image
of $b'_i$ and $b'_{i+1}$.
\end{itemize}
Now consider the diagrams
\[ \xymatrix{
	& \psi(x'_{i-1}) \ar[rd] & \\
	\psi(v'_i) \ar[ru] \ar[rd] & & \psi(z) \\
	& \psi(x'_i) \ar[ru] &
} \]
The associativity axiom applied to this diagram implies that $a'_{i-1}$ and $a'_i$
map to the same element $c\in \psi(z)$.
Therefore $a'_0$ and $a'_k$ both map to $c$.
But $a'_0$ and $a'_k$ are both elements of $\psi(x'_0)$ (because $x'_k = x'_0$).
So by the injectivity clause of the composition axiom, we must have that $a'_0 = a'_k$.
But this implies that $a = a_0 = a_k = \hat{a}$, contrary to our assumption that $a\ne \hat{a}$.
\end{proof}

%\nn{need to finish explaining why we have a system of fields;
%define $k$-cat $\cC(\cdot\times W)$}

\subsection{Modules}

Next we define ordinary and $A_\infty$ $n$-category modules.
The definition will be very similar to that of $n$-categories,
but with $k$-balls replaced by {\it marked $k$-balls,} defined below.

Our motivating example comes from an $(m{-}n{+}1)$-dimensional manifold $W$ with boundary
in the context of an $m{+}1$-dimensional TQFT.
Such a $W$ gives rise to a module for the $n$-category associated to $\bd W$.
This will be explained in more detail as we present the axioms.

Throughout, we fix an $n$-category $\cC$.
For all but one axiom, it doesn't matter whether $\cC$ is an ordinary $n$-category or an $A_\infty$ $n$-category.
We state the final axiom, regarding actions of homeomorphisms, differently in the two cases.

Define a {\it marked $k$-ball} to be a pair $(B, N)$ homeomorphic to the pair
$$(\text{standard $k$-ball}, \text{northern hemisphere in boundary of standard $k$-ball}).$$
We call $B$ the ball and $N$ the marking.
A homeomorphism between marked $k$-balls is a homeomorphism of balls which
restricts to a homeomorphism of markings.

\begin{module-axiom}[Module morphisms] \label{module-axiom-funct}
{For each $0 \le k \le n$, we have a functor $\cM_k$ from 
the category of marked $k$-balls and 
homeomorphisms to the category of sets and bijections.}
\end{module-axiom}

(As with $n$-categories, we will usually omit the subscript $k$.)

For example, let $\cD$ be the TQFT which assigns to a $k$-manifold $N$ the set 
of maps from $N$ to $T$ (for $k\le m$), modulo homotopy (and possibly linearized) if $k=m$.
Let $W$ be an $(m{-}n{+}1)$-dimensional manifold with boundary.
Let $\cC$ be the $n$-category with $\cC(X) \deq \cD(X\times \bd W)$.
Let $\cM(B, N) \deq \cD((B\times \bd W)\cup (N\times W))$
(see Example \ref{ex:maps-with-fiber}).
(The union is along $N\times \bd W$.)
%(If $\cD$ were a general TQFT, we would define $\cM(B, N)$ to be
%the subset of $\cD((B\times \bd W)\cup (N\times W))$ which is splittable along $N\times \bd W$.)

\begin{figure}[!ht]
$$\mathfig{.55}{ncat/boundary-collar}$$
\caption{From manifold with boundary collar to marked ball}\label{blah15}\end{figure}

Define the boundary of a marked $k$-ball $(B, N)$ to be the pair $(\bd B \setmin N, \bd N)$.
Call such a thing a {marked $k{-}1$-hemisphere}.

\begin{lem}
\label{lem:hemispheres}
{For each $0 \le k \le n-1$, we have a functor $\cl\cM_k$ from 
the category of marked $k$-hemispheres and 
homeomorphisms to the category of sets and bijections.}
\end{lem}
The proof is exactly analogous to that of Lemma \ref{lem:spheres}, and we omit the details.
We use the same type of colimit construction.

In our example, $\cl\cM(H) = \cD(H\times\bd W \cup \bd H\times W)$.

\begin{module-axiom}[Module boundaries (maps)]
{For each marked $k$-ball $M$ we have a map of sets $\bd: \cM(M)\to \cl\cM(\bd M)$.
These maps, for various $M$, comprise a natural transformation of functors.}
\end{module-axiom}

Given $c\in\cl\cM(\bd M)$, let $\cM(M; c) \deq \bd^{-1}(c)$.

If the $n$-category $\cC$ is enriched over some other category (e.g.\ vector spaces),
then $\cM(M; c)$ should be an object in that category for each marked $n$-ball $M$
and $c\in \cC(\bd M)$.

\begin{lem}[Boundary from domain and range]
{Let $H = M_1 \cup_E M_2$, where $H$ is a marked $k{-}1$-hemisphere ($1\le k\le n$),
$M_i$ is a marked $k{-}1$-ball, and $E = M_1\cap M_2$ is a marked $k{-}2$-hemisphere.
Let $\cM(M_1) \times_{\cM(E)} \cM(M_2)$ denote the fibered product of the 
two maps $\bd: \cM(M_i)\to \cl\cM(E)$.
Then we have an injective map
\[
	\gl_E : \cM(M_1) \times_{\cl\cM(E)} \cM(M_2) \hookrightarrow \cl\cM(H)
\]
which is natural with respect to the actions of homeomorphisms.}
\end{lem}
Again, this is in exact analogy with Lemma \ref{lem:domain-and-range}.

Let $\cl\cM(H)\trans E$ denote the image of $\gl_E$.
We will refer to elements of $\cl\cM(H)\trans E$ as ``splittable along $E$" or ``transverse to $E$". 

\begin{lem}[Module to category restrictions]
{For each marked $k$-hemisphere $H$ there is a restriction map
$\cl\cM(H)\to \cC(H)$.  
($\cC(H)$ means apply $\cC$ to the underlying $k$-ball of $H$.)
These maps comprise a natural transformation of functors.}
\end{lem}

Note that combining the various boundary and restriction maps above
(for both modules and $n$-categories)
we have for each marked $k$-ball $(B, N)$ and each $k{-}1$-ball $Y\sub \bd B \setmin N$
a natural map from a subset of $\cM(B, N)$ to $\cC(Y)$.
The subset is the subset of morphisms which are appropriately splittable (transverse to the
cutting submanifolds).
This fact will be used below.

In our example, the various restriction and gluing maps above come from
restricting and gluing maps into $T$.

We require two sorts of composition (gluing) for modules, corresponding to two ways
of splitting a marked $k$-ball into two (marked or plain) $k$-balls.
(See Figure \ref{zzz3}.)

\begin{figure}[!ht]
\begin{equation*}
\mathfig{.4}{ncat/zz3}
\end{equation*}
\caption{Module composition (top); $n$-category action (bottom).}
\label{zzz3}
\end{figure}

First, we can compose two module morphisms to get another module morphism.

\begin{module-axiom}[Module composition]
{Let $M = M_1 \cup_Y M_2$, where $M$, $M_1$ and $M_2$ are marked $k$-balls (with $0\le k\le n$)
and $Y = M_1\cap M_2$ is a marked $k{-}1$-ball.
Let $E = \bd Y$, which is a marked $k{-}2$-hemisphere.
Note that each of $M$, $M_1$ and $M_2$ has its boundary split into two marked $k{-}1$-balls by $E$.
We have restriction (domain or range) maps $\cM(M_i)_E \to \cM(Y)$.
Let $\cM(M_1)_E \times_{\cM(Y)} \cM(M_2)_E$ denote the fibered product of these two maps. 
Then (axiom) we have a map
\[
	\gl_Y : \cM(M_1)_E \times_{\cM(Y)} \cM(M_2)_E \to \cM(M)_E
\]
which is natural with respect to the actions of homeomorphisms, and also compatible with restrictions
to the intersection of the boundaries of $M$ and $M_i$.
If $k < n$,
or if $k=n$ and we are in the $A_\infty$ case, 
we require that $\gl_Y$ is injective.
(For $k=n$ in the ordinary (non-$A_\infty$) case, see below.)}
\end{module-axiom}


Second, we can compose an $n$-category morphism with a module morphism to get another
module morphism.
We'll call this the action map to distinguish it from the other kind of composition.

\begin{module-axiom}[$n$-category action]
{Let $M = X \cup_Y M'$, where $M$ and $M'$ are marked $k$-balls ($0\le k\le n$),
$X$ is a plain $k$-ball,
and $Y = X\cap M'$ is a $k{-}1$-ball.
Let $E = \bd Y$, which is a $k{-}2$-sphere.
We have restriction maps $\cM(M')_E \to \cC(Y)$ and $\cC(X)_E\to \cC(Y)$.
Let $\cC(X)\trans E \times_{\cC(Y)} \cM(M')_E$ denote the fibered product of these two maps. 
Then (axiom) we have a map
\[
	\gl_Y :\cC(X)\trans E \times_{\cC(Y)} \cM(M')_E \to \cM(M)_E
\]
which is natural with respect to the actions of homeomorphisms, and also compatible with restrictions
to the intersection of the boundaries of $X$ and $M'$.
If $k < n$,
or if $k=n$ and we are in the $A_\infty$ case, 
we require that $\gl_Y$ is injective.
(For $k=n$ in the ordinary (non-$A_\infty$) case, see below.)}
\end{module-axiom}

\begin{module-axiom}[Strict associativity]
The composition and action maps above are strictly associative.
Given any decomposition of a large marked ball into smaller marked and unmarked balls
any sequence of pairwise gluings yields (via composition and action maps) the same result.
\end{module-axiom}

Note that the above associativity axiom applies to mixtures of module composition,
action maps and $n$-category composition.
See Figure \ref{zzz1b}.

\begin{figure}[!ht]
\begin{equation*}
\mathfig{0.49}{ncat/zz0} \mathfig{0.49}{ncat/zz1}
\end{equation*}
\caption{Two examples of mixed associativity}
\label{zzz1b}
\end{figure}


The above three axioms are equivalent to the following axiom,
which we state in slightly vague form.

\xxpar{Module multi-composition:}
{Given any splitting 
\[
	X_1 \sqcup\cdots\sqcup X_p \sqcup M_1\sqcup\cdots\sqcup M_q \to M
\]
of a marked $k$-ball $M$
into small (marked and plain) $k$-balls $M_i$ and $X_j$, there is a 
map from an appropriate subset (like a fibered product) 
of 
\[
	\cC(X_1)\times\cdots\times\cC(X_p) \times \cM(M_1)\times\cdots\times\cM(M_q) 
\]
to $\cM(M)$,
and these various multifold composition maps satisfy an
operad-type strict associativity condition.}

The above operad-like structure is analogous to the swiss cheese operad
\cite{MR1718089}.

\medskip

We can define marked pinched products $\pi:E\to M$ of marked balls analogously to the 
plain ball case.
Note that a marked pinched product can be decomposed into either
two marked pinched products or a plain pinched product and a marked pinched product.
%\nn{should maybe give figure}

\begin{module-axiom}[Product (identity) morphisms]
For each pinched product $\pi:E\to M$, with $M$ a marked $k$-ball and $E$ a marked
$k{+}m$-ball ($m\ge 1$),
there is a map $\pi^*:\cM(M)\to \cM(E)$.
These maps must satisfy the following conditions.
\begin{enumerate}
\item
If $\pi:E\to M$ and $\pi':E'\to M'$ are marked pinched products, and
if $f:M\to M'$ and $\tilde{f}:E \to E'$ are maps such that the diagram
\[ \xymatrix{
	E \ar[r]^{\tilde{f}} \ar[d]_{\pi} & E' \ar[d]^{\pi'} \\
	M \ar[r]^{f} & M'
} \]
commutes, then we have 
\[
	\pi'^*\circ f = \tilde{f}\circ \pi^*.
\]
\item
Product morphisms are compatible with module composition and module action.
Let $\pi:E\to M$, $\pi_1:E_1\to M_1$, and $\pi_2:E_2\to M_2$ 
be pinched products with $E = E_1\cup E_2$.
Let $a\in \cM(M)$, and let $a_i$ denote the restriction of $a$ to $M_i\sub M$.
Then 
\[
	\pi^*(a) = \pi_1^*(a_1)\bullet \pi_2^*(a_2) .
\]
Similarly, if $\rho:D\to X$ is a pinched product of plain balls and
$E = D\cup E_1$, then
\[
	\pi^*(a) = \rho^*(a')\bullet \pi_1^*(a_1),
\]
where $a'$ is the restriction of $a$ to $D$.
\item
Product morphisms are associative.
If $\pi:E\to M$ and $\rho:D\to E$ are marked pinched products then
\[
	\rho^*\circ\pi^* = (\pi\circ\rho)^* .
\]
\item
Product morphisms are compatible with restriction.
If we have a commutative diagram
\[ \xymatrix{
	D \ar@{^(->}[r] \ar[d]_{\rho} & E \ar[d]^{\pi} \\
	Y \ar@{^(->}[r] & M
} \]
such that $\rho$ and $\pi$ are pinched products, then
\[
	\res_D\circ\pi^* = \rho^*\circ\res_Y .
\]
($Y$ could be either a marked or plain ball.)
\end{enumerate}
\end{module-axiom}

As in the $n$-category definition, once we have product morphisms we can define
collar maps $\cM(M)\to \cM(M)$.
Note that there are two cases:
the collar could intersect the marking of the marked ball $M$, in which case
we use a product on a morphism of $\cM$; or the collar could be disjoint from the marking,
in which case we use a product on a morphism of $\cC$.

In our example, elements $a$ of $\cM(M)$ maps to $T$, and $\pi^*(a)$ is the pullback of
$a$ along a map associated to $\pi$.

\medskip

There are two alternatives for the next axiom, according whether we are defining
modules for ordinary $n$-categories or $A_\infty$ $n$-categories.
In the ordinary case we require

\begin{module-axiom}[\textup{\textbf{[ordinary version]}} Extended isotopy invariance in dimension $n$]
{Let $M$ be a marked $n$-ball and $f: M\to M$ be a homeomorphism which restricts
to the identity on $\bd M$ and is isotopic (rel boundary) to the identity.
Then $f$ acts trivially on $\cM(M)$.}
In addition, collar maps act trivially on $\cM(M)$.
\end{module-axiom}

We emphasize that the $\bd M$ above means boundary in the marked $k$-ball sense.
In other words, if $M = (B, N)$ then we require only that isotopies are fixed 
on $\bd B \setmin N$.

For $A_\infty$ modules we require

%\addtocounter{module-axiom}{-1}
\begin{module-axiom}[\textup{\textbf{[$A_\infty$ version]}} Families of homeomorphisms act]
For each marked $n$-ball $M$ and each $c\in \cM(\bd M)$ we have a map of chain complexes
\[
	C_*(\Homeo_\bd(M))\ot \cM(M; c) \to \cM(M; c) .
\]
Here $C_*$ means singular chains and $\Homeo_\bd(M)$ is the space of homeomorphisms of $M$
which fix $\bd M$.
These action maps are required to be associative up to homotopy, as in Theorem \ref{thm:CH-associativity}, 
and also compatible with composition (gluing) in the sense that
a diagram like the one in Theorem \ref{thm:CH} commutes.
\end{module-axiom}

As with the $n$-category version of the above axiom, we should also have families of collar maps act.

\medskip

Note that the above axioms imply that an $n$-category module has the structure
of an $n{-}1$-category.
More specifically, let $J$ be a marked 1-ball, and define $\cE(X)\deq \cM(X\times J)$,
where $X$ is a $k$-ball and in the product $X\times J$ we pinch 
above the non-marked boundary component of $J$.
(More specifically, we collapse $X\times P$ to a single point, where
$P$ is the non-marked boundary component of $J$.)
Then $\cE$ has the structure of an $n{-}1$-category.

All marked $k$-balls are homeomorphic, unless $k = 1$ and our manifolds
are oriented or Spin (but not unoriented or $\text{Pin}_\pm$).
In this case ($k=1$ and oriented or Spin), there are two types
of marked 1-balls, call them left-marked and right-marked,
and hence there are two types of modules, call them right modules and left modules.
In all other cases ($k>1$ or unoriented or $\text{Pin}_\pm$),
there is no left/right module distinction.

\medskip

We now give some examples of modules over ordinary and $A_\infty$ $n$-categories.

\begin{example}[Examples from TQFTs]
\rm
Continuing Example \ref{ex:ncats-from-tqfts}, with $\cF$ a TQFT, $W$ an $n{-}j$-manifold,
and $\cF(W)$ the $j$-category associated to $W$.
Let $Y$ be an $(n{-}j{+}1)$-manifold with $\bd Y = W$.
Define a $\cF(W)$ module $\cF(Y)$ as follows.
If $M = (B, N)$ is a marked $k$-ball with $k<j$ let 
$\cF(Y)(M)\deq \cF((B\times W) \cup (N\times Y))$.
If $M = (B, N)$ is a marked $j$-ball and $c\in \cl{\cF(Y)}(\bd M)$ let
$\cF(Y)(M)\deq A_\cF((B\times W) \cup (N\times Y); c)$.
\end{example}

\begin{example}[Examples from the blob complex] \label{bc-module-example}
\rm
In the previous example, we can instead define
$\cF(Y)(M)\deq \bc_*((B\times W) \cup (N\times Y), c; \cF)$ (when $\dim(M) = n$)
and get a module for the $A_\infty$ $n$-category associated to $\cF$ as in 
Example \ref{ex:blob-complexes-of-balls}.
\end{example}


\begin{example}
\rm
Suppose $S$ is a topological space, with a subspace $T$.
We can define a module $\pi_{\leq n}(S,T)$ so that on each marked $k$-ball $(B,N)$ 
for $k<n$ the set $\pi_{\leq n}(S,T)(B,N)$ consists of all continuous maps of pairs 
$(B,N) \to (S,T)$ and on each marked $n$-ball $(B,N)$ it consists of all 
such maps modulo homotopies fixed on $\bdy B \setminus N$.
This is a module over the fundamental $n$-category $\pi_{\leq n}(S)$ of $S$, from Example \ref{ex:maps-to-a-space}.
\end{example}
Modifications corresponding to Examples \ref{ex:maps-to-a-space-with-a-fiber} and 
\ref{ex:linearized-maps-to-a-space} are also possible, and there is an $A_\infty$ version analogous to 
Example \ref{ex:chains-of-maps-to-a-space} given by taking singular chains.





\subsection{Modules as boundary labels (colimits for decorated manifolds)}
\label{moddecss}

Fix an ordinary $n$-category or $A_\infty$ $n$-category  $\cC$.
Let $W$ be a $k$-manifold ($k\le n$),
let $\{Y_i\}$ be a collection of disjoint codimension 0 submanifolds of $\bd W$,
and let $\cN = (\cN_i)$ be an assignment of a $\cC$ module $\cN_i$ to $Y_i$.

We will define a set $\cC(W, \cN)$ using a colimit construction very similar to 
the one appearing in \S \ref{ss:ncat_fields} above.
(If $k = n$ and our $n$-categories are enriched, then
$\cC(W, \cN)$ will have additional structure; see below.)

Define a permissible decomposition of $W$ to be a map
\[
	\left(\bigsqcup_a X_a\right) \sqcup \left(\bigsqcup_{i,b} M_{ib}\right)  \to W,
\]
where each $X_a$ is a plain $k$-ball disjoint, in $W$, from $\cup Y_i$, and
each $M_{ib}$ is a marked $k$-ball intersecting $Y_i$  (once mapped into $W$),
with $M_{ib}\cap Y_i$ being the marking, which extends to a ball decomposition in the sense of Definition \ref{defn:gluing-decomposition}.
(See Figure \ref{mblabel}.)
\begin{figure}[t]
\begin{equation*}
\mathfig{.4}{ncat/mblabel}
\end{equation*}
\caption{A permissible decomposition of a manifold
whose boundary components are labeled by $\cC$ modules $\{\cN_i\}$.
Marked balls are shown shaded, plain balls are unshaded.}\label{mblabel}
\end{figure}
Given permissible decompositions $x$ and $y$, we say that $x$ is a refinement
of $y$, or write $x \le y$, if each ball of $y$ is a union of balls of $x$.
This defines a partial ordering $\cell(W)$, which we will think of as a category.
(The objects of $\cell(D)$ are permissible decompositions of $W$, and there is a unique
morphism from $x$ to $y$ if and only if $x$ is a refinement of $y$.)

The collection of modules $\cN$ determines 
a functor $\psi_\cN$ from $\cell(W)$ to the category of sets 
(possibly with additional structure if $k=n$).
For a decomposition $x = (X_a, M_{ib})$ in $\cell(W)$, define $\psi_\cN(x)$ to be the subset
\[
	\psi_\cN(x) \sub \left(\prod_a \cC(X_a)\right) \times \left(\prod_{ib} \cN_i(M_{ib})\right)
\]
such that the restrictions to the various pieces of shared boundaries amongst the
$X_a$ and $M_{ib}$ all agree.
(That is, the fibered product over the boundary restriction maps.)
If $x$ is a refinement of $y$, define a map $\psi_\cN(x)\to\psi_\cN(y)$
via the gluing (composition or action) maps from $\cC$ and the $\cN_i$.

We now define the set $\cC(W, \cN)$ to be the colimit of the functor $\psi_\cN$.
(As in \S\ref{ss:ncat-coend}, if $k=n$ we take a colimit in whatever
category we are enriching over, and if additionally we are in the $A_\infty$ case, 
then we use a homotopy colimit.)

\medskip

If $D$ is an $m$-ball, $0\le m \le n-k$, then we can similarly define
$\cC(D\times W, \cN)$, where in this case $\cN_i$ labels the submanifold 
$D\times Y_i \sub \bd(D\times W)$.
It is not hard to see that the assignment $D \mapsto \cC(D\times W, \cN)$
has the structure of an $n{-}k$-category.

\medskip

We will use a simple special case of the above 
construction to define tensor products 
of modules.
Let $\cM_1$ and $\cM_2$ be modules for an $n$-category $\cC$.
(If $k=1$ and our manifolds are oriented, then one should be 
a left module and the other a right module.)
Choose a 1-ball $J$, and label the two boundary points of $J$ by $\cM_1$ and $\cM_2$.
Define the tensor product $\cM_1 \tensor \cM_2$ to be the 
$n{-}1$-category associated as above to $J$ with its boundary labeled by $\cM_1$ and $\cM_2$.
This of course depends (functorially)
on the choice of 1-ball $J$.

We will define a more general self tensor product (categorified coend) below.




\subsection{Morphisms of modules}
\label{ss:module-morphisms}

Modules are collections of functors together with some additional data, so we define morphisms
of modules to be collections of natural transformations which are compatible with this
additional data.

More specifically, let $\cX$ and $\cY$ be $\cC$ modules, i.e.\ collections of functors
$\{\cX_k\}$ and $\{\cY_k\}$, for $0\le k\le n$, from marked $k$-balls to sets 
as in Module Axiom \ref{module-axiom-funct}.
A morphism $g:\cX\to\cY$ is a collection of natural transformations $g_k:\cX_k\to\cY_k$
satisfying:
\begin{itemize}
\item Each $g_k$ commutes with $\bd$.
\item Each $g_k$ commutes with gluing (module composition and $\cC$ action).
\item Each $g_k$ commutes with taking products.
\item In the top dimension $k=n$, $g_n$ preserves whatever additional structure we are enriching over (e.g.\ vector
spaces).
In the $A_\infty$ case (e.g.\ enriching over chain complexes) $g_n$ should live in 
an appropriate derived hom space, as described below.
\end{itemize}

We will be mainly interested in the case $n=1$ and enriched over chain complexes,
since this is the case that's relevant to the generalized Deligne conjecture of \S\ref{sec:deligne}.
So we treat this case in more detail.

First we explain the remark about derived hom above.
Let $L$ be a marked 1-ball and let $\cl{\cX}(L)$ denote the local homotopy colimit construction
associated to $L$ by $\cX$ and $\cC$.
(See \S \ref{ss:ncat_fields} and \S \ref{moddecss}.)
Define $\cl{\cY}(L)$ similarly.
For $K$ an unmarked 1-ball let $\cl{\cC(K)}$ denote the local homotopy colimit
construction associated to $K$ by $\cC$.
Then we have an injective gluing map
\[
	\gl: \cl{\cX}(L) \ot \cl{\cC}(K) \to \cl{\cX}(L\cup K) 
\]
which is also a chain map.
(For simplicity we are suppressing mention of boundary conditions on the unmarked 
boundary components of the 1-balls.)
We define $\hom_\cC(\cX \to \cY)$ to be a collection of (graded linear) natural transformations
$g: \cl{\cX}(L)\to \cl{\cY}(L)$ such that the following diagram commutes for all $L$ and $K$:
\[ \xymatrix{
	\cl{\cX}(L) \ot \cl{\cC}(K) \ar[r]^{\gl} \ar[d]_{g\ot \id} & \cl{\cX}(L\cup K) \ar[d]^{g}\\
	\cl{\cY}(L) \ot \cl{\cC}(K) \ar[r]^{\gl} & \cl{\cY}(L\cup K)
} \]

The usual differential on graded linear maps between chain complexes induces a differential
on $\hom_\cC(\cX \to \cY)$, giving it the structure of a chain complex.

Let $\cZ$ be another $\cC$ module.
We define a chain map
\[
	a: \hom_\cC(\cX \to \cY) \ot (\cX \ot_\cC \cZ) \to \cY \ot_\cC \cZ
\]
as follows.
Recall that the tensor product $\cX \ot_\cC \cZ$  depends on a choice of interval $J$, labeled
by $\cX$ on one boundary component and $\cZ$ on the other.
Because we are using the {\it local} homotopy colimit, any generator
$D\ot x\ot \bar{c}\ot z$ of $\cX \ot_\cC \cZ$ can be written (perhaps non-uniquely) as a gluing
$(D'\ot x \ot \bar{c}') \bullet (D''\ot \bar{c}''\ot z)$, for some decomposition $J = L'\cup L''$
and with $D'\ot x \ot \bar{c}'$ a generator of $\cl{\cX}(L')$ and 
$D''\ot \bar{c}''\ot z$ a generator of $\cl{\cZ}(L'')$.
(Such a splitting exists because the blob diagram $D$ can be split into left and right halves, 
since no blob can include both the leftmost and rightmost intervals in the underlying decomposition.
This step would fail if we were using the usual hocolimit instead of the local hocolimit.)
We now define
\[
	a: g\ot (D\ot x\ot \bar{c}\ot z) \mapsto g(D'\ot x \ot \bar{c}')\bullet (D''\ot \bar{c}''\ot z) .
\]
This does not depend on the choice of splitting $D = D'\bullet D''$ because $g$ commutes with gluing.




\subsection{The \texorpdfstring{$n{+}1$}{n+1}-category of sphere modules}
\label{ssec:spherecat}

In this subsection we define $n{+}1$-categories $\cS$ of ``sphere modules" 
whose objects are $n$-categories.
With future applications in mind, we treat simultaneously the big category
of all $n$-categories and all sphere modules and also subcategories thereof.
When $n=1$ this is closely related to familiar $2$-categories consisting of 
algebras, bimodules and intertwiners (or a subcategory of that).

While it is appropriate to call an $S^0$ module a bimodule,
this is much less true for higher dimensional spheres, 
so we prefer the term ``sphere module" for the general case.

%The results of this subsection are not needed for the rest of the paper,
%so we will skimp on details in a couple of places. We have included this mostly 
%for the sake of comparing our notion of a disk-like $n$-category to other definitions.

For simplicity, we will assume that $n$-categories are enriched over $\c$-vector spaces.

The $0$- through $n$-dimensional parts of $\cS$ are various sorts of modules, and we describe
these first.
The $n{+}1$-dimensional part of $\cS$ consists of intertwiners
of  $1$-category modules associated to decorated $n$-balls.
We will see below that in order for these $n{+}1$-morphisms to satisfy all of
the axioms of an $n{+}1$-category (in particular, duality requirements), we will have to assume
that our $n$-categories and modules have non-degenerate inner products.
(In other words, we need to assume some extra duality on the $n$-categories and modules.)

\medskip

Our first task is to define an $n$-category $m$-sphere module, for $0\le m \le n-1$.
These will be defined in terms of certain classes of marked balls, very similarly
to the definition of $n$-category modules above.
(This, in turn, is very similar to our definition of $n$-category.)
Because of this similarity, we only sketch the definitions below.

We start with $0$-sphere modules, which also could reasonably be called (categorified) bimodules.
(For $n=1$ they are precisely bimodules in the usual, uncategorified sense.)
We prefer the more awkward term ``0-sphere module" to emphasize the analogy
with the higher sphere modules defined below.

Define a $0$-marked $k$-ball, $1\le k \le n$, to be a pair  $(X, M)$ homeomorphic to the standard
$(B^k, B^{k-1})$.
See Figure \ref{feb21a}.
Another way to say this is that $(X, M)$ is homeomorphic to $B^{k-1}\times([-1,1], \{0\})$.

\begin{figure}[t]
$$\tikz[baseline,line width=2pt]{\draw[blue] (-2,0)--(2,0); \fill[red] (0,0) circle (0.1);} \qquad \qquad \tikz[baseline,line width=2pt]{\draw[blue][fill=blue!30!white] (0,0) circle (2 and 1); \draw[red] (0,1)--(0,-1);}$$
\caption{0-marked 1-ball and 0-marked 2-ball}
\label{feb21a}
\end{figure}

The $0$-marked balls can be cut into smaller balls in various ways.
We only consider those decompositions in which the smaller balls are either
$0$-marked (i.e. intersect the $0$-marking of the large ball in a disc) 
or plain (don't intersect the $0$-marking of the large ball).
We can also take the boundary of a $0$-marked ball, which is $0$-marked sphere.

Fix $n$-categories $\cA$ and $\cB$.
These will label the two halves of a $0$-marked $k$-ball.

An $n$-category $0$-sphere module $\cM$ over the $n$-categories $\cA$ and $\cB$ is a collection of functors $\cM_k$ from the category
of $0$-marked $k$-balls, $1\le k \le n$,
(with the two halves labeled by $\cA$ and $\cB$) to the category of sets.
If $k=n$ these sets should be enriched to the extent $\cA$ and $\cB$ are.
Given a decomposition of a $0$-marked $k$-ball $X$ into smaller balls $X_i$, we have
morphism sets $\cA_k(X_i)$ (if $X_i$ lies on the $\cA$-labeled side)
or $\cB_k(X_i)$ (if $X_i$ lies on the $\cB$-labeled side)
or $\cM_k(X_i)$ (if $X_i$ intersects the marking and is therefore a smaller 0-marked ball).
Corresponding to this decomposition we have a composition (or ``gluing") map
from the product (fibered over the boundary data) of these various sets into $\cM_k(X)$.

\medskip

Part of the structure of an $n$-category 0-sphere module $\cM$  is captured by saying it is
a collection $\cD^{ab}$ of $n{-}1$-categories, indexed by pairs $(a, b)$ of objects (0-morphisms)
of $\cA$ and $\cB$.
Let $J$ be some standard 0-marked 1-ball (i.e.\ an interval with a marked point in its interior).
Given a $j$-ball $X$, $0\le j\le n-1$, we define
\[
	\cD(X) \deq \cM(X\times J) .
\]
The product is pinched over the boundary of $J$.
The set $\cD$ breaks into ``blocks" according to the restrictions to the pinched points of $X\times J$
(see Figure \ref{feb21b}).
These restrictions are 0-morphisms $(a, b)$ of $\cA$ and $\cB$.

\begin{figure}[t] \centering
\begin{tikzpicture}[blue,line width=2pt]
\draw (0,1) -- (0,-1) node[below] {$X$};

\draw (2,0) -- (4,0) node[below] {$J$};
\fill[red] (3,0) circle (0.1);

\draw[fill=blue!30!white] (6,0) node(a) {} arc (135:90:4) node(top) {} arc (90:45:4) node(b) {} arc (-45:-90:4) node(bottom) {} arc(-90:-135:4);
\draw[red] (top.center) -- (bottom.center);
\fill (a) circle (0.1) node[left] {\color{green!50!brown} $a$};
\fill (b) circle (0.1) node[right] {\color{green!50!brown} $b$};

\path (bottom) node[below]{$X \times J$};

\end{tikzpicture}
\caption{The pinched product $X\times J$}
\label{feb21b}
\end{figure}

More generally, consider an interval with interior marked points, and with the complements
of these points labeled by $n$-categories $\cA_i$ ($0\le i\le l$) and the marked points labeled
by $\cA_i$-$\cA_{i+1}$ 0-sphere modules $\cM_i$.
(See Figure \ref{feb21c}.)
To this data we can apply the coend construction as in \S\ref{moddecss} above
to obtain an $\cA_0$-$\cA_l$ $0$-sphere module and, forgetfully, an $n{-}1$-category.
This amounts to a definition of taking tensor products of $0$-sphere modules over $n$-categories.

\begin{figure}[t] \centering
\begin{tikzpicture}[baseline,line width = 2pt]
\draw[blue] (0,0) -- (6,0);
\foreach \x/\n in {0.5/0,1.5/1,3/2,4.5/3,5.5/4} {
	\path (\x,0)  node[below] {\color{green!50!brown}$\cA_{\n}$};
}
\foreach \x/\n in {1/0,2/1,4/2,5/3} {
	\fill[red] (\x,0) circle (0.1) node[above] {\color{green!50!brown}$\cM_{\n}$};
}
\end{tikzpicture}
\qquad
\qquad
\begin{tikzpicture}[baseline,line width = 2pt]
\draw[blue] (0,0) circle (2);
\foreach \q/\n in {-45/0,90/1,180/2} {
	\path (\q:2.4)  node {\color{green!50!brown}$\cA_{\n}$};
}
\foreach \q/\n in {60/0,120/1,-120/2} {
	\fill[red] (\q:2) circle (0.1);
	\path (\q:2.4) node {\color{green!50!brown}$\cM_{\n}$};
}
\end{tikzpicture}
\caption{Marked and labeled 1-manifolds}
\label{feb21c}
\end{figure}

We could also similarly mark and label a circle, obtaining an $n{-}1$-category
associated to the marked and labeled circle.
(See Figure \ref{feb21c}.)
If the circle is divided into two intervals, we can think of this $n{-}1$-category
as the 2-sided tensor product of the two 0-sphere modules associated to the two intervals.

\medskip

Next we define $n$-category 1-sphere modules.
These are just representations of (modules for) $n{-}1$-categories associated to marked and labeled 
circles (1-spheres) which we just introduced.

Equivalently, we can define 1-sphere modules in terms of 1-marked $k$-balls, $2\le k\le n$.
Fix a marked (and labeled) circle $S$.
Let $C(S)$ denote the cone of $S$, a marked 2-ball (Figure \ref{feb21d}).
%\nn{I need to make up my mind whether marked things are always labeled too.
%For the time being, let's say they are.}
A 1-marked $k$-ball is anything homeomorphic to $B^j \times C(S)$, $0\le j\le n-2$, 
where $B^j$ is the standard $j$-ball.
A 1-marked $k$-ball can be decomposed in various ways into smaller balls, which are either 
(a) smaller 1-marked $k$-balls, (b) 0-marked $k$-balls, or (c) plain $k$-balls.
(See Figure \ref{subdividing1marked}.)
We now proceed as in the above module definitions.

\begin{figure}[t] \centering
\begin{tikzpicture}[baseline,line width = 2pt]
\draw[blue][fill=blue!15!white] (0,0) circle (2);
\fill[red] (0,0) circle (0.1);
\foreach \qm/\qa/\n in {70/-30/0, 120/95/1, -120/180/2} {
	\draw[red] (0,0) -- (\qm:2);
	\path (\qa:1) node {\color{green!50!brown} $\cA_\n$};
	\path (\qm+20:2.5) node(M\n) {\color{green!50!brown} $\cM_\n$};
	\draw[line width=1pt, green!50!brown, ->] (M\n.\qm+135) to[out=\qm+135,in=\qm+90] (\qm+5:1.3);
}
\end{tikzpicture}
\caption{Cone on a marked circle, the prototypical 1-marked ball}
\label{feb21d}
\end{figure}

\begin{figure}[t] \centering
\begin{tikzpicture}[baseline,line width = 2pt]
\draw[blue][fill=blue!15!white] (0,0) circle (2);
\fill[red] (0,0) circle (0.1);
\foreach \qm/\qa/\n in {70/-30/0, 120/95/1, -120/180/2} {
	\draw[red] (0,0) -- (\qm:2);
%	\path (\qa:1) node {\color{green!50!brown} $\cA_\n$};
%	\path (\qm+20:2.5) node(M\n) {\color{green!50!brown} $\cM_\n$};
%	\draw[line width=1pt, green!50!brown, ->] (M\n.\qm+135) to[out=\qm+135,in=\qm+90] (\qm+5:1.3);
}


\begin{scope}[black, thin]
\clip (0,0) circle (2);
\draw (0:1) -- (90:1) -- (180:1) -- (270:1) -- cycle;
\draw (90:1) -- (90:2.1);
\draw (180:1) -- (180:2.1);
\draw (270:1) -- (270:2.1);
\draw (0:1) -- (15:2.1);
\draw (0:1) -- (315:1.5) -- (270:1);
\draw (315:1.5) -- (315:2.1);
\end{scope}

\node(0marked) at (2.5,2.25) {$0$-marked ball};
\node(1marked) at (3.5,1) {$1$-marked ball};
\node(plain) at (3,-1) {plain ball};
\draw[line width=1pt, green!50!brown, ->] (0marked.270) to[out=270,in=45] (50:1.1);
\draw[line width=1pt, green!50!brown, ->] (1marked.225) to[out=270,in=45] (0.4,0.1);
\draw[line width=1pt, green!50!brown, ->] (plain.90) to[out=135,in=45] (-45:1);

\end{tikzpicture}
\caption{Subdividing a $1$-marked ball into plain, $0$-marked and $1$-marked balls.}
\label{subdividing1marked}
\end{figure}

A $n$-category 1-sphere module is, among other things, an $n{-}2$-category $\cD$ with
\[
	\cD(X) \deq \cM(X\times C(S)) .
\]
The product is pinched over the boundary of $C(S)$.
$\cD$ breaks into ``blocks" according to the restriction to the 
image of $\bd C(S) = S$ in $X\times C(S)$.

More generally, consider a 2-manifold $Y$ 
(e.g.\ 2-ball or 2-sphere) marked by an embedded 1-complex $K$.
The components of $Y\setminus K$ are labeled by $n$-categories, 
the edges of $K$ are labeled by 0-sphere modules, 
and the 0-cells of $K$ are labeled by 1-sphere modules.
We can now apply the coend construction and obtain an $n{-}2$-category.
If $Y$ has boundary then this $n{-}2$-category is a module for the $n{-}1$-category
associated to the (marked, labeled) boundary of $Y$.
In particular, if $\bd Y$ is a 1-sphere then we get a 1-sphere module as defined above.

\medskip

It should now be clear how to define $n$-category $m$-sphere modules for $0\le m \le n-1$.
For example, there is an $n{-}2$-category associated to a marked, labeled 2-sphere,
and a 2-sphere module is a representation of such an $n{-}2$-category.

\medskip

We can now define the $n$-or-less-dimensional part of our $n{+}1$-category $\cS$.
Choose some collection of $n$-categories, then choose some collections of 0-sphere modules between
these $n$-categories, then choose some collection of 1-sphere modules for the various
possible marked 1-spheres labeled by the $n$-categories and 0-sphere modules, and so on.
Let $L_i$ denote the collection of $i{-}1$-sphere modules we have chosen.
(For convenience, we declare a $(-1)$-sphere module to be an $n$-category.)
There is a wide range of possibilities.
The set $L_0$ could contain infinitely many $n$-categories or just one.
For each pair of $n$-categories in $L_0$, $L_1$ could contain no 0-sphere modules at all or 
it could contain several.
The only requirement is that each $k$-sphere module be a module for a $k$-sphere $n{-}k$-category
constructed out of labels taken from $L_j$ for $j<k$.

We now define $\cS(X)$, for $X$ a ball of dimension at most $n$, to be the set of all 
cell-complexes $K$ embedded in $X$, with the codimension-$j$ parts of $(X, K)$ labeled
by elements of $L_j$.
As described above, we can think of each decorated $k$-ball as defining a $k{-}1$-sphere module
for the $n{-}k{+}1$-category associated to its decorated boundary.
Thus the $k$-morphisms of $\cS$ (for $k\le n$) can be thought 
of as $n$-category $k{-}1$-sphere modules 
(generalizations of bimodules).
On the other hand, we can equally well think of the $k$-morphisms as decorations on $k$-balls, 
and from this point of view it is clear that they satisfy all of the axioms of an
$n{+}1$-category.
(All of the axioms for the less-than-$n{+}1$-dimensional part of an $n{+}1$-category, that is.)

\medskip

Next we define the $n{+}1$-morphisms of $\cS$.
The construction of the 0- through $n$-morphisms was easy and tautological, but the 
$n{+}1$-morphisms will require some effort and combinatorial topology, as well as additional
duality assumptions on the lower morphisms. These are required because we define the spaces of $n{+}1$-morphisms by making arbitrary choices of incoming and outgoing boundaries for each $n$-ball. The additional duality assumptions are needed to prove independence of our definition form these choices.

Let $X$ be an $n{+}1$-ball, and let $c$ be a decoration of its boundary
by a cell complex labeled by 0- through $n$-morphisms, as above.
Choose an $n{-}1$-sphere $E\sub \bd X$ which divides
$\bd X$ into ``incoming" and ``outgoing" boundary $\bd_-X$ and $\bd_+X$.
Let $E_c$ denote $E$ decorated by the restriction of $c$ to $E$.
Recall from above the associated 1-category $\cS(E_c)$.
We can also have $\cS(E_c)$ modules $\cS(\bd_-X_c)$ and $\cS(\bd_+X_c)$.
Define
\[
	\cS(X; c; E) \deq \hom_{\cS(E_c)}(\cS(\bd_-X_c), \cS(\bd_+X_c)) .
\]

We will show that if the sphere modules are equipped with a ``compatible family of 
non-degenerate inner products", then there is a coherent family of isomorphisms
$\cS(X; c; E) \cong \cS(X; c; E')$ for all pairs of choices $E$ and $E'$.
This will allow us to define $\cS(X; c)$ independently of the choice of $E$.

First we must define ``inner product", ``non-degenerate" and ``compatible".
Let $Y$ be a decorated $n$-ball, and $\ol{Y}$ it's mirror image.
(We assume we are working in the unoriented category.)
Let $Y\cup\ol{Y}$ denote the decorated $n$-sphere obtained by gluing $Y$ and $\ol{Y}$
along their common boundary.
An {\it inner product} on $\cS(Y)$ is a dual vector
\[
	z_Y : \cS(Y\cup\ol{Y}) \to \c.
\]
We will also use the notation
\[
	\langle a, b\rangle \deq z_Y(a\bullet \ol{b}) \in \c .
\]
An inner product induces a linear map
\begin{eqnarray*}
	\varphi: \cS(Y) &\to& \cS(Y)^* \\
	a &\mapsto& \langle a, \cdot \rangle
\end{eqnarray*}
which satisfies, for all morphisms $e$ of $\cS(\bd Y)$,
\[
	\varphi(ae)(b) = \langle ae, b \rangle = z_Y(a\bullet e\bullet b) = 
			\langle a, eb \rangle = \varphi(a)(eb) .
\]
In other words, $\varphi$ is a map of $\cS(\bd Y)$ modules.
An inner product is {\it non-degenerate} if $\varphi$ is an isomorphism.
This implies that $\cS(Y; c)$ is finite dimensional for all boundary conditions $c$.
(One can think of these inner products as giving some duality in dimension $n{+}1$;
heretofore we have only assumed duality in dimensions 0 through $n$.)

Next we define compatibility.
Let $Y = Y_1\cup Y_2$ with $D = Y_1\cap Y_2$.
Let $X_1$ and $X_2$ be the two components of $Y\times I$ cut along
$D\times I$, in both cases using the pinched product.
(Here we are overloading notation and letting $D$ denote both a decorated and an undecorated
manifold.)
We have $\bd X_i = Y_i \cup \ol{Y}_i \cup (D\times I)$
(see Figure \ref{jun23a}).
\begin{figure}[t]
\begin{equation*}
\mathfig{.6}{ncat/YxI-sliced}
\end{equation*}
\caption{$Y\times I$ sliced open}
\label{jun23a}
\end{figure}
Given $a_i\in \cS(Y_i)$, $b_i\in \cS(\ol{Y}_i)$ and $v\in\cS(D\times I)$
which agree on their boundaries, we can evaluate
\[
	z_{Y_i}(a_i\bullet b_i\bullet v) \in \c .
\]
(This requires a choice of homeomorphism $Y_i \cup \ol{Y}_i \cup (D\times I) \cong
Y_i \cup \ol{Y}_i$, but the value of $z_{Y_i}$ is independent of this choice.)
We can think of $z_{Y_i}$ as giving a function
\[
	\psi_i : \cS(Y_i) \ot \cS(\ol{Y}_i) \to \cS(D\times I)^* 
					\stackrel{\varphi\inv}{\longrightarrow} \cS(D\times I) .
\]
We can now finally define a family of inner products to be {\it compatible} if
for all decompositions $Y = Y_1\cup Y_2$ as above and all $a_i\in \cS(Y_i)$, $b_i\in \cS(\ol{Y}_i)$
we have
\[
	z_Y(a_1\bullet a_2\bullet b_1\bullet b_2) = 
				z_{D\times I}(\psi_1(a_1\ot b_1)\bullet \psi_2(a_2\ot b_2)) .
\]
In other words, the inner product on $Y$ is determined by the inner products on
$Y_1$, $Y_2$ and $D\times I$.

Now we show how to unambiguously identify $\cS(X; c; E)$ and $\cS(X; c; E')$ for any
two choices of $E$ and $E'$.
Consider first the case where $\bd X$ is decomposed as three $n$-balls $A$, $B$ and $C$,
with $E = \bd(A\cup B)$ and $E' = \bd A$.
We must provide an isomorphism between $\cS(X; c; E) = \hom(\cS(C), \cS(A\cup B))$
and $\cS(X; c; E') = \hom(\cS(C\cup \ol{B}), \cS(A))$.
Let $D = B\cap A$.
Then as above we can construct a map
\[
	\psi: \cS(B)\ot\cS(\ol{B}) \to \cS(D\times I) .
\]
Given $f\in \hom(\cS(C), \cS(A\cup B))$ we define $f'\in \hom(\cS(C\cup \ol{B}), \cS(A))$
to be the composition
\[
	\cS(C\cup \ol{B}) \stackrel{f\ot\id}{\longrightarrow}
		\cS(A\cup B\cup \ol{B})  \stackrel{\id\ot\psi}{\longrightarrow}
			\cS(A\cup(D\times I)) \stackrel{\cong}{\longrightarrow} \cS(A) .
\]
(See Figure \ref{jun23b}.)
\begin{figure}[t]
$$
\begin{tikzpicture}[baseline,line width = 1pt,x=1.5cm,y=1.5cm]
\draw (0,0) node(R) {}
	-- (0.75,0) node[below] {$\bar{B}$}
	--(1.5,0)  node[circle,fill=black,inner sep=2pt] {}
	arc (0:80:1.5) node[above] {$D \times I$}
	arc (80:180:1.5);
\foreach \r in {0.3, 0.6, 0.9, 1.2} {
	\draw[blue!50, line width = 0.5pt] (\r,0) arc (0:180:\r);
}
\draw[fill=white]
	(R) node[circle,fill=black,inner sep=2pt] {}
	arc (45:65:3) node[below] {$B$}
	arc (65:90:3) node[below] {$A$}
	arc (90:135:3) node[circle,fill=black,inner sep=2pt] {}
	arc (-135:-90:3) node[below] {$C$}
	arc (-90:-45:3);
\draw[fill]  (150:1.5) circle (2pt) node[above=4pt] {$D$};
\node[green!50!brown] at (-2,0) {\scalebox{1.4}{$\uparrow f$}};
\node[green!50!brown] at (0.2,0.8) {\scalebox{1.4}{$\uparrow \psi$}};
\end{tikzpicture}
$$
\caption{Moving $B$ from top to bottom}
\label{jun23b}
\end{figure}
Let $D' = B\cap C$.
Using the inner products there is an adjoint map
\[
	\psi^\dagger: \cS(D'\times I) \to \cS(\ol{B})\ot\cS(B) .
\]
Given $f'\in \hom(\cS(C\cup \ol{B}), \cS(A))$ we define $f\in \hom(\cS(C), \cS(A\cup B))$
to be the composition
\[
	\cS(C) \stackrel{\cong}{\longrightarrow}
		\cS(C\cup(D'\times I)) \stackrel{\id\ot\psi^\dagger}{\longrightarrow}
			\cS(C\cup \ol{B}\cup B)   \stackrel{f'\ot\id}{\longrightarrow}
				\cS(A\cup B) .
\]
(See Figure \ref{jun23c}.)
\begin{figure}[t]
\begin{equation*}
\begin{tikzpicture}[baseline,line width = 1pt,x=1.5cm,y=-1.5cm]
\draw (0,0) node(R) {}
	-- (0.75,0) node[above] {$B$}
	--(1.5,0)  node[circle,fill=black,inner sep=2pt] {}
	arc (0:80:1.5) node[below] {$D' \times I$}
	arc (80:180:1.5);
\foreach \r in {0.3, 0.6, 0.9, 1.2} {
	\draw[blue!50, line width = 0.5pt] (\r,0) arc (0:180:\r);
}
\draw[fill=white]
	(R) node[circle,fill=black,inner sep=2pt] {}
	arc (45:65:3) node[above] {$\bar{B}$}
	arc (65:90:3) node[below] {$C$}
	arc (90:135:3) node[circle,fill=black,inner sep=2pt] {}
	arc (-135:-90:3) node[below] {$A$}
	arc (-90:-45:3);
\draw[fill]  (150:1.5) circle (2pt) node[below=4pt] {$D'$};
\node[green!50!brown] at (-2,0) {\scalebox{1.4}{$f'\uparrow $}};
\node[green!50!brown] at (0.2,0.8) {\scalebox{1.4}{$\psi^\dagger \uparrow $}};
\end{tikzpicture}
\end{equation*}
\caption{Moving $B$ from bottom to top}
\label{jun23c}
\end{figure}
Let $D' = B\cap C$.
It is not hard too show that the above two maps are mutually inverse.

\begin{lem} \label{equator-lemma}
Any two choices of $E$ and $E'$ are related by a series of modifications as above.
\end{lem}

\begin{proof}
(Sketch)
$E$ and $E'$ are isotopic, and any isotopy is 
homotopic to a composition of small isotopies which are either
(a) supported away from $E$, or (b) modify $E$ in the simple manner described above.
\end{proof}

It follows from the lemma that we can construct an isomorphism
between $\cS(X; c; E)$ and $\cS(X; c; E')$ for any pair $E$, $E'$.
This construction involves on a choice of simple ``moves" (as above) to transform
$E$ to $E'$.
We must now show that the isomorphism does not depend on this choice.
We will show below that it suffice to check two ``movie moves".

The first movie move is to push $E$ across an $n$-ball $B$ as above, then push it back.
The result is equivalent to doing nothing.
As we remarked above, the isomorphisms corresponding to these two pushes are mutually
inverse, so we have invariance under this movie move.

The second movie move replaces two successive pushes in the same direction,
across $B_1$ and $B_2$, say, with a single push across $B_1\cup B_2$.
(See Figure \ref{jun23d}.)
\begin{figure}[t]
\begin{tikzpicture}
\node(L) {
\scalebox{0.5}{
\begin{tikzpicture}[baseline,line width = 1pt,x=1.5cm,y=1.5cm]
\draw[red] (0.75,0) -- +(2,0);
\draw[red] (0,0) node(R) {}
	-- (0.75,0) node[below] {}
	--(1.5,0)  node[circle,fill=black,inner sep=2pt] {};
\draw[fill]  (150:1.5) circle (2pt) node[above=4pt] {};
\draw (1.5,0) arc (0:149:1.5);
\draw[red]
	(R) node[circle,fill=black,inner sep=2pt] {}
	arc (-45:-135:3) node[circle,fill=black,inner sep=2pt] {};
\draw[red] (-5.5,0) -- (-4.2,0);
\draw (R) arc (45:75:3);
\draw (150:1.5) arc (74:135:3);
\node at (-2,0) {\scalebox{2.0}{$B_1$}};
\node at (0.2,0.8) {\scalebox{2.0}{$B_2$}};
\node at (-4,1.2) {\scalebox{2.0}{$A$}};
\node at (-4,-1.2) {\scalebox{2.0}{$C$}};
\node[red] at (2.53,0.35) {\scalebox{2.0}{$E$}};
\end{tikzpicture}
}
};
\node(M) at (5,4) {
\scalebox{0.5}{
\begin{tikzpicture}[baseline,line width = 1pt,x=1.5cm,y=1.5cm]
\draw[red] (0.75,0) -- +(2,0);
\draw[red] (0,0) node(R) {}
	-- (0.75,0) node[below] {}
	--(1.5,0)  node[circle,fill=black,inner sep=2pt] {};
\draw[fill]  (150:1.5) circle (2pt) node[above=4pt] {};
\draw(1.5,0) arc (0:149:1.5);
\draw
	(R) node[circle,fill=black,inner sep=2pt] {}
	arc (-45:-135:3) node[circle,fill=black,inner sep=2pt] {};
\draw[red] (-5.5,0) -- (-4.2,0);
\draw[red] (R) arc (45:75:3);
\draw[red] (150:1.5) arc (74:135:3);
\node at (-2,0) {\scalebox{2.0}{$B_1$}};
\node at (0.2,0.8) {\scalebox{2.0}{$B_2$}};
\node at (-4,1.2) {\scalebox{2.0}{$A$}};
\node at (-4,-1.2) {\scalebox{2.0}{$C$}};
\node[red] at (2.53,0.35) {\scalebox{2.0}{$E$}};
\end{tikzpicture}
}
};
\node(R) at (10,0) {
\scalebox{0.5}{
\begin{tikzpicture}[baseline,line width = 1pt,x=1.5cm,y=1.5cm]
\draw[red] (0.75,0) -- +(2,0);
\draw (0,0) node(R) {}
	-- (0.75,0) node[below] {}
	--(1.5,0)  node[circle,fill=black,inner sep=2pt] {};
\draw[fill]  (150:1.5) circle (2pt) node[above=4pt] {};
\draw[red] (1.5,0) arc (0:149:1.5);
\draw
	(R) node[circle,fill=black,inner sep=2pt] {}
	arc (-45:-135:3) node[circle,fill=black,inner sep=2pt] {};
\draw[red] (-5.5,0) -- (-4.2,0);
\draw (R) arc (45:75:3);
\draw[red] (150:1.5) arc (74:135:3);
\node at (-2,0) {\scalebox{2.0}{$B_1$}};
\node at (0.2,0.8) {\scalebox{2.0}{$B_2$}};
\node at (-4,1.2) {\scalebox{2.0}{$A$}};
\node at (-4,-1.2) {\scalebox{2.0}{$C$}};
\node[red] at (2.53,0.35) {\scalebox{2.0}{$E$}};
\end{tikzpicture}
}
};
\draw[->] (L) to[out=90,in=225] node[sloped, above] {push $B_1$} (M);
\draw[->] (M)  to[out=-45,in=90] node[sloped, above] {push $B_2$} (R);
\draw[->] (L) to[out=-35,in=-145] node[sloped, below] {push $B_1 \cup B_2$} (R);
\end{tikzpicture}
\caption{A movie move}
\label{jun23d}
\end{figure}
Invariance under this movie move follows from the compatibility of the inner
product for $B_1\cup B_2$ with the inner products for $B_1$ and $B_2$.

%The third movie move could be called ``locality" or ``disjoint commutativity".
%\nn{...}

If $n\ge 2$, these two movie move suffice:

\begin{lem}
Assume $n\ge 2$ and fix $E$ and $E'$ as above.
Then any two sequences of elementary moves connecting $E$ to $E'$
are related by a sequence of the two movie moves defined above.
\end{lem}

\begin{proof}
(Sketch)
Consider a two parameter family of diffeomorphisms (one parameter family of isotopies) 
of $\bd X$.
Up to homotopy,
such a family is homotopic to a family which can be decomposed 
into small families which are either
(a) supported away from $E$, 
(b) have boundaries corresponding to the two movie moves above.
Finally, observe that the space of $E$'s is simply connected.
(This fails for $n=1$.)
\end{proof}

For $n=1$ we have to check an additional ``global" relations corresponding to 
rotating the 0-sphere $E$ around the 1-sphere $\bd X$.
But if $n=1$, then we are in the case of ordinary algebroids and bimodules,
and this is just the well-known ``Frobenius reciprocity" result for bimodules \cite{MR1424954}.

\medskip

We have now defined $\cS(X; c)$ for any $n{+}1$-ball $X$ with boundary decoration $c$.
We must also define, for any homeomorphism $X\to X'$, an action $f: \cS(X; c) \to \cS(X', f(c))$.
Choosing an equator $E\sub \bd X$ we have 
\[
	\cS(X; c) \cong \cS(X; c; E) \deq \hom_{\cS(E_c)}(\cS(\bd_-X_c), \cS(\bd_+X_c)) .
\]
We define $f: \cS(X; c) \to \cS(X', f(c))$ to be the tautological map
\[
	f: \cS(X; c; E) \to \cS(X'; f(c); f(E)) .
\]
It is easy to show that this is independent of the choice of $E$.
Note also that this map depends only on the restriction of $f$ to $\bd X$.
In particular, if $F: X\to X$ is the identity on $\bd X$ then $f$ acts trivially, as required by
Axiom \ref{axiom:extended-isotopies}.

We define product $n{+}1$-morphisms to be identity maps of modules.

To define (binary) composition of $n{+}1$-morphisms, choose the obvious common equator
then compose the module maps.
The proof that this composition rule is associative is similar to the proof of Lemma \ref{equator-lemma}.


%!TEX root = ../blob1.tex

\section{The blob complex for $A_\infty$ $n$-categories}
\label{sec:ainfblob}

Given an $A_\infty$ $n$-category $\cC$ and an $n$-manifold $M$, we define the blob
complex $\bc_*(M)$ to the be the homotopy colimit $\cC(M)$ of Section \ref{sec:ncats}.
\nn{say something about this being anticlimatically tautological?}
We will show below 
\nn{give ref}
that this agrees (up to homotopy) with our original definition of the blob complex
in the case of plain $n$-categories.
When we need to distinguish between the new and old definitions, we will refer to the 
new-fangled and old-fashioned blob complex.

\medskip

Let $M^n = Y^k\times F^{n-k}$.  
Let $C$ be a plain $n$-category.
Let $\cF$ be the $A_\infty$ $k$-category which assigns to a $k$-ball
$X$ the old-fashioned blob complex $\bc_*(X\times F)$.

\begin{thm} \label{product_thm}
The old-fashioned blob complex $\bc_*^C(Y\times F)$ is homotopy equivalent to the
new-fangled blob complex $\bc_*^\cF(Y)$.
\end{thm}

%!TEX root = ../blob1.tex
\nn{Not sure where this goes yet: small blobs, unfinished:}

Fix $\cU$, an open cover of $M$. Define the `small blob complex' $\bc^{\cU}_*(M)$ to be the subcomplex of $\bc_*(M)$ of all blob diagrams in which every blob is contained in some open set of $\cU$.

\begin{thm}[Small blobs]
The inclusion $i: \bc^{\cU}_*(M) \into \bc_*(M)$ is a homotopy equivalence.
\end{thm}
\begin{proof}
We begin by describing the homotopy inverse in small degrees, to illustrate the general technique.
We will construct a chain map $s:  \bc_*(M) \to \bc^{\cU}_*(M)$ and a homotopy $h:\bc_*(M) \to \bc_{*+1}(M)$ so that $\bdy h+h \bdy=\id - i\circ s$. The composition $s \circ i$ will just be the identity.

On $0$-blobs, $s$ is just the identity; a blob diagram without any blobs is compatible with any open cover. Nevertheless, we'll begin introducing nomenclature at this point: for configuration $\beta$ of disjoint embedded balls in $M$ we'll associate a one parameter family of homeomorphisms $\phi_\beta : \Delta^1 \to \Homeo(M)$ (here $\Delta^m$ is the standard simplex $\setc{\mathbf{x} \in \Real^{m+1}}{\sum_{i=0}^m x_i = 1}$). For $0$-blobs, where $\beta = \eset$, all these homeomorphisms are just the identity.

On a $1$-blob $b$, with ball $\beta$, $s$ is defined as the sum of two terms. Essentially, the first term `makes $\beta$ small', while the other term `gets the boundary right'. First, pick a one-parameter family $\phi_\beta : \Delta^1 \to \Homeo(M)$ of homeomorphisms, so $\phi_\beta(1,0)$ is the identity and $\phi_\beta(0,1)$ makes the ball $\beta$ small. Next, pick a two-parameter family $\phi_{\eset \prec \beta} : \Delta^2 \to \Homeo(M)$ so that $\phi_{\eset \prec \beta}(0,x_1,x_2)$ makes the ball $\beta$ small for all $x_1+x_2=1$, while $\phi_{\eset \prec \beta}(x_0,0,x_2) = \phi_\eset(x_0,x_2)$ and $\phi_{\eset \prec \beta}(x_0,x_1,0) = \phi_\beta(x_0,x_1)$. (It's perhaps not obvious that this is even possible --- see Lemma \ref{lem:extend-small-homeomorphisms} below.) We now define $s$ by
$$s(b) = \restrict{\phi_\beta}{x_0=0}(b) + \restrict{\phi_{\eset \prec \beta}}{x_0=0}(\bdy b).$$
Here, $\restrict{\phi_\beta}{x_0=0} = \phi_\beta(0,1)$ is just a homeomorphism, which we apply to $b$, while $\restrict{\phi_{\eset \prec \beta}}{x_0=0}$ is a one parameter family of homeomorphisms which acts on the $0$-blob $\bdy b$ to give a $1$-blob. We now check that $s$, as defined so far, is a chain map, calculating
\begin{align*}
\bdy (s(b)) & = \restrict{\phi_\beta}{x_0=0}(\bdy b) + (\bdy \restrict{\phi_{\eset \prec \beta}}{x_0=0})(\bdy b) \\
		 & = \restrict{\phi_\beta}{x_0=0}(\bdy b) + \restrict{\phi_\eset}{x_0=0}(\bdy b) - \restrict{\phi_\beta}{x_0=0}(\bdy b) \\
		 & = \restrict{\phi_\eset}{x_0=0}(\bdy b) \\
		 & = s(\bdy b)
\end{align*}
Next, we compute the compositions $s \circ i$ and $i \circ s$. If we start with a small $1$-blob diagram $b$, first include it up to the full blob complex then apply $s$, we get exactly back to $b$, at least assuming we adopt the convention that for any ball $\beta$ which is already small, we choose the families of homeomorphisms $\phi_\beta$ and $\phi_{\eset \prec \beta}$ to always be the identity. In the other direction, $i \circ s$, we will need to construct a homotopy $h:\bc_*(M) \to \bc_{*+1}(M)$ for $*=0$ or $1$. This is defined by $h(b) = \phi_\eset(b)$ when $b$ is a $0$-blob (here $\phi_\eset$ is a one parameter family of homeomorphisms, so this is a $1$-blob), and $h(b) = \phi_\beta(b) + \phi_{\eset \prec \beta}(\bdy b)$ when $b$ is a $1$-blob (here $\beta$ is the ball in $b$, and the first term is the action of a one parameter family of homeomorphisms on a $1$-blob, and the second term is the action of a two parameter family of homeomorphisms on a $0$-blob, so both are $2$-blobs).
\begin{align*}
(\bdy h+h \bdy)(b) & = \bdy (\phi_{\beta}(b) + \phi_{\eset \prec \beta}{\bdy b}) + \phi_\eset(\bdy b)  \\
	& =  \restrict{\phi_\beta}{x_0=0}(b) - \restrict{\phi_\beta}{x_1=0}(b) - \phi_\beta(\bdy b) + (\bdy \phi_{\eset \prec \beta})(\bdy b) + \phi_\eset(\bdy b) \\
	& =  \restrict{\phi_\beta}{x_0=0}(b) - b - \phi_\beta(\bdy b) + \restrict{\phi_{\eset \prec \beta}}{x_0=0}(\bdy b) -  \phi_\eset(\bdy b) + \phi_\beta(\bdy b) + \phi_\eset(\bdy b) \\
	& = \restrict{\phi_\beta}{x_0=0}(b) - b + \restrict{\phi_{\eset \prec \beta}}{x_0=0}(\bdy b) \\
	& = (i \circ s - \id)(b)
\end{align*}


Given a blob diagram $b \in \bc_k(M)$, denote by $b_\cS$ for $\cS \subset \{1, \ldots, k\}$ the blob diagram obtained by erasing the corresponding blobs. In particular, $b_\eset = b$, $b_{\{1,\ldots,k\}} \in \bc_0(M)$, and $d b_\cS = \sum_{\cS' = \cS'\sqcup\{i\}} \pm  b_{\cS'}$.
Similarly, for a disjoint embedding of $k$ balls $\beta$ (that is, a blob diagram but without the labels on regions), $\beta_\cS$ denotes the result of erasing a subset of blobs. We'll write $\beta' \prec \beta$ if $\beta' = \beta_\cS$ for some $\cS$. Finally, for finite sequences, we'll write $i \prec i'$ if $i$ is subsequence of $i'$, and $i \prec_1 i$ if the lengths differ by exactly 1.

Next, we'll choose a `shrinking system' for $\cU$, namely for each increasing sequence of blob configurations
$\beta_0 \prec \beta_1 \prec \cdots \prec \beta_m$, an $m$ parameter family of diffeomorphisms
$\phi_{\beta_0 \prec \cdots \prec \beta_m} : \Delta^m \to \Diff{M}$ (here $\Delta^m$ is the standard simplex $\setc{\mathbf{x} \in \Real^{m+1}}{\sum_i x_i = 1}$), such that
\begin{itemize}
\item if $\beta$ is the empty configuration, $\phi_{\beta}(1) = \id_M$,
\item if $\beta$ is a single configuration of blobs, then $\phi_{\beta}(1)(\beta)$ (which is another configuration of blobs: $\phi_{\beta}(1)$ is a diffeomorphism of $M$) is subordinate to $\cU$,
\item (more generally) for any $x$ with $x_0 = 0$, $\phi_{\beta_0 \prec \cdots \prec \beta_m}(x)(\beta)$ is subordinate to $\cU$, and
\item for each $i = 1, \ldots, m$,
\begin{align*}
\phi_{\beta_0 \prec \cdots \prec \beta_m}(x_0, \ldots, x_{i-1},0,x_{i+1},\ldots,x_m) & = \phi_{\beta_0 \prec \cdots \beta_{i-1} \prec \beta_{i+1} \prec \beta_m}(x_0,\ldots, x_{i-1},x_{i+1},\ldots,x_m).
\end{align*}
\end{itemize}
It's not immediately obvious that it's possible to make such choices, but it follows readily from the following Lemma.

When $\beta$ is a collection of disjoint embedded balls in $M$, we say that a homeomorphism of $M$ `makes $\beta$ small' if the image of each ball in $\beta$ under the homeomorphism is contained in some open set of $\cU$.

\begin{lem}
\label{lem:extend-small-homeomorphisms}
Fix a collection of disjoint embedded balls $\beta$ in $M$. Suppose we have a map $f :  X \to \Homeo(M)$ on some compact $X$ such that for each $x \in \bdy X$, $f(x)$ makes $\beta$ small. Then we can extend $f$ to a map $\tilde{f} : X \times [0,1] \to \Homeo(M)$ so that $\tilde{f}(x,0) = f(x)$ and for every $x \in \bdy X \times [0,1] \cup X \times \{1\}$, $\tilde{f}(x)$ makes $\beta$ small.
\end{lem}
\begin{proof}
Fix a metric on $M$, and pick $\epsilon > 0$ so every $\epsilon$ ball in $M$ is contained in some open set of $\cU$. First construct a family of homeomorphisms $g_s : M \to M$, $s \in [1,\infty)$ so $g_1$ is the identity, and $g_s(\beta_i) \subset \beta_i$ and $\rad g_s(\beta_i) \leq \frac{1}{s} \rad \beta_i$ for each ball $\beta_i$. 
There is some $K$ which uniformly bounds the expansion factors of all the homeomorphisms $f(x)$, that is $d(f(x)(a), f(x)(b)) < K d(a,b)$ for all $x \in X, a,b \in M$. Write $S=\epsilon^{-1} K \max_i \{\rad \beta_i\}$ (note that is $S<1$, we can just take $S=1$, as already $f(x)$ makes $\beta$ small for all $x$). Now define $\tilde{f}(t, x) = f(x) \compose g_{(S-1)t+1}$.

If $x \in \bdy X$, then $g_{(S-1)t+1}(\beta_i) \subset \beta_i$, and by hypothesis $f(x)$ makes $\beta_i$ small, so $\tilde{f}(t, x)$ makes $\beta$ small for all $t \in [0,1]$. Alternatively, $\rad g_S(\beta_i) \leq \frac{1}{S} \rad \beta_i \leq \frac{\epsilon}{K}$, so $\rad \tilde{f}(1,x)(\beta_i) \leq \epsilon$, and so $\tilde{f}(1,x)$ makes $\beta$ small for all $x \in X$.
\end{proof}

We'll need a stronger version of Property \ref{property:evaluation}; while the evaluation map $ev: \CD{M} \tensor \bc_*(M) \to \bc_*(M)$ is not unique, it has an up-to-homotopy representative (satisfying the usual conditions) which restricts to become a chain map $ev: \CD{M} \tensor \bc^{\cU}_*(M) \to \bc^{\cU}_*(M)$. The proof is straightforward: when deforming the family of diffeomorphisms to shrink its supports to a union of open sets, do so such that those open sets are subordinate to the cover.

Now define a map $s: \bc_*(M) \to \bc^{\cU}_*(M)$, and then a homotopy $h:\bc_*(M) \to \bc_{*+1}(M)$ so that $dh+hd=i\circ s$. The map $s: \bc_0(M) \to \bc^{\cU}_0(M)$ is just the identity; blob diagrams without blobs are automatically compatible with any cover. Given a blob diagram $b$, we'll abuse notation and write $\phi_b$ to mean $\phi_\beta$ for the blob configuration $\beta$ underlying $b$. We have
$$s(b) = \sum_{i} ev(\restrict{\phi_{i(b)}}{x_0 = 0} \tensor b_i)$$
where the sum is over sequences $i=(i_1,\ldots,i_m)$ in $\{1,\ldots,k\}$, with $0\leq m < k$, $i(b)$ denotes the increasing sequence of blob configurations
$$\beta_{(i_1,\ldots,i_m)} \prec \beta_{(i_2,\ldots,i_m)} \prec \cdots \prec \beta_{()},$$
and, as usual, $i(b)$ denotes $b$ with blobs $i_1, \ldots i_m$ erased. We'll also write
$$s(b) = \sum_{m=0}^{k-1} \sum_{\norm{i}=m} ev(\restrict{\phi_{i(b)}}{x_0 = 0} \tensor b_i),$$
arranging the sum according to the length $\norm{i}$ of $i$.


We need to check that $s$ is a chain map, and that the image of $s$ in fact lies in $\bc^{\cU}_*(M)$. \todo{} Calculate
\begin{align*}
\bdy(s(b)) & = \sum_{m=0}^{k-1} \sum_{\norm{i}=m} \ev\left(\bdy(\restrict{\phi_{i(b)}}{x_0 = 0})\tensor b_i\right) + (-1)^m \ev\left(\restrict{\phi_{i(b)}}{x_0 = 0} \tensor \bdy b_i\right) \\
                & = \sum_{m=0}^{k-1} \sum_{\norm{i}=m} \ev\left(\sum_{i' \prec_1 i} \pm \restrict{\phi_{i'(b)}}{x_0 = 0})\tensor b_i\right) + (-1)^m \ev\left(\restrict{\phi_{i(b)}}{x_0 = 0}\tensor \sum_{i \prec_1 i'} \pm b_{i'}\right) \\
\intertext{and telescoping the sum}
		& = \sum_{m=0}^{k-2} \left(\sum_{\norm{i}=m}  (-1)^m \ev\left(\restrict{\phi_{i(b)}}{x_0 = 0} \tensor \sum_{i \prec_1 i'} \pm b_{i'}\right) \right) + \left(\sum_{\norm{i}=m+1} \ev\left(\sum_{i' \prec_1 i} \pm \restrict{\phi_{i'(b)}}{x_0 = 0} \tensor b_i\right) \right) + \\
		& \qquad + (-1)^{k-1} \sum_{\norm{i}=k-1} \ev\left(\restrict{\phi_{i(b)}}{x_0 = 0} \tensor \sum_{i \prec_1 i'} \pm b_{i'}\right) \\
		& = (-1)^{k-1} \sum_{\norm{i}=k-1} \ev\left(\restrict{\phi_{i(b)}}{x_0 = 0} \tensor \sum_{i \prec_1 i'} \pm b_{i'}\right)
\end{align*}

Next, we define the homotopy $h:\bc_*(M) \to \bc_{*+1}(M)$ by
$$h(b) = \sum_{i} ev(\phi_{i(b)}, b_i).$$
\todo{and check that it's the right one...}
\end{proof}


\begin{proof}[Proof of Theorem \ref{product_thm}]
We will use the concrete description of the colimit from Subsection \ref{ss:ncat_fields}.

First we define a map 
\[
	\psi: \bc_*^\cF(Y) \to \bc_*^C(Y\times F) .
\]
In filtration degree 0 we just glue together the various blob diagrams on $X\times F$
(where $X$ is a component of a permissible decomposition of $Y$) to get a blob diagram on
$Y\times F$.
In filtration degrees 1 and higher we define the map to be zero.
It is easy to check that this is a chain map.

Next we define a map 
\[
	\phi: \bc_*^C(Y\times F) \to \bc_*^\cF(Y) .
\]
Actually, we will define it on the homotopy equivalent subcomplex
$\cS_* \sub \bc_*^C(Y\times F)$ generated by blob diagrams which are small with 
respect to some open cover
of $Y\times F$.
\nn{need reference to small blob lemma}
We will have to show eventually that this is independent (up to homotopy) of the choice of cover.
Also, for a fixed choice of cover we will only be able to define the map for blob degree less than
some bound, but this bound goes to infinity as the cover become finer.

Given a decomposition $K$ of $Y$ into $k$-balls $X_i$, let $K\times F$ denote the corresponding
decomposition of $Y\times F$ into the pieces $X_i\times F$.

%We will define $\phi$ inductively, starting at blob degree 0.
%Given a 0-blob diagram $x$ on $Y\times F$, we can choose a decomposition $K$ of $Y$
%such that $x$ is splittable with respect to $K\times F$.
%This defines a filtration degree 0 element of $\bc_*^\cF(Y)$

We will define $\phi$ using a variant of the method of acyclic models.
Let $a\in \cS_m$ be a blob diagram on $Y\times F$.
For $m$ sufficiently small there exists a decomposition $K$ of $Y$ into $k$-balls such that the
codimension 1 cells of $K\times F$ miss the blobs of $a$, and more generally such that $a$ is splittable along (the codimension-1 part of) $K\times F$.
Let $D(a)$ denote the subcomplex of $\bc_*^\cF(Y)$ generated by all $(a, \bar{K})$
such that each $K_i$ has the aforementioned splittable property
(see Subsection \ref{ss:ncat_fields}).
\nn{need to define $D(a)$ more clearly; also includes $(b_j, \bar{K})$ where
$\bd(a) = \sum b_j$.}
(By $(a, \bar{K})$ we really mean $(a^\sharp, \bar{K})$, where $a^\sharp$ is 
$a$ split according to $K_0\times F$.
To simplify notation we will just write plain $a$ instead of $a^\sharp$.)
Roughly speaking, $D(a)$ consists of filtration degree 0 stuff which glues up to give
$a$, filtration degree 1 stuff which makes all of the filtration degree 0 stuff homologous, 
filtration degree 2 stuff which kills the homology created by the 
filtration degree 1 stuff, and so on.
More formally,
 
\begin{lemma}
$D(a)$ is acyclic.
\end{lemma}

\begin{proof}
We will prove acyclicity in the first couple of degrees, and \nn{in this draft, at least}
leave the general case to the reader.

Let $K$ and $K'$ be two decompositions of $Y$ compatible with $a$.
We want to show that $(a, K)$ and $(a, K')$ are homologous via filtration degree 1 stuff.
\nn{need to say this better; these two chains don't have the same boundary.}
We might hope that $K$ and $K'$ have a common refinement, but this is not necessarily
the case.
(Consider the $x$-axis and the graph of $y = x^2\sin(1/x)$ in $\r^2$.)
However, we {\it can} find another decomposition $L$ such that $L$ shares common
refinements with both $K$ and $K'$.
Let $KL$ and $K'L$ denote these two refinements.
Then filtration degree 1 chains associated to the four anti-refinemnts
$KL\to K$, $KL\to L$, $K'L\to L$ and $K'L\to K'$
give the desired chain connecting $(a, K)$ and $(a, K')$
(see Figure \ref{zzz4}).

\begin{figure}[!ht]
\begin{equation*}
\begin{tikzpicture}
\foreach \x/\label in {-3/K, 0/L, 3/K'} {
	\node(\label) at (\x,0) {$\label$};
}
\foreach \x/\la/\lb in {-1.5/K/L, 1.5/K'/L} {
	\node(\la \lb) at (\x,-1.5) {$\la \lb$};
	\draw[->] (\la \lb) -- (\la);
	\draw[->] (\la \lb) -- (\lb); 
}
\end{tikzpicture}
\end{equation*}
\caption{Connecting $K$ and $K'$ via $L$}
\label{zzz4}
\end{figure}

Consider a different choice of decomposition $L'$ in place of $L$ above.
This leads to a cycle consisting of filtration degree 1 stuff.
We want to show that this cycle bounds a chain of filtration degree 2 stuff.
Choose a decomposition $M$ which has common refinements with each of 
$K$, $KL$, $L$, $K'L$, $K'$, $K'L'$, $L'$ and $KL'$.
\nn{need to also require that $KLM$ antirefines to $KM$, etc.}
Then we have a filtration degree 2 chain, as shown in Figure \ref{zzz5}, which does the trick.
(Each small triangle in Figure \ref{zzz5} can be filled with a filtration degree 2 chain.)

\begin{figure}[!ht]
%\begin{equation*}
%\mathfig{1.0}{tempkw/zz5}
%\end{equation*}
\begin{equation*}
\begin{tikzpicture}
\node(M) at (0,0) {$M$};
\foreach \angle/\label in {0/K', 45/K'L, 90/L, 135/KL, 180/K, 225/KL', 270/L', 315/K'L'} {
	\node(\label) at (\angle:4) {$\label$};
}
\foreach \label in {K', L, K, L'} {
	\node(\label M) at ($(M)!0.6!(\label)$) {$\label M$};
	\draw[->] (\label M)--(M);
	\draw[->] (\label M)--(\label);
}
\foreach \k in {K, K'} {
	\foreach \l in {L, L'} {
		\node(\k \l M) at (intersection cs: first line={(\k M)--(\l)}, second line={(\l M)--(\k)}) {$\k \l M$};
		\draw[->] (\k \l M)--(M);
		\draw[->] (\k \l M)--(\k \l );
		\draw[->] (\k \l M)--(\k M);
		\draw[->] (\k \l M)--(\l);
		\draw[->] (\k \l M)--(\l M);
		\draw[->] (\k \l M)--(\k);
	}
}
\draw[->] (K'L') to[bend right=10] (K');
\draw[->] (K'L') to[bend left=10] (L');
\draw[->] (KL') to[bend left=10] (K);
\draw[->] (KL') to[bend right=10] (L');
\draw[->] (K'L) to[bend left=10] (K');
\draw[->] (K'L) to[bend right=10] (L);
\draw[->] (KL) to[bend right=10] (K);
\draw[->] (KL) to[bend left=10] (L);
\end{tikzpicture}
\end{equation*}
\caption{Filling in $K$-$KL$-$L$-$K'L$-$K'$-$K'L'$-$L'$-$KL'$-$K$}
\label{zzz5}
\end{figure}

Continuing in this way we see that $D(a)$ is acyclic.
\end{proof}

We are now in a position to apply the method of acyclic models to get a map
$\phi:\cS_* \to \bc_*^\cF(Y)$.
This map is defined in sufficiently low degrees, sends a blob diagram $a$ to $D(a)$, 
and is well-defined up to (iterated) homotopy.

The subcomplex $\cS_* \subset \bc_*^C(Y\times F)$ depends on choice of cover of $Y\times F$.
If we refine that cover, we get a complex $\cS'_* \subset \cS_*$
and a map $\phi':\cS'_* \to \bc_*^\cF(Y)$.
$\phi'$ is defined only on homological degrees below some bound, but this bound is higher than 
the corresponding bound for $\phi$.
We must show that $\phi$ and $\phi'$ agree, up to homotopy,
on the intersection of the subcomplexes on which they are defined.
This is clear, since the acyclic subcomplexes $D(a)$ above used in the definition of 
$\phi$ and $\phi'$ do not depend on the choice of cover.

\nn{need to say (and justify) that we now have a map $\phi$ indep of choice of cover}

We now show that $\phi\circ\psi$ and $\psi\circ\phi$ are homotopic to the identity.

$\psi\circ\phi$ is the identity.  $\phi$ takes a blob diagram $a$ and chops it into pieces 
according to some decomposition $K$ of $Y$.
$\psi$ glues those pieces back together, yielding the same $a$ we started with.

$\phi\circ\psi$ is the identity up to homotopy by another MoAM argument...

This concludes the proof of Theorem \ref{product_thm}.
\nn{at least I think it does; it's pretty rough at this point.}
\end{proof}

\nn{need to say something about dim $< n$ above}

\medskip

\begin{cor}
The new-fangled and old-fashioned blob complexes are homotopic.
\end{cor}
\begin{proof}
Apply Theorem \ref{product_thm} with the fiber $F$ equal to a point.
\end{proof}

\medskip

Next we prove a gluing theorem.
Let $X$ be a closed $k$-manifold with a splitting $X = X'_1\cup_Y X'_2$.
We will need an explicit collar on $Y$, so rewrite this as
$X = X_1\cup (Y\times J) \cup X_2$.
\nn{need figure}
Given this data we have: \nn{need refs to above for these}
\begin{itemize}
\item An $A_\infty$ $n{-}k$-category $\bc(X)$, which assigns to an $m$-ball
$D$ fields on $D\times X$ (for $m+k < n$) or the blob complex $\bc_*(D\times X; c)$
(for $m+k = n$). \nn{need to explain $c$}.
\item An $A_\infty$ $n{-}k{+}1$-category $\bc(Y)$, defined similarly.
\item Two $\bc(Y)$ modules $\bc(X_1)$ and $\bc(X_2)$, which assign to a marked
$m$-ball $(D, H)$ either fields on $(D\times Y) \cup (H\times X_i)$ (if $m+k < n$)
or the blob complex $\bc_*((D\times Y) \cup (H\times X_i))$ (if $m+k = n$).
\end{itemize}

\begin{thm}
$\bc(X) \cong \bc(X_1) \otimes_{\bc(Y), J} \bc(X_2)$.
\end{thm}

\begin{proof}
The proof is similar to that of Theorem \ref{product_thm}.
\nn{need to say something about dimensions less than $n$, 
but for now concentrate on top dimension.}

Let $\cT$ denote the $n{-}k$-category $\bc(X_1) \otimes_{\bc(Y), J} \bc(X_2)$.
Let $D$ be an $n{-}k$-ball.
There is an obvious map from $\cT(D)$ to $\bc_*(D\times X)$.
To get a map in the other direction, we replace $\bc_*(D\times X)$ with a subcomplex
$\cS_*$ which is adapted to a fine open cover of $D\times X$.
For sufficiently small $j$ (depending on the cover), we can find, for each $j$-blob diagram $b$
on $D\times X$, a decomposition of $J$ such that $b$ splits on the corresponding
decomposition of $D\times X$.
The proof that these two maps are inverse to each other is the same as in
Theorem \ref{product_thm}.
\end{proof}


\medskip

The next theorem shows how to reconstruct a mapping space from local data.
Let $T$ be a topological space, let $M$ be an $n$-manifold, 
and recall the $A_\infty$ $n$-category $\pi^\infty_{\leq n}(T)$ 
of Example \ref{ex:chains-of-maps-to-a-space}.
Think of $\pi^\infty_{\leq n}(T)$ as encoding everything you would ever
want to know about spaces of maps of $k$-balls into $T$ ($k\le n$).
To simplify notation, let $\cT = \pi^\infty_{\leq n}(T)$.

\begin{thm} \label{thm:map-recon}
$\cB^\cT(M) \simeq C_*(\Maps(M\to T))$.
\end{thm}
\begin{proof}
We begin by constructing chain map $g: \cB^\cT(M) \to C_*(\Maps(M\to T))$.
We then use \ref{extension_lemma_b} to show that $g$ induces isomorphisms on homology.

Recall that the homotopy colimit $\cB^\cT(M)$ is constructed out of a series of
$j$-fold mapping cylinders, $j \ge 0$.
So, as an abelian group (but not as a chain complex), 
\[
	\cB^\cT(M) = \bigoplus_{j\ge 0} C^j,
\]
where $C^j$ denotes the new chains introduced by the $j$-fold mapping cylinders.

Recall that $C^0$ is a direct sum of chain complexes with the summands indexed by
decompositions of $M$ which have their $n{-}1$-skeletons labeled by $n{-}1$-morphisms
of $\cT$.
Since $\cT = \pi^\infty_{\leq n}(T)$, this means that the summands are indexed by pairs
$(K, \vphi)$, where $K$ is a decomposition of $M$ and $\vphi$ is a continuous
maps from the $n{-}1$-skeleton of $K$ to $T$.
The summand indexed by $(K, \vphi)$ is
\[
	\bigotimes_b D_*(b, \vphi),
\]
where $b$ runs through the $n$-cells of $K$ and $D_*(b, \vphi)$ denotes
chains of maps from $b$ to $T$ compatible with $\vphi$.
We can take the product of these chains of maps to get a chains of maps from
all of $M$ to $K$.
This defines $g$ on $C^0$.

We define $g(C^j) = 0$ for $j > 0$.
It is not hard to see that this defines a chain map from 
$\cB^\cT(M)$ to $C_*(\Maps(M\to T))$.


%%%%%%%%%%%%%%%%%
\noop{
Next we show that $g$ induces a surjection on homology.
Fix $k > 0$ and choose an open cover $\cU$ of $M$ fine enough so that the union 
of any $k$ open sets of $\cU$ is contained in a disjoint union of balls in $M$.
\nn{maybe should refer to elsewhere in this paper where we made a very similar argument}
Let $S_*$ be the subcomplex of $C_*(\Maps(M\to T))$ generated by chains adapted to $\cU$.
It follows from Lemma \ref{extension_lemma_b} that $C_*(\Maps(M\to T))$
retracts onto $S_*$.

Let $S_{\le k}$ denote the chains of $S_*$ of degree less than or equal to $k$.
We claim that $S_{\le k}$ lies in the image of $g$.
Let $c$ be a generator of $S_{\le k}$ --- that is, a $j$-parameter family of maps $M\to T$,
$j \le k$.
We chose $\cU$ fine enough so that the support of $c$ is contained in a disjoint union of balls
in $M$.
It follow that we can choose a decomposition $K$ of $M$ so that the support of $c$ is 
disjoint from the $n{-}1$-skeleton of $K$.
It is now easy to see that $c$ is in the image of $g$.

Next we show that $g$ is injective on homology.
}



\nn{...}



\end{proof}

\nn{maybe should also mention version where we enrich over
spaces rather than chain complexes; should comment on Lurie's (and others') similar result
for the $E_\infty$ case, and mention that our version does not require 
any connectivity assumptions}

\medskip
\hrule
\medskip

\nn{to be continued...}
\medskip
\nn{still to do: fiber bundles, general maps}

\todo{}
Various citations we might want to make:
\begin{itemize}
\item \cite{MR2061854} McClure and Smith's review article
\item \cite{MR0420610} May, (inter alia, definition of $E_\infty$ operad)
\item \cite{MR0236922,MR0420609} Boardman and Vogt
\item \cite{MR1256989} definition of framed little-discs operad
\end{itemize}

We now turn to establishing the gluing formula for blob homology, restated from Property \ref{property:gluing} in the Introduction
\begin{itemize}
%\mbox{}% <-- gets the indenting right
\item For any $(n-1)$-manifold $Y$, the blob homology of $Y \times I$ is
naturally an $A_\infty$ category. % We'll write $\bc_*(Y)$ for $\bc_*(Y \times I)$ below.

\item For any $n$-manifold $X$, with $Y$ a codimension $0$-submanifold of its boundary, the blob homology of $X$ is naturally an
$A_\infty$ module for $\bc_*(Y \times I)$.

\item For any $n$-manifold $X$, with $Y \cup Y^{\text{op}}$ a codimension
$0$-submanifold of its boundary, the blob homology of $X'$, obtained from
$X$ by gluing along $Y$, is the $A_\infty$ self-tensor product of
$\bc_*(X)$ as an $\bc_*(Y \times I)$-bimodule.
\begin{equation*}
\bc_*(X') \iso \bc_*(X) \Tensor^{A_\infty}_{\mathclap{\bc_*(Y \times I)}} \!\!\!\!\!\!\xymatrix{ \ar@(ru,rd)@<-1ex>[]}
\end{equation*}
\end{itemize}



%!TEX root = ../blob1.tex

\section{Higher-dimensional Deligne conjecture}
\label{sec:deligne}
In this section we 
sketch
\nn{revisit ``sketch" after proof is done} 
the proof of a higher dimensional version of the Deligne conjecture
about the action of the little disks operad on Hochschild cohomology.
The first several paragraphs lead up to a precise statement of the result
(Proposition \ref{prop:deligne} below).
Then we sketch the proof.

\nn{Does this generalization encompass Kontsevich's proposed generalization from \cite[\S2.5]{MR1718044}, 
that (I think...) the Hochschild homology of an $E_n$ algebra is an $E_{n+1}$ algebra? -S}

%from http://www.ams.org/mathscinet-getitem?mr=1805894
%Different versions of the geometric counterpart of Deligne's conjecture have been proven by Tamarkin [``Formality of chain operad of small squares'', preprint, http://arXiv.org/abs/math.QA/9809164], the reviewer [in Conf�rence Mosh� Flato 1999, Vol. II (Dijon), 307--331, Kluwer Acad. Publ., Dordrecht, 2000; MR1805923 (2002d:55009)], and J. E. McClure and J. H. Smith [``A solution of Deligne's conjecture'', preprint, http://arXiv.org/abs/math.QA/9910126] (see also a later simplified version [J. E. McClure and J. H. Smith, ``Multivariable cochain operations and little $n$-cubes'', preprint, http://arXiv.org/abs/math.QA/0106024]). The paper under review gives another proof of Deligne's conjecture, which, as the authors indicate, may be generalized to a proof of a higher-dimensional generalization of Deligne's conjecture, suggested in [M. Kontsevich, Lett. Math. Phys. 48 (1999), no. 1, 35--72; MR1718044 (2000j:53119)]. 


The usual Deligne conjecture (proved variously in \cite{MR1805894, MR2064592, hep-th/9403055, MR1805923} gives a map
\[
	C_*(LD_k)\otimes \overbrace{Hoch^*(C, C)\otimes\cdots\otimes Hoch^*(C, C)}^{\text{$k$ copies}}
			\to  Hoch^*(C, C) .
\]
Here $LD_k$ is the $k$-th space of the little disks operad, and $Hoch^*(C, C)$ denotes Hochschild
cochains.
The little disks operad is homotopy equivalent to the 
(transversely orient) fat graph operad
\nn{need ref, or say more precisely what we mean}, 
and Hochschild cochains are homotopy equivalent to $A_\infty$ endomorphisms
of the blob complex of the interval, thought of as a bimodule for itself.
\nn{need to make sure we prove this above}.
So the 1-dimensional Deligne conjecture can be restated as
\[
	C_*(FG_k)\otimes \hom(\bc^C_*(I), \bc^C_*(I))\otimes\cdots
	\otimes \hom(\bc^C_*(I), \bc^C_*(I))
	  \to  \hom(\bc^C_*(I), \bc^C_*(I)) .
\]
See Figure \ref{delfig1}.
\begin{figure}[t]
$$\mathfig{.9}{deligne/intervals}$$
\caption{A fat graph}\label{delfig1}\end{figure}
We emphasize that in $\hom(\bc^C_*(I), \bc^C_*(I))$ we are thinking of $\bc^C_*(I)$ as a module
for the $A_\infty$ 1-category associated to $\bd I$, and $\hom$ means the 
morphisms of such modules as defined in 
Subsection \ref{ss:module-morphisms}.

We can think of a fat graph as encoding a sequence of surgeries, starting at the bottommost interval
of Figure \ref{delfig1} and ending at the topmost interval.
The surgeries correspond to the $k$ bigon-shaped ``holes" in the fat graph.
We remove the bottom interval of the bigon and replace it with the top interval.
To convert this topological operation to an algebraic one, we need, for each hole, an element of
$\hom(\bc^C_*(I_{\text{bottom}}), \bc^C_*(I_{\text{top}}))$.
So for each fixed fat graph we have a map
\[
	 \hom(\bc^C_*(I), \bc^C_*(I))\otimes\cdots
	\otimes \hom(\bc^C_*(I), \bc^C_*(I))  \to  \hom(\bc^C_*(I), \bc^C_*(I)) .
\]
If we deform the fat graph, corresponding to a 1-chain in $C_*(FG_k)$, we get a homotopy
between the maps associated to the endpoints of the 1-chain.
Similarly, higher-dimensional chains in $C_*(FG_k)$ give rise to higher homotopies.

It should now be clear how to generalize this to higher dimensions.
In the sequence-of-surgeries description above, we never used the fact that the manifolds
involved were 1-dimensional.
Thus we can define an $n$-dimensional fat graph to be a sequence of general surgeries
on an $n$-manifold (Figure \ref{delfig2}).
\begin{figure}[t]
$$\mathfig{.9}{deligne/manifolds}$$
\caption{An $n$-dimensional fat graph}\label{delfig2}
\end{figure}

More specifically, an $n$-dimensional fat graph ($n$-FG for short) consists of:
\begin{itemize}
\item ``Upper" $n$-manifolds $M_0,\ldots,M_k$ and ``lower" $n$-manifolds $N_0,\ldots,N_k$,
with $\bd M_i = \bd N_i = E_i$ for all $i$.
We call $M_0$ and $N_0$ the outer boundary and the remaining $M_i$'s and $N_i$'s the inner
boundaries.
\item Additional manifolds $R_1,\ldots,R_{k}$, with $\bd R_i = E_0\cup \bd M_i = E_0\cup \bd N_i$.
%(By convention, $M_i = N_i = \emptyset$ if $i <1$ or $i>k$.)
\item Homeomorphisms 
\begin{eqnarray*}
	f_0: M_0 &\to& R_1\cup M_1 \\
	f_i: R_i\cup N_i &\to& R_{i+1}\cup M_{i+1}\;\; \mbox{for}\, 1\le i \le k-1 \\
	f_k: R_k\cup N_k &\to& N_0 .
\end{eqnarray*}
Each $f_i$ should be the identity restricted to $E_0$.
\end{itemize}
We can think of the above data as encoding the union of the mapping cylinders $C(f_0),\ldots,C(f_k)$,
with $C(f_i)$ glued to $C(f_{i+1})$ along $R_{i+1}$
(see Figure \ref{xdfig2}).
\begin{figure}[t]
$$\mathfig{.9}{deligne/mapping-cylinders}$$
\caption{An $n$-dimensional fat graph constructed from mapping cylinders}\label{xdfig2}
\end{figure}
The $n$-manifolds are the ``$n$-dimensional graph" and the $I$ direction of the mapping cylinders is the ``fat" part.
We regard two such fat graphs as the same if there is a homeomorphism between them which is the 
identity on the boundary and which preserves the 1-dimensional fibers coming from the mapping
cylinders.
More specifically, we impose the following two equivalence relations:
\begin{itemize}
\item If $g: R_i\to R'_i$ is a homeomorphism, we can replace
\begin{eqnarray*}
	(\ldots, R_{i-1}, R_i, R_{i+1}, \ldots) &\to& (\ldots, R_{i-1}, R'_i, R_{i+1}, \ldots) \\
	(\ldots, f_{i-1}, f_i, \ldots) &\to& (\ldots, g\circ f_{i-1}, f_i\circ g^{-1}, \ldots),
\end{eqnarray*}
leaving the $M_i$ and $N_i$ fixed.
(Keep in mind the case $R'_i = R_i$.)
(See Figure \ref{xdfig3}.)
\begin{figure}[t]
$$\mathfig{.4}{deligne/dfig3a} \to \mathfig{.4}{deligne/dfig3b} $$
\caption{Conjugating by a homeomorphism}\label{xdfig3}
\end{figure}
\item If $M_i = M'_i \du M''_i$ and $N_i = N'_i \du N''_i$ (and there is a
compatible disjoint union of $\bd M = \bd N$), we can replace
\begin{eqnarray*}
	(\ldots, M_{i-1}, M_i, M_{i+1}, \ldots) &\to& (\ldots, M_{i-1}, M'_i, M''_i, M_{i+1}, \ldots) \\
	(\ldots, N_{i-1}, N_i, N_{i+1}, \ldots) &\to& (\ldots, N_{i-1}, N'_i, N''_i, N_{i+1}, \ldots) \\
	(\ldots, R_{i-1}, R_i, R_{i+1}, \ldots) &\to& 
						(\ldots, R_{i-1}, R_i\cup M''_i, R_i\cup N'_i, R_{i+1}, \ldots) \\
	(\ldots, f_{i-1}, f_i, \ldots) &\to& (\ldots, f_{i-1}, \rm{id}, f_i, \ldots) .
\end{eqnarray*}
(See Figure \ref{xdfig1}.)
\begin{figure}[t]
$$\mathfig{.3}{deligne/dfig1a} \leftarrow \mathfig{.3}{deligne/dfig1b} \rightarrow \mathfig{.3}{deligne/dfig1c}$$
\caption{Changing the order of a surgery}\label{xdfig1}
\end{figure}
\end{itemize}

Note that the second equivalence increases the number of holes (or arity) by 1.
We can make a similar identification with the roles of $M'_i$ and $M''_i$ reversed.
In terms of the ``sequence of surgeries" picture, this says that if two successive surgeries
do not overlap, we can perform them in reverse order or simultaneously.

There is an operad structure on $n$-dimensional fat graphs, given by gluing the outer boundary
of one graph into one of the inner boundaries of another graph.
We leave it to the reader to work out a more precise statement in terms of $M_i$'s, $f_i$'s etc.

For fixed $\ol{M} = (M_0,\ldots,M_k)$ and $\ol{N} = (N_0,\ldots,N_k)$, we let
$FG^n_{\ol{M}\ol{N}}$ denote the topological space of all $n$-dimensional fat graphs as above.
(Note that in different parts of $FG^n_{\ol{M}\ol{N}}$ the $M_i$'s and $N_i$'s
are ordered differently.)
The topology comes from the spaces
\[
	\Homeo(M_0\to R_1\cup M_1)\times \Homeo(R_1\cup N_1\to R_2\cup M_2)\times
			\cdots\times \Homeo(R_k\cup N_k\to N_0)
\]
and the above equivalence relations.
We will denote the typical element of $FG^n_{\ol{M}\ol{N}}$ by $\ol{f} = (f_0,\ldots,f_k)$.

\medskip

%The little $n{+}1$-ball operad injects into the $n$-FG operad.
The $n$-FG operad contains the little $n{+}1$-balls operad.
Roughly speaking, given a configuration of $k$ little $n{+}1$-balls in the standard
$n{+}1$-ball, we fiber the complement of the balls by vertical intervals
and let $M_i$ [$N_i$] be the southern [northern] hemisphere of the $i$-th ball.
More precisely, let $x_1,\ldots,x_{n+1}$ be the coordinates of $\r^{n+1}$.
Let $z$ be a point of the $k$-th space of the little $n{+}1$-ball operad, with
little balls $D_1,\ldots,D_k$ inside the standard $n{+}1$-ball.
We assume the $D_i$'s are ordered according to the $x_{n+1}$ coordinate of their centers.
Let $\pi:\r^{n+1}\to \r^n$ be the projection corresponding to $x_{n+1}$.
Let $B\sub\r^n$ be the standard $n$-ball.
Let $M_i$ and $N_i$ be $B$ for all $i$.
Identify $\pi(D_i)$ with $B$ (a.k.a.\ $M_i$ or $N_i$) via translations and dilations (no rotations).
Let $R_i = B\setmin \pi(D_i)$.
Let $f_i = \rm{id}$ for all $i$.
We have now defined a map from the little $n{+}1$-ball operad to the $n$-FG operad,
with contractible fibers.
(The fibers correspond to moving the $D_i$'s in the $x_{n+1}$ 
direction without changing their ordering.)
\nn{issue: we've described this by varying the $R_i$'s, but above we emphasize varying the $f_i$'s.
does this need more explanation?}

Another familiar subspace of the $n$-FG operad is $\Homeo(M\to N)$, which corresponds to 
case $k=0$ (no holes).

\medskip

Let $\ol{f} \in FG^n_{\ol{M}\ol{N}}$.
Let $\hom(\bc_*(M_i), \bc_*(N_i))$ denote the morphisms from $\bc_*(M_i)$ to $\bc_*(N_i)$,
as modules of the $A_\infty$ 1-category $\bc_*(E_i)$.
We define a map
\[
	p(\ol{f}): \hom(\bc_*(M_1), \bc_*(N_1))\ot\cdots\ot\hom(\bc_*(M_k), \bc_*(N_k))
				\to \hom(\bc_*(M_0), \bc_*(N_0)) .
\]
Given $\alpha_i\in\hom(\bc_*(M_i), \bc_*(N_i))$, we define $p(\ol{f}$) to be the composition
\[
	\bc_*(M_0)  \stackrel{f_0}{\to} \bc_*(R_1\cup M_1)
				 \stackrel{\id\ot\alpha_1}{\to} \bc_*(R_1\cup N_1)
				 \stackrel{f_1}{\to} \bc_*(R_2\cup M_2) \stackrel{\id\ot\alpha_2}{\to}
				 \cdots  \stackrel{\id\ot\alpha_k}{\to} \bc_*(R_k\cup N_k)
				 \stackrel{f_k}{\to} \bc_*(N_0)
\]
(Recall that the maps $\id\ot\alpha_i$ were defined in \nn{need ref}.)
\nn{need to double check case where $\alpha_i$'s are not closed.}
It is easy to check that the above definition is compatible with the equivalence relations
and also the operad structure.
We can reinterpret the above as a chain map
\[
	p: C_0(FG^n_{\ol{M}\ol{N}})\ot \hom(\bc_*(M_1), \bc_*(N_1))\ot\cdots\ot\hom(\bc_*(M_k), \bc_*(N_k))
				\to \hom(\bc_*(M_0), \bc_*(N_0)) .
\]
The main result of this section is that this chain map extends to the full singular
chain complex $C_*(FG^n_{\ol{M}\ol{N}})$.

\begin{prop}
\label{prop:deligne}
There is a collection of chain maps
\[
	C_*(FG^n_{\overline{M}, \overline{N}})\otimes \hom(\bc_*(M_1), \bc_*(N_1))\otimes\cdots\otimes 
\hom(\bc_*(M_{k}), \bc_*(N_{k})) \to  \hom(\bc_*(M_0), \bc_*(N_0))
\]
which satisfy the operad compatibility conditions.
On $C_0(FG^n_{\ol{M}\ol{N}})$ this agrees with the chain map $p$ defined above.
When $k=0$, this coincides with the $C_*(\Homeo(M_0\to N_0))$ action of Section \ref{sec:evaluation}.
\end{prop}

If, in analogy to Hochschild cochains, we define elements of $\hom(M, N)$
to be ``blob cochains", we can summarize the above proposition by saying that the $n$-FG operad acts on
blob cochains.
As noted above, the $n$-FG operad contains the little $n{+}1$-ball operad, so this constitutes
a higher dimensional version of the Deligne conjecture for Hochschild cochains and the little 2-disk operad.

\begin{proof}
As described above, $FG^n_{\overline{M}, \overline{N}}$ is equal to the disjoint
union of products of homeomorphism spaces, modulo some relations.
By Proposition \ref{CHprop} and the Eilenberg-Zilber theorem, we have for each such product $P$
a chain map
\[
	C_*(P)\otimes \hom(\bc_*(M_1), \bc_*(N_1))\otimes\cdots\otimes 
\hom(\bc_*(M_{k}), \bc_*(N_{k})) \to  \hom(\bc_*(M_0), \bc_*(N_0)) .
\]
It suffices to show that the above maps are compatible with the relations whereby
$FG^n_{\overline{M}, \overline{N}}$ is constructed from the various $P$'s.
This in turn follows easily from the fact that
the actions of $C_*(\Homeo(\cdot\to\cdot))$ are local (compatible with gluing) and associative.

\nn{should add some detail to above}
\end{proof}

\nn{maybe point out that even for $n=1$ there's something new here.}


\appendix

%!TEX root = ../../blob1.tex

\section{The method of acyclic models}  \label{sec:moam}

In this section we recall the method of acyclic models for the reader's convenience. The material presented here is closely modeled on  \cite[Chapter 4]{MR0210112}.
We use this method throughout the paper (c.f. Lemma \ref{support-shrink}, Theorem \ref{thm:product}, Theorem \ref{thm:gluing} and Theorem \ref{thm:map-recon}), as it provides a very convenient way to show the existence of a chain map with desired properties, even when many non-canonical choices are required in order to construct one, and further to show the up-to-homotopy uniqueness of such maps.

Let $F_*$ and $G_*$ be chain complexes.
Assume $F_k$ has a basis $\{x_{kj}\}$
(that is, $F_*$ is free and we have specified a basis).
(In our applications, $\{x_{kj}\}$ will typically be singular $k$-simplices or 
$k$-blob diagrams.)
For each basis element $x_{kj}$ assume we have specified a ``target" $D^{kj}_*\sub G_*$.

We say that a chain map $f:F_*\to G_*$ is {\it compatible} with the above data (basis and targets)
if $f(x_{kj})\in D^{kj}_*$ for all $k$ and $j$.
Let $\Compat(D^\bullet_*)$ denote the subcomplex of maps from $F_*$ to $G_*$
such that the image of each higher homotopy applied to $x_{kj}$ lies in $D^{kj}_*$.

\begin{thm}[Acyclic models]  \label{moam-thm}
Suppose 
\begin{itemize}
\item $D^{k-1,l}_* \sub D^{kj}_*$ whenever $x_{k-1,l}$ occurs in $\bd x_{kj}$
with non-zero coefficient;
\item $D^{0j}_0$ is non-empty for all $j$; and
\item $D^{kj}_*$ is $(k{-}1)$-acyclic (i.e.\ $H_{k-1}(D^{kj}_*) = 0$) for all $k,j$ .
\end{itemize}
Then $\Compat(D^\bullet_*)$ is non-empty.
If, in addition,
\begin{itemize}
\item $D^{kj}_*$ is $m$-acyclic for $k\le m \le k+i$ and for all $k,j$,
\end{itemize}
then $\Compat(D^\bullet_*)$ is $i$-connected.
\end{thm}

\begin{proof}
(Sketch)
This is a standard result; see, for example, \cite[Chapter 4]{MR0210112}.

We will build a chain map $f\in \Compat(D^\bullet_*)$ inductively.
Choose $f(x_{0j})\in D^{0j}_0$ for all $j$
(possible since $D^{0j}_0$ is non-empty).
Choose $f(x_{1j})\in D^{1j}_1$ such that $\bd f(x_{1j}) = f(\bd x_{1j})$
(possible since $D^{0l}_* \sub D^{1j}_*$ for each $x_{0l}$ in $\bd x_{1j}$
and $D^{1j}_*$ is 0-acyclic).
Continue in this way, choosing $f(x_{kj})\in D^{kj}_k$ such that $\bd f(x_{kj}) = f(\bd x_{kj})$
We have now constructed $f\in \Compat(D^\bullet_*)$, proving the first claim of the theorem.

Now suppose that $D^{kj}_*$ is $k$-acyclic for all $k$ and $j$.
Let $f$ and $f'$ be two chain maps (0-chains) in $\Compat(D^\bullet_*)$.
Using a technique similar to above we can construct a homotopy (1-chain) in $\Compat(D^\bullet_*)$
between $f$ and $f'$.
Thus $\Compat(D^\bullet_*)$ is 0-connected.
Similarly, if $D^{kj}_*$ is $(k{+}i)$-acyclic then we can show that $\Compat(D^\bullet_*)$ is $i$-connected.
\end{proof}



%!TEX root = ../../blob1.tex

\section{Adapting families of maps to open covers}  \label{sec:localising}


Let $X$ and $T$ be topological spaces.
Let $\cU = \{U_\alpha\}$ be an open cover of $X$ which affords a partition of
unity $\{r_\alpha\}$.
(That is, $r_\alpha : X \to [0,1]$; $r_\alpha(x) = 0$ if $x\notin U_\alpha$;
for fixed $x$, $r_\alpha(x) \ne 0$ for only finitely many $\alpha$; and $\sum_\alpha r_\alpha = 1$.)

Let
\[
	CM_*(X, T) \deq C_*(\Maps(X\to T)) ,
\]
the singular chains on the space of continuous maps from $X$ to $T$.
$CM_k(X, T)$ is generated by continuous maps
\[
	f: P\times X \to T ,
\]
where $P$ is some linear polyhedron in $\r^k$.
Recall that $f$ is {\it supported} on $S\sub X$ if $f(p, x)$ does not depend on $p$ when
$x \notin S$, and that $f$ is {\it adapted} to $\cU$ if 
$f$ is supported on the union of at most $k$ of the $U_\alpha$'s.
A chain $c \in CM_*(X, T)$ is adapted to $\cU$ if it is a linear combination of 
generators which are adapted.

\begin{lemma} \label{basic_adaptation_lemma}
The $f: P\times X \to T$, as above.
The there exists
\[
	F: I \times P\times X \to T
\]
such that
\begin{enumerate}
\item $F(0, \cdot, \cdot) = f$ .
\item We can decompose $P = \cup_i D_i$ so that
the restrictions $F(1, \cdot, \cdot) : D_i\times X\to T$ are all adapted to $\cU$.
\item If $f$ restricted to $Q\sub P$ has support $S\sub X$, then the restriction
$F: (I\times Q)\times X\to T$ also has support $S$.
\item If for all $p\in P$ we have $f(p, \cdot):X\to T$ is a 
[submersion, diffeomorphism, PL homeomorphism, bi-Lipschitz homeomorphism]
then the same is true of $F(t, p, \cdot)$ for all $t\in I$ and $p\in P$.
(Of course we must assume that $X$ and $T$ are the appropriate 
sort of manifolds for this to make sense.)
\end{enumerate}
\end{lemma}




\noop{

\nn{move this to later:}

\begin{lemma}  \label{extension_lemma_b}
Let $x \in CM_k(X, T)$ be a singular chain such that $\bd x$ is adapted to $\cU$.
Then $x$ is homotopic (rel boundary) to some $x' \in CM_k(S, T)$ which is adapted to $\cU$.
Furthermore, one can choose the homotopy so that its support is equal to the support of $x$.
If $S$ and $T$ are manifolds, the statement remains true if we replace $CM_*(S, T)$ with
chains of smooth maps or immersions.
\end{lemma}

\medskip
\hrule
\medskip


In this appendix we provide the proof of
\nn{should change this to the more general \ref{extension_lemma_b}}

\begin{lem*}[Restatement of Lemma \ref{extension_lemma}]
Let $x \in CD_k(X)$ be a singular chain such that $\bd x$ is adapted to $\cU$.
Then $x$ is homotopic (rel boundary) to some $x' \in CD_k(X)$ which is adapted to $\cU$.
Furthermore, one can choose the homotopy so that its support is equal to the support of $x$.
\end{lem*}

\nn{for pedagogical reasons, should do $k=1,2$ cases first; probably do this in
later draft}

\nn{not sure what the best way to deal with boundary is; for now just give main argument, worry
about boundary later}

}


\nn{**** resume revising here ****}


\begin{proof}

Recall that we are given
an open cover $\cU = \{U_\alpha\}$ and an
$x \in CD_k(X)$ such that $\bd x$ is adapted to $\cU$.
We must find a homotopy of $x$ (rel boundary) to some $x' \in CD_k(X)$ which is adapted to $\cU$.

Let $\{r_\alpha : X \to [0,1]\}$ be a partition of unity for $\cU$.

As a first approximation to the argument we will eventually make, let's replace $x$
with a single singular cell
\eq{
    f: P \times X \to X .
}
Also, we'll ignore for now issues around $\bd P$.

Our homotopy will have the form
\eqar{
    F: I \times P \times X &\to& X \\
    (t, p, x) &\mapsto& f(u(t, p, x), x)
}
for some function
\eq{
    u : I \times P \times X \to P .
}
First we describe $u$, then we argue that it does what we want it to do.

For each cover index $\alpha$ choose a cell decomposition $K_\alpha$ of $P$.
The various $K_\alpha$ should be in general position with respect to each other.
We will see below that the $K_\alpha$'s need to be sufficiently fine in order
to insure that $F$ above is a homotopy through diffeomorphisms of $X$ and not
merely a homotopy through maps $X\to X$.

Let $L$ be the union of all the $K_\alpha$'s.
$L$ is itself a cell decomposition of $P$.
\nn{next two sentences not needed?}
To each cell $a$ of $L$ we associate the tuple $(c_\alpha)$,
where $c_\alpha$ is the codimension of the cell of $K_\alpha$ which contains $c$.
Since the $K_\alpha$'s are in general position, we have $\sum c_\alpha \le k$.

Let $J$ denote the handle decomposition of $P$ corresponding to $L$.
Each $i$-handle $C$ of $J$ has an $i$-dimensional tangential coordinate and,
more importantly, a $k{-}i$-dimensional normal coordinate.

For each (top-dimensional) $k$-cell $c$ of each $K_\alpha$, choose a point $p_c \in c \sub P$.
Let $D$ be a $k$-handle of $J$, and let $D$ also denote the corresponding
$k$-cell of $L$.
To $D$ we associate the tuple $(c_\alpha)$ of $k$-cells of the $K_\alpha$'s
which contain $d$, and also the corresponding tuple $(p_{c_\alpha})$ of points in $P$.

For $p \in D$ we define
\eq{
    u(t, p, x) = (1-t)p + t \sum_\alpha r_\alpha(x) p_{c_\alpha} .
}
(Recall that $P$ is a single linear cell, so the weighted average of points of $P$
makes sense.)

So far we have defined $u(t, p, x)$ when $p$ lies in a $k$-handle of $J$.
For handles of $J$ of index less than $k$, we will define $u$ to
interpolate between the values on $k$-handles defined above.

If $p$ lies in a $k{-}1$-handle $E$, let $\eta : E \to [0,1]$ be the normal coordinate
of $E$.
In particular, $\eta$ is equal to 0 or 1 only at the intersection of $E$
with a $k$-handle.
Let $\beta$ be the index of the $K_\beta$ containing the $k{-}1$-cell
corresponding to $E$.
Let $q_0, q_1 \in P$ be the points associated to the two $k$-cells of $K_\beta$
adjacent to the $k{-}1$-cell corresponding to $E$.
For $p \in E$, define
\eq{
    u(t, p, x) = (1-t)p + t \left( \sum_{\alpha \ne \beta} r_\alpha(x) p_{c_\alpha}
            + r_\beta(x) (\eta(p) q_1 + (1-\eta(p)) q_0) \right) .
}

In general, for $E$ a $k{-}j$-handle, there is a normal coordinate
$\eta: E \to R$, where $R$ is some $j$-dimensional polyhedron.
The vertices of $R$ are associated to $k$-cells of the $K_\alpha$, and thence to points of $P$.
If we triangulate $R$ (without introducing new vertices), we can linearly extend
a map from the vertices of $R$ into $P$ to a map of all of $R$ into $P$.
Let $\cN$ be the set of all $\beta$ for which $K_\beta$ has a $k$-cell whose boundary meets
the $k{-}j$-cell corresponding to $E$.
For each $\beta \in \cN$, let $\{q_{\beta i}\}$ be the set of points in $P$ associated to the aforementioned $k$-cells.
Now define, for $p \in E$,
\begin{equation}
\label{eq:u}
    u(t, p, x) = (1-t)p + t \left(
            \sum_{\alpha \notin \cN} r_\alpha(x) p_{c_\alpha}
                + \sum_{\beta \in \cN} r_\beta(x) \left( \sum_i \eta_{\beta i}(p) \cdot q_{\beta i} \right)
             \right) .
\end{equation}
Here $\eta_{\beta i}(p)$ is the weight given to $q_{\beta i}$ by the linear extension
mentioned above.

This completes the definition of $u: I \times P \times X \to P$.

\medskip

Next we verify that $u$ has the desired properties.

Since $u(0, p, x) = p$ for all $p\in P$ and $x\in X$, $F(0, p, x) = f(p, x)$ for all $p$ and $x$.
Therefore $F$ is a homotopy from $f$ to something.

Next we show that if the $K_\alpha$'s are sufficiently fine cell decompositions,
then $F$ is a homotopy through diffeomorphisms.
We must show that the derivative $\pd{F}{x}(t, p, x)$ is non-singular for all $(t, p, x)$.
We have
\eq{
%   \pd{F}{x}(t, p, x) = \pd{f}{x}(u(t, p, x), x) + \pd{f}{p}(u(t, p, x), x) \pd{u}{x}(t, p, x) .
    \pd{F}{x} = \pd{f}{x} + \pd{f}{p} \pd{u}{x} .
}
Since $f$ is a family of diffeomorphisms, $\pd{f}{x}$ is non-singular and
\nn{bounded away from zero, or something like that}.
(Recall that $X$ and $P$ are compact.)
Also, $\pd{f}{p}$ is bounded.
So if we can insure that $\pd{u}{x}$ is sufficiently small, we are done.
It follows from Equation \eqref{eq:u} above that $\pd{u}{x}$ depends on $\pd{r_\alpha}{x}$
(which is bounded)
and the differences amongst the various $p_{c_\alpha}$'s and $q_{\beta i}$'s.
These differences are small if the cell decompositions $K_\alpha$ are sufficiently fine.
This completes the proof that $F$ is a homotopy through diffeomorphisms.

\medskip

Next we show that for each handle $D \sub P$, $F(1, \cdot, \cdot) : D\times X \to X$
is a singular cell adapted to $\cU$.
This will complete the proof of the lemma.
\nn{except for boundary issues and the `$P$ is a cell' assumption}

Let $j$ be the codimension of $D$.
(Or rather, the codimension of its corresponding cell.  From now on we will not make a distinction
between handle and corresponding cell.)
Then $j = j_1 + \cdots + j_m$, $0 \le m \le k$,
where the $j_i$'s are the codimensions of the $K_\alpha$
cells of codimension greater than 0 which intersect to form $D$.
We will show that
if the relevant $U_\alpha$'s are disjoint, then
$F(1, \cdot, \cdot) : D\times X \to X$
is a product of singular cells of dimensions $j_1, \ldots, j_m$.
If some of the relevant $U_\alpha$'s intersect, then we will get a product of singular
cells whose dimensions correspond to a partition of the $j_i$'s.
We will consider some simple special cases first, then do the general case.

First consider the case $j=0$ (and $m=0$).
A quick look at Equation xxxx above shows that $u(1, p, x)$, and hence $F(1, p, x)$,
is independent of $p \in P$.
So the corresponding map $D \to \Diff(X)$ is constant.

Next consider the case $j = 1$ (and $m=1$, $j_1=1$).
Now Equation yyyy applies.
We can write $D = D'\times I$, where the normal coordinate $\eta$ is constant on $D'$.
It follows that the singular cell $D \to \Diff(X)$ can be written as a product
of a constant map $D' \to \Diff(X)$ and a singular 1-cell $I \to \Diff(X)$.
The singular 1-cell is supported on $U_\beta$, since $r_\beta = 0$ outside of this set.

Next case: $j=2$, $m=1$, $j_1 = 2$.
This is similar to the previous case, except that the normal bundle is 2-dimensional instead of
1-dimensional.
We have that $D \to \Diff(X)$ is a product of a constant singular $k{-}2$-cell
and a 2-cell with support $U_\beta$.

Next case: $j=2$, $m=2$, $j_1 = j_2 = 1$.
In this case the codimension 2 cell $D$ is the intersection of two
codimension 1 cells, from $K_\beta$ and $K_\gamma$.
We can write $D = D' \times I \times I$, where the normal coordinates are constant
on $D'$, and the two $I$ factors correspond to $\beta$ and $\gamma$.
If $U_\beta$ and $U_\gamma$ are disjoint, then we can factor $D$ into a constant $k{-}2$-cell and
two 1-cells, supported on $U_\beta$ and $U_\gamma$ respectively.
If $U_\beta$ and $U_\gamma$ intersect, then we can factor $D$ into a constant $k{-}2$-cell and
a 2-cell supported on $U_\beta \cup U_\gamma$.
\nn{need to check that this is true}

\nn{finally, general case...}

\nn{this completes proof}

\end{proof}




\medskip
\hrule
\medskip
\nn{the following was removed from earlier section; it should be reincorporated somehwere
in this section}

Let $\cU = \{U_\alpha\}$ be an open cover of $X$.
A $k$-parameter family of homeomorphisms $f: P \times X \to X$ is
{\it adapted to $\cU$} if there is a factorization
\eq{
    P = P_1 \times \cdots \times P_m
}
(for some $m \le k$)
and families of homeomorphisms
\eq{
    f_i :  P_i \times X \to X
}
such that
\begin{itemize}
\item each $f_i$ is supported on some connected $V_i \sub X$;
\item the sets $V_i$ are mutually disjoint;
\item each $V_i$ is the union of at most $k_i$ of the $U_\alpha$'s,
where $k_i = \dim(P_i)$; and
\item $f(p, \cdot) = g \circ f_1(p_1, \cdot) \circ \cdots \circ f_m(p_m, \cdot)$
for all $p = (p_1, \ldots, p_m)$, for some fixed $g \in \Homeo(X)$.
\end{itemize}
A chain $x \in CH_k(X)$ is (by definition) adapted to $\cU$ if it is the sum
of singular cells, each of which is adapted to $\cU$.
\medskip
\hrule
\medskip





%!TEX root = ../../blob1.tex

Here's an alternative proof of the special case in which $P$, the parameter space for the family of diffeomorphisms, is a cube. It is much more explicit, for better or worse.

\begin{proof}[Alternative, more explicit proof of Lemma \ref{extension_lemma}]


Fix a finite open cover of $X$, say $(U_l)_{l=1}^L$, along with an
associated partition of unity $(r_l)$.

We'll define the homotopy $H:I \times P \times X \to X$ via a function
$u:I \times P \times X \to P$, with
\begin{equation*}
H(t,p,x) = F(u(t,p,x),x).
\end{equation*}

To begin, we'll define a function $u'' : I \times P \times X \to P$, and
a corresponding homotopy $H''$. This homotopy will just be a homotopy of
$F$ through families of maps, not through families of diffeomorphisms. On
the other hand, it will be quite simple to describe, and we'll later
explain how to build the desired function $u$ out of it.

For each $l = 1, \ldots, L$, pick some $C^\infty$ function $f_l : I \to
I$ which is identically $0$ on a neighborhood of the closed interval $[0,\frac{l-1}{L}]$
and identically $1$ on a neighborhood of the closed interval $[\frac{l}{L},1]$. (Monotonic?
Fix a bound for the derivative?) We'll extend it to a function on
$k$-tuples $f_l : I^k \to I^k$ pointwise.

Define $$u''(t,p,x) = \sum_{l=1}^L r_l(x) u_l(t,p),$$ with
$$u_l(t,p) = t f_l(p) + (1-t)p.$$ Notice that the $i$-th component of $u''(t,p,x)$ depends only on the $i$-th component of $p$.

Let's now establish some properties of $u''$ and $H''$. First,
\begin{align*}
H''(0,p,x) & = F(u''(0,p,x),x) \\
           & = F(\sum_{l=1}^L r_l(x) p, x) \\
           & = F(p,x).
\end{align*}
Next, calculate the derivatives
\begin{align*}
\partial_{p_i} H''(1,p,x) & = \partial_{p_i}u''(1,p,x) \partial_1 F(u(1,p,x),x) \\
\intertext{and}
\partial_{p_i}u''(1,p,x) & = \sum_{l=1}^L r_l(x) \partial_{p_i} f_l(p).
\end{align*}
Now $\partial_{p_i} f_l(p) = 0$ unless $\frac{l-1}{L} < p_i < \frac{l}{L}$, and $r_l(x) = 0$ unless $x \in U_l$,
so we conclude that for a fixed $p$, $\partial_p H''(1,p,x) = 0$ for all $x$ outside the union of $k$ open sets from the open cover, namely
$\bigcup_{i=1}^k U_{l_i}$ where for each $i$, we choose $l_i$ so $\frac{l_i -1}{L} \leq p_i \leq \frac{l_i}{L}$. It may be helpful to refer to Figure \ref{fig:supports}.

\begin{figure}[!ht]
\begin{equation*}
\mathfig{0.5}{explicit/supports}
\end{equation*}
\caption{The supports of the derivatives {\color{green}$\partial_p f_1$}, {\color{blue}$\partial_p f_2$} and {\color{red}$\partial_p f_3$}, illustrating the case $k=2$, $L=3$. Notice that any
point $p$ lies in the intersection of at most $k$ supports. The support of $\partial_p u''(1,p,x)$ is contained in the union of these supports.}
\label{fig:supports}
\end{figure}

Unfortunately, $H''$ does not have the desired property that it's a homotopy through diffeomorphisms. To achieve this, we'll paste together several copies
of the map $u''$. First, glue together $2^k$ copies, defining $u':I \times P \times X$ by
\begin{align*}
u'(t,p,x)_i & =
\begin{cases}
\frac{1}{2} u''(t, 2p_i, x)_i & \text{if $0 \leq p_i \leq \frac{1}{2}$} \\
1-\frac{1}{2} u''(t, 2-2p_i, x)_i & \text{if $\frac{1}{2} \leq p_i \leq 1$}.
\end{cases}
\end{align*}
(Note that we're abusing notation somewhat, using the fact that $u''(t,p,x)_i$ depends on $p$ only through $p_i$.)
To see what's going on here, it may be helpful to look at Figure \ref{fig:supports_4}, which shows the support of $\partial_p u'(1,p,x)$.
\begin{figure}[!ht]
\begin{equation*}
\mathfig{0.4}{explicit/supports_4} \qquad \qquad \mathfig{0.4}{explicit/supports_36}
\end{equation*}
\caption{The supports of $\partial_p u'(1,p,x)$ and of $\partial_p u(1,p,x)$ (with $K=3$) are subsets of the indicated region.}
\label{fig:supports_4}
\end{figure}

Second, pick some $K$, and define
\begin{align*}
u(t,p,x) & = \frac{\floor{K p}}{K} + \frac{1}{K} u'\left(t, K \left(p - \frac{\floor{K p}}{K}\right), x\right).
\end{align*}

\todo{Explain that the localisation property survives for $u'$ and $u$.}

We now check that by making $K$ large enough, $H$ becomes a homotopy through diffeomorphisms. We start with
$$\partial_x H(t,p,x) = \partial_x u(t,p,x) \partial_1 F(u(t,p,x), x) + \partial_2 F(u(t,p,x), x)$$
and observe that since $F(p, -)$ is a diffeomorphism, the second term $\partial_2 F(u(t,p,x), x)$ is bounded away from $0$. Thus if we can control the
size of the first term $\partial_x u(t,p,x) \partial_1 F(u(t,p,x), x)$ we're done. The factor $\partial_1 F(u(t,p,x), x)$ is bounded, and we
calculate \todo{err... this is a mess, and probably wrong.}
\begin{align*}
\partial_x u(t,p,x)_i & = \partial_x \frac{1}{K} u'\left(t, K\left(p - \frac{\floor{K p}}{K}\right), x\right)_i \\
                      & = \pm \frac{1}{2 K} \partial_x u''\left(t, (1\mp1)\pm 2K\left(p_i-\frac{\floor{K p}}{K}\right), x\right)_i \\
                      & = \pm \frac{1}{2 K} \sum_{l=1}^L (\partial_x r_l(x)) u_l\left(t, (1\mp1)\pm 2K\left(p_i-\frac{\floor{K p}}{K}\right)\right)_i. \\
\intertext{Since the target of $u_l$ is just the unit cube $I^k$, we can make the estimate}
\norm{\partial_x u(t,p,x)_i} & \leq \frac{1}{2 K} \sum_{l=1}^L \norm{\partial_x r_l(x)}.
\end{align*}
The sum here is bounded, so for large enough $K$ this is small enough that $\partial_x H(t,p,x)$ is never zero.

\end{proof}



%%!TEX root = ../../blob1.tex
\section{The small blob complex}
\label{appendix:small-blobs}

Before proving Theorem \ref{thm:small-blobs}, we need a lemma allowing us to choose a convenient action of families of diffeomorphisms.
Say that an open cover $\cV$ is strictly subordinate to $\cU$ if the closure of every open set of $\cV$ is contained in some open set of $\cU$.

\begin{lem}
\label{lem:CH-small-blobs}
Fix an open cover $\cU$, and a sequence $\cV_k$ of open covers which are each strictly subordinate to $\cU$. For a given $k$, consider $\cG_k$ the subspace of $C_k(\Homeo(M)) \tensor \bc_*(M)$ spanned by $f \tensor b$, where $f:P^k \times M \to M$ is a $k$-parameter family of homeomorphisms such that for each $p \in P$, $f(p, -)$ makes $b$ small with respect to $\cV_k$. We can choose an up-to-homotopy representative $\ev$ of the chain map of Theorem \ref{thm:evaluation} which gives the action of families of homeomorphisms, which restricts to give a map
$$\ev : \cG_k \subset C_k(\Homeo(M)) \tensor \bc_*(M) \to \bc^{\cU}_*(M)$$
for each $k$.
\end{lem}
\begin{rem}
We can't quite do the same with all $\cV_k$ just equal to $\cU$, but we can get by if we give ourselves arbitrarily little room to maneuver, by making the blobs we act on slightly smaller.
\end{rem}
\begin{proof}
This follows from Remark \ref{rem:for-small-blobs} following the proof of 
Proposition \ref{CHprop}.
\end{proof}

\begin{proof}[Proof of Theorem \ref{thm:small-blobs}]
We begin by describing the homotopy inverse in small degrees, to illustrate the general technique.
We will construct a chain map $s:  \bc_*(M) \to \bc^{\cU}_*(M)$ and a homotopy $h:\bc_*(M) \to \bc_{*+1}(M)$ so that $\bdy h+h \bdy=i\circ s - \id$. The composition $s \circ i$ will just be the identity.

On $0$-blobs, $s$ is just the identity; a blob diagram without any blobs is compatible with any open cover. 
\nn{KW: For some systems of fields this is not true.
For example, consider a planar algebra with boxes of size greater than zero.
So I think we should do the homotopy even in degree zero.
But as noted above, maybe it's best to ignore this.}
Nevertheless, we'll begin introducing nomenclature at this point: for configuration $\beta$ of disjoint embedded balls in $M$ we'll associate a one parameter family of homeomorphisms $\phi_\beta : \Delta^1 \to \Homeo(M)$ (here $\Delta^m$ is the standard simplex $\setc{\mathbf{x} \in \Real^{m+1}}{\sum_{i=0}^m x_i = 1}$). For $0$-blobs, where $\beta = \eset$, all these homeomorphisms are just the identity.

When $\beta$ is a collection of disjoint embedded balls in $M$, we say that a homeomorphism of $M$ `makes $\beta$ small' if the image of each ball in $\beta$ under the homeomorphism is contained in some open set of $\cU$. Further, we'll say a homeomorphism `makes $\beta$ $\epsilon$-small' if the image of each ball is contained in some open ball of radius $\epsilon$.

On a $1$-blob $b$, with ball $\beta$, $s$ is defined as the sum of two terms. Essentially, the first term `makes $\beta$ small', while the other term `gets the boundary right'. First, pick a one-parameter family $\phi_\beta : \Delta^1 \to \Homeo(M)$ of homeomorphisms, so $\phi_\beta(1,0)$ is the identity and $\phi_\beta(0,1)$ makes the ball $\beta$ small --- in fact, not just small with respect to $\cU$, but $\epsilon/2$-small, where $\epsilon > 0$ is such that every $\epsilon$-ball is contained in some open set of $\cU$. Next, pick a two-parameter family $\phi_{\eset \prec \beta} : \Delta^2 \to \Homeo(M)$ so that $\phi_{\eset \prec \beta}(0,x_1,x_2)$ makes the ball $\beta$ $\frac{3\epsilon}{4}$-small for all $x_1+x_2=1$, while $\phi_{\eset \prec \beta}(x_0,0,x_2) = \phi_\beta(x_0,x_2)$ and $\phi_{\eset \prec \beta}(x_0,x_1,0) = \phi_\eset(x_0,x_1)$. (It's perhaps not obvious that this is even possible --- see Lemma \ref{lem:extend-small-homeomorphisms} below.) We now define $s$ by
$$s(b) = \restrict{\phi_\beta}{x_0=0}(b) - \restrict{\phi_{\eset \prec \beta}}{x_0=0}(\bdy b).$$
Here, $\restrict{\phi_\beta}{x_0=0} = \phi_\beta(0,1)$ is just a homeomorphism, which we apply to $b$, while $\restrict{\phi_{\eset \prec \beta}}{x_0=0}$ is a one parameter family of homeomorphisms which acts on the $0$-blob $\bdy b$ to give a $1$-blob. To be precise, this action is via the chain map identified in Lemma \ref{lem:CH-small-blobs} with $\cV_0$ the open cover by $\epsilon/2$-balls and $\cV_1$ the open cover by $\frac{3\epsilon}{4}$-balls. From this, it is immediate that $s(b) \in \bc^{\cU}_1(M)$, as desired.

We now check that $s$, as defined so far, is a chain map, calculating
\begin{align*}
\bdy (s(b)) & = \restrict{\phi_\beta}{x_0=0}(\bdy b) - (\bdy \restrict{\phi_{\eset \prec \beta}}{x_0=0})(\bdy b) \\
		 & = \restrict{\phi_\beta}{x_0=0}(\bdy b) - \restrict{\phi_\beta}{x_0=0}(\bdy b) + \restrict{\phi_\eset}{x_0=0}(\bdy b) \\
		 & = \restrict{\phi_\eset}{x_0=0}(\bdy b) \\
		 & = s(\bdy b)
\end{align*}
Next, we compute the compositions $s \circ i$ and $i \circ s$. If we start with a small $1$-blob diagram $b$, first include it up to the full blob complex then apply $s$, we get exactly back to $b$, at least assuming we adopt the convention that for any ball $\beta$ which is already small, we choose the families of homeomorphisms $\phi_\beta$ and $\phi_{\eset \prec \beta}$ to always be the identity. In the other direction, $i \circ s$, we will need to construct a homotopy $h:\bc_*(M) \to \bc_{*+1}(M)$ for $*=0$ or $1$.

The homotopy $h$ is defined by $$h(b) = \phi_\eset(b)$$ when $b$ is a $0$-blob (here $\phi_\eset$ is a one parameter family of homeomorphisms, so this is a $1$-blob), and $$h(b) = \phi_\beta(b) - \phi_{\eset \prec \beta}(\bdy b)$$ when $b$ is a $1$-blob (here $\beta$ is the ball in $b$, and the first term is the action of a one parameter family of homeomorphisms on a $1$-blob, and the second term is the action of a two parameter family of homeomorphisms on a $0$-blob, so both are $2$-blobs). We then calculate
\begin{align*}
(\bdy h+h \bdy)(b) & = \bdy (\phi_{\beta}(b) - \phi_{\eset \prec \beta}(\bdy b)) + \phi_\eset(\bdy b)  \\
	& =  \restrict{\phi_\beta}{x_0=0}(b) - \restrict{\phi_\beta}{x_1=0}(b) - \phi_\beta(\bdy b) - (\bdy \phi_{\eset \prec \beta})(\bdy b) + \phi_\eset(\bdy b) \\
	& =  \restrict{\phi_\beta}{x_0=0}(b) - b - \phi_\beta(\bdy b) - \restrict{\phi_{\eset \prec \beta}}{x_0=0}(\bdy b) +  \phi_\beta(\bdy b) - \phi_\eset(\bdy b) + \phi_\eset(\bdy b) \\
	& = \restrict{\phi_\beta}{x_0=0}(b) - b - \restrict{\phi_{\eset \prec \beta}}{x_0=0}(\bdy b) \\
	& = (i \circ s - \id)(b).
\end{align*}


In order to define $s$ on arbitrary blob diagrams, we first fix a sequence of strictly subordinate covers for $\cU$. First choose an $\epsilon > 0$ so every $\epsilon$ ball is contained in some open set of $\cU$. For $k \geq 1$, let $\cV_{k}$ be the open cover of $M$ by $\epsilon (1-2^{-k})$ balls, and $\cV_0 = \cU$. Certainly $\cV_k$ is strictly subordinate to $\cU$. We now chose the chain map $\ev$ provided by Lemma \ref{lem:CH-small-blobs} for the open covers $\cV_k$ strictly subordinate to $\cU$. Note that $\cV_1$ and $\cV_2$ have already implicitly appeared in the description above.

Next, we choose a `shrinking system' for $\left(\cU,\{\cV_k\}_{k \geq 1}\right)$, namely for each increasing sequence of blob configurations
$\beta_1 \prec \cdots \prec \beta_n$, an $n$ parameter family of diffeomorphisms
$\phi_{\beta_1 \prec \cdots \prec \beta_n} : \Delta^{n+1} \to \Diff{M}$, such that
\begin{itemize}
\item for any $x$ with $x_0 = 0$, $\phi_{\beta_1 \prec \cdots \prec \beta_n}(x)(\beta_1)$ is subordinate to $\cV_{n+1}$, and
\item for each $i = 1, \ldots, n$,
\begin{align*}
\phi_{\beta_1 \prec \cdots \prec \beta_n}&(x_0, \ldots, x_{i-1},0,x_{i+1},\ldots,x_n) = \\ &\phi_{\beta_1 \prec \cdots \prec \beta_{i-1} \prec \beta_{i+1} \prec \cdots \prec \beta_n}(x_0,\ldots, x_{i-1},x_{i+1},\ldots,x_n).
\end{align*}
\end{itemize}
Again, we've already made the choices for $\phi_{\beta}$ and for $\phi_{\eset \prec \beta}$, where $\beta$ is a single ball. It's not immediately obvious that it's possible to make such choices, but it follows readily from the following.

\begin{lem}
\label{lem:extend-small-homeomorphisms}
Fix a collection of disjoint embedded balls $\beta$ in $M$ and some open cover $\cV$. Suppose we have a map $f :  X \to \Homeo(M)$ on some compact $X$ such that for each $x \in \bdy X$, $f(x)$ makes $\beta$ $\cV$-small. Then we can extend $f$ to a map $\tilde{f} : X \times [0,1] \to \Homeo(M)$ so that $\tilde{f}(x,0) = f(x)$ and for every $x \in \bdy X \times [0,1] \cup X \times \{1\}$, $\tilde{f}(x)$ makes $\beta$ $\cV$-small.
\end{lem}
\begin{proof}
Fix a metric on $M$, and pick $\epsilon > 0$ so every $\epsilon$ ball in $M$ is contained in some open set of $\cV$. First construct a family of homeomorphisms $g_s : M \to M$, $s \in [1,\infty)$ so $g_1$ is the identity, and $g_s(\beta_i) \subset \beta_i$ and $\rad g_s(\beta_i) \leq \frac{1}{s} \rad \beta_i$ for each ball $\beta_i$. 
There is some $K$ which uniformly bounds the expansion factors of all the homeomorphisms $f(x)$, that is $d(f(x)(a), f(x)(b)) < K d(a,b)$ for all $x \in X, a,b \in M$. Write $S=\epsilon^{-1} K \max_i \{\rad \beta_i\}$ (note that is $S<1$, we can just take $S=1$, as already $f(x)$ makes $\beta$ small for all $x$). Now define $\tilde{f}(t, x) = f(x) \compose g_{(S-1)t+1}$.

If $x \in \bdy X$, then $g_{(S-1)t+1}(\beta_i) \subset \beta_i$, and by hypothesis $f(x)$ makes $\beta_i$ small, so $\tilde{f}(t, x)$ makes $\beta$ $\cV$-small for all $t \in [0,1]$. Alternatively, $\rad g_S(\beta_i) \leq \frac{1}{S} \rad \beta_i \leq \frac{\epsilon}{K}$, so $\rad \tilde{f}(1,x)(\beta_i) \leq \epsilon$, and so $\tilde{f}(1,x)$ makes $\beta$ $\cV$-small for all $x \in X$.
\end{proof}

In fact, the application of this Lemma would allow us to choose the families of diffeomorphisms $\phi_{\beta_1 \prec \cdots \prec \beta_n}$ so that for any $x$ with $x_0 = 0$, $\phi_{\beta_1 \prec \cdots \prec \beta_n}(x)(\beta_1)$ is subordinate to any fixed open cover, for example $\cV_1$ (that is, the covering by $\epsilon/2$ balls), not just $\cV_{n+1}$, which is a weaker condition. Regardless, because of the way we have chosen the $\ev$ map, we only ensure that $\ev(\restrict{\phi_{\beta_1 \prec \cdots \prec \beta_n}}{x_0 = 0} \tensor \beta_1) \in \bc_{\deg \beta_1 + n}^{\cU}(M)$, so the distinction is not important.

We now describe the general case. For a $k$-blob diagram $b \in \bc_k(M)$, denote by $b_\cS$ for $\cS \subset \{1, \ldots, k\}$ the blob diagram obtained by erasing the corresponding blobs. In particular, $b_\eset = b$, $b_{\{1,\ldots,k\}} \in \bc_0(M)$, and $\bdy b_\cS = \sum_{i \notin \cS} (-1)^{i+1+\card{\setc{j \in \cS}{j < i}}}  b_{\cS \cup \{i\}}$.
Similarly, for a disjoint embedding of $k$ balls $\beta$ (that is, a blob diagram but without the labels on regions), $\beta_\cS$ denotes the result of erasing a subset of blobs. 

\newcommand{\length}[1]{\operatorname{length}(#1)}

We've finally reached the point where we can define a map $s: \bc_*(M) \to \bc^{\cU}_*(M)$, and then a homotopy $h:\bc_*(M) \to \bc_{*+1}(M)$ so that $dh+hd=i\circ s$.  We have
$$s(b) = \sum_{m=0}^{k} \sum_{i \in \{1, \ldots, k\}^{m} \setminus \Delta} (-1)^{\sigma(i)}  \ev(\restrict{\phi_{i(b)}}{x_0 = 0} \tensor b_i),$$
where the sum is over sequences without repeats $i=(i_1,\ldots,i_m)$ in $\{1,\ldots,k\}$, with $0\leq m \leq k$ (we're using $\Delta$ here to indicate the generalized diagonal, where any two entries coincide), $\sigma(i)$ is defined blow, $i(b)$ denotes the increasing sequence of blob configurations
$$\beta_{(i_1,\ldots,i_m)} \prec \beta_{(i_2,\ldots,i_m)} \prec \cdots \prec \beta_{()},$$
and, as usual, $b_i$ denotes $b$ with blobs $i_1, \ldots, i_m$ erased.
The homotopy $h:\bc_*(M) \to \bc_{*+1}(M)$ is similarly given by
$$h(b) = \sum_{m=0}^{k} \sum_{i} (-1)^{\sigma(i)} \ev(\phi_{i(b)}, b_i).$$

 The signs $\sigma: \{1, \ldots, k\}^{m} \setminus \Delta \to \Integer/2\Integer$ are defined by
$$\sigma(i)= \lceil\frac{m-1}{2}\rceil  +o(i) + \sum i$$
where $o(i)$ is the number of transpositions required to bring $i$ into increasing order. Given a sequence $a \in \{1, \ldots, k\}^{m-1} \setminus \Delta$ and $1 \leq b \leq k$ with $b \not \in a$, denote by $a \!\downarrow_b\in \{1, \ldots, k-1\}^{m-1} \setminus \Delta$ the sequence obtained by reducing by 1 each entry of $a$ which is greater than $b$. We'll need the identities
\begin{align} 
\sigma(ab) & = \sigma(a\!\downarrow_b) +m + b+1 \label{eq:sigma(ab)}\\
%		  & =  \sigma(a) + m+b + 1 + \card{\setc{x \in a}{x > b}} \\
\intertext{and}
\sigma(ba) & = \sigma(a\!\downarrow_b) + b \notag \\
		 & = \sigma(a) + m+b + 1+\card{\setc{x \in a}{x < b}}. \label{eq:sigma(ba)}
\end{align}

Before completing the proof, we unpack this definition for $b \in \bc_2(M)$, a $2$-blob. We'll write $\beta$ for the underlying balls (either nested or disjoint).
Now $s$ is the sum of $5$ terms, split into three groups depending on with the length of the sequence $i$ is $0, 1$ or $2$. Thus
\begin{align*}
s(b) & = (-1)^{\sigma()} \restrict{\phi_{\beta}}{x_0 = 0}(b) + \\
	& \quad + (-1)^{\sigma(1)}  \restrict{\phi_{\beta_1 \prec \beta}}{x_0 = 0}(b_1) + (-1)^{\sigma(2)} \restrict{\phi_{\beta_2 \prec \beta}}{x_0 = 0}(b_2) + \\
	& \quad + (-1)^{\sigma(12)}  \restrict{\phi_{\eset \prec \beta_2 \prec \beta}}{x_0 = 0}(b_{12}) + (-1)^{\sigma(21)}  \restrict{\phi_{\eset \prec \beta_1 \prec \beta}}{x_0 = 0}(b_{12}).
\end{align*}

As in the $k=1$ case, the first term, corresponding to $i(b) = \eset$, makes the all balls in $\beta$ $\cV_1$-small. However, if this were the only term $s$ would not be a chain map, because we have no control over $\restrict{\phi_{\beta}}{x_0 = 0}(\bdy b)$. This necessitates the other terms, which fix the boundary at successively higher codimensions.

It may be useful to look at Figure \ref{fig:erectly-a-tent-badly} to help understand the arrangement. The red, blue and orange $2$-cells there correspond to the $m=0$, $m=1$ and $m=2$ terms respectively, while the $3$-cells (only one of each type is shown) correspond to the terms in the homotopy $h$.
\begin{figure}[!ht]
$$\mathfig{0.5}{smallblobs/tent}$$
\caption{``Erecting a tent badly.'' We know where we want to send a simplex, and each of the iterated boundary components. However, these do not agree, and we need to stitch the pieces together. Note that these diagrams don't exactly match the situation in the text: a $k$-simplex has $k+1$ boundary components, while a $k$-blob has $k$ boundary terms.}
\label{fig:erectly-a-tent-badly}
\end{figure}

Now
\begin{align*}
\bdy s(b) & =  (-1)^{\sigma()} \restrict{\phi_{\beta}}{x_0 = 0}(\bdy b) + \\
	& \quad + (-1)^{\sigma(1)} \left( \restrict{\phi_{\beta}}{x_0 = 0}(b_1) - \restrict{\phi_{\beta_1}}{x_0 = 0}(b_1) - \restrict{\phi_{\beta_1 \prec \beta}}{x_0 = 0}(b_{12}) \right) + \\
	& \quad + (-1)^{\sigma(2)} \left( \restrict{\phi_{\beta}}{x_0 = 0}(b_2) - \restrict{\phi_{\beta_2}}{x_0 = 0}(b_2) - \restrict{\phi_{\beta_2 \prec \beta}}{x_0 = 0}(b_{12}) \right) + \\
	& \quad + (-1)^{\sigma(12)} \left( \restrict{\phi_{\beta_2 \prec \beta}}{x_0 = 0}(b_{12}) - \restrict{\phi_{\eset \prec \beta}}{x_0 = 0}(b_{12}) + \restrict{\phi_{\eset \prec \beta_2}}{x_0 = 0}(b_{12}) \right) + \\ 
	& \quad + (-1)^{\sigma(21)} \left( \restrict{\phi_{\beta_1 \prec \beta}}{x_0 = 0}(b_{12}) - \restrict{\phi_{\eset  \prec \beta}}{x_0 = 0}(b_{12}) + \restrict{\phi_{\eset \prec \beta_1}}{x_0 = 0}(b_{12}) \right), \\
\intertext{while}
s(\bdy(b)) & = s(b_1) - s(b_2) \\
		& = \restrict{\phi_{\beta_1}}{x_0=0}(b_1) - \restrict{\phi_{\eset \prec \beta_1}}{x_0=0}(b_{12}) - \restrict{\phi_{\beta_2}}{x_0=0}(b_2) + \restrict{\phi_{\eset \prec \beta_2}}{x_0=0}(b_{12}) .
\end{align*}
This gives what we want, since $\sigma() = 0,\sigma(1)=1, \sigma(2)=0, \sigma(21)=1$ and $\sigma(12)=0$.

We now return to the general case. Certainly, the image of $s$ in fact lies in $\bc^{\cU}_*(M)$: each of the families of diffeomorphisms $\phi_{i(b)}$ has been chosen so with $x_0=0$ they pointwise make $b_i$ sufficiently small that the $\ev$ map we've chosen has image in $\bc^{\cU}_*(M)$.

We need to check that $s$ is a chain map.
\begin{align*}
\bdy(s(b)) & = \sum_{m=0}^{k} \sum_{i \in \{1, \ldots, k\}^{m} \setminus \Delta} (-1)^{\sigma(i)} \ev\left(\bdy(\restrict{\phi_{i(b)}}{x_0 = 0})\tensor b_i\right) + (-1)^{\sigma(i) + m} \ev\left(\restrict{\phi_{i(b)}}{x_0 = 0} \tensor \bdy b_i\right) \\
\intertext{and begin by expanding out $\bdy(\restrict{\phi_{i(b)}}{x_0 = 0})$,}
	& = \sum_{m=0}^{k} \sum_{i \in \{1, \ldots, k\}^{m} \setminus \Delta} \Bigg(\sum_{p=1}^{m+1} (-1)^{\sigma(i)+p+1} \ev\left(\restrict{\phi_{i(b)}}{x_0 = x_p = 0}\tensor b_i\right) \Bigg) + \\
	& \qquad + (-1)^{\sigma(i) + m} \ev\left(\restrict{\phi_{i(b)}}{x_0 = 0} \tensor \bdy b_i\right)
\end{align*}
Now, write $s_{p-1,p}(i)$ to indicate the sequence obtained from $i$ by transposing its $p-1$-th and $p$-th entries and note that for $2 \leq p \leq m$,
\begin{align*}
\restrict{\phi_{i(b)}}{x_0=x_p=0} & = \restrict{\phi_{\beta_{i_1\cdots i_m} \prec \beta_{i_2 \cdots i_m} \prec \cdots \prec \beta}}{x_0=x_p=0} \\
	& = \restrict{\phi_{\beta_{i_1\cdots i_m} \prec \beta_{i_2 \cdots i_m} \prec \cdots \prec \beta_{i_{p-1} i_p \cdots i_m} \prec \beta_{i_{p+1} \cdots i_m} \prec \cdots \prec \beta}}{x_0=0} \\
	& = \restrict{\phi_{s_{p-1,p}(i)(b)}}{x_0=x_p=0}.
\end{align*}
Since $\sigma(i) = - \sigma(s_{p-1,p}(i))$, we can cancel out in pairs all the terms above except those with $p=1$ or $p=m+1$. Thus
\begin{align*}
\bdy(s(b)) & = \sum_{m=0}^{k} \sum_{i \in \{1, \ldots, k\}^{m} \setminus \Delta} \Bigg((-1)^{\sigma(i)} \ev\left(\restrict{\phi_{\rest(i)(b)}}{x_0 = 0}\tensor b_i\right) + (-1)^{\sigma(i) + m} \ev\left(\restrict{\phi_{\most(i)\!\downarrow_{i_m}(b_{i_m})}}{x_0 = 0}\tensor b_i\right)\Bigg) + \\
	& \qquad + (-1)^{\sigma(i) + m} \ev\left(\restrict{\phi_{i(b)}}{x_0 = 0} \tensor \bdy b_i\right)
\end{align*}
where we use the notations $\rest(i_1 i_2 \cdots i_m) = (i_2 \cdots i_m)$ and $\most(i_1 \cdots i_{m-1} i_m) = (i_1 \cdots i_{m-1})$. Next, we note that $b_i = (b_{i_1})_{\rest(i)} = (b_{i_m})_{\most(i)}$, and then rewrite the sum of $i$ as a double sum over $i_1$ and $\rest(i)$, with $i = i_1\rest(i)$, for the first term, and as a double sum over $\most(i)$ and $i_m$, with $i = \most(i)i_m$, for the second term.
\begin{align*}
\bdy(s(b)) & = \sum_{m=0}^{k} \Bigg( \sum_{\rest(i) \in \{1, \ldots, k\}^{m-1} \setminus \Delta} \sum_{\substack{i_1 = 1 \\ i_1 \not\in \rest(i)}}^{k}  (-1)^{\sigma(i_1\rest(i))} \ev\left(\restrict{\phi_{\rest(i)(b)}}{x_0 = 0}\tensor b_{i_1\rest(i)}\right) \Bigg)+ \\
	& \qquad \Bigg( \sum_{\most(i) \in \{1, \ldots, k\}^{m-1} \setminus \Delta} \sum_{\substack{i_m = 1 \\ i_1 \not\in \most(i)}}^{k}  (-1)^{\sigma(\most(i) i_m) + m} \ev\left(\restrict{\phi_{\most(i)\!\downarrow_{i_m}(b_{i_m})}}{x_0 = 0}\tensor b_{\most(i) i_m}\right)\Bigg) + \\
	& \qquad \Bigg( \sum_{i \in \{1, \ldots, k\}^{m} \setminus \Delta} (-1)^{\sigma(i) + m} \ev\left(\restrict{\phi_{i(b)}}{x_0 = 0} \tensor \bdy b_i\right)\Bigg)
\end{align*}
We will show that the first and third rows cancel, and that the second row gives with $s(\bdy b)$.
First,
\begin{align*}
\sum_{\substack{i_1 = 1 \\ i_1 \not\in \rest(i)}}^{k} (-1)^{\sigma(i_1\rest(i))} b_{i_1\rest(i)} & =
	\sum_{\substack{i_1 = 1 \\ i_1 \not\in \rest(i)}}^{k} (-1)^{\sigma(\rest(i))+m+1+i_1 + \card{\setc{x \in \rest(i)}{x<i_1}}}  (b_{\rest(i)})_{i_1} \\
\intertext{(using Equation \eqref{eq:sigma(ba)})}
& = (-1)^{\sigma(\rest(i))+m} \bdy (b_{\rest(i)}),
\end{align*}
and this cancels exactly with the term indexed by $\rest(i)$ (with a value of $m$ off by one) in the third row.
The second row gives 
\begin{align*}
& \sum_{m=0}^k  \sum_{\most(i) \in \{1, \ldots, k\}^{m-1} \setminus \Delta} \sum_{\substack{i_m = 1 \\ i_1 \not\in \most(i)}}^{k}  (-1)^{\sigma(\most(i) i_m) + m} \ev\left(\restrict{\phi_{\most(i)\!\downarrow_{i_m}(b_{i_m})}}{x_0 = 0}\tensor b_{\most(i) i_m}\right) \\
& \quad = \sum_{m=0}^{k-1} \sum_{q=1}^k \sum_{i \in \{1, \ldots, k-1\}^m\setminus \Delta} (-1)^{\sigma(i)+q+1} \ev\left(\restrict{\phi_{i(b_{q})}}{x_0 = 0}\tensor (b_q)_i\right) \\
\intertext{(here we've used Equation \eqref{eq:sigma(ab)} and renamed $i_m$ to $q$ and $most(i)$ to $i$, as well as shifted $m$ by one), which is just}
& \quad = \sum_{q=1}^k (-1)^{q+1} s(b_q) \\
& \quad = s(\bdy b).
\end{align*}

Finally, the calculation that $\bdy h+h \bdy=i\circ s - \id$ is very similar, and we omit it.
\end{proof}
 

%!TEX root = ../../blob1.tex

\section{Comparing $n$-category definitions}
\label{sec:comparing-defs}

In \S\ref{sec:example:traditional-n-categories(fields)} we showed how to construct
a topological $n$-category from a traditional $n$-category; the morphisms of the 
topological $n$-category are string diagrams labeled by the traditional $n$-category.
In this appendix we sketch how to go the other direction, for $n=1$ and 2.
The basic recipe, given a topological $n$-category $\cC$, is to define the $k$-morphisms
of the corresponding traditional $n$-category to be $\cC(B^k)$, where
$B^k$ is the {\it standard} $k$-ball.
One must then show that the axioms of \S\ref{ss:n-cat-def} imply the traditional $n$-category axioms.
One should also show that composing the two arrows (between traditional and topological $n$-categories)
yields the appropriate sort of equivalence on each side.
Since we haven't given a definition for functors between topological $n$-categories
(the paper is already too long!), we do not pursue this here.
\nn{say something about modules and tensor products?}

We emphasize that we are just sketching some of the main ideas in this appendix ---
it falls well short of proving the definitions are equivalent.

%\nn{cases to cover: (a) plain $n$-cats for $n=1,2$; (b) $n$-cat modules for $n=1$, also 2?;
%(c) $A_\infty$ 1-cat; (b) $A_\infty$ 1-cat module?; (e) tensor products?}

\subsection{$1$-categories over $\Set$ or $\Vect$}
\label{ssec:1-cats}
Given a topological $1$-category $\cX$ we construct a $1$-category in the conventional sense, $c(\cX)$.
This construction is quite straightforward, but we include the details for the sake of completeness, 
because it illustrates the role of structures (e.g. orientations, spin structures, etc) 
on the underlying manifolds, and 
to shed some light on the $n=2$ case, which we describe in \S \ref{ssec:2-cats}.

Let $B^k$ denote the \emph{standard} $k$-ball.
Let the objects of $c(\cX)$ be $c(\cX)^0 = \cX(B^0)$ and the morphisms of $c(\cX)$ be $c(\cX)^1 = \cX(B^1)$.
The boundary and restriction maps of $\cX$ give domain and range maps from $c(\cX)^1$ to $c(\cX)^0$.

Choose a homeomorphism $B^1\cup_{pt}B^1 \to B^1$.
Define composition in $c(\cX)$ to be the induced map $c(\cX)^1\times c(\cX)^1 \to c(\cX)^1$ 
(defined only when range and domain agree).
By isotopy invariance in $\cX$, any other choice of homeomorphism gives the same composition rule.
Also by isotopy invariance, composition is strictly associative.

Given $a\in c(\cX)^0$, define $\id_a \deq a\times B^1$.
By extended isotopy invariance in $\cX$, this has the expected properties of an identity morphism.


If the underlying manifolds for $\cX$ have further geometric structure, then we obtain certain functors.
The base case is for oriented manifolds, where we obtain no extra algebraic data.

For 1-categories based on unoriented manifolds, 
there is a map $*:c(\cX)^1\to c(\cX)^1$
coming from $\cX$ applied to an orientation-reversing homeomorphism (unique up to isotopy) 
from $B^1$ to itself.
Topological properties of this homeomorphism imply that 
$a^{**} = a$ (* is order 2), * reverses domain and range, and $(ab)^* = b^*a^*$
(* is an anti-automorphism).

For 1-categories based on Spin manifolds,
the the nontrivial spin homeomorphism from $B^1$ to itself which covers the identity
gives an order 2 automorphism of $c(\cX)^1$.

For 1-categories based on $\text{Pin}_-$ manifolds,
we have an order 4 antiautomorphism of $c(\cX)^1$.
For 1-categories based on $\text{Pin}_+$ manifolds,
we have an order 2 antiautomorphism and also an order 2 automorphism of $c(\cX)^1$,
and these two maps commute with each other.
%\nn{need to also consider automorphisms of $B^0$ / objects}

\noop{
\medskip

In the other direction, given a $1$-category $C$
(with objects $C^0$ and morphisms $C^1$) we will construct a topological
$1$-category $t(C)$.

If $X$ is a 0-ball (point), let $t(C)(X) \deq C^0$.
If $S$ is a 0-sphere, let $t(C)(S) \deq C^0\times C^0$.
If $X$ is a 1-ball, let $t(C)(X) \deq C^1$.
Homeomorphisms isotopic to the identity act trivially.
If $C$ has extra structure (e.g.\ it's a *-1-category), we use this structure
to define the action of homeomorphisms not isotopic to the identity
(and get, e.g., an unoriented topological 1-category).

The domain and range maps of $C$ determine the boundary and restriction maps of $t(C)$.

Gluing maps for $t(C)$ are determined by composition of morphisms in $C$.

For $X$ a 0-ball, $D$ a 1-ball and $a\in t(C)(X)$, define the product morphism 
$a\times D \deq \id_a$.
It is not hard to verify that this has the desired properties.

\medskip

The compositions of the constructions above, $$\cX\to c(\cX)\to t(c(\cX))$$ 
and $$C\to t(C)\to c(t(C)),$$ give back 
more or less exactly the same thing we started with.  

As we will see below, for $n>1$ the compositions yield a weaker sort of equivalence.
} %end \noop

\medskip

Similar arguments show that modules for topological 1-categories are essentially
the same thing as traditional modules for traditional 1-categories.


\subsection{Plain 2-categories}
\label{ssec:2-cats}
Let $\cC$ be a topological 2-category.
We will construct from $\cC$ a traditional pivotal 2-category.
(The ``pivotal" corresponds to our assumption of strong duality for $\cC$.)

We will try to describe the construction in such a way the the generalization to $n>2$ is clear,
though this will make the $n=2$ case a little more complicated than necessary.

Before proceeding, we must decide whether the 2-morphisms of our
pivotal 2-category are shaped like rectangles or bigons.
Each approach has advantages and disadvantages.
For better or worse, we choose bigons here.

Define the $k$-morphisms $C^k$ of $C$ to be $\cC(B^k)_E$, where $B^k$ denotes the standard
$k$-ball, which we also think of as the standard bihedron (a.k.a.\ globe).
(For $k=1$ this is an interval, and for $k=2$ it is a bigon.)
Since we are thinking of $B^k$ as a bihedron, we have a standard decomposition of the $\bd B^k$
into two copies of $B^{k-1}$ which intersect along the ``equator" $E \cong S^{k-2}$.
Recall that the subscript in $\cC(B^k)_E$ means that we consider the subset of $\cC(B^k)$
whose boundary is splittable along $E$.
This allows us to define the domain and range of morphisms of $C$ using
boundary and restriction maps of $\cC$.

Choosing a homeomorphism $B^1\cup B^1 \to B^1$ defines a composition map on $C^1$.
This is not associative, but we will see later that it is weakly associative.

Choosing a homeomorphism $B^2\cup B^2 \to B^2$ defines a ``vertical" composition map 
on $C^2$ (Figure \ref{fzo1}).
Isotopy invariance implies that this is associative.
We will define a ``horizontal" composition later.

\begin{figure}[t]
\begin{equation*}
\mathfig{.73}{tempkw/zo1}
\end{equation*}
\caption{Vertical composition of 2-morphisms}
\label{fzo1}
\end{figure}

Given $a\in C^1$, define $\id_a = a\times I \in C^2$ (pinched boundary).
Extended isotopy invariance for $\cC$ shows that this morphism is an identity for 
vertical composition.

Given $x\in C^0$, define $\id_x = x\times B^1 \in C^1$.
We will show that this 1-morphism is a weak identity.
This would be easier if our 2-morphisms were shaped like rectangles rather than bigons.

In showing that identity 1-morphisms have the desired properties, we will
rely heavily on the extended isotopy invariance of 2-morphisms in $\cC$.
This means we are free to add or delete product regions from 2-morphisms.

Let $a: y\to x$ be a 1-morphism.
Define 2-morphsims $a \to a\bullet \id_x$ and $a\bullet \id_x \to a$
as shown in Figure \ref{fzo2}.
\begin{figure}[t]
\begin{equation*}
\mathfig{.73}{tempkw/zo2}
\end{equation*}
\caption{blah blah}
\label{fzo2}
\end{figure}
As suggested by the figure, these are two different reparameterizations
of a half-pinched version of $a\times I$.
We must show that the two compositions of these two maps give the identity 2-morphisms
on $a$ and $a\bullet \id_x$, as defined above.
Figure \ref{fzo3} shows one case.
\begin{figure}[t]
\begin{equation*}
\mathfig{.83}{tempkw/zo3}
\end{equation*}
\caption{blah blah}
\label{fzo3}
\end{figure}
In the first step we have inserted a copy of $(x\times I)\times I$.
Figure \ref{fzo4} shows the other case.
\begin{figure}[t]
\begin{equation*}
\mathfig{.83}{tempkw/zo4}
\end{equation*}
\caption{blah blah}
\label{fzo4}
\end{figure}
We identify a product region and remove it.

We define horizontal composition of 2-morphisms as shown in Figure \ref{fzo5}.
It is not hard to show that this is independent of the arbitrary (left/right) 
choice made in the definition, and that it is associative.
\begin{figure}[t]
\begin{equation*}
\mathfig{.83}{tempkw/zo5}
\end{equation*}
\caption{Horizontal composition of 2-morphisms}
\label{fzo5}
\end{figure}

%\nn{need to find a list of axioms for pivotal 2-cats to check}


\subsection{$A_\infty$ $1$-categories}
\label{sec:comparing-A-infty}
In this section, we make contact between the usual definition of an $A_\infty$ category 
and our definition of a topological $A_\infty$ $1$-category, from \S \ref{ss:n-cat-def}.

\medskip

Given a topological $A_\infty$ $1$-category $\cC$, we define an ``$m_k$-style" 
$A_\infty$ $1$-category $A$ as follows.
The objects of $A$ are $\cC(pt)$.
The morphisms of $A$, from $x$ to $y$, are $\cC(I; x, y)$
($\cC$ applied to the standard interval with boundary labeled by $x$ and $y$).
For simplicity we will now assume there is only one object and suppress it from the notation.

A choice of homeomorphism $I\cup I \to I$ induces a chain map $m_2: A\times A\to A$.
We now have two different homeomorphisms $I\cup I\cup I \to I$, but they are isotopic.
Choose a specific 1-parameter family of homeomorphisms connecting them; this induces
a degree 1 chain homotopy $m_3:A\ot A\ot A\to A$.
Proceeding in this way we define the rest of the $m_i$'s.
It is straightforward to verify that they satisfy the necessary identities.

\medskip

In the other direction, we start with an alternative conventional definition of an $A_\infty$ algebra:
an algebra $A$ for the $A_\infty$ operad.
(For simplicity, we are assuming our $A_\infty$ 1-category has only one object.)
We are free to choose any operad with contractible spaces, so we choose the operad
whose $k$-th space is the space of decompositions of the standard interval $I$ into $k$
parameterized copies of $I$.
Note in particular that when $k=1$ this implies a $C_*(\Homeo(I))$ action on $A$.
(Compare with Example \ref{ex:e-n-alg} and the discussion which precedes it.)
Given a non-standard interval $J$, we define $\cC(J)$ to be
$(\Homeo(I\to J) \times A)/\Homeo(I\to I)$,
where $\beta \in \Homeo(I\to I)$ acts via $(f, a) \mapsto (f\circ \beta\inv, \beta_*(a))$.
\nn{check this}
Note that $\cC(J) \cong A$ (non-canonically) for all intervals $J$.
We define a $\Homeo(J)$ action on $\cC(J)$ via $g_*(f, a) = (g\circ f, a)$.
The $C_*(\Homeo(J))$ action is defined similarly.

Let $J_1$ and $J_2$ be intervals.
We must define a map $\cC(J_1)\ot\cC(J_2)\to\cC(J_1\cup J_2)$.
Choose a homeomorphism $g:I\to J_1\cup J_2$.
Let $(f_i, a_i)\in \cC(J_i)$.
We have a parameterized decomposition of $I$ into two intervals given by
$g\inv \circ f_i$, $i=1,2$.
Corresponding to this decomposition the operad action gives a map $\mu: A\ot A\to A$.
Define the gluing map to send $(f_1, a_1)\ot (f_2, a_2)$ to $(g, \mu(a_1\ot a_2))$.
Operad associativity for $A$ implies that this gluing map is independent of the choice of
$g$ and the choice of representative $(f_i, a_i)$.

It is straightforward to verify the remaining axioms for a topological $A_\infty$ 1-category.







\noop { %%%%%%%%%%%%%%%%%%%%%%%%%%%%%%%%%%

That definition associates a chain complex to every interval, and we begin by giving an alternative definition that is entirely in terms of the chain complex associated to the standard interval $[0,1]$. 
\begin{defn}
A \emph{topological $A_\infty$ category on $[0,1]$} $\cC$ has a set of objects $\Obj(\cC)$, 
and for each $a,b \in \Obj(\cC)$, a chain complex $\cC_{a,b}$, along with
\begin{itemize}
\item an action of the operad of $\Obj(\cC)$-labeled cell decompositions
\item and a compatible action of $\CD{[0,1]}$.
\end{itemize}
\end{defn}
Here the operad of cell decompositions of $[0,1]$ has operations indexed by a finite set of 
points $0 < x_1< \cdots < x_k < 1$, cutting $[0,1]$ into subintervals.
An $X$-labeled cell decomposition labels $\{0, x_1, \ldots, x_k, 1\}$ by $X$.
Given two cell decompositions $\cJ^{(1)}$ and $\cJ^{(2)}$, and an index $m$, we can compose 
them to form a new cell decomposition $\cJ^{(1)} \circ_m \cJ^{(2)}$ by inserting the points 
of $\cJ^{(2)}$ linearly into the $m$-th interval of $\cJ^{(1)}$.
In the $X$-labeled case, we insist that the appropriate labels match up.
Saying we have an action of this operad means that for each labeled cell decomposition 
$0 < x_1< \cdots < x_k < 1$, $a_0, \ldots, a_{k+1} \subset \Obj(\cC)$, there is a chain 
map $$\cC_{a_0,a_1} \tensor \cdots \tensor \cC_{a_k,a_{k+1}} \to \cC_{a_0,a_{k+1}}$$ and these 
chain maps compose exactly as the cell decompositions.
An action of $\CD{[0,1]}$ is compatible with an action of the cell decomposition operad 
if given a decomposition $\pi$, and a family of diffeomorphisms $f \in \CD{[0,1]}$ which 
is supported on the subintervals determined by $\pi$, then the two possible operations 
(glue intervals together, then apply the diffeomorphisms, or apply the diffeormorphisms 
separately to the subintervals, then glue) commute (as usual, up to a weakly unique homotopy).

Translating between this notion and the usual definition of an $A_\infty$ category is now straightforward.
To restrict to the standard interval, define $\cC_{a,b} = \cC([0,1];a,b)$.
Given a cell decomposition $0 < x_1< \cdots < x_k < 1$, we use the map (suppressing labels)
$$\cC([0,1])^{\tensor k+1} \to \cC([0,x_1]) \tensor \cdots \tensor \cC[x_k,1] \to \cC([0,1])$$
where the factors of the first map are induced by the linear isometries $[0,1] \to [x_i, x_{i+1}]$, and the second map is just gluing.
The action of $\CD{[0,1]}$ carries across, and is automatically compatible.
Going the other way, we just declare $\cC(J;a,b) = \cC_{a,b}$, pick a diffeomorphism 
$\phi_J : J \isoto [0,1]$ for every interval $J$, define the gluing map 
$\cC(J_1) \tensor \cC(J_2) \to \cC(J_1 \cup J_2)$ by the first applying 
the cell decomposition map for $0 < \frac{1}{2} < 1$, then the self-diffeomorphism of $[0,1]$ 
given by $\frac{1}{2} (\phi_{J_1} \cup (1+ \phi_{J_2})) \circ \phi_{J_1 \cup J_2}^{-1}$.
You can readily check that this gluing map is associative on the nose. \todo{really?}

%First recall the \emph{coloured little intervals operad}. Given a set of labels $\cL$, the operations are indexed by \emph{decompositions of the interval}, each of which is a collection of disjoint subintervals $\{(a_i,b_i)\}_{i=1}^k$ of $[0,1]$, along with a labeling of the complementary regions by $\cL$, $\{l_0, \ldots, l_k\}$.  Given two decompositions $\cJ^{(1)}$ and $\cJ^{(2)}$, and an index $m$ such that $l^{(1)}_{m-1} = l^{(2)}_0$ and $l^{(1)}_{m} = l^{(2)}_{k^{(2)}}$, we can form a new decomposition by inserting the intervals of $\cJ^{(2)}$ linearly inside the $m$-th interval of $\cJ^{(1)}$. We call the resulting decomposition $\cJ^{(1)} \circ_m \cJ^{(2)}$.

%\begin{defn}
%A \emph{topological $A_\infty$ category} $\cC$ has a set of objects $\Obj(\cC)$ and for each $a,b \in \Obj(\cC)$ a chain complex $\cC_{a,b}$, along with a compatible `composition map' and an `action of families of diffeomorphisms'.

%A \emph{composition map} $f$ is a family of chain maps, one for each decomposition of the interval, $f_\cJ : A^{\tensor k} \to A$, making $\cC$ into a category over the coloured little intervals operad, with labels $\cL = \Obj(\cC)$. Thus the chain maps satisfy the identity 
%\begin{equation*}
%f_{\cJ^{(1)} \circ_m \cJ^{(2)}} = f_{\cJ^{(1)}} \circ (\id^{\tensor m-1} \tensor f_{\cJ^{(2)}} \tensor \id^{\tensor k^{(1)} - m}).
%\end{equation*}

%An \emph{action of families of diffeomorphisms} is a chain map $ev: \CD{[0,1]} \tensor A \to A$, such that 
%\begin{enumerate}
%\item The diagram 
%\begin{equation*}
%\xymatrix{
%\CD{[0,1]} \tensor \CD{[0,1]} \tensor A \ar[r]^{\id \tensor ev} \ar[d]^{\circ \tensor \id} & \CD{[0,1]} \tensor A \ar[d]^{ev} \\
%\CD{[0,1]} \tensor A \ar[r]^{ev} & A
%}
%\end{equation*}
%commutes up to weakly unique homotopy.
%\item If $\phi \in \Diff([0,1])$ and $\cJ$ is a decomposition of the interval, we obtain a new decomposition $\phi(\cJ)$ and a collection $\phi_m \in \Diff([0,1])$ of diffeomorphisms obtained by taking the restrictions $\restrict{\phi}{[a_m,b_m]} : [a_m,b_m] \to [\phi(a_m),\phi(b_m)]$ and pre- and post-composing these with the linear diffeomorphisms $[0,1] \to [a_m,b_m]$ and $[\phi(a_m),\phi(b_m)] \to [0,1]$. We require that
%\begin{equation*}
%\phi(f_\cJ(a_1, \cdots, a_k)) = f_{\phi(\cJ)}(\phi_1(a_1), \cdots, \phi_k(a_k)).
%\end{equation*}
%\end{enumerate}
%\end{defn}

From a topological $A_\infty$ category on $[0,1]$ $\cC$ we can produce a `conventional' 
$A_\infty$ category $(A, \{m_k\})$ as defined in, for example, \cite{MR1854636}.
We'll just describe the algebra case (that is, a category with only one object), 
as the modifications required to deal with multiple objects are trivial.
Define $A = \cC$ as a chain complex (so $m_1 = d$).
Define $m_2 : A\tensor A \to A$ by $f_{\{(0,\frac{1}{2}),(\frac{1}{2},1)\}}$.
To define $m_3$, we begin by taking the one parameter family $\phi_3$ of diffeomorphisms 
of $[0,1]$ that interpolates linearly between the identity and the piecewise linear 
diffeomorphism taking $\frac{1}{4}$ to $\frac{1}{2}$ and $\frac{1}{2}$ to $\frac{3}{4}$, and then define
\begin{equation*}
m_3(a,b,c) = ev(\phi_3, m_2(m_2(a,b), c)).
\end{equation*}

It's then easy to calculate that
\begin{align*}
d(m_3(a,b,c)) & = ev(d \phi_3, m_2(m_2(a,b),c)) - ev(\phi_3 d m_2(m_2(a,b), c)) \\
 & = ev( \phi_3(1), m_2(m_2(a,b),c)) - ev(\phi_3(0), m_2 (m_2(a,b),c)) - \\ & \qquad - ev(\phi_3, m_2(m_2(da, b), c) + (-1)^{\deg a} m_2(m_2(a, db), c) + \\ & \qquad \quad + (-1)^{\deg a+\deg b} m_2(m_2(a, b), dc) \\
 & = m_2(a , m_2(b,c)) - m_2(m_2(a,b),c) - \\ & \qquad - m_3(da,b,c) + (-1)^{\deg a + 1} m_3(a,db,c) + \\ & \qquad \quad + (-1)^{\deg a + \deg b + 1} m_3(a,b,dc), \\
\intertext{and thus that}
m_1 \circ m_3 & =  m_2 \circ (\id \tensor m_2) - m_2 \circ (m_2 \tensor \id) - \\ & \qquad - m_3 \circ (m_1 \tensor \id \tensor \id) - m_3 \circ (\id \tensor m_1 \tensor \id) - m_3 \circ (\id \tensor \id \tensor m_1)
\end{align*}
as required (c.f. \cite[p. 6]{MR1854636}).
\todo{then the general case.}
We won't describe a reverse construction (producing a topological $A_\infty$ category 
from a ``conventional" $A_\infty$ category), but we presume that this will be easy for the experts.

} %%%%% end \noop %%%%%%%%%%%%%%%%%%%%%%%%%%%%%%%%%

%%!TEX root = ../blob1.tex

\section{Commutative algebras as $n$-categories}
\label{sec:comm_alg}

\nn{this should probably not be a section by itself.  i'm just trying to write down the outline 
while it's still fresh in my mind.}

\nn{I strongly suspect that [blob complex
for $M^n$ based on comm alg $C$ thought of as an $n$-category]
is homotopy equivalent to [higher Hochschild complex for $M^n$ with coefficients in $C$].
(Thomas Tradler's idea.)
Should prove (or at least conjecture) that here.}

\nn{also, this section needs a little updating to be compatible with the rest of the paper.}

If $C$ is a commutative algebra it
can also be thought of as an $n$-category whose $j$-morphisms are trivial for
$j<n$ and whose $n$-morphisms are $C$. 
The goal of this \nn{subsection?} is to compute
$\bc_*(M^n, C)$ for various commutative algebras $C$.

\medskip

Let $k[t]$ denote the ring of polynomials in $t$ with coefficients in $k$.

Let $\Sigma^i(M)$ denote the $i$-th symmetric power of $M$, the configuration space of $i$
unlabeled points in $M$.
Note that $\Sigma^0(M)$ is a point.
Let $\Sigma^\infty(M) = \coprod_{i=0}^\infty \Sigma^i(M)$.

Let $C_*(X, k)$ denote the singular chain complex of the space $X$ with coefficients in $k$.

\begin{prop} \label{sympowerprop}
$\bc_*(M, k[t])$ is homotopy equivalent to $C_*(\Sigma^\infty(M), k)$.
\end{prop}

\begin{proof}
To define the chain maps between the two complexes we will use the following lemma:

\begin{lemma}
Let $A_*$ and $B_*$ be chain complexes, and assume $A_*$ is equipped with
a basis (e.g.\ blob diagrams or singular simplices).
For each basis element $c \in A_*$ assume given a contractible subcomplex $R(c)_* \sub B_*$
such that $R(c')_* \sub R(c)_*$ whenever $c'$ is a basis element which is part of $\bd c$.
Then the complex of chain maps (and (iterated) homotopies) $f:A_*\to B_*$ such that
$f(c) \in R(c)_*$ for all $c$ is contractible (and in particular non-empty).
\end{lemma}

\begin{proof}
\nn{easy, but should probably write the details eventually}
\nn{this is just the standard ``method of acyclic models" set up, so we should just give a reference for that}
\end{proof}

Our first task: For each blob diagram $b$ define a subcomplex $R(b)_* \sub C_*(\Sigma^\infty(M))$
satisfying the conditions of the above lemma.
If $b$ is a 0-blob diagram, then it is just a $k[t]$ field on $M$, which is a 
finite unordered collection of points of $M$ with multiplicities, which is
a point in $\Sigma^\infty(M)$.
Define $R(b)_*$ to be the singular chain complex of this point.
If $(B, u, r)$ is an $i$-blob diagram, let $D\sub M$ be its support (the union of the blobs).
The path components of $\Sigma^\infty(D)$ are contractible, and these components are indexed 
by the numbers of points in each component of $D$.
We may assume that the blob labels $u$ have homogeneous $t$ degree in $k[t]$, and so
$u$ picks out a component $X \sub \Sigma^\infty(D)$.
The field $r$ on $M\setminus D$ can be thought of as a point in $\Sigma^\infty(M\setminus D)$,
and using this point we can embed $X$ in $\Sigma^\infty(M)$.
Define $R(B, u, r)_*$ to be the singular chain complex of $X$, thought of as a 
subspace of $\Sigma^\infty(M)$.
It is easy to see that $R(\cdot)_*$ satisfies the condition on boundaries from the above lemma.
Thus we have defined (up to homotopy) a map from 
$\bc_*(M^n, k[t])$ to $C_*(\Sigma^\infty(M))$.

Next we define, for each simplex $c$ of $C_*(\Sigma^\infty(M))$, a contractible subspace
$R(c)_* \sub \bc_*(M^n, k[t])$.
If $c$ is a 0-simplex we use the identification of the fields $\cC(M)$ and 
$\Sigma^\infty(M)$ described above.
Now let $c$ be an $i$-simplex of $\Sigma^j(M)$.
Choose a metric on $M$, which induces a metric on $\Sigma^j(M)$.
We may assume that the diameter of $c$ is small --- that is, $C_*(\Sigma^j(M))$
is homotopy equivalent to the subcomplex of small simplices.
How small?  $(2r)/3j$, where $r$ is the radius of injectivity of the metric.
Let $T\sub M$ be the ``track" of $c$ in $M$.
\nn{do we need to define this precisely?}
Choose a neighborhood $D$ of $T$ which is a disjoint union of balls of small diameter.
\nn{need to say more precisely how small}
Define $R(c)_*$ to be $\bc_*(D, k[t]) \sub \bc_*(M^n, k[t])$.
This is contractible by \ref{bcontract}.
We can arrange that the boundary/inclusion condition is satisfied if we start with
low-dimensional simplices and work our way up.
\nn{need to be more precise}

\nn{still to do: show indep of choice of metric; show compositions are homotopic to the identity
(for this, might need a lemma that says we can assume that blob diameters are small)}
\end{proof}


\begin{prop} \label{ktcdprop}
The above maps are compatible with the evaluation map actions of $C_*(\Diff(M))$.
\end{prop}

\begin{proof}
The actions agree in degree 0, and both are compatible with gluing.
(cf. uniqueness statement in \ref{CDprop}.)
\end{proof}

\medskip

In view of \ref{hochthm}, we have proved that $HH_*(k[t]) \cong C_*(\Sigma^\infty(S^1), k)$,
and that the cyclic homology of $k[t]$ is related to the action of rotations
on $C_*(\Sigma^\infty(S^1), k)$.
\nn{probably should put a more precise statement about cyclic homology and $S^1$ actions in the Hochschild section}
Let us check this directly.

According to \cite[3.2.2]{MR1600246}, $HH_i(k[t]) \cong k[t]$ for $i=0,1$ and is zero for $i\ge 2$.
\nn{say something about $t$-degree?  is this in Loday?}

We can define a flow on $\Sigma^j(S^1)$ by having the points repel each other.
The fixed points of this flow are the equally spaced configurations.
This defines a map from $\Sigma^j(S^1)$ to $S^1/j$ ($S^1$ modulo a $2\pi/j$ rotation).
The fiber of this map is $\Delta^{j-1}$, the $(j-1)$-simplex, 
and the holonomy of the $\Delta^{j-1}$ bundle
over $S^1/j$ is induced by the cyclic permutation of its $j$ vertices.

In particular, $\Sigma^j(S^1)$ is homotopy equivalent to a circle for $j>0$, and
of course $\Sigma^0(S^1)$ is a point.
Thus the singular homology $H_i(\Sigma^\infty(S^1))$ has infinitely many generators for $i=0,1$
and is zero for $i\ge 2$.
\nn{say something about $t$-degrees also matching up?}

By xxxx and \ref{ktcdprop}, 
the cyclic homology of $k[t]$ is the $S^1$-equivariant homology of $\Sigma^\infty(S^1)$.
Up to homotopy, $S^1$ acts by $j$-fold rotation on $\Sigma^j(S^1) \simeq S^1/j$.
If $k = \z$, $\Sigma^j(S^1)$ contributes the homology of an infinite lens space: $\z$ in degree
0, $\z/j \z$ in odd degrees, and 0 in positive even degrees.
The point $\Sigma^0(S^1)$ contributes the homology of $BS^1$ which is $\z$ in even 
degrees and 0 in odd degrees.
This agrees with the calculation in \cite[3.1.7]{MR1600246}.

\medskip

Next we consider the case $C = k[t_1, \ldots, t_m]$, commutative polynomials in $m$ variables.
Let $\Sigma_m^\infty(M)$ be the $m$-colored infinite symmetric power of $M$, that is, configurations
of points on $M$ which can have any of $m$ distinct colors but are otherwise indistinguishable.
The components of $\Sigma_m^\infty(M)$ are indexed by $m$-tuples of natural numbers
corresponding to the number of points of each color of a configuration.
A proof similar to that of \ref{sympowerprop} shows that

\begin{prop}
$\bc_*(M, k[t_1, \ldots, t_m])$ is homotopy equivalent to $C_*(\Sigma_m^\infty(M), k)$.
\end{prop}

According to \cite[3.2.2]{MR1600246},
\[
	HH_n(k[t_1, \ldots, t_m]) \cong \Lambda^n(k^m) \otimes k[t_1, \ldots, t_m] .
\]
Let us check that this is also the singular homology of $\Sigma_m^\infty(S^1)$.
We will content ourselves with the case $k = \z$.
One can define a flow on $\Sigma_m^\infty(S^1)$ where points of the same color repel each other and points of different colors do not interact.
This shows that a component $X$ of $\Sigma_m^\infty(S^1)$ is homotopy equivalent
to the torus $(S^1)^l$, where $l$ is the number of non-zero entries in the $m$-tuple
corresponding to $X$.
The homology calculation we desire follows easily from this.

\nn{say something about cyclic homology in this case?  probably not necessary.}

\medskip

Next we consider the case $C$ is the truncated polynomial
algebra $k[t]/t^l$ --- polynomials in $t$ with $t^l = 0$.
Define $\Delta_l \sub \Sigma^\infty(M)$ to be configurations of points in $M$ with $l$ or
more of the points coinciding.

\begin{prop}
$\bc_*(M, k[t]/t^l)$ is homotopy equivalent to $C_*(\Sigma^\infty(M), \Delta_l, k)$
(relative singular chains with coefficients in $k$).
\end{prop}

\begin{proof}
\nn{...}
\end{proof}

\medskip
\hrule
\medskip

Still to do:
\begin{itemize}
\item compare the topological computation for truncated polynomial algebra with \cite{MR1600246}
\item multivariable truncated polynomial algebras (at least mention them)
\item ideally, say something more about higher hochschild homology (maybe sketch idea for proof of equivalence)
\end{itemize}



% ----------------------------------------------------------------
%\newcommand{\urlprefix}{}
\bibliographystyle{plain}
\bibliography{bibliography/bibliography}
% ----------------------------------------------------------------

This paper is available online at \arxiv{?????}, and at
\url{http://tqft.net/blobs},
and at \url{http://canyon23.net/math/}.

% ----------------------------------------------------------------
\end{document}
% ----------------------------------------------------------------

