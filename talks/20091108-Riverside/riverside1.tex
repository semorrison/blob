% use options
%  '[beamer]' for a digital projector
%  '[trans]' for an overhead projector
%  '[handout]' for 4-up printed notes
\documentclass[beamer]{beamer}

% change talk_preamble if you want to modify the slide theme, colours, and settings for trans and handout modes.
\newcommand{\pathtotrunk}{../../}
\input{\pathtotrunk preamble.tex}
%!TEX root = ../blob1.tex

%%%%% excerpts from KW's include file of standard macros
%%% (with various new ones added)

\def\z{\mathbb{Z}}
\def\r{\mathbb{R}}
\def\c{\mathbb{C}}
\def\t{\mathbb{T}}
\def\ebb{\mathbb{E}}

\def\du{\sqcup}
\def\bd{\partial}
\def\sub{\subset}
\def\subeq{\subseteq}
\def\sup{\supset}
%\def\setmin{\smallsetminus}
\def\setmin{\setminus}
\def\ep{\epsilon}
\def\sgl{_\mathrm{gl}}
\def\op{^\mathrm{op}}
\def\deq{\stackrel{\mathrm{def}}{=}}
\def\pd#1#2{\frac{\partial #1}{\partial #2}}
\def\lf{\overline{\cC}}
\def\ot{\otimes}
\def\inv{^{-1}}

\def\spl{_\pitchfork}

%\def\nn#1{{{\it \small [#1]}}}
\def\nn#1{{{\color[rgb]{.2,.5,.6} \small [#1]}}}
\long\def\noop#1{}

% equations
\newcommand{\eq}[1]{\begin{displaymath}#1\end{displaymath}}
\newcommand{\eqar}[1]{\begin{eqnarray*}#1\end{eqnarray*}}
\newcommand{\eqspl}[1]{\begin{displaymath}\begin{split}#1\end{split}\end{displaymath}}

% tricky way to iterate macros over a list
\def\semicolon{;}
\def\applytolist#1{
    \expandafter\def\csname multi#1\endcsname##1{
        \def\multiack{##1}\ifx\multiack\semicolon
            \def\next{\relax}
        \else
            \csname #1\endcsname{##1}
            \def\next{\csname multi#1\endcsname}
        \fi
        \next}
    \csname multi#1\endcsname}

% \def\cA{{\cal A}} for A..Z
\def\calc#1{\expandafter\def\csname c#1\endcsname{{\mathcal #1}}}
\applytolist{calc}QWERTYUIOPLKJHGFDSAZXCVBNM;

% \DeclareMathOperator{\pr}{pr} etc.
\def\declaremathop#1{\expandafter\DeclareMathOperator\csname #1\endcsname{#1}}
\applytolist{declaremathop}{pr}{im}{gl}{ev}{coinv}{tr}{rot}{Eq}{obj}{mor}{ob}{Rep}{Tet}{cat}{Maps}{Diff}{Homeo}{sign}{supp}{Nbd}{res};


%%%%%% end excerpt



\usepackage{etex}
\usepackage{pgfpages}

\usepackage{color}

\usepackage{tikz}
\usetikzlibrary{shapes}
\usetikzlibrary{backgrounds}

% beamer mode
\mode<beamer>{
\useinnertheme[shadow=true]{rounded}
\useoutertheme{shadow}
\usecolortheme{orchid}
\usecolortheme{whale}
\setbeamerfont{block title}{size={}}

\setbeamertemplate{headline}
{%
}

\setbeamertemplate{footline}
{%
  \leavevmode%
  \hbox{\begin{beamercolorbox}[wd=.25\paperwidth,ht=2.5ex,dp=1.125ex,leftskip=.3cm,rightskip=.3cm]{author in head/foot}%
    \usebeamerfont{author in head/foot}\insertshortauthor
  \end{beamercolorbox}%
  \begin{beamercolorbox}[wd=.25\paperwidth,ht=2.5ex,dp=1.125ex,leftskip=.3cm,rightskip=.3cm]{title in head/foot}%
    \usebeamerfont{title in head/foot}\insertshorttitle
  \end{beamercolorbox}}%
    \begin{beamercolorbox}[wd=.5\paperwidth,ht=2.5ex,dp=1.125ex]{section in head/foot}%
    \insertsectionnavigationhorizontal{.5\paperwidth}{}{\hskip0pt plus1filll}%
  \end{beamercolorbox}%
  \vskip0pt%
}

}



% transparency mode
\mode<trans>{
 \usetheme{Warsaw}
}

% handout mode
\mode<handout>{
 \usetheme{default}
 \setbeamercolor{background canvas}{bg=black!5}
 \pgfpagesuselayout{4 on 1}[letterpaper,landscape,border shrink=2.5mm]
}

\newcommand{\return}[2]{\hyperlink{#1}{\beamerreturnbutton{#2}}}
\newcommand{\goto}[2]{\hyperlink{#1}{\beamergotobutton{#2}}}
\newcommand{\skipto}[2]{\hyperlink{#1}{\beamerskipbutton{#2}}}

\beamertemplatetransparentcovered 
\setbeamertemplate{navigation symbols}{}  % no navigation symbols, please
\mode<beamer>{\setbeamercolor{block title}{bg=green!40!black}}
\beamersetuncovermixins 
{\opaqueness<1->{60}} 
{} 


% This switches fonts to the Palatino family.
%  \renewcommand{\familydefault}{ppl}

\usepackage{array}


%\setbeameroption{previous slide on second screen=right}

\author[Scott Morrison]{Scott Morrison \\ \texttt{http://tqft.net/} \\ joint work with Kevin Walker}
\institute{UC Berkeley / Miller Institute for Basic Research}
\title{Blob homology, part $\mathbb{I}$}
\date{Homotopy Theory and Higher Algebraic Structures, UC Riverside, November 10 2009 \\ \url{http://tqft.net/UCR-blobs1}}

\begin{document}

\frame{\titlepage}

\begin{frame}
       \frametitle{Outline}
       \tableofcontents
\end{frame}

\beamertemplatetransparentcovered 

\mode<beamer>{\setbeamercolor{block title}{bg=green!40!black}}

\beamersetuncovermixins 
{\opaqueness<1->{60}} 
{} 



\section{Overview}

\AtBeginSection[]
{
   \begin{frame}<beamer>
       \frametitle{Outline}
       \tableofcontents[currentsection]
   \end{frame}
}

\begin{frame}{What is \emph{blob homology}?}
\begin{block}{}
The blob complex takes an $n$-manifold $\cM$ and an `$n$-category with strong duality' $\cC$ and produces a chain complex, $\bc_*(\cM; \cC)$.
\end{block}
\tikzstyle{description}=[gray, font=\tiny, text centered, text width=2cm]
\begin{tikzpicture}[]
\setbeamercovered{%
 transparent=5,
% still covered={\opaqueness<1>{15}\opaqueness<2>{10}\opaqueness<3>{5}\opaqueness<4->{2}},
 again covered={\opaqueness<1->{50}}
}

\node[red] (blobs) at (0,0) {$H(\bc_*(\cM; \cC))$};
\uncover<1>{
\node[blue] (skein) at (4,0) {$A(\cM; \cC)$};
\node[below=5pt, description] (skein-label) at (skein) {(the usual TQFT Hilbert space)};
\path[->](blobs) edge node[above] {$*= 0$} (skein);
}

\uncover<2>{
  \node[blue] (hoch) at (0,3) {$HH_*(\cC)$};
  \node[right=20pt, description] (hoch-label) at (hoch) {(the Hochschild homology)};
  \path[->](blobs) edge node[right] {$\cM = S^1$} (hoch);
}

\uncover<3>{
  \node[blue] (comm) at (-2.4, -1.8) {$H_*(\Delta^\infty(\cM), k)$};
  \node[description, below=5pt] (comm-label) at (comm) {(singular homology of the infinite configuration space)};
  \path[->](blobs) edge node[right=5pt] {$\cC = k[t]$} (comm);
}

\end{tikzpicture}
\end{frame}

\begin{frame}{$n$-categories}
\begin{block}{Defining $n$-categories is fraught with difficulties}
I'm not going to go into details; I'll draw $2$-dimensional pictures, and rely on your intuition for pivotal $2$-categories.
\end{block}
\begin{block}{}
\begin{itemize}
\item
Kevin's talk (part $\mathbb{II}$) will explain the notions of `topological $n$-categories' and `$A_\infty$ $n$-categories'.\item
Defining $n$-categories: a choice of `shape' for morphisms.
\item
We allow all shapes! A vector space for every ball.
\item
`Strong duality' is integral in our definition.
\end{itemize}
\end{block}
\end{frame}

\newcommand{\roundframe}[1]{\begin{tikzpicture}[baseline]\node[rectangle,inner sep=1pt,rounded corners,fill=white] {#1};\end{tikzpicture}}


\begin{frame}{Fields and pasting diagrams}
\begin{block}{Pasting diagrams}
Fix an $n$-category with strong duality $\cC$. A \emph{field} on $\cM$ is a pasting diagram drawn on $\cM$, with cells labelled by morphisms from $\cC$.
\end{block}
\begin{example}[$\cC = \text{TL}_d$ the Temperley-Lieb category]
$$\roundframe{\mathfig{0.35}{definition/example-pasting-diagram}} \in \cF^{\text{TL}_d}\left(T^2\right)$$
\end{example}
\begin{block}{}
Given a field on a ball, we can evaluate it to a morphism. We call the kernel the \emph{null fields}.
\vspace{-3mm}
$$\text{ev}\Bigg(\roundframe{\mathfig{0.12}{definition/evaluation1}} - \frac{1}{d}\roundframe{\mathfig{0.12}{definition/evaluation2}}\Bigg) = 0$$
\end{block}
\end{frame}

\begin{frame}{\emph{Definition} of the blob complex, $k=0,1$}
\begin{block}{Motivation}
A \emph{local} construction, such that when $\cM$ is a ball, $\bc_*(\cM; \cC)$ is a resolution of $A(\cM,; \cC)$.
\end{block}

\begin{block}{}
\center
$\bc_0(\cM; \cC) = \cF(\cM)$, arbitrary fields on $\cM$.
\end{block}

\begin{block}{}
\vspace{-1mm}
$$\bc_1(\cM; \cC) = \setc{(B, u, r)}{\begin{array}{c}\text{$B$ an embedded ball}\\\text{$u \in \cF(B)$ in the kernel}\\ r \in \cF(\cM \setminus B)\end{array}}.$$
\end{block}
\vspace{-3.5mm}
$$\mathfig{.5}{definition/single-blob}$$
\vspace{-3mm}
\begin{block}{}
\vspace{-6mm}
\begin{align*}
d_1 : (B, u, r) & \mapsto u \circ r & \bc_0 / \im(d_1) \iso A(\cM; \cC)
\end{align*}
\end{block}
\end{frame}

\begin{frame}{Definition, $k=2$}
\begin{block}{}
\vspace{-1mm}
$$\bc_2 = \bc_2^{\text{disjoint}} \oplus \bc_2^{\text{nested}}$$
\end{block}
\begin{block}{}
\vspace{-5mm}
\begin{align*}
\bc_2^{\text{disjoint}} & =  \roundframe{\mathfig{0.5}{definition/disjoint-blobs}} & u_i \in \ker{\text{ev}_{B_i}}
\end{align*}
\vspace{-4mm}
$$d_2 : (B_1, B_2, u_1, u_2, r) \mapsto (B_2, u_2, r \circ u_1) - (B_1, u_1, r \circ u_2)$$
\end{block}
\begin{block}{}
\vspace{-5mm}
\begin{align*}
\bc_2^{\text{nested}} & = \roundframe{\mathfig{0.5}{definition/nested-blobs}} & u \in \ker{\text{ev}_{B_1}}
\end{align*}
\vspace{-4mm}
$$d_2 : (B_1, B_2, u, r', r) \mapsto (B_2, u \circ r', r) - (B_1, u, r \circ r')$$
\end{block}
\end{frame}

\end{document}
% ----------------------------------------------------------------

