%auto-ignore
%!TEX root = pnas.tex

\def\bc{{\mathcal B}}
\def\btc{{\mathcal{BT}}}

\newcommand{\HC}{\operatorname{Hoch}}
\newcommand{\HH}{\operatorname{HH}}

\newcommand{\CM}[2]{C_*(\Maps(#1 \to #2))}
\newcommand{\CD}[1]{C_*(\Diff(#1))}
\newcommand{\CH}[1]{CH_*(#1)}

\newcommand{\cl}[1]{\underrightarrow{#1}}

\newcommand{\Set}{\text{\textbf{Set}}}
\newcommand{\Vect}{\text{\textbf{Vect}}}
\newcommand{\Kom}{\text{\textbf{Kom}}}
\newcommand{\Cat}{\mathcal{C}}

\newcommand{\cell}{\mathfrak{D}}

\newcommand{\into}{\hookrightarrow}
\newcommand{\onto}{\twoheadrightarrow}
\newcommand{\iso}{\cong}
\newcommand{\quism}{\underset{\text{q.i.}}{\simeq}}
\newcommand{\htpy}{\simeq}
\newcommand{\actsOn}{\circlearrowright}
\newcommand{\xto}[1]{\xrightarrow{#1}}
\newcommand{\isoto}{\xto{\iso}}
\newcommand{\quismto}{\xrightarrow[\text{q.i.}]{\iso}}
\newcommand{\diffeoto}{\xrightarrow[\text{diffeo}]{\iso}}
\newcommand{\htpyto}{\xrightarrow[\text{htpy}]{\htpy}}

\newcommand{\directSum}{\oplus}
\newcommand{\DirectSum}{\bigoplus}
\newcommand{\tensor}{\otimes}
\newcommand{\Tensor}{\bigotimes}

\newcommand{\selfarrow}{\ensuremath{\smash{\tikz[baseline]{\clip (0,0.36) rectangle (0.48,-0.16); \draw[->] (0,0.2) .. controls (0.6,0.8) and (0.6,-0.6) .. (0,0);}}}}

\newcommand{\bdy}{\partial}
\newcommand{\compose}{\circ}
\newcommand{\eset}{\emptyset}

\newcommand{\id}{\boldsymbol{1}}

\newtheorem{property}{Property}
\newtheorem{prop}{Proposition}
\newtheorem{thm}[prop]{Theorem}
\newtheorem{defn}[prop]{Definition}

\newenvironment{rem}{\noindent\textsl{Remark.}}{}

% \mathrlap -- a horizontal \smash--------------------------------
% For comparison, the existing overlap macros:
% \def\llap#1{\hbox to 0pt{\hss#1}}
% \def\rlap#1{\hbox to 0pt{#1\hss}}
\def\clap#1{\hbox to 0pt{\hss#1\hss}}
\def\mathllap{\mathpalette\mathllapinternal}
\def\mathrlap{\mathpalette\mathrlapinternal}
\def\mathclap{\mathpalette\mathclapinternal}
\def\mathllapinternal#1#2{%
\llap{$\mathsurround=0pt#1{#2}$}}
\def\mathrlapinternal#1#2{%
\rlap{$\mathsurround=0pt#1{#2}$}}
\def\mathclapinternal#1#2{%
\clap{$\mathsurround=0pt#1{#2}$}}

% references

\newcommand{\arxiv}[1]{\href{http://arxiv.org/abs/#1}{\tt arXiv:\nolinkurl{#1}}}
\newcommand{\doi}[1]{\href{http://dx.doi.org/#1}{{\tt DOI:#1}}}
\newcommand{\euclid}[1]{\href{http://projecteuclid.org/euclid.cmp/#1}{{\tt at Project Euclid: #1}}}
\newcommand{\mathscinet}[1]{\href{http://www.ams.org/mathscinet-getitem?mr=#1}{\tt #1}}
\newcommand{\googlebooks}[1]{(preview at \href{http://books.google.com/books?id=#1}{google books})}


% figures

\newcommand{\mathfig}[2]{\ensuremath{\hspace{-3pt}\begin{array}{c}%
  \raisebox{-2.5pt}{\includegraphics[width=#1\textwidth]{diagrams/#2}}%
\end{array}\hspace{-3pt}}}


% packages

\usepackage{tikz}
\usetikzlibrary{shapes}
\usetikzlibrary{backgrounds}
\usetikzlibrary{decorations,decorations.pathreplacing}
\usetikzlibrary{fit,calc,through}

\usepackage[all,color]{xy}
\SelectTips{cm}{}

\usepackage[pdftex,plainpages=false,hypertexnames=false,pdfpagelabels]{hyperref}

\usepackage{xcolor}
\definecolor{dark-red}{rgb}{0.7,0.25,0.25}
\definecolor{dark-blue}{rgb}{0.15,0.15,0.55}
\definecolor{medium-blue}{rgb}{0,0,0.65}

\hypersetup{
    colorlinks, linkcolor={dark-red},
    citecolor={dark-blue}, urlcolor={medium-blue}
}

