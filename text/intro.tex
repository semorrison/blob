%!TEX root = ../blob1.tex

\section{Introduction}

We construct a chain complex $\bc_*(M; \cC)$ --- the ``blob complex'' --- 
associated to an $n$-manifold $M$ and a linear $n$-category $\cC$ with strong duality.
This blob complex provides a simultaneous generalization of several well known constructions:
\begin{itemize}
\item The 0-th homology $H_0(\bc_*(M; \cC))$ is isomorphic to the usual 
topological quantum field theory invariant of $M$ associated to $\cC$.
(See Proposition \ref{thm:skein-modules} later in the introduction and \S \ref{sec:constructing-a-tqft}.)
\item When $n=1$ and $\cC$ is just a 1-category (e.g.\ an associative algebra), 
the blob complex $\bc_*(S^1; \cC)$ is quasi-isomorphic to the Hochschild complex $\HC_*(\cC)$.
(See Theorem \ref{thm:hochschild} and \S \ref{sec:hochschild}.)
%\item When $\cC$ is the polynomial algebra $k[t]$, thought of as an n-category, we have 
%that $\bc_*(M; k[t])$ is homotopy equivalent to $C_*(\Sigma^\infty(M), k)$, the singular chains
%on the configuration space of unlabeled points in $M$. (See \S \ref{sec:comm_alg}.)
\item When $\cC$ is $\pi^\infty_{\leq n}(T)$, the $A_\infty$ version of the fundamental $n$-groupoid of
the space $T$ (Example \ref{ex:chains-of-maps-to-a-space}), 
$\bc_*(M; \cC)$ is homotopy equivalent to $C_*(\Maps(M\to T))$,
the singular chains on the space of maps from $M$ to $T$.
(See Theorem \ref{thm:map-recon}.)
\end{itemize}

The blob complex definition is motivated by the desire for a derived analogue of the usual TQFT Hilbert space 
(replacing the quotient of fields by local relations with some sort of resolution), 
and for a generalization of Hochschild homology to higher $n$-categories.
One can think of it as the push-out of these two familiar constructions.
More detailed motivations are described in \S \ref{sec:motivations}.

The blob complex has good formal properties, summarized in \S \ref{sec:properties}.
These include an action of $\CH{M}$, 
extending the usual $\Homeo(M)$ action on the TQFT space $H_0$ (Theorem \ref{thm:evaluation}) and a gluing 
formula allowing calculations by cutting manifolds into smaller parts (Theorem \ref{thm:gluing}).

We expect applications of the blob complex to contact topology and Khovanov homology 
but do not address these in this paper.
%See \S \ref{sec:future} for slightly more detail.

Throughout, we have resisted the temptation to work in the greatest possible generality.
%(Don't worry, it wasn't that hard.)
In most of the places where we say ``set" or ``vector space", any symmetric monoidal category 
with sufficient limits and colimits would do.
We could also replace many of our chain complexes with topological spaces (or indeed, work at the generality of model categories).


\subsection{Structure of the paper}
The subsections of the introduction explain our motivations in defining the blob complex (see \S \ref{sec:motivations}), 
summarize the formal properties of the blob complex (see \S \ref{sec:properties}), 
describe known specializations (see \S \ref{sec:specializations}), 
and outline the major results of the paper (see \S \ref{sec:structure} and \S \ref{sec:applications}).
%and outline anticipated future directions (see \S \ref{sec:future}).
%\nn{recheck this list after done editing intro}

The first part of the paper (sections \S \ref{sec:fields}--\S \ref{sec:evaluation}) gives the definition of the blob complex, 
and establishes some of its properties.
There are many alternative definitions of $n$-categories, and part of the challenge of defining the blob complex is 
simply explaining what we mean by an ``$n$-category with strong duality'' as one of the inputs.
At first we entirely avoid this problem by introducing the notion of a ``system of fields", and define the blob complex 
associated to an $n$-manifold and an $n$-dimensional system of fields.
We sketch the construction of a system of fields from a *-$1$-category and from a pivotal $2$-category.

Nevertheless, when we attempt to establish all of the observed properties of the blob complex, 
we find this situation unsatisfactory.
Thus, in the second part of the paper (\S\S \ref{sec:ncats}-\ref{sec:ainfblob}) we give yet another 
definition of an $n$-category, or rather a definition of an $n$-category with strong duality.
(Removing the duality conditions from our definition would make it more complicated rather than less.) 
We call these ``disk-like $n$-categories'', to differentiate them from previous versions.
Moreover, we find that we need analogous $A_\infty$ $n$-categories, and we define these as well following very similar axioms.
(See \S \ref{n-cat-names} below for a discussion of $n$-category terminology.)

The basic idea is that each potential definition of an $n$-category makes a choice about the ``shape" of morphisms.
We try to be as lax as possible: a disk-like $n$-category associates a 
vector space to every $B$ homeomorphic to the $n$-ball.
These vector spaces glue together associatively, and we require that there is an action of the homeomorphism groupoid.
For an $A_\infty$ $n$-category, we associate a chain complex instead of a vector space to 
each such $B$ and ask that the action of 
homeomorphisms extends to a suitably defined action of the complex of singular chains of homeomorphisms.
The axioms for an $A_\infty$ $n$-category are designed to capture two main examples: 
the blob complexes of $n$-balls labelled by a 
disk-like $n$-category, and the complex $\CM{-}{T}$ of maps to a fixed target space $T$.

In \S \ref{ssec:spherecat} we explain how $n$-categories can be viewed as objects in an $n{+}1$-category 
of sphere modules.
When $n=1$ this just the familiar 2-category of 1-categories, bimodules and intertwiners.

In \S \ref{ss:ncat_fields}  we explain how to construct a system of fields from a disk-like $n$-category 
(using a colimit along certain decompositions of a manifold into balls). 
With this in hand, we write $\bc_*(M; \cC)$ to indicate the blob complex of a manifold $M$ 
with the system of fields constructed from the $n$-category $\cC$. 
%\nn{KW: I don't think we use this notational convention any more, right?}
In \S \ref{sec:ainfblob} we give an alternative definition 
of the blob complex for an $A_\infty$ $n$-category on an $n$-manifold (analogously, using a homotopy colimit).
Using these definitions, we show how to use the blob complex to ``resolve" any ordinary $n$-category as an 
$A_\infty$ $n$-category, and relate the first and second definitions of the blob complex.
We use the blob complex for $A_\infty$ $n$-categories to establish important properties of the blob complex (in both variants), 
in particular the ``gluing formula" of Theorem \ref{thm:gluing} below.

The relationship between all these ideas is sketched in Figure \ref{fig:outline}.

% NB: the following tikz requires a *more recent* version of PGF than is distributed with MacTex 2010.
% grab the latest build from http://www.texample.net/tikz/builds/
% unzip it in your personal tex tree, and run "mktexlsr ." there
\tikzstyle{box} = [rectangle, rounded corners, draw,outer sep = 5pt, inner sep = 5pt, line width=0.5pt]

\begin{figure}[t]
{\center
\beginpgfgraphicnamed{gadgets-external}%
\begin{tikzpicture}[align=center,line width = 1.5pt]
\newcommand{\xxa}{2}
\newcommand{\xxb}{8}
\newcommand{\yya}{14}
\newcommand{\yyb}{10}
\newcommand{\yyc}{6}

\node[box] at (-4,\yyb) (tC) {$C$ \\ a `traditional' \\ weak $n$-category};
\node[box] at (\xxa,\yya) (C) {$\cC$ \\ a disk-like \\ $n$-category};
\node[box] at (\xxb,\yya) (A) {$A(M)$ \\ the (dual) TQFT \\ Hilbert space};
\node[box] at (\xxa,\yyb) (FU) {$(\cF, U)$ \\ fields and\\ local relations};
\node[box] at (\xxb,\yyb) (BC) {$\bc_*(M; \cF)$ \\ the blob complex};
\node[box] at (\xxa,\yyc) (Cs) {$\cC_*$ \\ an $A_\infty$ \\$n$-category};
\node[box] at (\xxb,\yyc) (BCs) {$\underrightarrow{\cC_*}(M)$};



\draw[->] (C) -- node[above] {$\displaystyle \colim_{\cell(M)} \cC$} node[below] {\S\S \ref{sec:constructing-a-tqft} \& \ref{ss:ncat_fields}} (A);

\draw[->] (FU) -- node[above] {blob complex \\ for $M$} node[below]{\S \ref{sec:blob-definition}} (BC);
\draw[->] (Cs) -- node[above] {$\displaystyle \hocolim_{\cell(M)} \cC_*$} node[below] {\S \ref{ss:ncat_fields}} (BCs);

\draw[->] (FU) -- node[right=10pt] {$\cF(M)/U$ \\ Defn \ref{defn:TQFT-invariant}} (A);

\draw[->] (tC) -- node[below] {Example \ref{ex:traditional-n-categories(fields)}\\ and \S \ref{sec:example:traditional-n-categories(fields)}} (FU);


\draw[->] (C.-100) -- node[left] {
	\S \ref{ss:ncat_fields}
	%$\displaystyle \cF(M) = \DirectSum_{c \in\cell(M)} \cC(c)$ \\ $\displaystyle U(B) = \DirectSum_{c \in \cell(B)} \ker \ev: \cC(c) \to \cC(B)$
   } (FU.100);
\draw[->] (C.210) -- node[above left=3pt] {restrict to \\ standard balls} (tC.42);
\draw[->] (tC) -- node[below=4.5pt] {c.f. \S \ref{sec:comparing-defs}} (C.220);
\draw[->] (FU.80) -- +(0,0.5) -- node[right] {restrict to balls \\ Lem \ref{lem:ncat-from-fields}} (C.-80);
\draw[->] (BC) -- node[right] {$H_0$ \\ c.f. Proposition \ref{thm:skein-modules}} (A);

\draw[->] (FU) -- node[left] {blob complex \\ for balls \\ Example \ref{ex:blob-complexes-of-balls}} (Cs);
\draw[<->] (BC) -- node[right] {$\iso$ by \\ Corollary \ref{cor:new-old}} (BCs);
\end{tikzpicture}
\endpgfgraphicnamed%
\mbox{} % <-- weird, doesn't compile unless I put something here after the \endpgfgraphicnamed...? -S
}
\caption{The main gadgets and constructions of the paper.}
\label{fig:outline}
\end{figure}

Later sections address other topics.
Section \S \ref{sec:deligne} gives
a higher dimensional generalization of the Deligne conjecture 
(that the little discs operad acts on Hochschild cochains) in terms of the blob complex.
The appendices prove technical results about $\CH{M}$ and
make connections between our definitions of $n$-categories and familiar definitions for $n=1$ and $n=2$, 
as well as relating the $n=1$ case of our $A_\infty$ $n$-categories with usual $A_\infty$ algebras. 
%Appendix \ref{sec:comm_alg} describes the blob complex when $\cC$ is a commutative algebra, 
%thought of as a disk-like $n$-category, in terms of the topology of $M$.



\subsection{Motivation}
\label{sec:motivations}

We will briefly sketch our original motivation for defining the blob complex.

As a starting point, consider TQFTs constructed via fields and local relations.
(See \S\ref{sec:tqftsviafields} or \cite{kw:tqft}.)
This gives a satisfactory treatment for semisimple TQFTs
(i.e.\ TQFTs for which the cylinder 1-category associated to an
$n{-}1$-manifold $Y$ is semisimple for all $Y$).

For non-semi-simple TQFTs, this approach is less satisfactory.
Our main motivating example (though we will not develop it in this paper)
is the (decapitated) $4{+}1$-dimensional TQFT associated to Khovanov homology.
It associates a bigraded vector space $A_{Kh}(W^4, L)$ to a 4-manifold $W$ together
with a link $L \subset \bd W$.
The original Khovanov homology of a link in $S^3$ is recovered as $A_{Kh}(B^4, L)$.
%\todo{I'm tempted to replace $A_{Kh}$ with $\cl{Kh}$ throughout this page -S}

How would we go about computing $A_{Kh}(W^4, L)$?
For the Khovanov homology of a link in $S^3$ the main tool is the exact triangle (long exact sequence)
relating resolutions of a crossing.
Unfortunately, the exactness breaks if we glue $B^4$ to itself and attempt
to compute $A_{Kh}(S^1\times B^3, L)$.
According to the gluing theorem for TQFTs, gluing along $B^3 \subset \bd B^4$
corresponds to taking a coend (self tensor product) over the cylinder category
associated to $B^3$ (with appropriate boundary conditions).
The coend is not an exact functor, so the exactness of the triangle breaks.

The obvious solution to this problem is to replace the coend with its derived counterpart, 
Hochschild homology.
This presumably works fine for $S^1\times B^3$ (the answer being the Hochschild homology
of an appropriate bimodule), but for more complicated 4-manifolds this leaves much to be desired.
If we build our manifold up via a handle decomposition, the computation
would be a sequence of derived coends.
A different handle decomposition of the same manifold would yield a different
sequence of derived coends.
To show that our definition in terms of derived coends is well-defined, we
would need to show that the above two sequences of derived coends yield 
isomorphic answers, and that the isomorphism does not depend on any
choices we made along the way.
This is probably not easy to do.

Instead, we would prefer a definition for a derived version of $A_{Kh}(W^4, L)$
which is manifestly invariant.
(That is, a definition that does not
involve choosing a decomposition of $W$.
After all, one of the virtues of our starting point --- TQFTs via field and local relations ---
is that it has just this sort of manifest invariance.)

The solution is to replace $A_{Kh}(W^4, L)$, which is a quotient
\[
 \text{linear combinations of fields} \;\big/\; \text{local relations} ,
\]
with an appropriately free resolution (the blob complex)
\[
	\cdots\to \bc_2(W, L) \to \bc_1(W, L) \to \bc_0(W, L) .
\]
Here $\bc_0$ is linear combinations of fields on $W$,
$\bc_1$ is linear combinations of local relations on $W$,
$\bc_2$ is linear combinations of relations amongst relations on $W$,
and so on. We now have a short exact sequence of chain complexes relating resolutions of the link $L$ 
(c.f. Lemma \ref{lem:hochschild-exact} which shows exactness 
with respect to boundary conditions in the context of Hochschild homology).


\subsection{Formal properties}
\label{sec:properties}
The blob complex enjoys the following list of formal properties.

\begin{property}[Functoriality]
\label{property:functoriality}%
The blob complex is functorial with respect to homeomorphisms.
That is, 
for a fixed $n$-dimensional system of fields $\cF$, the association
\begin{equation*}
X \mapsto \bc_*(X; \cF)
\end{equation*}
is a functor from $n$-manifolds and homeomorphisms between them to chain 
complexes and isomorphisms between them.
\end{property}
As a consequence, there is an action of $\Homeo(X)$ on the chain complex $\bc_*(X; \cF)$; 
this action is extended to all of $C_*(\Homeo(X))$ in Theorem \ref{thm:evaluation} below.

The blob complex is also functorial with respect to $\cF$, 
although we will not address this in detail here.

\begin{property}[Disjoint union]
\label{property:disjoint-union}
The blob complex of a disjoint union is naturally isomorphic to the tensor product of the blob complexes.
\begin{equation*}
\bc_*(X_1 \du X_2) \iso \bc_*(X_1) \tensor \bc_*(X_2)
\end{equation*}
\end{property}

If an $n$-manifold $X$ contains $Y \sqcup Y^\text{op}$ as a codimension $0$ submanifold of its boundary, 
write $X_\text{gl} = X \bigcup_{Y}\selfarrow$ for the manifold obtained by gluing together $Y$ and $Y^\text{op}$.
Note that this includes the case of gluing two disjoint manifolds together.
\begin{property}[Gluing map]
\label{property:gluing-map}%
Given a gluing $X \to X_\mathrm{gl}$, there is an injective natural map
\[
	\bc_*(X) \to \bc_*(X_\mathrm{gl}) 
\]
(natural with respect to homeomorphisms, and also associative with respect to iterated gluings).
\end{property}

\begin{property}[Contractibility]
\label{property:contractibility}%
With field coefficients, the blob complex on an $n$-ball is contractible in the sense 
that it is homotopic to its $0$-th homology.
Moreover, the $0$-th homology of balls can be canonically identified with the vector spaces 
associated by the system of fields $\cF$ to balls.
\begin{equation*}
\xymatrix{\bc_*(B^n;\cF) \ar[r]^(0.4){\iso}_(0.4){\text{qi}} & H_0(\bc_*(B^n;\cF)) \ar[r]^(0.6)\iso & A_\cF(B^n)}
\end{equation*}
\end{property}

Property \ref{property:functoriality} will be immediate from the definition given in
\S \ref{sec:blob-definition}, and we'll recall it at the appropriate point there.
Properties \ref{property:disjoint-union}, \ref{property:gluing-map} and 
\ref{property:contractibility} are established in \S \ref{sec:basic-properties}.

\subsection{Specializations}
\label{sec:specializations}

The blob complex is a simultaneous generalization of the TQFT skein module construction and of Hochschild homology.

\newtheorem*{thm:skein-modules}{Proposition \ref{thm:skein-modules}}

\begin{thm:skein-modules}[Skein modules]
The $0$-th blob homology of $X$ is the usual 
(dual) TQFT Hilbert space (a.k.a.\ skein module) associated to $X$
by $\cF$.
(See \S \ref{sec:local-relations}.)
\begin{equation*}
H_0(\bc_*(X;\cF)) \iso A_{\cF}(X)
\end{equation*}
\end{thm:skein-modules}

\newtheorem*{thm:hochschild}{Theorem \ref{thm:hochschild}}

\begin{thm:hochschild}[Hochschild homology when $X=S^1$]
The blob complex for a $1$-category $\cC$ on the circle is
quasi-isomorphic to the Hochschild complex.
\begin{equation*}
\xymatrix{\bc_*(S^1;\cC) \ar[r]^(0.47){\iso}_(0.47){\text{qi}} & \HC_*(\cC).}
\end{equation*}
\end{thm:hochschild}

Proposition \ref{thm:skein-modules} is immediate from the definition, and
Theorem \ref{thm:hochschild} is established in \S \ref{sec:hochschild}.
%We also note \S \ref{sec:comm_alg} which describes the blob complex when $\cC$ is a one of 
%certain commutative algebras thought of as $n$-categories.


\subsection{Structure of the blob complex}
\label{sec:structure}

In the following $\CH{X}$ is the singular chain complex of the space of homeomorphisms of $X$, fixed on $\bdy X$.

\newtheorem*{thm:CH}{Theorem \ref{thm:CH}}

\begin{thm:CH}[$C_*(\Homeo(-))$ action]
There is a chain map
\begin{equation*}
e_X: \CH{X} \tensor \bc_*(X) \to \bc_*(X).
\end{equation*}
such that
\begin{enumerate}
\item Restricted to $C_0(\Homeo(X))$ this is the action of homeomorphisms described in Property \ref{property:functoriality}. 

\item For
any codimension $0$-submanifold $Y \sqcup Y^\text{op} \subset \bdy X$ the following diagram
(using the gluing maps described in Property \ref{property:gluing-map}) commutes (up to homotopy).
\begin{equation*}
\xymatrix@C+2cm{
     \CH{X} \otimes \bc_*(X)
        \ar[r]_{e_{X}}  \ar[d]^{\gl^{\Homeo}_Y \otimes \gl_Y}  &
            \bc_*(X) \ar[d]_{\gl_Y} \\
     \CH{X \bigcup_Y \selfarrow} \otimes \bc_*(X \bigcup_Y \selfarrow) \ar[r]^<<<<<<<<<<<<{e_{(X \bigcup_Y \scalebox{0.5}{\selfarrow})}}    & \bc_*(X \bigcup_Y \selfarrow)
}
\end{equation*}
\end{enumerate}
%Moreover any such chain map is unique, up to an iterated homotopy.
%(That is, any pair of homotopies have a homotopy between them, and so on.)
%\nn{revisit this after proof below has stabilized}
\end{thm:CH}

\newtheorem*{thm:CH-associativity}{Theorem \ref{thm:CH-associativity}}


Further,
\begin{thm:CH-associativity}
The chain map of Theorem \ref{thm:CH} is associative, in the sense that the following diagram commutes (up to homotopy).
\begin{equation*}
\xymatrix{
\CH{X} \tensor \CH{X} \tensor \bc_*(X) \ar[r]^<<<<<{\id \tensor e_X} \ar[d]^{\compose \tensor \id} & \CH{X} \tensor \bc_*(X) \ar[d]^{e_X} \\
\CH{X} \tensor \bc_*(X) \ar[r]^{e_X} & \bc_*(X)
}
\end{equation*}
\end{thm:CH-associativity}

Since the blob complex is functorial in the manifold $X$, this is equivalent to having chain maps
$$ev_{X \to Y} : \CH{X \to Y} \tensor \bc_*(X) \to \bc_*(Y)$$
for any homeomorphic pair $X$ and $Y$, 
satisfying corresponding conditions.

In \S \ref{sec:ncats} we introduce the notion of disk-like $n$-categories, 
from which we can construct systems of fields.
Traditional $n$-categories can be converted to disk-like $n$-categories by taking string diagrams
(see \S\ref{sec:example:traditional-n-categories(fields)}).
Below, when we talk about the blob complex for a disk-like $n$-category, 
we are implicitly passing first to this associated system of fields.
Further, in \S \ref{sec:ncats} we also have the notion of a disk-like $A_\infty$ $n$-category. 
In that section we describe how to use the blob complex to 
construct disk-like $A_\infty$ $n$-categories from ordinary disk-like $n$-categories:

\newtheorem*{ex:blob-complexes-of-balls}{Example \ref{ex:blob-complexes-of-balls}}

\begin{ex:blob-complexes-of-balls}[Blob complexes of products with balls form a disk-like $A_\infty$ $n$-category]
%\label{thm:blobs-ainfty}
Let $\cC$ be  an ordinary disk-like $n$-category.
Let $Y$ be an $n{-}k$-manifold. 
There is a disk-like $A_\infty$ $k$-category $\bc_*(Y;\cC)$, defined on each $m$-ball $D$, for $0 \leq m < k$, 
to be the set $$\bc_*(Y;\cC)(D) = \cC(Y \times D)$$ and on $k$-balls $D$ to be the set 
$$\bc_*(Y;\cC)(D) = \bc_*(Y \times D; \cC).$$ 
(When $m=k$ the subsets with fixed boundary conditions form a chain complex.) 
These sets have the structure of a disk-like $A_\infty$ $k$-category, with compositions coming from the gluing map in 
Property \ref{property:gluing-map} and with the action of families of homeomorphisms given in Theorem \ref{thm:evaluation}.
\end{ex:blob-complexes-of-balls}

\begin{rem}
Perhaps the most interesting case is when $Y$ is just a point; 
then we have a way of building a disk-like $A_\infty$ $n$-category from an ordinary $n$-category. % disk-like or not
We think of this disk-like $A_\infty$ $n$-category as a free resolution of the ordinary $n$-category.
\end{rem}

There is a version of the blob complex for $\cC$ a disk-like $A_\infty$ $n$-category
instead of an ordinary $n$-category; this is described in \S \ref{sec:ainfblob}.
The definition is in fact simpler, almost tautological, and we use a different notation, $\cl{\cC}(M)$. 
The next theorem describes the blob complex for product manifolds
in terms of the $A_\infty$ blob complex of the disk-like $A_\infty$ $n$-categories constructed as in the previous example.
%The notation is intended to reflect the close parallel with the definition of the TQFT skein module via a colimit.

\newtheorem*{thm:product}{Theorem \ref{thm:product}}

\begin{thm:product}[Product formula]
Let $W$ be a $k$-manifold and $Y$ be an $n-k$ manifold.
Let $\cC$ be an $n$-category.
Let $\bc_*(Y;\cC)$ be the disk-like $A_\infty$ $k$-category associated to $Y$ via blob homology 
(see Example \ref{ex:blob-complexes-of-balls}).
Then
\[
	\bc_*(Y\times W; \cC) \simeq \cl{\bc_*(Y;\cC)}(W).
\]
\end{thm:product}
The statement can be generalized to arbitrary fibre bundles, and indeed to arbitrary maps
(see \S \ref{ss:product-formula}).

Fix a disk-like $n$-category $\cC$, which we'll omit from the notation.
Recall that for any $(n-1)$-manifold $Y$, the blob complex $\bc_*(Y)$ is naturally an $A_\infty$ 1-category.
(See Appendix \ref{sec:comparing-A-infty} for the translation between disk-like $A_\infty$ $1$-categories 
and the usual algebraic notion of an $A_\infty$ category.)

\newtheorem*{thm:gluing}{Theorem \ref{thm:gluing}}

\begin{thm:gluing}[Gluing formula]
\mbox{}% <-- gets the indenting right
\begin{itemize}
\item For any $n$-manifold $X$, with $Y$ a codimension $0$-submanifold of its boundary, the blob complex of $X$ is naturally an
$A_\infty$ module for $\bc_*(Y)$.

\item For any $n$-manifold $X_\text{gl} = X\bigcup_Y \selfarrow$, the blob complex $\bc_*(X_\text{gl})$ 
is the $A_\infty$ self-tensor product of
$\bc_*(X)$ as an $\bc_*(Y)$-bimodule:
\begin{equation*}
\bc_*(X_\text{gl}) \simeq \bc_*(X) \Tensor^{A_\infty}_{\mathclap{\bc_*(Y)}} \selfarrow
\end{equation*}
\end{itemize}
\end{thm:gluing}

Theorem \ref{thm:product} is proved in \S \ref{ss:product-formula}, and Theorem \ref{thm:gluing} in \S \ref{sec:gluing}.

\subsection{Applications}
\label{sec:applications}
Finally, we give two applications of the above machinery.

\newtheorem*{thm:map-recon}{Theorem \ref{thm:map-recon}}

\begin{thm:map-recon}[Mapping spaces]
Let $\pi^\infty_{\le n}(T)$ denote the disk-like $A_\infty$ $n$-category based on singular chains on maps 
$B^n \to T$.
(The case $n=1$ is the usual $A_\infty$-category of paths in $T$.)
Then 
\[
	\bc_*(X; \pi^\infty_{\le n}(T)) \simeq \CM{X}{T},
\]
where $C_*$ denotes singular chains.
\end{thm:map-recon}

This says that we can recover (up to homotopy) the space of maps to $T$ via blob homology from local data. 
Note that there is no restriction on the connectivity of $T$.
The proof appears in \S \ref{sec:map-recon}.

\newtheorem*{thm:deligne}{Theorem \ref{thm:deligne}}

\begin{thm:deligne}[Higher dimensional Deligne conjecture]
The singular chains of the $n$-dimensional surgery cylinder operad act on blob cochains
(up to coherent homotopy).
Since the little $n{+}1$-balls operad is a suboperad of the $n$-dimensional surgery cylinder operad,
this implies that the little $n{+}1$-balls operad acts on blob cochains of the $n$-ball.
\end{thm:deligne}
See \S \ref{sec:deligne} for a full explanation of the statement, and the proof.



\noop{ %%%%%%%%%%%%%%%%%%%%%%%%%%%%%%
\subsection{Future directions}
\label{sec:future}
\nn{KW: Perhaps we should delete this subsection and salvage only the first few sentences.}
Throughout, we have resisted the temptation to work in the greatest generality possible.
(Don't worry, it wasn't that hard.)
In most of the places where we say ``set" or ``vector space", any symmetric monoidal category would do.
We could also replace many of our chain complexes with topological spaces (or indeed, work at the generality of model categories).
%%%%%%
And likely it will prove useful to think about the connections between what we do here and $(\infty,k)$-categories.
More could be said about finite characteristic 
(there appears in be $2$-torsion in $\bc_1(S^2; \cC)$ for any spherical $2$-category $\cC$, for example).
Much more could be said about other types of manifolds, in particular oriented, 
$\operatorname{Spin}$ and $\operatorname{Pin}^{\pm}$ manifolds, where boundary issues become more complicated.
(We'd recommend thinking about boundaries as germs, rather than just codimension $1$ manifolds.) 
We've also take the path of least resistance by concentrating on PL manifolds; 
there may be some differences for topological manifolds and smooth manifolds.

The paper ``Skein homology'' \cite{MR1624157} has similar motivations, and it may be 
interesting to investigate if there is a connection with the material here.

Many results in Hochschild homology can be understood ``topologically" via the blob complex.
For example, we expect that the shuffle product on the Hochschild homology of a commutative algebra $A$ 
(see \cite[\S 4.2]{MR1600246}) simply corresponds to the gluing operation on $\bc_*(S^1 \times [0,1]; A)$, 
but haven't investigated the details.

Most importantly, however, \nn{applications!} \nn{cyclic homology, $n=2$ cases, contact, Kh} \nn{stabilization} 
\nn{stable categories, generalized cohomology theories}
} %%% end \noop %%%%%%%%%%%%%%%%%%%%%

\subsection{\texorpdfstring{$n$}{n}-category terminology}
\label{n-cat-names}

Section \S \ref{sec:ncats} adds to the zoo of $n$-category definitions, and the new creatures need names.
Unfortunately, we have found it difficult to come up with terminology which satisfies all
of the colleagues whom we have consulted, or even satisfies just ourselves.

One distinction we need to make is between $n$-categories which are associative in dimension $n$ and those
that are associative only up to higher homotopies.
The latter are closely related to $(\infty, n)$-categories (i.e.\ $\infty$-categories where all morphisms
of dimension greater than $n$ are invertible), but we don't want to use that name
since we think of the higher homotopies not as morphisms of the $n$-category but
rather as belonging to some auxiliary category (like chain complexes)
that we are enriching in.
We have decided to call them ``$A_\infty$ $n$-categories", since they are a natural generalization 
of the familiar $A_\infty$ 1-categories.
We also considered the names ``homotopy $n$-categories" and ``infinity $n$-categories".
When we need to emphasize that we are talking about an $n$-category which is not $A_\infty$ in this sense
we will say ``ordinary $n$-category".
% small problem: our n-cats are of course strictly associative, since we have more morphisms.
% when we say ``associative only up to homotopy" above we are thinking about
% what would happen we we tried to convert to a more traditional n-cat with fewer morphisms

Another distinction we need to make is between our style of definition of $n$-categories and
more traditional and combinatorial definitions.
We will call instances of our definition ``disk-like $n$-categories", since $n$-dimensional disks
play a prominent role in the definition.
(In general we prefer ``$k$-ball" to ``$k$-disk", but ``ball-like" doesn't roll off 
the tongue as well as ``disk-like''.)

Another thing we need a name for is the ability to rotate morphisms around in various ways.
For 2-categories, ``strict pivotal" is a standard term for what we mean.
A more general term is ``duality", but duality comes in various flavors and degrees.
We are mainly interested in a very strong version of duality, where the available ways of
rotating $k$-morphisms correspond to all the ways of rotating $k$-balls.
We sometimes refer to this as ``strong duality", and sometimes we consider it to be implied
by ``disk-like".
(But beware: disks can come in various flavors, and some of them, such as framed disks,
don't actually imply much duality.)
Another possibility considered here was ``pivotal $n$-category", but we prefer to preserve pivotal for its usual sense. 
It will thus be a theorem that our disk-like 2-categories 
are equivalent to pivotal 2-categories, c.f. \S \ref{ssec:2-cats}.

Finally, we need a general name for isomorphisms between balls, where the balls could be
piecewise linear or smooth or Spin or framed or etc., or some combination thereof.
We have chosen to use ``homeomorphism" for the appropriate sort of isomorphism, so the reader should
keep in mind that ``homeomorphism" could mean PL homeomorphism or diffeomorphism (and so on)
depending on context.

\subsection{Thanks and acknowledgements}
% attempting to make this chronological rather than alphabetical
We'd like to thank 
Justin Roberts (for helpful discussions in the very early stages of this work), 
Michael Freedman, 
Peter Teichner (for helping us improve an earlier version of the $n$-category definition), 
David Ben-Zvi, 
Vaughan Jones, 
Chris Schommer-Pries, 
Thomas Tradler,
Kevin Costello, 
Chris Douglas,
Alexander Kirillov,
Michael Shulman,
and
Rob Kirby
for many interesting and useful conversations. 
Peter Teichner ran a reading course based on an earlier draft of this paper, and the detailed feedback
we got from the student lecturers lead to very many improvements in later drafts.
So big thanks to
Aaron Mazel-Gee,
Nate Watson,
Alan Wilder,
Dmitri Pavlov,
Ansgar Schneider,
and
Dan Berwick-Evans.
We thank the anonymous referee for numerous suggestions which improved this paper.
During this work, Kevin Walker has been at Microsoft Station Q, and Scott Morrison has been at 
Microsoft Station Q and the Miller Institute for Basic Research at UC Berkeley. 
We'd like to thank the Aspen Center for Physics for the pleasant and productive 
environment provided there during the final preparation of this manuscript.

