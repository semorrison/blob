%!TEX root = ../blob1.tex

\section{Introduction}

[some things to cover in the intro]
\begin{itemize}
\item explain relation between old and new blob complex definitions
\item overview of sections
\item state main properties of blob complex (already mostly done below)
\item give multiple motivations/viewpoints for blob complex: (1) derived cat
version of TQFT Hilbert space; (2) generalization of Hochschild homology to higher $n$-cats;
(3) ? sort-of-obvious colimit type construction;
(4) ? a generalization of $C_*(\Maps(M, T))$ to the case where $T$ is
a category rather than a manifold
\item hope to apply to Kh, contact, (other examples?) in the future
\item ?? we have resisted the temptation 
(actually, it was not a temptation) to state things in the greatest
generality possible
\item related: we are being unsophisticated from a homotopy theory point of
view and using chain complexes in many places where we could be by with spaces
\item ? one of the points we make (far) below is that there is not really much
difference between (a) systems of fields and local relations and (b) $n$-cats;
thus we tend to switch between talking in terms of one or the other
\end{itemize}

\medskip\hrule\medskip

[Old outline for intro]
\begin{itemize}
\item Starting point: TQFTs via fields and local relations.
This gives a satisfactory treatment for semisimple TQFTs
(i.e.\ TQFTs for which the cylinder 1-category associated to an
$n{-}1$-manifold $Y$ is semisimple for all $Y$).
\item For non-semiemple TQFTs, this approach is less satisfactory.
Our main motivating example (though we will not develop it in this paper)
is the $4{+}1$-dimensional TQFT associated to Khovanov homology.
It associates a bigraded vector space $A_{Kh}(W^4, L)$ to a 4-manifold $W$ together
with a link $L \subset \bd W$.
The original Khovanov homology of a link in $S^3$ is recovered as $A_{Kh}(B^4, L)$.
\item How would we go about computing $A_{Kh}(W^4, L)$?
For $A_{Kh}(B^4, L)$, the main tool is the exact triangle (long exact sequence)
\nn{... $L_1, L_2, L_3$}.
Unfortunately, the exactness breaks if we glue $B^4$ to itself and attempt
to compute $A_{Kh}(S^1\times B^3, L)$.
According to the gluing theorem for TQFTs-via-fields, gluing along $B^3 \subset \bd B^4$
corresponds to taking a coend (self tensor product) over the cylinder category
associated to $B^3$ (with appropriate boundary conditions).
The coend is not an exact functor, so the exactness of the triangle breaks.
\item The obvious solution to this problem is to replace the coend with its derived counterpart.
This presumably works fine for $S^1\times B^3$ (the answer being the Hochschild homology
of an appropriate bimodule), but for more complicated 4-manifolds this leaves much to be desired.
If we build our manifold up via a handle decomposition, the computation
would be a sequence of derived coends.
A different handle decomposition of the same manifold would yield a different
sequence of derived coends.
To show that our definition in terms of derived coends is well-defined, we
would need to show that the above two sequences of derived coends yield the same answer.
This is probably not easy to do.
\item Instead, we would prefer a definition for a derived version of $A_{Kh}(W^4, L)$
which is manifestly invariant.
(That is, a definition that does not
involve choosing a decomposition of $W$.
After all, one of the virtues of our starting point --- TQFTs via field and local relations ---
is that it has just this sort of manifest invariance.)
\item The solution is to replace $A_{Kh}(W^4, L)$, which is a quotient
\[
 \text{linear combinations of fields} \;\big/\; \text{local relations} ,
\]
with an appropriately free resolution (the ``blob complex")
\[
	\cdots\to \bc_2(W, L) \to \bc_1(W, L) \to \bc_0(W, L) .
\]
Here $\bc_0$ is linear combinations of fields on $W$,
$\bc_1$ is linear combinations of local relations on $W$,
$\bc_2$ is linear combinations of relations amongst relations on $W$,
and so on.
\item None of the above ideas depend on the details of the Khovanov homology example,
so we develop the general theory in the paper and postpone specific applications
to later papers.
\item The blob complex enjoys the following nice properties \nn{...}
\end{itemize}

\bigskip
\hrule
\bigskip

We then show that blob homology enjoys the following properties.

\begin{property}[Functoriality]
\label{property:functoriality}%
Blob homology is functorial with respect to homeomorphisms. That is, 
for fixed $n$-category / fields $\cC$, the association
\begin{equation*}
X \mapsto \bc_*^{\cC}(X)
\end{equation*}
is a functor from $n$-manifolds and homeomorphisms between them to chain complexes and isomorphisms between them.
\end{property}

\nn{should probably also say something about being functorial in $\cC$}

\begin{property}[Disjoint union]
\label{property:disjoint-union}
The blob complex of a disjoint union is naturally the tensor product of the blob complexes.
\begin{equation*}
\bc_*(X_1 \du X_2) \iso \bc_*(X_1) \tensor \bc_*(X_2)
\end{equation*}
\end{property}

\begin{property}[Gluing map]
\label{property:gluing-map}%
If $X_1$ and $X_2$ are $n$-manifolds, with $Y$ a codimension $0$-submanifold of $\bdy X_1$, and $Y^{\text{op}}$ a codimension $0$-submanifold of $\bdy X_2$,
there is a chain map
\begin{equation*}
\gl_Y: \bc_*(X_1) \tensor \bc_*(X_2) \to \bc_*(X_1 \cup_Y X_2).
\end{equation*}
\nn{alternate version:}Given a gluing $X_\mathrm{cut} \to X_\mathrm{gl}$, there is
a natural map
\[
	\bc_*(X_\mathrm{cut}) \to \bc_*(X_\mathrm{gl}) .
\]
(Natural with respect to homeomorphisms, and also associative with respect to iterated gluings.)
\end{property}

\begin{property}[Contractibility]
\label{property:contractibility}%
\todo{Err, requires a splitting?}
The blob complex for an $n$-category on an $n$-ball is quasi-isomorphic to its $0$-th homology.
\begin{equation}
\xymatrix{\bc_*^{\cC}(B^n) \ar[r]^{\iso}_{\text{qi}} & H_0(\bc_*^{\cC}(B^n))}
\end{equation}
\todo{Say that this is just the original $n$-category?}
\end{property}

\begin{property}[Skein modules]
\label{property:skein-modules}%
The $0$-th blob homology of $X$ is the usual 
(dual) TQFT Hilbert space (a.k.a.\ skein module) associated to $X$
by $\cC$. (See \S \ref{sec:local-relations}.)
\begin{equation*}
H_0(\bc_*^{\cC}(X)) \iso A^{\cC}(X)
\end{equation*}
\end{property}

\begin{property}[Hochschild homology when $X=S^1$]
\label{property:hochschild}%
The blob complex for a $1$-category $\cC$ on the circle is
quasi-isomorphic to the Hochschild complex.
\begin{equation*}
\xymatrix{\bc_*^{\cC}(S^1) \ar[r]^{\iso}_{\text{qi}} & HC_*(\cC)}
\end{equation*}
\end{property}

\begin{property}[Evaluation map]
\label{property:evaluation}%
There is an `evaluation' chain map
\begin{equation*}
\ev_X: \CD{X} \tensor \bc_*(X) \to \bc_*(X).
\end{equation*}
(Here $\CD{X}$ is the singular chain complex of the space of diffeomorphisms of $X$, fixed on $\bdy X$.)

Restricted to $C_0(\Diff(X))$ this is just the action of diffeomorphisms described in Property \ref{property:functoriality}. Further, for
any codimension $1$-submanifold $Y \subset X$ dividing $X$ into $X_1 \cup_Y X_2$, the following diagram
(using the gluing maps described in Property \ref{property:gluing-map}) commutes.
\begin{equation*}
\xymatrix{
     \CD{X} \otimes \bc_*(X) \ar[r]^{\ev_X}    & \bc_*(X) \\
     \CD{X_1} \otimes \CD{X_2} \otimes \bc_*(X_1) \otimes \bc_*(X_2)
        \ar@/_4ex/[r]_{\ev_{X_1} \otimes \ev_{X_2}}  \ar[u]^{\gl^{\Diff}_Y \otimes \gl_Y}  &
            \bc_*(X_1) \otimes \bc_*(X_2) \ar[u]_{\gl_Y}
}
\end{equation*}
\nn{should probably say something about associativity here (or not?)}
\end{property}


\begin{property}[Gluing formula]
\label{property:gluing}%
\mbox{}% <-- gets the indenting right
\begin{itemize}
\item For any $(n-1)$-manifold $Y$, the blob homology of $Y \times I$ is
naturally an $A_\infty$ category. % We'll write $\bc_*(Y)$ for $\bc_*(Y \times I)$ below.

\item For any $n$-manifold $X$, with $Y$ a codimension $0$-submanifold of its boundary, the blob homology of $X$ is naturally an
$A_\infty$ module for $\bc_*(Y \times I)$.

\item For any $n$-manifold $X$, with $Y \cup Y^{\text{op}}$ a codimension
$0$-submanifold of its boundary, the blob homology of $X'$, obtained from
$X$ by gluing along $Y$, is the $A_\infty$ self-tensor product of
$\bc_*(X)$ as an $\bc_*(Y \times I)$-bimodule.
\begin{equation*}
\bc_*(X') \iso \bc_*(X) \Tensor^{A_\infty}_{\mathclap{\bc_*(Y \times I)}} \!\!\!\!\!\!\xymatrix{ \ar@(ru,rd)@<-1ex>[]}
\end{equation*}
\end{itemize}
\end{property}



\begin{property}[Relation to mapping spaces]
There is a version of the blob complex for $C$ an $A_\infty$ $n$-category
instead of a garden variety $n$-category.

Let $\pi^\infty_{\le n}(W)$ denote the $A_\infty$ $n$-category based on maps 
$B^n \to W$.
(The case $n=1$ is the usual $A_\infty$ category of paths in $W$.)
Then $\bc_*(M, \pi^\infty_{\le n}(W))$ is 
homotopy equivalent to $C_*(\{\text{maps}\; M \to W\})$.
\end{property}




\begin{property}[Product formula]
Let $M^n = Y^{n-k}\times W^k$ and let $C$ be an $n$-category.
Let $A_*(Y)$ be the $A_\infty$ $k$-category associated to $Y$ via blob homology.
Then
\[
	\bc_*(Y^{n-k}\times W^k, C) \simeq \bc_*(W, A_*(Y)) .
\]
\nn{say something about general fiber bundles?}
\end{property}




\begin{property}[Higher dimensional Deligne conjecture]
The singular chains of the $n$-dimensional fat graph operad act on blob cochains.

The $n$-dimensional fat graph operad can be thought of as a sequence of general surgeries
of $n$-manifolds
$R_i \cup A_i \leadsto R_i \cup B_i$ together with mapping cylinders of diffeomorphisms
$f_i: R_i\cup B_i \to R_{i+1}\cup A_{i+1}$.
(Note that the suboperad where $A_i$, $B_i$ and $R_i\cup A_i$ are all diffeomorphic to 
the $n$-ball is equivalent to the little $n{+}1$-disks operad.)

If $A$ and $B$ are $n$-manifolds sharing the same boundary, define
the blob cochains $\bc^*(A, B)$ (analogous to Hochschild cohomology) to be
$A_\infty$ maps from $\bc_*(A)$ to $\bc_*(B)$, where we think of both
(collections of) complexes as modules over the $A_\infty$ category associated to $\bd A = \bd B$.
The ``holes" in the above 
$n$-dimensional fat graph operad are labeled by $\bc^*(A_i, B_i)$.
\end{property}








Properties \ref{property:functoriality}, \ref{property:gluing-map} and \ref{property:skein-modules} will be immediate from the definition given in
\S \ref{sec:blob-definition}, and we'll recall them at the appropriate points there. \todo{Make sure this gets done.}
Properties \ref{property:disjoint-union} and \ref{property:contractibility} are established in \S \ref{sec:basic-properties}.
Property \ref{property:hochschild} is established in \S \ref{sec:hochschild}, Property \ref{property:evaluation} in \S \ref{sec:evaluation},
and Property \ref{property:gluing} in \S \ref{sec:gluing}.
\nn{need to say where the remaining properties are proved.}