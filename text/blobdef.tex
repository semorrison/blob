%!TEX root = ../blob1.tex

\section{The blob complex}
\label{sec:blob-definition}

Let $X$ be an $n$-manifold.
Let $(\cF,U)$ be a fixed system of fields and local relations.
We'll assume it is enriched over \textbf{Vect}; 
if it is not we can make it so by allowing finite
linear combinations of elements of $\cF(X; c)$, for fixed $c\in \cF(\bd X)$.

%In this section we will usually suppress boundary conditions on $X$ from the notation, e.g. by writing $\cF(X)$ instead of $\cF(X; c)$.

We want to replace the quotient
\[
	A(X) \deq \cF(X) / U(X)
\]
of Definition \ref{defn:TQFT-invariant} with a resolution
\[
	\cdots \to \bc_2(X) \to \bc_1(X) \to \bc_0(X) .
\]

We will define $\bc_0(X)$, $\bc_1(X)$ and $\bc_2(X)$, then give the general case $\bc_k(X)$. 
In fact, on the first pass we will intentionally describe the definition in a misleadingly simple way, 
then explain the technical difficulties, and finally give a cumbersome but complete definition in 
Definition \ref{defn:blobs}. 
If (we don't recommend it) you want to keep track of the ways in which 
this initial description is misleading, or you're reading through a second time to understand the 
technical difficulties, keep note that later we will give precise meanings to ``a ball in $X$'', 
``nested'' and ``disjoint'', that are not quite the intuitive ones. 
Moreover some of the pieces 
into which we cut manifolds below are not themselves manifolds, and it requires special attention 
to define fields on these pieces.

We of course define $\bc_0(X) = \cF(X)$.
(If $X$ has nonempty boundary, instead define $\bc_0(X; c) = \cF(X; c)$ for each $c \in \cF(\bdy X)$.
We'll omit such boundary conditions from the notation in the rest of this section.)
In other words, $\bc_0(X)$ is just the vector space of all fields on $X$.

We want the vector space $\bc_1(X)$ to capture 
``the space of all local relations that can be imposed on $\bc_0(X)$".
Thus we say  a $1$-blob diagram consists of:
\begin{itemize}
\item An closed ball in $X$ (``blob") $B \sub X$.
\item A boundary condition $c \in \cF(\bdy B) = \cF(\bd(X \setmin B))$.
\item A field $r \in \cF(X \setmin B; c)$.
\item A local relation field $u \in U(B; c)$.
\end{itemize}
(See Figure \ref{blob1diagram}.) Since $c$ is implicitly determined by $u$ or $r$, we usually omit it from the notation.
\begin{figure}[t]\begin{equation*}
\mathfig{.6}{definition/single-blob}
\end{equation*}\caption{A 1-blob diagram.}\label{blob1diagram}\end{figure}
In order to get the linear structure correct, we define
\[
	\bc_1(X) \deq \bigoplus_B \bigoplus_c U(B; c) \otimes \cF(X \setmin B; c) .
\]
The first direct sum is indexed by all blobs $B\subset X$, and the second
by all boundary conditions $c \in \cF(\bd B)$.
Note that $\bc_1(X)$ is spanned by 1-blob diagrams $(B, u, r)$.

Define the boundary map $\bd : \bc_1(X) \to \bc_0(X)$ by 
\[ 
	(B, u, r) \mapsto u\bullet r, 
\]
where $u\bullet r$ denotes the field on $X$ obtained by gluing $u$ to $r$.
In other words $\bd : \bc_1(X) \to \bc_0(X)$ is given by
just erasing the blob from the picture
(but keeping the blob label $u$).

Note that directly from the definition we have
\begin{prop}
\label{thm:skein-modules}
The skein module $A(X)$ is naturally isomorphic to $\bc_0(X)/\bd(\bc_1(X))) = H_0(\bc_*(X))$.
\end{prop}
This also establishes the second 
half of Property \ref{property:contractibility}.

Next, we want the vector space $\bc_2(X)$ to capture ``the space of all relations 
(redundancies, syzygies) among the 
local relations encoded in $\bc_1(X)$''.
A $2$-blob diagram, comes in one of two types, disjoint and nested.
A disjoint 2-blob diagram consists of
\begin{itemize}
\item A pair of closed balls (blobs) $B_1, B_2 \sub X$ with disjoint interiors.
\item A field $r \in \cF(X \setmin (B_1 \cup B_2); c_1, c_2)$
(where $c_i \in \cF(\bd B_i)$).
\item Local relation fields $u_i \in U(B_i; c_i)$, $i=1,2$.
\end{itemize}
(See Figure \ref{blob2ddiagram}.)
\begin{figure}[t]\begin{equation*}
\mathfig{.6}{definition/disjoint-blobs}
\end{equation*}\caption{A disjoint 2-blob diagram.}\label{blob2ddiagram}\end{figure}
We also identify $(B_1, B_2, u_1, u_2, r)$ with $-(B_2, B_1, u_2, u_1, r)$;
reversing the order of the blobs changes the sign.
Define $\bd(B_1, B_2, u_1, u_2, r) = 
(B_2, u_2, u_1\bullet r) - (B_1, u_1, u_2\bullet r) \in \bc_1(X)$.
In other words, the boundary of a disjoint 2-blob diagram
is the sum (with alternating signs)
of the two ways of erasing one of the blobs.
It's easy to check that $\bd^2 = 0$.

A nested 2-blob diagram consists of
\begin{itemize}
\item A pair of nested balls (blobs) $B_1 \subseteq B_2 \subseteq X$.
\item A field $r' \in \cF(B_2 \setminus B_1; c_1, c_2)$ 
(for some $c_1 \in \cF(\bdy B_1)$ and $c_2 \in \cF(\bdy B_2)$).
\item A field $r \in \cF(X \setminus B_2; c_2)$.
\item A local relation field $u \in U(B_1; c_1)$.
\end{itemize}
(See Figure \ref{blob2ndiagram}.)
\begin{figure}[t]\begin{equation*}
\mathfig{.6}{definition/nested-blobs}
\end{equation*}\caption{A nested 2-blob diagram.}\label{blob2ndiagram}\end{figure}
Define $\bd(B_1, B_2, u, r', r) = (B_2, u\bullet r', r) - (B_1, u, r' \bullet r)$.
As in the disjoint 2-blob case, the boundary of a nested 2-blob is the alternating
sum of the two ways of erasing one of the blobs.
When  we erase the inner blob, the outer blob inherits the label $u\bullet r'$.
It is again easy to check that $\bd^2 = 0$. Note that the requirement that
local relations are an ideal with respect to gluing guarantees that $u\bullet r' \in U(B_2)$.

As with the $1$-blob diagrams, in order to get the linear structure correct the actual definition is 
\begin{eqnarray*}
	\bc_2(X) & \deq &
	\left( 
		\bigoplus_{B_1, B_2\; \text{disjoint}} \bigoplus_{c_1, c_2}
			U(B_1; c_1) \otimes U(B_2; c_2) \otimes \cF(X\setmin (B_1\cup B_2); c_1, c_2)
	\right)  \bigoplus \\
	&& \quad\quad  \left( 
		\bigoplus_{B_1 \subset B_2} \bigoplus_{c_1, c_2}
			U(B_1; c_1) \otimes \cF(B_2 \setmin B_1; c_1, c_2) \tensor \cF(X \setminus B_2; c_2)
	\right) .
\end{eqnarray*}
% __ (already said this above)
%For the disjoint blobs, reversing the ordering of $B_1$ and $B_2$ introduces a minus sign
%(rather than a new, linearly independent, 2-blob diagram). 

\medskip

Roughly, $\bc_k(X)$ is generated by configurations of $k$ blobs, pairwise disjoint or nested, 
along with fields on all the components that the blobs divide $X$ into. 
Blobs which have no other blobs inside are called `twig blobs', 
and the fields on the twig blobs must be local relations.
The boundary is the alternating sum of erasing one of the blobs.
In order to describe this general case in full detail, we must give a more precise description of
which configurations of balls inside $X$ we permit.
These configurations are generated by two operations:
\begin{itemize}
\item For any (possibly empty) configuration of blobs on an $n$-ball $D$, we can add
$D$ itself as an outermost blob.
(This is used in the proof of Proposition \ref{bcontract}.)
\item If $X'$ is obtained from $X$ by gluing, then any permissible configuration of blobs
on $X$ gives rise to a permissible configuration on $X'$.
(This is necessary for Proposition \ref{blob-gluing}.)
\end{itemize}
Combining these two operations can give rise to configurations of blobs whose complement in $X$ is not
a manifold.
Thus will need to be more careful when speaking of a field $r$ on the complement of the blobs.

\begin{example}
Consider the four subsets of $\Real^3$,
\begin{align*}
A & = [0,1] \times [0,1] \times [0,1] \\
B & = [0,1] \times [-1,0] \times [0,1] \\
C & = [-1,0] \times \setc{(y,z)}{z \sin(1/z) \leq y \leq 1, z \in [0,1]} \\
D & = [-1,0] \times \setc{(y,z)}{-1 \leq y \leq z \sin(1/z), z \in [0,1]}.
\end{align*}
Here $A \cup B = [0,1] \times [-1,1] \times [0,1]$ and $C \cup D = [-1,0] \times [-1,1] \times [0,1]$. 
Now, $\{A\}$ is a valid configuration of blobs in $A \cup B$, 
and $\{C\}$ is a valid configuration of blobs in $C \cup D$, 
so we must allow $\{A, C\}$ as a configuration of blobs in $[-1,1]^2 \times [0,1]$. 
Note however that the complement is not a manifold.
\end{example}

\begin{defn}
\label{defn:gluing-decomposition}
A \emph{gluing decomposition} of an $n$-manifold $X$ is a sequence of manifolds 
$M_0 \to M_1 \to \cdots \to M_m = X$ such that each $M_k$ is obtained from $M_{k-1}$ 
by gluing together some disjoint pair of homeomorphic $n{-}1$-manifolds in the boundary of $M_{k-1}$.
If, in addition, $M_0$ is a disjoint union of balls, we call it a \emph{ball decomposition}.
\end{defn}
Given a gluing decomposition $M_0 \to M_1 \to \cdots \to M_m = X$, we say that a field is 
splittable along it if it is the image of a field on $M_0$.

In the example above, note that
\[
	A \sqcup B \sqcup C \sqcup D \to (A \cup B) \sqcup (C \cup D) \to A \cup B \cup C \cup D
\]
is a  ball decomposition, but other sequences of gluings starting from $A \sqcup B \sqcup C \sqcup D$
have intermediate steps which are not manifolds.

We'll now slightly restrict the possible configurations of blobs.
%%%%% oops -- I missed the similar discussion after the definition
%The basic idea is that each blob in a configuration 
%is the image a ball, with embedded interior and possibly glued-up boundary;
%distinct blobs should either have disjoint interiors or be nested;
%and the entire configuration should be compatible with some gluing decomposition of $X$.
\begin{defn}
\label{defn:configuration}
A configuration of $k$ blobs in $X$ is an ordered collection of $k$ subsets $\{B_1, \ldots B_k\}$ 
of $X$ such that there exists a gluing decomposition $M_0  \to \cdots \to M_m = X$ of $X$ and 
for each subset $B_i$ there is some $0 \leq r \leq m$ and some connected component $M_r'$ of 
$M_r$ which is a ball, so $B_i$ is the image of $M_r'$ in $X$. 
We say that such a gluing decomposition 
is \emph{compatible} with the configuration. 
A blob $B_i$ is a twig blob if no other blob $B_j$ is a strict subset of it. 
\end{defn}
In particular, this implies what we said about blobs above: 
that for any two blobs in a configuration of blobs in $X$, 
they either have disjoint interiors, or one blob is contained in the other. 
We describe these as disjoint blobs and nested blobs. 
Note that nested blobs may have boundaries that overlap, or indeed coincide. 
Blobs may meet the boundary of $X$.
Further, note that blobs need not actually be embedded balls in $X$, since parts of the 
boundary of the ball $M_r'$ may have been glued together.

Note that often the gluing decomposition for a configuration of blobs may just be the trivial one: 
if the boundaries of all the blobs cut $X$ into pieces which are all manifolds, 
we can just take $M_0$ to be these pieces, and $M_1 = X$.

In the informal description above, in the definition of a $k$-blob diagram we asked for any 
collection of $k$ balls which were pairwise disjoint or nested. 
We now further insist that the balls are a configuration in the sense of Definition \ref{defn:configuration}. 
Also, we asked for a local relation on each twig blob, and a field on the complement of the twig blobs; 
this is unsatisfactory because that complement need not be a manifold. Thus, the official definitions are
\begin{defn}
\label{defn:blob-diagram}
A $k$-blob diagram on $X$ consists of
\begin{itemize}
\item a configuration $\{B_1, \ldots B_k\}$ of $k$ blobs in $X$,
\item and a field $r \in \cF(X)$ which is splittable along some gluing decomposition compatible with that configuration,
\end{itemize}
such that
the restriction $u_i$ of $r$ to each twig blob $B_i$ lies in the subspace 
$U(B_i) \subset \cF(B_i)$. 
(See Figure \ref{blobkdiagram}.) 
More precisely, each twig blob $B_i$ is the image of some ball $M_r'$ as above, 
and it is really the restriction to $M_r'$ that must lie in the subspace $U(M_r')$.
\end{defn}
\begin{figure}[t]\begin{equation*}
\mathfig{.7}{definition/k-blobs}
\end{equation*}\caption{A $k$-blob diagram.}\label{blobkdiagram}\end{figure}
and
\begin{defn}
\label{defn:blobs}
The $k$-th vector space $\bc_k(X)$ of the \emph{blob complex} of $X$ is the direct sum over all 
configurations of $k$ blobs in $X$ of the vector space of $k$-blob diagrams with that configuration, 
modulo identifying the vector spaces for configurations that only differ by a permutation of the balls 
by the sign of that permutation. 
The differential $\bc_k(X) \to \bc_{k-1}(X)$ is, as above, the signed sum of ways of 
forgetting one blob from the configuration, preserving the field $r$:
\begin{equation*}
\bdy(\{B_1, \ldots B_k\}, r) = \sum_{i=1}^{k} (-1)^{i+1} (\{B_1, \ldots, \widehat{B_i}, \ldots, B_k\}, r)
\end{equation*}
\end{defn}
We readily see that if a gluing decomposition is compatible with some configuration of blobs, 
then it is also compatible with any configuration obtained by forgetting some blobs, 
ensuring that the differential in fact lands in the space of $k{-}1$-blob diagrams.
A slight compensation to the complication of the official definition arising from attention 
to splitting is that the differential now just preserves the entire field $r$ without 
having to say anything about gluing together fields on smaller components.

Note that Property \ref{property:functoriality}, that the blob complex is functorial with respect to homeomorphisms, 
is immediately obvious from the definition.
A homeomorphism acts in an obvious way on blobs and on fields.

We define the {\it support} of a blob diagram $b$, $\supp(b) \sub X$, 
to be the union of the blobs of $b$.
For $y \in \bc_*(X)$ with $y = \sum c_i b_i$ ($c_i$ a non-zero number, $b_i$ a blob diagram),
we define $\supp(y) \deq \bigcup_i \supp(b_i)$.

\begin{remark} \label{blobsset-remark} \rm
We note that blob diagrams in $X$ have a structure similar to that of a simplicial set,
but with simplices replaced by a more general class of combinatorial shapes.
Let $P$ be the minimal set of (isomorphisms classes of) polyhedra which is closed under products
and cones, and which contains the point.
We can associate an element $p(b)$ of $P$ to each blob diagram $b$ 
(equivalently, to each rooted tree) according to the following rules:
\begin{itemize}
\item $p(\emptyset) = pt$, where $\emptyset$ denotes a 0-blob diagram or empty tree;
\item $p(a \du b) = p(a) \times p(b)$, where $a \du b$ denotes the distant (non-overlapping) union 
of two blob diagrams (equivalently, join two trees at the roots); and
\item $p(\bar{b}) = \kone(p(b))$, where $\bar{b}$ is obtained from $b$ by adding an outer blob which 
encloses all the others (equivalently, add a new edge to the root, with the new vertex becoming the root).
\end{itemize}
For example, a diagram of $k$ strictly nested blobs corresponds to a $k$-simplex, while
a diagram of $k$ disjoint blobs corresponds to a $k$-cube.
(When the fields come from an $n$-category, this correspondence works best if we think of each 
twig label $u_i$ as having the form
$x - s(e(x))$, where $x$ is an arbitrary field on $B_i$, $e: \cF(B_i) \to C$ is the evaluation map, 
and $s:C \to \cF(B_i)$ is some fixed section of $e$.)

For lack of a better name, 
we'll call elements of $P$ cone-product polyhedra, 
and say that blob diagrams have the structure of a cone-product set (analogous to simplicial set).
\end{remark}

