%!TEX root = ../blob1.tex
\nn{Not sure where this goes yet: small blobs, unfinished:}

Fix $\cU$, an open cover of $M$. Define the `small blob complex' $\bc^{\cU}_*(M)$ to be the subcomplex of $\bc_*(M)$ of all blob diagrams in which every blob is contained in some open set of $\cU$.

\begin{thm}[Small blobs]
The inclusion $i: \bc^{\cU}_*(M) \into \bc_*(M)$ is a homotopy equivalence.
\end{thm}
\begin{proof}
We begin by describing the homotopy inverse in small degrees, to illustrate the general technique.
We will construct a chain map $s:  \bc_*(M) \to \bc^{\cU}_*(M)$ and a homotopy $h:\bc_*(M) \to \bc_{*+1}(M)$ so that $\bdy h+h \bdy=\id - i\circ s$. The composition $s \circ i$ will just be the identity.

On $0$-blobs, $s$ is just the identity; a blob diagram without any blobs is compatible with any open cover. Nevertheless, we'll begin introducing nomenclature at this point: for configuration $\beta$ of disjoint embedded balls in $M$ we'll associate a one parameter family of homeomorphisms $\phi_\beta : \Delta^1 \to \Homeo(M)$ (here $\Delta^m$ is the standard simplex $\setc{\mathbf{x} \in \Real^{m+1}}{\sum_i x_i = 1}$). For $0$-blobs, where $\beta = \eset$, all these homeomorphisms are just the identity.

On a $1$-blob $b$, with ball $\beta$, $s$ is defined as the sum of two terms. Essentially, the first term `makes $\beta$ small', while the other term `gets the boundary right'. First, pick a one-parameter family $\phi_\beta : \Delta^1 \to \Homeo(M)$ of homeomorphisms, so $\phi_\beta(0,1)$ is the identity and $\phi_\beta(1,0)$ makes the ball $\beta$ small. Next, pick a two-parameter family $\phi_{\eset \prec \beta} : \Delta^2 \to \Homeo(M)$ so that $\phi_{\eset \prec \beta}(s,t,0)$ makes the ball $\beta$ small for all $s+t=1$, while $\phi_{\eset \prec \beta}(0,t,u) = \phi_\eset(t,u)$ and $\phi_{\eset \prec \beta}(s,0,u) = \phi_\beta(s,u)$. (It's perhaps not obvious that this is even possible --- see Lemma \ref{lem:extend-small-homeomorphisms} below.) We now define $s$ by
$$s(b) = \phi_\beta(1,0)(b) + \restrict{\phi_{\eset \prec \beta}}{u=0}(\bdy b).$$
Here, $\phi_\beta(1,0)$ is just a homeomorphism, which we apply to $b$, while $\restrict{\phi_{\eset \prec \beta}}{u=0}$ is a one parameter family of homeomorphisms which acts on the $0$-blob $\bdy b$ to give a $1$-blob. We now check that $s$, as defined so far, is a chain map, calculating
\begin{align*}
\bdy (s(b)) & = \phi_\beta(1,0)(\bdy b) + (\bdy \restrict{\phi_{\eset \prec \beta}}{u=0})(\bdy b) \\
		 & = \phi_\beta(1,0)(\bdy b) + \phi_\eset(1,0)(\bdy b) - \phi_\beta(1,0)(\bdy b) \\
		 & = \phi_\eset(1,0)(\bdy b) \\
		 & = s(\bdy b)
\end{align*}
Next, we compute the compositions $s \circ i$ and $i \circ s$. If we start with a small $1$-blob diagram $b$, first include it up to the full blob complex then apply $s$, we get exactly back to $b$, at least assuming we adopt the convention that for any ball $\beta$ which is already small, we choose the families of homeomorphisms $\phi_\beta$ and $\phi_{\eset \prec \beta}$ to always be the identity. In the other direction, $i \circ s$, we will need to construct the homotopy $h:\bc_*(M) \to \bc_{*+1}(M)$ for $*=0$ or $1$. This is defined by $h(b) = \phi_\eset(b)$ when $b$ is a $0$-blob (here $\phi_\eset$ is a one parameter family of homeomorphisms, so this is a $1$-blob), and $h(b) = \phi_\beta(b) - \phi_{\eset \prec \beta}(\bdy b)$ when $b$ is a $1$-blob (here $\beta$ is the ball in $b$, and this is the action of a one parameter family of homeomorphisms on a $1$-blob, so a $2$-blob).

\begin{align*}
(\bdy h+h \bdy)(b) & = \bdy (\phi_{\beta}(b) - \phi_{\eset \prec \beta}{\bdy b}) + \phi_\eset(\bdy b)  \\
	& = b - \phi_\beta(1,0)(b) - \phi_\beta(\bdy b) - (\bdy \phi_{\eset \prec \beta})(\bdy b) + \phi_\eset(\bdy b) \\
	& =  b - \phi_\beta(1,0)(b) - \phi_\beta(\bdy b) -  \phi_\eset(\bdy b) + \phi_\beta(\bdy b) - \restrict{\phi_{\eset \prec \beta}}{u=0}(\bdy b) + \phi_\eset(\bdy b) \\
	& = b - \phi_\beta(1,0)(b) - \restrict{\phi_{\eset \prec \beta}}{u=0}(\bdy b) \\
	& = (\id - i \circ s)(b)
\end{align*}


Given a blob diagram $b \in \bc_k(M)$, denote by $b_\cS$ for $\cS \subset \{1, \ldots, k\}$ the blob diagram obtained by erasing the corresponding blobs. In particular, $b_\eset = b$, $b_{\{1,\ldots,k\}} \in \bc_0(M)$, and $d b_\cS = \sum_{\cS' = \cS'\sqcup\{i\}} \pm  b_{\cS'}$.
Similarly, for a disjoint embedding of $k$ balls $\beta$ (that is, a blob diagram but without the labels on regions), $\beta_\cS$ denotes the result of erasing a subset of blobs. We'll write $\beta' \prec \beta$ if $\beta' = \beta_\cS$ for some $\cS$. Finally, for finite sequences, we'll write $i \prec i'$ if $i$ is subsequence of $i'$, and $i \prec_1 i$ if the lengths differ by exactly 1.

Next, we'll choose a `shrinking system' for $\cU$, namely for each increasing sequence of blob configurations
$\beta_0 \prec \beta_1 \prec \cdots \prec \beta_m$, an $m$ parameter family of diffeomorphisms
$\phi_{\beta_0 \prec \cdots \prec \beta_m} : \Delta^m \to \Diff{M}$ (here $\Delta^m$ is the standard simplex $\setc{\mathbf{x} \in \Real^{m+1}}{\sum_i x_i = 1}$), such that
\begin{itemize}
\item if $\beta$ is the empty configuration, $\phi_{\beta}(1) = \id_M$,
\item if $\beta$ is a single configuration of blobs, then $\phi_{\beta}(1)(\beta)$ (which is another configuration of blobs: $\phi_{\beta}(1)$ is a diffeomorphism of $M$) is subordinate to $\cU$,
\item (more generally) for any $x$ with $x_0 = 0$, $\phi_{\beta_0 \prec \cdots \prec \beta_m}(x)(\beta)$ is subordinate to $\cU$, and
\item for each $i = 1, \ldots, m$,
\begin{align*}
\phi_{\beta_0 \prec \cdots \prec \beta_m}(x_0, \ldots, x_{i-1},0,x_{i+1},\ldots,x_m) & = \phi_{\beta_0 \prec \cdots \beta_{i-1} \prec \beta_{i+1} \prec \beta_m}(x_0,\ldots, x_{i-1},x_{i+1},\ldots,x_m).
\end{align*}
\end{itemize}
It's not immediately obvious that it's possible to make such choices, but it follows readily from the following Lemma.

When $\beta$ is a collection of disjoint embedded balls in $M$, we say that a homeomorphism of $M$ `makes $\beta$ small' if the image of each ball in $\beta$ under the homeomorphism is contained in some open set of $\cU$.

\begin{lem}
\label{lem:extend-small-homeomorphisms}
Fix a collection of disjoint embedded balls $\beta$ in $M$. Suppose we have a map $f :  X \to \Homeo(M)$ on some compact $X$ such that for each $x \in \bdy X$, $f(x)$ makes $\beta$ small. Then we can extend $f$ to a map $\tilde{f} : X \times [0,1] \to \Homeo(M)$ so that $\tilde{f}(x,0) = f(x)$ and for every $x \in \bdy X \times [0,1] \cup X \times \{1\}$, $\tilde{f}(x)$ makes $\beta$ small.
\end{lem}
\begin{proof}
Fix a metric on $M$, and pick $\epsilon > 0$ so every $\epsilon$ ball in $M$ is contained in some open set of $\cU$. First construct a family of homeomorphisms $g_s : M \to M$, $s \in [1,\infty)$ so $g_1$ is the identity, and $g_s(\beta_i) \subset \beta_i$ and $\rad g_s(\beta_i) \leq \frac{1}{s} \rad \beta_i$ for each ball $\beta_i$. 
There is some $K$ which uniformly bounds the expansion factors of all the homeomorphisms $f(x)$, that is $d(f(x)(a), f(x)(b)) < K d(a,b)$ for all $x \in X, a,b \in M$. Write $S=\epsilon^{-1} K \max_i \{\rad \beta_i\}$ (note that is $S<1$, we can just take $S=1$, as already $f(x)$ makes $\beta$ small for all $x$). Now define $\tilde{f}(t, x) = f(x) \compose g_{(S-1)t+1}$.

If $x \in \bdy X$, then $g_{(S-1)t+1}(\beta_i) \subset \beta_i$, and by hypothesis $f(x)$ makes $\beta_i$ small, so $\tilde{f}(t, x)$ makes $\beta$ small for all $t \in [0,1]$. Alternatively, $\rad g_S(\beta_i) \leq \frac{1}{S} \rad \beta_i \leq \frac{\epsilon}{K}$, so $\rad \tilde{f}(1,x)(\beta_i) \leq \epsilon$, and so $\tilde{f}(1,x)$ makes $\beta$ small for all $x \in X$.
\end{proof}

We'll need a stronger version of Property \ref{property:evaluation}; while the evaluation map $ev: \CD{M} \tensor \bc_*(M) \to \bc_*(M)$ is not unique, it has an up-to-homotopy representative (satisfying the usual conditions) which restricts to become a chain map $ev: \CD{M} \tensor \bc^{\cU}_*(M) \to \bc^{\cU}_*(M)$. The proof is straightforward: when deforming the family of diffeomorphisms to shrink its supports to a union of open sets, do so such that those open sets are subordinate to the cover.

Now define a map $s: \bc_*(M) \to \bc^{\cU}_*(M)$, and then a homotopy $h:\bc_*(M) \to \bc_{*+1}(M)$ so that $dh+hd=i\circ s$. The map $s: \bc_0(M) \to \bc^{\cU}_0(M)$ is just the identity; blob diagrams without blobs are automatically compatible with any cover. Given a blob diagram $b$, we'll abuse notation and write $\phi_b$ to mean $\phi_\beta$ for the blob configuration $\beta$ underlying $b$. We have
$$s(b) = \sum_{i} ev(\restrict{\phi_{i(b)}}{x_0 = 0} \tensor b_i)$$
where the sum is over sequences $i=(i_1,\ldots,i_m)$ in $\{1,\ldots,k\}$, with $0\leq m < k$, $i(b)$ denotes the increasing sequence of blob configurations
$$\beta_{(i_1,\ldots,i_m)} \prec \beta_{(i_2,\ldots,i_m)} \prec \cdots \prec \beta_{()},$$
and, as usual, $i(b)$ denotes $b$ with blobs $i_1, \ldots i_m$ erased. We'll also write
$$s(b) = \sum_{m=0}^{k-1} \sum_{\norm{i}=m} ev(\restrict{\phi_{i(b)}}{x_0 = 0} \tensor b_i),$$
arranging the sum according to the length $\norm{i}$ of $i$.


We need to check that $s$ is a chain map, and that the image of $s$ in fact lies in $\bc^{\cU}_*(M)$. \todo{} Calculate
\begin{align*}
\bdy(s(b)) & = \sum_{m=0}^{k-1} \sum_{\norm{i}=m} \ev\left(\bdy(\restrict{\phi_{i(b)}}{x_0 = 0})\tensor b_i\right) + (-1)^m \ev\left(\restrict{\phi_{i(b)}}{x_0 = 0} \tensor \bdy b_i\right) \\
                & = \sum_{m=0}^{k-1} \sum_{\norm{i}=m} \ev\left(\sum_{i' \prec_1 i} \pm \restrict{\phi_{i'(b)}}{x_0 = 0})\tensor b_i\right) + (-1)^m \ev\left(\restrict{\phi_{i(b)}}{x_0 = 0}\tensor \sum_{i \prec_1 i'} \pm b_{i'}\right) \\
\intertext{and telescoping the sum}
		& = \sum_{m=0}^{k-2} \left(\sum_{\norm{i}=m}  (-1)^m \ev\left(\restrict{\phi_{i(b)}}{x_0 = 0} \tensor \sum_{i \prec_1 i'} \pm b_{i'}\right) \right) + \left(\sum_{\norm{i}=m+1} \ev\left(\sum_{i' \prec_1 i} \pm \restrict{\phi_{i'(b)}}{x_0 = 0} \tensor b_i\right) \right) + \\
		& \qquad + (-1)^{k-1} \sum_{\norm{i}=k-1} \ev\left(\restrict{\phi_{i(b)}}{x_0 = 0} \tensor \sum_{i \prec_1 i'} \pm b_{i'}\right) \\
		& = (-1)^{k-1} \sum_{\norm{i}=k-1} \ev\left(\restrict{\phi_{i(b)}}{x_0 = 0} \tensor \sum_{i \prec_1 i'} \pm b_{i'}\right)
\end{align*}

Next, we define the homotopy $h:\bc_*(M) \to \bc_{*+1}(M)$ by
$$h(b) = \sum_{i} ev(\phi_{i(b)}, b_i).$$
\todo{and check that it's the right one...}
\end{proof}
