%!TEX root = ../blob1.tex

\section{Comparing $n$-category definitions}
\label{sec:comparing-defs}

In this appendix we relate the ``topological" category definitions of Section \ref{sec:ncats}
to more traditional definitions, for $n=1$ and 2.

\subsection{Plain 1-categories}

Given a topological 1-category $\cC$, we construct a traditional 1-category $C$.
(This is quite straightforward, but we include the details for the sake of completeness and
to shed some light on the $n=2$ case.)

Let the objects of $C$ be $C^0 \deq \cC(B^0)$ and the morphisms of $C$ be $C^1 \deq \cC(B^1)$, 
where $B^k$ denotes the standard $k$-ball.
The boundary and restriction maps of $\cC$ give domain and range maps from $C^1$ to $C^0$.

Choose a homeomorphism $B^1\cup_{pt}B^1 \to B^1$.
Define composition in $C$ to be the induced map $C^1\times C^1 \to C^1$ (defined only when range and domain agree).
By isotopy invariance in $\cC$, any other choice of homeomorphism gives the same composition rule.
Also by isotopy invariance, composition is associative.

Given $a\in C^0$, define $\id_a \deq a\times B^1$.
By extended isotopy invariance in $\cC$, this has the expected properties of an identity morphism.

\nn{(slash)id seems to rendering a a boldface 1 --- is this what we want?}

\medskip

For 1-categories based on oriented manifolds, there is no additional structure.

For 1-categories based on unoriented manifolds, there is a map $*:C^1\to C^1$
coming from $\cC$ applied to an orientation-reversing homeomorphism (unique up to isotopy) 
from $B^1$ to itself.
Topological properties of this homeomorphism imply that 
$a^{**} = a$ (* is order 2), * reverses domain and range, and $(ab)^* = b^*a^*$
(* is an anti-automorphism).

For 1-categories based on Spin manifolds,
the the nontrivial spin homeomorphism from $B^1$ to itself which covers the identity
gives an order 2 automorphism of $C^1$.

For 1-categories based on $\text{Pin}_-$ manifolds,
we have an order 4 antiautomorphism of $C^1$.

For 1-categories based on $\text{Pin}_+$ manifolds,
we have an order 2 antiautomorphism and also an order 2 automorphism of $C^1$,
and these two maps commute with each other.

\nn{need to also consider automorphisms of $B^0$ / objects}

\medskip

In the other direction, given a traditional 1-category $C$
(with objects $C^0$ and morphisms $C^1$) we will construct a topological
1-category $\cC$.

If $X$ is a 0-ball (point), let $\cC(X) \deq C^0$.
If $S$ is a 0-sphere, let $\cC(S) \deq C^0\times C^0$.
If $X$ is a 1-ball, let $\cC(X) \deq C^1$.
Homeomorphisms isotopic to the identity act trivially.
If $C$ has extra structure (e.g.\ it's a *-1-category), we use this structure
to define the action of homeomorphisms not isotopic to the identity
(and get, e.g., an unoriented topological 1-category).

The domain and range maps of $C$ determine the boundary and restriction maps of $\cC$.

Gluing maps for $\cC$ are determined my composition of morphisms in $C$.

For $X$ a 0-ball, $D$ a 1-ball and $a\in \cC(X)$, define the product morphism 
$a\times D \deq \id_a$.
It is not hard to verify that this has the desired properties.

\medskip

The compositions of the above two ``arrows" ($\cC\to C\to \cC$ and $C\to \cC\to C$) give back 
more or less exactly the same thing we started with.  
\nn{need better notation here}
As we will see below, for $n>1$ the compositions yield a weaker sort of equivalence.

\medskip

Similar arguments show that modules for topological 1-categories are essentially
the same thing as traditional modules for traditional 1-categories.

\subsection{Plain 2-categories}

Let $\cC$ be a topological 2-category.
We will construct a traditional pivotal 2-category.
(The ``pivotal" corresponds to our assumption of strong duality for $\cC$.)

We will try to describe the construction in such a way the the generalization to $n>2$ is clear,
though this will make the $n=2$ case a little more complicated than necessary.

Define the $k$-morphisms $C^k$ of $C$ to be $\cC(B^k)_E$, where $B^k$ denotes the standard
$k$-ball, which we also think of as the standard bihedron.
Since we are thinking of $B^k$ as a bihedron, we have a standard decomposition of the $\bd B^k$
into two copies of $B^{k-1}$ which intersect along the ``equator" $E \cong S^{k-2}$.
Recall that the subscript in $\cC(B^k)_E$ means that we consider the subset of $\cC(B^k)$
whose boundary is splittable along $E$.
This allows us to define the domain and range of morphisms of $C$ using
boundary and restriction maps of $\cC$.

Choosing a homeomorphism $B^1\cup B^1 \to B^1$ defines a composition map on $C^1$.
This is not associative, but we will see later that it is weakly associative.

Choosing a homeomorphism $B^2\cup B^2 \to B^2$ defines a ``vertical" composition map on $C^2$.
Isotopy invariance implies that this is associative.
We will define a ``horizontal" composition later.





\nn{...}

\medskip
\hrule
\medskip

\nn{to be continued...}
\medskip
