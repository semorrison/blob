%!TEX root = ../blob1.tex

\def\xxpar#1#2{\smallskip\noindent{\bf #1} {\it #2} \smallskip}

\section{$n$-categories (maybe)}
\label{sec:ncats}

\nn{experimental section.  maybe this should be rolled into other sections.
maybe it should be split off into a separate paper.}

Before proceeding, we need more appropriate definitions of $n$-categories, 
$A_\infty$ $n$-categories, modules for these, and tensor products of these modules.
(As is the case throughout this paper, by ``$n$-category" we mean
a weak $n$-category with strong duality.)

Consider first ordinary $n$-categories.
We need a set (or sets) of $k$-morphisms for each $0\le k \le n$.
We must decide on the ``shape" of the $k$-morphisms.
Some $n$-category definitions model $k$-morphisms on the standard bihedron (interval, bigon, ...).
Other definitions have a separate set of 1-morphisms for each interval $[0,l] \sub \r$, 
a separate set of 2-morphisms for each rectangle $[0,l_1]\times [0,l_2] \sub \r^2$,
and so on.
(This allows for strict associativity.)
Still other definitions \nn{need refs for all these; maybe the Leinster book}
model the $k$-morphisms on more complicated combinatorial polyhedra.

We will allow our $k$-morphisms to have any shape, so long as it is homeomorphic to a $k$-ball.
In other words,

\xxpar{Morphisms (preliminary version):}{For any $k$-manifold $X$ homeomorphic 
to a $k$-ball, we have a set of $k$-morphisms
$\cC(X)$.}

Given a homeomorphism $f:X\to Y$ between such $k$-manifolds, we want a corresponding
bijection of sets $f:\cC(X)\to \cC(Y)$.
So we replace the above with

\xxpar{Morphisms:}{For each $0 \le k \le n$, we have a functor $\cC_k$ from 
the category of manifolds homeomorphic to the $k$-ball and 
homeomorphisms to the category of sets and bijections.}

(Note: We usually omit the subscript $k$.)

We are being deliberately vague about what flavor of manifolds we are considering.
They could be unoriented or oriented or Spin or $\mbox{Pin}_\pm$.
They could be topological or PL or smooth.
(If smooth, ``homeomorphism" should be read ``diffeomorphism", and we would need
to be fussier about corners.)
For each flavor of manifold there is a corresponding flavor of $n$-category.
We will concentrate of the case of PL unoriented manifolds.

Next we consider domains and ranges of morphisms (or, as we prefer to say, boundaries
of morphisms).
The 0-sphere is unusual among spheres in that it is disconnected.
Correspondingly, for 1-morphisms it makes sense to distinguish between domain and range.
For $k>1$ and in the presence of strong duality the domain/range division makes less sense.
\nn{maybe say more here; rotate disk, Frobenius reciprocity blah blah}
We prefer to combine the domain and range into a single entity which we call the 
boundary of a morphism.
Morphisms are modeled on balls, so their boundaries are modeled on spheres:

\xxpar{Boundaries (domain and range), part 1:}
{For each $0 \le k \le n-1$, we have a functor $\cC_k$ from 
the category of manifolds homeomorphic to the $k$-sphere and 
homeomorphisms to the category of sets and bijections.}

(In order to conserve symbols, we use the same symbol $\cC_k$ for both morphisms and boundaries.)

\xxpar{Boundaries, part 2:}
{For each $X$ homeomorphic to a $k$-ball, we have a map of sets $\bd: \cC(X)\to \cC(\bd X)$.
These maps, for various $X$, comprise a natural transformation of functors.}

(Note that the first ``$\bd$" above is part of the data for the category, 
while the second is the ordinary boundary of manifolds.)

Given $c\in\cC(\bd(X))$, let $\cC(X; c) = \bd^{-1}(c)$.

Most of the examples of $n$-categories we are interested in are enriched in the following sense.
The various sets of $n$-morphisms $\cC(X; c)$, for all $X$ homeomorphic to an $n$-ball and
all $c\in \cC(\bd X)$, have the structure of an object in some auxiliary category
(e.g.\ vector spaces, or modules over some ring, or chain complexes),
and all the structure maps of the $n$-category should be compatible with the auxiliary
category structure.
Note that this auxiliary structure is only in dimension $n$;
$\cC(Y; c)$ is just a plain set if $\dim(Y) < n$.

\medskip
\nn{At the moment I'm a little confused about orientations, and more specifically
about the role of orientation-reversing maps of boundaries when gluing oriented manifolds.
Tentatively, I think we need to redefine the oriented boundary of an oriented $n$-manifold.
Instead of an ordinary oriented $(n-1)$-manifold via the inward (or outward) normal 
first (or last) convention, perhaps it is better to define the boundary to be an $(n-1)$-manifold
equipped with an orientation of its once-stabilized tangent bundle.
Similarly, in dimension $n-k$ we would have manifolds equipped with an orientation of 
their $k$ times stabilized tangent bundles.
For the moment just stick with unoriented manifolds.}
\medskip

We have just argued that the boundary of a morphism has no preferred splitting into
domain and range, but the converse meets with our approval.
That is, given compatible domain and range, we should be able to combine them into
the full boundary of a morphism:

\xxpar{Domain $+$ range $\to$ boundary:}
{Let $S = B_1 \cup_E B_2$, where $S$ is homeomorphic to a $k$-sphere ($0\le k\le n-1$),
$B_i$ is homeomorphic to a $k$-ball, and $E = B_1\cap B_2$ is homeomorphic to  a $k{-}1$-sphere.
Let $\cC(B_1) \times_{\cC(E)} \cC(B_2)$ denote the fibered product of the 
two maps $\bd: \cC(B_i)\to \cC(E)$.
Then (axiom) we have an injective map
\[
	\gl_E : \cC(B_1) \times_{\cC(E)} \cC(B_2) \to \cC(S)
\]
which is natural with respect to the actions of homeomorphisms.}

Note that we insist on injectivity above.
Let $\cC(S)_E$ denote the image of $\gl_E$.
We have ``restriction" maps $\cC(S)_E \to \cC(B_i)$, which can be thought of as
domain and range maps, relative to the choice of splitting $S = B_1 \cup_E B_2$.

If $B$ is homeomorphic to a $k$-ball and $E \sub \bd B$ splits $\bd B$ into two $k{-}1$-balls
as above, then we define $\cC(B)_E = \bd^{-1}(\cC(\bd B)_E)$.

Next we consider composition of morphisms.
For $n$-categories which lack strong duality, one usually considers
$k$ different types of composition of $k$-morphisms, each associated to a different direction.
(For example, vertical and horizontal composition of 2-morphisms.)
In the presence of strong duality, these $k$ distinct compositions are subsumed into 
one general type of composition which can be in any ``direction".

\xxpar{Composition:}
{Let $B = B_1 \cup_Y B_2$, where $B$, $B_1$ and $B_2$ are homeomorphic to $k$-balls ($0\le k\le n$)
and $Y = B_1\cap B_2$ is homeomorphic to a $k{-}1$-ball.
Let $E = \bd Y$, which is homeomorphic to a $k{-}2$-sphere.
Note that each of $B$, $B_1$ and $B_2$ has its boundary split into two $k{-}1$-balls by $E$.
We have restriction (domain or range) maps $\cC(B_i)_E \to \cC(Y)$.
Let $\cC(B_1)_E \times_{\cC(Y)} \cC(B_2)_E$ denote the fibered product of these two maps. 
Then (axiom) we have a map
\[
	\gl_Y : \cC(B_1)_E \times_{\cC(Y)} \cC(B_2)_E \to \cC(B)_E
\]
which is natural with respect to the actions of homeomorphisms, and also compatible with restrictions
to the intersection of the boundaries of $B$ and $B_i$.
If $k < n$ we require that $\gl_Y$ is injective.
(For $k=n$, see below.)}

\xxpar{Strict associativity:}
{The composition (gluing) maps above are strictly associative.
It follows that given a decomposition $B = B_1\cup\cdots\cup B_m$ of a $k$-ball
into small $k$-balls, there is a well-defined
map from an appropriate subset of $\cC(B_1)\times\cdots\times\cC(B_m)$ to $\cC(B)$,
and these various $m$-fold composition maps satisfy an
operad-type associativity condition.}

\nn{above maybe needs some work}

The next axiom is related to identity morphisms, though that might not be immediately obvious.

\xxpar{Product (identity) morphisms:}
{Let $X$ be homeomorphic to a $k$-ball and $D$ be homeomorphic to an $m$-ball, with $k+m \le n$.
Then we have a map $\cC(X)\to \cC(X\times D)$, usually denoted $a\mapsto a\times D$ for $a\in \cC(X)$.
If $f:X\to X'$ and $\tilde{f}:X\times D \to X'\times D'$ are maps such that the diagram
\[ \xymatrix{
	X\times D \ar[r]^{\tilde{f}} \ar[d]^{\pi} & X'\times D' \ar[d]^{\pi} \\
	X \ar[r]^{f} & X'
} \]
commutes, then we have $\tilde{f}(a\times D) = f(a)\times D'$.}

\nn{Need to say something about compatibility with gluing (of both $X$ and $D$) above.}

All of the axioms listed above hold for both ordinary $n$-categories and $A_\infty$ $n$-categories.
The last axiom (below), concerning actions of 
homeomorphisms in the top dimension $n$, distinguishes the two cases.

We start with the plain $n$-category case.

\xxpar{Isotopy invariance in dimension $n$ (preliminary version):}
{Let $X$ be homeomorphic to the $n$-ball and $f: X\to X$ be a homeomorphism which restricts
to the identity on $\bd X$ and is isotopic (rel boundary) to the identity.
Then $f(a) = a$ for all $a\in \cC(X)$.}





\medskip

\hrule

\medskip

\nn{to be continued...}

