%!TEX root = ../blob1.tex

\section{Basic properties of the blob complex}
\label{sec:basic-properties}

In this section we complete the proofs of Properties 2-4.
Throughout the paper, where possible, we prove results using Properties 1-4, 
rather than the actual definition of blob homology.
This allows the possibility of future improvements to or alternatives on our definition.
In fact, we hope that there may be a characterisation of blob homology in 
terms of Properties 1-4, but at this point we are unaware of one.

Recall Property \ref{property:disjoint-union}, 
that there is a natural isomorphism $\bc_*(X \du Y) \cong \bc_*(X) \otimes \bc_*(Y)$.

\begin{proof}[Proof of Property \ref{property:disjoint-union}]
Given blob diagrams $b_1$ on $X$ and $b_2$ on $Y$, we can combine them
(putting the $b_1$ blobs before the $b_2$ blobs in the ordering) to get a
blob diagram $(b_1, b_2)$ on $X \du Y$.
Because of the blob reordering relations, all blob diagrams on $X \du Y$ arise this way.
In the other direction, any blob diagram on $X\du Y$ is equal (up to sign)
to one that puts $X$ blobs before $Y$ blobs in the ordering, and so determines
a pair of blob diagrams on $X$ and $Y$.
These two maps are compatible with our sign conventions.
(We follow the usual convention for tensors products of complexes, 
as in e.g. \cite{MR1438306}: $d(a \tensor b) = da \tensor b + (-1)^{\deg(a)} a \tensor db$.)
The two maps are inverses of each other.
\end{proof}

For the next proposition we will temporarily restore $n$-manifold boundary
conditions to the notation.

Suppose that for all $c \in \cC(\bd B^n)$
we have a splitting $s: H_0(\bc_*(B^n, c)) \to \bc_0(B^n; c)$
of the quotient map
$p: \bc_0(B^n; c) \to H_0(\bc_*(B^n, c))$.
For example, this is always the case if the coefficient ring is a field.
Then
\begin{prop} \label{bcontract}
For all $c \in \cC(\bd B^n)$ the natural map $p: \bc_*(B^n, c) \to H_0(\bc_*(B^n, c))$
is a chain homotopy equivalence
with inverse $s: H_0(\bc_*(B^n, c)) \to \bc_*(B^n; c)$.
Here we think of $H_0(\bc_*(B^n, c))$ as a 1-step complex concentrated in degree 0.
\end{prop}
\begin{proof}
By assumption $p\circ s = \id$, so all that remains is to find a degree 1 map
$h : \bc_*(B^n; c) \to \bc_*(B^n; c)$ such that $\bd h + h\bd = \id - s \circ p$.
For $i \ge 1$, define $h_i : \bc_i(B^n; c) \to \bc_{i+1}(B^n; c)$ by adding
an $(i{+}1)$-st blob equal to all of $B^n$.
In other words, add a new outermost blob which encloses all of the others.
Define $h_0 : \bc_0(B^n; c) \to \bc_1(B^n; c)$ by setting $h_0(x)$ equal to
the 1-blob with blob $B^n$ and label $x - s(p(x)) \in U(B^n; c)$.
\end{proof}
This proves Property \ref{property:contractibility} (the second half of the 
statement of this Property was immediate from the definitions).
Note that even when there is no splitting $s$, we can let $h_0 = 0$ and get a homotopy
equivalence to the 2-step complex $U(B^n; c) \to \cC(B^n; c)$.

For fields based on $n$-categories, $H_0(\bc_*(B^n; c)) \cong \mor(c', c'')$,
where $(c', c'')$ is some (any) splitting of $c$ into domain and range.

\begin{cor} \label{disj-union-contract}
If $X$ is a disjoint union of $n$-balls, then $\bc_*(X; c)$ is contractible.
\end{cor}

\begin{proof}
This follows from Properties \ref{property:disjoint-union} and \ref{property:contractibility}.
\end{proof}

Define the {\it support} of a blob diagram to be the union of all the 
blobs of the diagram.
Define the support of a linear combination of blob diagrams to be the union of the 
supports of the constituent diagrams.
For future use we prove the following lemma.

\begin{lemma} \label{support-shrink}
Let $L_* \sub \bc_*(X)$ be a subcomplex generated by some
subset of the blob diagrams on $X$, and let $f: L_* \to L_*$
be a chain map which does not increase supports and which induces an isomorphism on
$H_0(L_*)$.
Then $f$ is homotopic (in $\bc_*(X)$) to the identity $L_*\to L_*$.
\end{lemma}

\begin{proof}
We will use the method of acyclic models.
Let $b$ be a blob diagram of $L_*$, let $S\sub X$ be the support of $b$, and let
$r$ be the restriction of $b$ to $X\setminus S$.
Note that $S$ is a disjoint union of balls.
Assign to $b$ the acyclic (in positive degrees) subcomplex $T(b) \deq r\bullet\bc_*(S)$.
Note that if a diagram $b'$ is part of $\bd b$ then $T(B') \sub T(b)$.
Both $f$ and the identity are compatible with $T$ (in the sense of acyclic models), 
so $f$ and the identity map are homotopic. \nn{We should actually have a section with a definition of ``compatible" and this statement as a lemma}
\end{proof}

For the next proposition we will temporarily restore $n$-manifold boundary
conditions to the notation.

Let $X$ be an $n$-manifold, $\bd X = Y \cup Y \cup Z$.
Gluing the two copies of $Y$ together yields an $n$-manifold $X\sgl$
with boundary $Z\sgl$.
Given compatible fields (boundary conditions) $a$, $b$ and $c$ on $Y$, $Y$ and $Z$,
we have the blob complex $\bc_*(X; a, b, c)$.
If $b = a$, then we can glue up blob diagrams on
$X$ to get blob diagrams on $X\sgl$.
This proves Property \ref{property:gluing-map}, which we restate here in more detail.

\textbf{Property \ref{property:gluing-map}.}\emph{
There is a natural chain map
\eq{
    \gl: \bigoplus_a \bc_*(X; a, a, c) \to \bc_*(X\sgl; c\sgl).
}
The sum is over all fields $a$ on $Y$ compatible at their
($n{-}2$-dimensional) boundaries with $c$.
``Natural" means natural with respect to the actions of diffeomorphisms.
}

This map is very far from being an isomorphism, even on homology.
We fix this deficit in Section \ref{sec:gluing} below.
