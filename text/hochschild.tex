In this section we analyze the blob complex in dimension $n=1$
and find that for $S^1$ the homology of the blob complex is the
Hochschild homology of the category (algebroid) that we started with.
\nn{or maybe say here that the complexes are quasi-isomorphic?  in general,
should perhaps put more emphasis on the complexes and less on the homology.}

Notation: $HB_i(X) = H_i(\bc_*(X))$.

Let us first note that there is no loss of generality in assuming that our system of
fields comes from a category.
(Or maybe (???) there {\it is} a loss of generality.
Given any system of fields, $A(I; a, b) = \cC(I; a, b)/U(I; a, b)$ can be
thought of as the morphisms of a 1-category $C$.
More specifically, the objects of $C$ are $\cC(pt)$, the morphisms from $a$ to $b$
are $A(I; a, b)$, and composition is given by gluing.
If we instead take our fields to be $C$-pictures, the $\cC(pt)$ does not change
and neither does $A(I; a, b) = HB_0(I; a, b)$.
But what about $HB_i(I; a, b)$ for $i > 0$?
Might these higher blob homology groups be different?
Seems unlikely, but I don't feel like trying to prove it at the moment.
In any case, we'll concentrate on the case of fields based on 1-category
pictures for the rest of this section.)

(Another question: $\bc_*(I)$ is an $A_\infty$-category.
How general of an $A_\infty$-category is it?
Given an arbitrary $A_\infty$-category can one find fields and local relations so
that $\bc_*(I)$ is in some sense equivalent to the original $A_\infty$-category?
Probably not, unless we generalize to the case where $n$-morphisms are complexes.)

Continuing...

Let $C$ be a *-1-category.
Then specializing the definitions from above to the case $n=1$ we have:
\begin{itemize}
\item $\cC(pt) = \ob(C)$ .
\item Let $R$ be a 1-manifold and $c \in \cC(\bd R)$.
Then an element of $\cC(R; c)$ is a collection of (transversely oriented)
points in the interior
of $R$, each labeled by a morphism of $C$.
The intervals between the points are labeled by objects of $C$, consistent with
the boundary condition $c$ and the domains and ranges of the point labels.
\item There is an evaluation map $e: \cC(I; a, b) \to \mor(a, b)$ given by
composing the morphism labels of the points.
Note that we also need the * of *-1-category here in order to make all the morphisms point
the same way.
\item For $x \in \mor(a, b)$ let $\chi(x) \in \cC(I; a, b)$ be the field with a single
point (at some standard location) labeled by $x$.
Then the kernel of the evaluation map $U(I; a, b)$ is generated by things of the
form $y - \chi(e(y))$.
Thus we can, if we choose, restrict the blob twig labels to things of this form.
\end{itemize}

We want to show that $HB_*(S^1)$ is naturally isomorphic to the
Hochschild homology of $C$.
\nn{Or better that the complexes are homotopic
or quasi-isomorphic.}
In order to prove this we will need to extend the blob complex to allow points to also
be labeled by elements of $C$-$C$-bimodules.
%Given an interval (1-ball) so labeled, there is an evaluation map to some tensor product
%(over $C$) of $C$-$C$-bimodules.
%Define the local relations $U(I; a, b)$ to be the direct sum of the kernels of these maps.
%Now we can define the blob complex for $S^1$.
%This complex is the sum of complexes with a fixed cyclic tuple of bimodules present.
%If $M$ is a $C$-$C$-bimodule, let $G_*(M)$ denote the summand of $\bc_*(S^1)$ corresponding
%to the cyclic 1-tuple $(M)$.
%In other words, $G_*(M)$ is a blob-like complex where exactly one point is labeled
%by an element of $M$ and the remaining points are labeled by morphisms of $C$.
%It's clear that $G_*(C)$ is isomorphic to the original bimodule-less
%blob complex for $S^1$.
%\nn{Is it really so clear?  Should say more.}

%\nn{alternative to the above paragraph:}
Fix points $p_1, \ldots, p_k \in S^1$ and $C$-$C$-bimodules $M_1, \ldots M_k$.
We define a blob-like complex $K_*(S^1, (p_i), (M_i))$.
The fields have elements of $M_i$ labeling $p_i$ and elements of $C$ labeling
other points.
The blob twig labels lie in kernels of evaluation maps.
(The range of these evaluation maps is a tensor product (over $C$) of $M_i$'s.)
Let $K_*(M) = K_*(S^1, (*), (M))$, where $* \in S^1$ is some standard base point.
In other words, fields for $K_*(M)$ have an element of $M$ at the fixed point $*$
and elements of $C$ at variable other points.

\todo{Some orphaned questions:}
\nn{Or maybe we should claim that $M \to K_*(M)$ is the/a derived coend.
Or maybe that $K_*(M)$ is quasi-isomorphic (or perhaps homotopic) to the Hochschild
complex of $M$.}

\nn{What else needs to be said to establish quasi-isomorphism to Hochschild complex?
Do we need a map from hoch to blob?
Does the above exactness and contractibility guarantee such a map without writing it
down explicitly?
Probably it's worth writing down an explicit map even if we don't need to.}


We claim that
\begin{thm}
The blob complex $\bc_*(S^1; C)$ on the circle is quasi-isomorphic to the
usual Hochschild complex for $C$.
\end{thm}

This follows from two results. First, we see that
\begin{lem}
\label{lem:module-blob}%
The complex $K_*(C)$ (here $C$ is being thought of as a
$C$-$C$-bimodule, not a category) is quasi-isomorphic to the blob complex
$\bc_*(S^1; C)$. (Proof later.)
\end{lem}

Next, we show that for any $C$-$C$-bimodule $M$,
\begin{prop}
The complex $K_*(M)$ is quasi-isomorphic to $HC_*(M)$, the usual
Hochschild complex of $M$.
\end{prop}
\begin{proof}
%First, since we're working over $\Complex$, note that saying two complexes are quasi-isomorphic simply means they have isomorphic homologies.
%\todo{We really don't want to work over $\Complex$, though; it would be nice to talk about torsion!}

Recall that the usual Hochschild complex of $M$ is uniquely determined, up to quasi-isomorphism, by the following properties:
\begin{enumerate}
\item \label{item:hochschild-additive}%
$HC_*(M_1 \oplus M_2) \cong HC_*(M_1) \oplus HC_*(M_2)$.
\item \label{item:hochschild-exact}%
An exact sequence $0 \to M_1 \into M_2 \onto M_3 \to 0$ gives rise to an
exact sequence $0 \to HC_*(M_1) \into HC_*(M_2) \onto HC_*(M_3) \to 0$.
\item \label{item:hochschild-free}%
$HC_*(C\otimes C)$ (the free $C$-$C$-bimodule with one generator) is
quasi-isomorphic to the 0-step complex $C$.
\item \label{item:hochschild-coinvariants}%
$HH_0(M)$ is isomorphic to the coinvariants of $M$, $M/\langle cm-mc \rangle$.
\end{enumerate}
(Together, these just say that Hochschild homology is `the derived functor of coinvariants'.)
We'll first recall why these properties are characteristic.

Take some $C$-$C$ bimodule $M$, and choose a free resolution
\begin{equation*}
\cdots \to F_2 \xrightarrow{f_2} F_1 \xrightarrow{f_1} F_0.
\end{equation*}
There's a quotient map $\pi: F_0 \onto M$, and by construction the cone of the chain map $\pi: F_j \to M$ is acyclic. Now construct the total complex
$HC_i(F_j)$, with $i,j \geq 0$, graded by $i+j$.

Observe that we have two chain maps
\begin{align*}
HC_i(F_j) & \xrightarrow{HC_i(\pi)} HC_i(M) \\
\intertext{and}
HC_i(F_j) & \xrightarrow{HC_0(F_j) \onto HH_0(F_j)} \operatorname{coinv}(F_j).
\end{align*}
The cone of each chain map is acyclic. In the first case, this is because the `rows' indexed by $i$ are acyclic since $HC_i$ is exact.
In the second case, this is because the `columns' indexed by $j$ are acyclic, since $F_j$ is free.

Because the cones are acyclic, the chain maps are quasi-isomorphisms. Composing one with the inverse of the other, we obtain the desired quasi-isomorphism
$$HC_*(M) \iso \operatorname{coinv}(F_*).$$

%If $M$ is free, that is, a direct sum of copies of
%$C \tensor C$, then properties \ref{item:hochschild-additive} and
%\ref{item:hochschild-free} determine $HC_*(M)$. Otherwise, choose some
%free cover $F \onto M$, and define $K$ to be this map's kernel. Thus we
%have a short exact sequence $0 \to K \into F \onto M \to 0$, and hence a
%short exact sequence of complexes $0 \to HC_*(K) \into HC_*(F) \onto HC_*(M)
%\to 0$. Such a sequence gives a long exact sequence on homology
%\begin{equation*}
%%\begin{split}
%\cdots \to HH_{i+1}(F) \to HH_{i+1}(M) \to HH_i(K) \to HH_i(F) \to \cdots % \\
%%\cdots \to HH_1(F) \to HH_1(M) \to HH_0(K) \to HH_0(F) \to HH_0(M).
%%\end{split}
%\end{equation*}
%For any $i \geq 1$, $HH_{i+1}(F) = HH_i(F) = 0$, by properties
%\ref{item:hochschild-additive} and \ref{item:hochschild-free}, and so
%$HH_{i+1}(M) \iso HH_i(F)$. For $i=0$, \todo{}.
%
%This tells us how to
%compute every homology group of $HC_*(M)$; we already know $HH_0(M)$
%(it's just coinvariants, by property \ref{item:hochschild-coinvariants}),
%and higher homology groups are determined by lower ones in $HC_*(K)$, and
%hence recursively as coinvariants of some other bimodule.

The proposition then follows from the following lemmas, establishing that $K_*$ has precisely these required properties.
\begin{lem}
\label{lem:hochschild-additive}%
Directly from the definition, $K_*(M_1 \oplus M_2) \cong K_*(M_1) \oplus K_*(M_2)$.
\end{lem}
\begin{lem}
\label{lem:hochschild-exact}%
An exact sequence $0 \to M_1 \into M_2 \onto M_3 \to 0$ gives rise to an
exact sequence $0 \to K_*(M_1) \into K_*(M_2) \onto K_*(M_3) \to 0$.
\end{lem}
\begin{lem}
\label{lem:hochschild-free}%
$K_*(C\otimes C)$ is quasi-isomorphic to the 0-step complex $C$.
\end{lem}
\begin{lem}
\label{lem:hochschild-coinvariants}%
$H_0(K_*(M))$ is isomorphic to the coinvariants of $M$, $M/\langle cm-mc \rangle$.
\end{lem}

The remainder of this section is devoted to proving Lemmas
\ref{lem:module-blob},
\ref{lem:hochschild-exact}, \ref{lem:hochschild-free} and
\ref{lem:hochschild-coinvariants}.
\end{proof}

\begin{proof}[Proof of Lemma \ref{lem:module-blob}]
We show that $K_*(C)$ is quasi-isomorphic to $\bc_*(S^1)$.
$K_*(C)$ differs from $\bc_*(S^1)$ only in that the base point *
is always a labeled point in $K_*(C)$, while in $\bc_*(S^1)$ it may or may not be.
In other words, there is an inclusion map $i: K_*(C) \to \bc_*(S^1)$.

We define a quasi-inverse \nn{right term?} $s: \bc_*(S^1) \to K_*(C)$ to the inclusion as follows.
If $y$ is a field defined on a neighborhood of *, define $s(y) = y$ if
* is a labeled point in $y$.
Otherwise, define $s(y)$ to be the result of adding a label 1 (identity morphism) at *.
Let $x \in \bc_*(S^1)$.
Let $s(x)$ be the result of replacing each field $y$ (containing *) mentioned in
$x$ with $y$.
It is easy to check that $s$ is a chain map and $s \circ i = \id$.

Let $L_*^\ep \sub \bc_*(S^1)$ be the subcomplex where there are no labeled points
in a neighborhood $B_\ep$ of *, except perhaps *.
Note that for any chain $x \in \bc_*(S^1)$, $x \in L_*^\ep$ for sufficiently small $\ep$.
\nn{rest of argument goes similarly to above}
\end{proof}
\begin{proof}[Proof of Lemma \ref{lem:hochschild-exact}]
\todo{}
\end{proof}
\begin{proof}[Proof of Lemma \ref{lem:hochschild-free}]
We show that $K_*(C\otimes C)$ is
quasi-isomorphic to the 0-step complex $C$.

Let $K'_* \sub K_*(C\otimes C)$ be the subcomplex where the label of
the point $*$ is $1 \otimes 1 \in C\otimes C$.
We will show that the inclusion $i: K'_* \to K_*(C\otimes C)$ is a quasi-isomorphism.

Fix a small $\ep > 0$.
Let $B_\ep$ be the ball of radius $\ep$ around $* \in S^1$.
Let $K_*^\ep \sub K_*(C\otimes C)$ be the subcomplex
generated by blob diagrams $b$ such that $B_\ep$ is either disjoint from
or contained in each blob of $b$, and the two boundary points of $B_\ep$ are not labeled points of $b$.
For a field (picture) $y$ on $B_\ep$, let $s_\ep(y)$ be the equivalent picture with~$*$
labeled by $1\otimes 1$ and the only other labeled points at distance $\pm\ep/2$ from $*$.
(See Figure xxxx.)
Note that $y - s_\ep(y) \in U(B_\ep)$.
\nn{maybe it's simpler to assume that there are no labeled points, other than $*$, in $B_\ep$.}

Define a degree 1 chain map $j_\ep : K_*^\ep \to K_*^\ep$ as follows.
Let $x \in K_*^\ep$ be a blob diagram.
If $*$ is not contained in any twig blob, $j_\ep(x)$ is obtained by adding $B_\ep$ to
$x$ as a new twig blob, with label $y - s_\ep(y)$, where $y$ is the restriction of $x$ to $B_\ep$.
If $*$ is contained in a twig blob $B$ with label $u = \sum z_i$, $j_\ep(x)$ is obtained as follows.
Let $y_i$ be the restriction of $z_i$ to $B_\ep$.
Let $x_i$ be equal to $x$ outside of $B$, equal to $z_i$ on $B \setmin B_\ep$,
and have an additional blob $B_\ep$ with label $y_i - s_\ep(y_i)$.
Define $j_\ep(x) = \sum x_i$.
\nn{need to check signs coming from blob complex differential}

Note that if $x \in K'_* \cap K_*^\ep$ then $j_\ep(x) \in K'_*$ also.

The key property of $j_\ep$ is
\eq{
    \bd j_\ep + j_\ep \bd = \id - \sigma_\ep ,
}
where $\sigma_\ep : K_*^\ep \to K_*^\ep$ is given by replacing the restriction $y$ of each field
mentioned in $x \in K_*^\ep$ with $s_\ep(y)$.
Note that $\sigma_\ep(x) \in K'_*$.

If $j_\ep$ were defined on all of $K_*(C\otimes C)$, it would show that $\sigma_\ep$
is a homotopy inverse to the inclusion $K'_* \to K_*(C\otimes C)$.
One strategy would be to try to stitch together various $j_\ep$ for progressively smaller
$\ep$ and show that $K'_*$ is homotopy equivalent to $K_*(C\otimes C)$.
Instead, we'll be less ambitious and just show that
$K'_*$ is quasi-isomorphic to $K_*(C\otimes C)$.

If $x$ is a cycle in $K_*(C\otimes C)$, then for sufficiently small $\ep$ we have
$x \in K_*^\ep$.
(This is true for any chain in $K_*(C\otimes C)$, since chains are sums of
finitely many blob diagrams.)
Then $x$ is homologous to $s_\ep(x)$, which is in $K'_*$, so the inclusion map
$K'_* \sub K_*(C\otimes C)$ is surjective on homology.
If $y \in K_*(C\otimes C)$ and $\bd y = x \in K'_*$, then $y \in K_*^\ep$ for some $\ep$
and
\eq{
    \bd y = \bd (\sigma_\ep(y) + j_\ep(x)) .
}
Since $\sigma_\ep(y) + j_\ep(x) \in F'$, it follows that the inclusion map is injective on homology.
This completes the proof that $K'_*$ is quasi-isomorphic to $K_*(C\otimes C)$.

Let $K''_* \sub K'_*$ be the subcomplex of $K'_*$ where $*$ is not contained in any blob.
We will show that the inclusion $i: K''_* \to K'_*$ is a homotopy equivalence.

First, a lemma:  Let $G''_*$ and $G'_*$ be defined the same as $K''_*$ and $K'_*$, except with
$S^1$ replaced some (any) neighborhood of $* \in S^1$.
Then $G''_*$ and $G'_*$ are both contractible
and the inclusion $G''_* \sub G'_*$ is a homotopy equivalence.
For $G'_*$ the proof is the same as in (\ref{bcontract}), except that the splitting
$G'_0 \to H_0(G'_*)$ concentrates the point labels at two points to the right and left of $*$.
For $G''_*$ we note that any cycle is supported \nn{need to establish terminology for this; maybe
in ``basic properties" section above} away from $*$.
Thus any cycle lies in the image of the normal blob complex of a disjoint union
of two intervals, which is contractible by (\ref{bcontract}) and (\ref{disjunion}).
Actually, we need the further (easy) result that the inclusion
$G''_* \to G'_*$ induces an isomorphism on $H_0$.

Next we construct a degree 1 map (homotopy) $h: K'_* \to K'_*$ such that
for all $x \in K'_*$ we have
\eq{
    x - \bd h(x) - h(\bd x) \in K''_* .
}
Since $K'_0 = K''_0$, we can take $h_0 = 0$.
Let $x \in K'_1$, with single blob $B \sub S^1$.
If $* \notin B$, then $x \in K''_1$ and we define $h_1(x) = 0$.
If $* \in B$, then we work in the image of $G'_*$ and $G''_*$ (with respect to $B$).
Choose $x'' \in G''_1$ such that $\bd x'' = \bd x$.
Since $G'_*$ is contractible, there exists $y \in G'_2$ such that $\bd y = x - x''$.
Define $h_1(x) = y$.
The general case is similar, except that we have to take lower order homotopies into account.
Let $x \in K'_k$.
If $*$ is not contained in any of the blobs of $x$, then define $h_k(x) = 0$.
Otherwise, let $B$ be the outermost blob of $x$ containing $*$.
By xxxx above, $x = x' \bullet p$, where $x'$ is supported on $B$ and $p$ is supported away from $B$.
So $x' \in G'_l$ for some $l \le k$.
Choose $x'' \in G''_l$ such that $\bd x'' = \bd (x' - h_{l-1}\bd x')$.
Choose $y \in G'_{l+1}$ such that $\bd y = x' - x'' - h_{l-1}\bd x'$.
Define $h_k(x) = y \bullet p$.
This completes the proof that $i: K''_* \to K'_*$ is a homotopy equivalence.
\nn{need to say above more clearly and settle on notation/terminology}

Finally, we show that $K''_*$ is contractible.
\nn{need to also show that $H_0$ is the right thing; easy, but I won't do it now}
Let $x$ be a cycle in $K''_*$.
The union of the supports of the diagrams in $x$ does not contain $*$, so there exists a
ball $B \subset S^1$ containing the union of the supports and not containing $*$.
Adding $B$ as a blob to $x$ gives a contraction.
\nn{need to say something else in degree zero}
\end{proof}
\begin{proof}[Proof of Lemma \ref{lem:hochschild-coinvariants}]
\todo{}
\end{proof}

We can also describe explicitly a map from the standard Hochschild
complex to the blob complex on the circle. \nn{What properties does this
map have?}

\begin{figure}%
$$\mathfig{0.6}{barycentric/barycentric}$$
\caption{The Hochschild chain $a \tensor b \tensor c$ is sent to
the sum of six blob $2$-chains, corresponding to a barycentric subdivision of a $2$-simplex.}
\label{fig:Hochschild-example}%
\end{figure}

As an example, Figure \ref{fig:Hochschild-example} shows the image of the Hochschild chain $a \tensor b \tensor c$. Only the $0$-cells are shown explicitly.
The edges marked $x, y$ and $z$ carry the $1$-chains
\begin{align*}
x & = \mathfig{0.1}{barycentric/ux} & u_x = \mathfig{0.1}{barycentric/ux_ca} - \mathfig{0.1}{barycentric/ux_c-a} \\
y & = \mathfig{0.1}{barycentric/uy} & u_y = \mathfig{0.1}{barycentric/uy_cab} - \mathfig{0.1}{barycentric/uy_ca-b} \\
z & = \mathfig{0.1}{barycentric/uz} & u_z = \mathfig{0.1}{barycentric/uz_c-a-b} - \mathfig{0.1}{barycentric/uz_cab}
\end{align*}
and the $2$-chain labelled $A$ is
\begin{equation*}
A = \mathfig{0.1}{barycentric/Ax}+\mathfig{0.1}{barycentric/Ay}.
\end{equation*}
Note that we then have
\begin{equation*}
\bdy A = x+y+z.
\end{equation*}

In general, the Hochschild chain $\Tensor_{i=1}^n a_i$ is sent to the sum of $n!$ blob $(n-1)$-chains, indexed by permutations,
$$\phi\left(\Tensor_{i=1}^n a_i\right) = \sum_{\pi} \phi^\pi(a_1, \ldots, a_n)$$
with ... (hmmm, problems making this precise; you need to decide where to put the labels, but then it's hard to make an honest chain map!)
