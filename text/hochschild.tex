%!TEX root = ../blob1.tex

In this section we analyze the blob complex in dimension $n=1$
and find that for $S^1$ the blob complex is homotopy equivalent to the 
Hochschild complex of the category (algebroid) that we started with.

\nn{need to be consistent about quasi-isomorphic versus homotopy equivalent
in this section.
since the various complexes are free, q.i. implies h.e.}

Let $C$ be a *-1-category.
Then specializing the definitions from above to the case $n=1$ we have:
\begin{itemize}
\item $\cC(pt) = \ob(C)$ .
\item Let $R$ be a 1-manifold and $c \in \cC(\bd R)$.
Then an element of $\cC(R; c)$ is a collection of (transversely oriented)
points in the interior
of $R$, each labeled by a morphism of $C$.
The intervals between the points are labeled by objects of $C$, consistent with
the boundary condition $c$ and the domains and ranges of the point labels.
\item There is an evaluation map $e: \cC(I; a, b) \to \mor(a, b)$ given by
composing the morphism labels of the points.
Note that we also need the * of *-1-category here in order to make all the morphisms point
the same way.
\item For $x \in \mor(a, b)$ let $\chi(x) \in \cC(I; a, b)$ be the field with a single
point (at some standard location) labeled by $x$.
Then the kernel of the evaluation map $U(I; a, b)$ is generated by things of the
form $y - \chi(e(y))$.
Thus we can, if we choose, restrict the blob twig labels to things of this form.
\end{itemize}

We want to show that $\bc_*(S^1)$ is homotopy equivalent to the
Hochschild complex of $C$.
Note that both complexes are free (and hence projective), so it suffices to show that they
are quasi-isomorphic.
In order to prove this we will need to extend the blob complex to allow points to also
be labeled by elements of $C$-$C$-bimodules.

Fix points $p_1, \ldots, p_k \in S^1$ and $C$-$C$-bimodules $M_1, \ldots M_k$.
We define a blob-like complex $K_*(S^1, (p_i), (M_i))$.
The fields have elements of $M_i$ labeling $p_i$ and elements of $C$ labeling
other points.
The blob twig labels lie in kernels of evaluation maps.
(The range of these evaluation maps is a tensor product (over $C$) of $M_i$'s.)
Let $K_*(M) = K_*(S^1, (*), (M))$, where $* \in S^1$ is some standard base point.
In other words, fields for $K_*(M)$ have an element of $M$ at the fixed point $*$
and elements of $C$ at variable other points.


We claim that
\begin{thm} \label{hochthm}
The blob complex $\bc_*(S^1; C)$ on the circle is quasi-isomorphic to the
usual Hochschild complex for $C$.
\end{thm}

This follows from two results. First, we see that
\begin{lem}
\label{lem:module-blob}%
The complex $K_*(C)$ (here $C$ is being thought of as a
$C$-$C$-bimodule, not a category) is homotopy equivalent to the blob complex
$\bc_*(S^1; C)$. (Proof later.)
\end{lem}

Next, we show that for any $C$-$C$-bimodule $M$,
\begin{prop} \label{prop:hoch}
The complex $K_*(M)$ is quasi-isomorphic to $HC_*(M)$, the usual
Hochschild complex of $M$.
\end{prop}
\begin{proof}
Recall that the usual Hochschild complex of $M$ is uniquely determined,
up to quasi-isomorphism, by the following properties:
\begin{enumerate}
\item \label{item:hochschild-additive}%
$HC_*(M_1 \oplus M_2) \cong HC_*(M_1) \oplus HC_*(M_2)$.
\item \label{item:hochschild-exact}%
An exact sequence $0 \to M_1 \into M_2 \onto M_3 \to 0$ gives rise to an
exact sequence $0 \to HC_*(M_1) \into HC_*(M_2) \onto HC_*(M_3) \to 0$.
\item \label{item:hochschild-coinvariants}%
$HH_0(M)$ is isomorphic to the coinvariants of $M$, $\coinv(M) =
M/\langle cm-mc \rangle$.
\item \label{item:hochschild-free}%
$HC_*(C\otimes C)$ is contractible.
(Here $C\otimes C$ denotes
the free $C$-$C$-bimodule with one generator.)
That is, $HC_*(C\otimes C)$ is
quasi-isomorphic to its $0$-th homology (which in turn, by \ref{item:hochschild-coinvariants}, is just $C$) via the quotient map $HC_0 \onto HH_0$.
\end{enumerate}
(Together, these just say that Hochschild homology is `the derived functor of coinvariants'.)
We'll first recall why these properties are characteristic.

Take some $C$-$C$ bimodule $M$, and choose a free resolution
\begin{equation*}
\cdots \to F_2 \xrightarrow{f_2} F_1 \xrightarrow{f_1} F_0.
\end{equation*}
We will show that for any functor $\cP$ satisfying properties
\ref{item:hochschild-additive}, \ref{item:hochschild-exact},
\ref{item:hochschild-coinvariants} and \ref{item:hochschild-free}, there
is a quasi-isomorphism
$$\cP_*(M) \iso \coinv(F_*).$$
%
Observe that there's a quotient map $\pi: F_0 \onto M$, and by
construction the cone of the chain map $\pi: F_* \to M$ is acyclic. Now
construct the total complex $\cP_i(F_j)$, with $i,j \geq 0$, graded by
$i+j$. We have two chain maps
\begin{align*}
\cP_i(F_*) & \xrightarrow{\cP_i(\pi)} \cP_i(M) \\
\intertext{and}
\cP_*(F_j) & \xrightarrow{\cP_0(F_j) \onto H_0(\cP_*(F_j))} \coinv(F_j).
\end{align*}
The cone of each chain map is acyclic. In the first case, this is because the `rows' indexed by $i$ are acyclic since $HC_i$ is exact.
In the second case, this is because the `columns' indexed by $j$ are acyclic, since $F_j$ is free.
Because the cones are acyclic, the chain maps are quasi-isomorphisms. Composing one with the inverse of the other, we obtain the desired quasi-isomorphism
$$\cP_*(M) \quismto \coinv(F_*).$$

%If $M$ is free, that is, a direct sum of copies of
%$C \tensor C$, then properties \ref{item:hochschild-additive} and
%\ref{item:hochschild-free} determine $HC_*(M)$. Otherwise, choose some
%free cover $F \onto M$, and define $K$ to be this map's kernel. Thus we
%have a short exact sequence $0 \to K \into F \onto M \to 0$, and hence a
%short exact sequence of complexes $0 \to HC_*(K) \into HC_*(F) \onto HC_*(M)
%\to 0$. Such a sequence gives a long exact sequence on homology
%\begin{equation*}
%%\begin{split}
%\cdots \to HH_{i+1}(F) \to HH_{i+1}(M) \to HH_i(K) \to HH_i(F) \to \cdots % \\
%%\cdots \to HH_1(F) \to HH_1(M) \to HH_0(K) \to HH_0(F) \to HH_0(M).
%%\end{split}
%\end{equation*}
%For any $i \geq 1$, $HH_{i+1}(F) = HH_i(F) = 0$, by properties
%\ref{item:hochschild-additive} and \ref{item:hochschild-free}, and so
%$HH_{i+1}(M) \iso HH_i(F)$. For $i=0$, \todo{}.
%
%This tells us how to
%compute every homology group of $HC_*(M)$; we already know $HH_0(M)$
%(it's just coinvariants, by property \ref{item:hochschild-coinvariants}),
%and higher homology groups are determined by lower ones in $HC_*(K)$, and
%hence recursively as coinvariants of some other bimodule.

Proposition \ref{prop:hoch} then follows from the following lemmas, establishing that $K_*$ has precisely these required properties.
\begin{lem}
\label{lem:hochschild-additive}%
Directly from the definition, $K_*(M_1 \oplus M_2) \cong K_*(M_1) \oplus K_*(M_2)$.
\end{lem}
\begin{lem}
\label{lem:hochschild-exact}%
An exact sequence $0 \to M_1 \into M_2 \onto M_3 \to 0$ gives rise to an
exact sequence $0 \to K_*(M_1) \into K_*(M_2) \onto K_*(M_3) \to 0$.
\end{lem}
\begin{lem}
\label{lem:hochschild-coinvariants}%
$H_0(K_*(M))$ is isomorphic to the coinvariants of $M$.
\end{lem}
\begin{lem}
\label{lem:hochschild-free}%
$K_*(C\otimes C)$ is quasi-isomorphic to $H_0(K_*(C \otimes C)) \iso C$.
\end{lem}

The remainder of this section is devoted to proving Lemmas
\ref{lem:module-blob},
\ref{lem:hochschild-exact}, \ref{lem:hochschild-coinvariants} and
\ref{lem:hochschild-free}.
\end{proof}

\begin{proof}[Proof of Lemma \ref{lem:module-blob}]
We show that $K_*(C)$ is quasi-isomorphic to $\bc_*(S^1)$.
$K_*(C)$ differs from $\bc_*(S^1)$ only in that the base point *
is always a labeled point in $K_*(C)$, while in $\bc_*(S^1)$ it may or may not be.
In particular, there is an inclusion map $i: K_*(C) \to \bc_*(S^1)$.

We define a left inverse $s: \bc_*(S^1) \to K_*(C)$ to the inclusion as follows.
If $y$ is a field defined on a neighborhood of *, define $s(y) = y$ if
* is a labeled point in $y$.
Otherwise, define $s(y)$ to be the result of adding a label 1 (identity morphism) at *.
Extending linearly, we get the desired map $s: \bc_*(S^1) \to K_*(C)$.
%Let $x \in \bc_*(S^1)$.
%Let $s(x)$ be the result of replacing each field $y$ (containing *) mentioned in
%$x$ with $s(y)$.
It is easy to check that $s$ is a chain map and $s \circ i = \id$.

Let $N_\ep$ denote the ball of radius $\ep$ around *.
Let $L_*^\ep \sub \bc_*(S^1)$ be the subcomplex 
spanned by blob diagrams
where there are no labeled points
in $N_\ep$, except perhaps $*$, and $N_\ep$ is either disjoint from or contained in 
every blob in the diagram.
Note that for any chain $x \in \bc_*(S^1)$, $x \in L_*^\ep$ for sufficiently small $\ep$.
\nn{what if * is on boundary of a blob?  need preliminary homotopy to prevent this.}

We define a degree $1$ chain map $j_\ep: L_*^\ep \to L_*^\ep$ as follows. Let $x \in L_*^\ep$ be a blob diagram.
\nn{maybe add figures illustrating $j_\ep$?}
If $*$ is not contained in any twig blob, we define $j_\ep(x)$ by adding $N_\ep$ as a new twig blob, with label $y - s(y)$ where $y$ is the restriction
of $x$ to $N_\ep$. If $*$ is contained in a twig blob $B$ with label $u=\sum z_i$,
write $y_i$ for the restriction of $z_i$ to $N_\ep$, and let
$x_i$ be equal to $x$ on $S^1 \setmin B$, equal to $z_i$ on $B \setmin N_\ep$,
and have an additional blob $N_\ep$ with label $y_i - s(y_i)$.
Define $j_\ep(x) = \sum x_i$.

It is not hard to show that on $L_*^\ep$
\[
	\bd j_\ep  + j_\ep \bd = \id - i \circ s .
\]
\nn{need to check signs coming from blob complex differential}
Since for $\ep$ small enough $L_*^\ep$ captures all of the
homology of $\bc_*(S^1)$, 
it follows that the mapping cone of $i \circ s$ is acyclic and therefore (using the fact that
these complexes are free) $i \circ s$ is homotopic to the identity.
\end{proof}

\begin{proof}[Proof of Lemma \ref{lem:hochschild-exact}]
We now prove that $K_*$ is an exact functor.

%\todo{p. 1478 of scott's notes}
Essentially, this comes down to the unsurprising fact that the functor on $C$-$C$ bimodules
\begin{equation*}
M \mapsto \ker(C \tensor M \tensor C \xrightarrow{c_1 \tensor m \tensor c_2 \mapsto c_1 m c_2} M)
\end{equation*}
is exact. For completeness we'll explain this below.

Suppose we have a short exact sequence of $C$-$C$ bimodules $$\xymatrix{0 \ar[r] & K \ar@{^{(}->}[r]^f & E \ar@{->>}[r]^g & Q \ar[r] & 0}.$$
We'll write $\hat{f}$ and $\hat{g}$ for the image of $f$ and $g$ under the functor.
Most of what we need to check is easy.
If $(a \tensor k \tensor b) \in \ker(C \tensor K \tensor C \to K)$, to have $\hat{f}(a \tensor k \tensor b) = 0 \in \ker(C \tensor E \tensor C \to E)$, we must have $f(k) = 0 \in E$, so
be $k=0$ itself. If $(a \tensor e \tensor b) \in \ker(C \tensor E \tensor C \to E)$ is in the image of $\ker(C \tensor K \tensor C \to K)$ under $\hat{f}$, clearly
$e$ is in the image of the original $f$, so is in the kernel of the original $g$, and so $\hat{g}(a \tensor e \tensor b) = 0$.
If $\hat{g}(a \tensor e \tensor b) = 0$, then $g(e) = 0$, so $e = f(\widetilde{e})$ for some $\widetilde{e} \in K$, and $a \tensor e \tensor b = \hat{f}(a \tensor \widetilde{e} \tensor b)$.
Finally, the interesting step is in checking that any $q = \sum_i a_i \tensor q_i \tensor b_i$ such that $\sum_i a_i q_i b_i = 0$ is in the image of $\ker(C \tensor E \tensor C \to C)$ under $\hat{g}$.
For each $i$, we can find $\widetilde{q_i}$ so $g(\widetilde{q_i}) = q_i$. However $\sum_i a_i \widetilde{q_i} b_i$ need not be zero.
Consider then $$\widetilde{q} = \sum_i (a_i \tensor \widetilde{q_i} \tensor b_i) - 1 \tensor (\sum_i a_i \widetilde{q_i} b_i) \tensor 1.$$ Certainly
$\widetilde{q} \in \ker(C \tensor E \tensor C \to E)$. Further,
\begin{align*}
\hat{g}(\widetilde{q}) & = \sum_i (a_i \tensor g(\widetilde{q_i}) \tensor b_i) - 1 \tensor (\sum_i a_i g(\widetilde{q_i}) b_i) \tensor 1 \\
                       & = q - 0
\end{align*}
(here we used that $g$ is a map of $C$-$C$ bimodules, and that $\sum_i a_i q_i b_i = 0$).

Identical arguments show that the functors
\begin{equation}
\label{eq:ker-functor}%
M \mapsto \ker(C^{\tensor k} \tensor M \tensor C^{\tensor l} \to M)
\end{equation}
are all exact too. Moreover, tensor products of such functors with each
other and with $C$ or $\ker(C^{\tensor k} \to C)$ (e.g., producing the functor $M \mapsto \ker(M \tensor C \to M)
\tensor C \tensor \ker(C \tensor C \to M)$) are all still exact.

Finally, then we see that the functor $K_*$ is simply an (infinite)
direct sum of copies of this sort of functor. The direct sum is indexed by
configurations of nested blobs and of labels; for each such configuration, we have one of the above tensor product functors,
with the labels of twig blobs corresponding to tensor factors as in \eqref{eq:ker-functor} or $\ker(C^{\tensor k} \to C)$, and all other labelled points corresponding
to tensor factors of $C$.
\end{proof}
\begin{proof}[Proof of Lemma \ref{lem:hochschild-coinvariants}]
We show that $H_0(K_*(M))$ is isomorphic to the coinvariants of $M$.

We define a map $\ev: K_0(M) \to M$. If $x \in K_0(M)$ has the label $m \in M$ at $*$, and labels $c_i \in C$ at the other labeled points of $S^1$, reading clockwise from $*$,
we set $\ev(x) = m c_1 \cdots c_k$. We can think of this as $\ev : M \tensor C^{\tensor k} \to M$, for each direct summand of $K_0(M)$ indexed by a configuration of labeled points.
There is a quotient map $\pi: M \to \coinv{M}$, and the composition $\pi \compose \ev$ is well-defined on the quotient $H_0(K_*(M))$; if $y \in K_1(M)$, the blob in $y$ either contains $*$ or does not. If it doesn't, then
suppose $y$ has label $m$ at $*$, labels $c_i$ at other labeled points outside the blob, and the field inside the blob is a sum, with the $j$-th term having
labeled points $d_{j,i}$. Then $\sum_j d_{j,1} \tensor \cdots \tensor d_{j,k_j} \in \ker(\DirectSum_k C^{\tensor k} \to C)$, and so
$\ev(\bdy y) = 0$, because $$C^{\tensor \ell_1} \tensor \ker(\DirectSum_k C^{\tensor k} \to C) \tensor C^{\tensor \ell_2} \subset \ker(\DirectSum_k C^{\tensor k} \to C).$$
Similarly, if $*$ is contained in the blob, then the blob label is a sum, with the $j$-th term have labelled points $d_{j,i}$ to the left of $*$, $m_j$ at $*$, and $d_{j,i}'$ to the right of $*$,
and there are labels $c_i$ at the labeled points outside the blob. We know that
$$\sum_j d_{j,1} \tensor \cdots \tensor d_{j,k_j} \tensor m_j \tensor d_{j,1}' \tensor \cdots \tensor d_{j,k'_j}' \in \ker(\DirectSum_{k,k'} C^{\tensor k} \tensor M \tensor C^{\tensor k'} \tensor \to M),$$
and so
\begin{align*}
\ev(\bdy y) & = \sum_j m_j d_{j,1}' \cdots d_{j,k'_j}' c_1 \cdots c_k d_{j,1} \cdots d_{j,k_j} \\
            & = \sum_j d_{j,1} \cdots d_{j,k_j} m_j d_{j,1}' \cdots d_{j,k'_j}' c_1 \cdots c_k \\
            & = 0
\end{align*}
where this time we use the fact that we're mapping to $\coinv{M}$, not just $M$.

The map $\pi \compose \ev: H_0(K_*(M)) \to \coinv{M}$ is clearly surjective ($\ev$ surjects onto $M$); we now show that it's injective. \todo{}
\end{proof}
\begin{proof}[Proof of Lemma \ref{lem:hochschild-free}]
We show that $K_*(C\otimes C)$ is
quasi-isomorphic to the 0-step complex $C$. We'll do this in steps, establishing quasi-isomorphisms and homotopy equivalences
$$K_*(C \tensor C) \quismto K'_* \htpyto K''_* \quismto C.$$

Let $K'_* \sub K_*(C\otimes C)$ be the subcomplex where the label of
the point $*$ is $1 \otimes 1 \in C\otimes C$.
We will show that the inclusion $i: K'_* \to K_*(C\otimes C)$ is a quasi-isomorphism.

Fix a small $\ep > 0$.
Let $N_\ep$ be the ball of radius $\ep$ around $* \in S^1$.
Let $K_*^\ep \sub K_*(C\otimes C)$ be the subcomplex
generated by blob diagrams $b$ such that $N_\ep$ is either disjoint from
or contained in each blob of $b$, and the only labeled point inside $N_\ep$ is $*$.
%and the two boundary points of $N_\ep$ are not labeled points of $b$.
For a field $y$ on $N_\ep$, let $s_\ep(y)$ be the equivalent picture with~$*$
labeled by $1\otimes 1$ and the only other labeled points at distance $\pm\ep/2$ from $*$.
(See Figure \ref{fig:sy}.) Note that $y - s_\ep(y) \in U(N_\ep)$. We can think of
$\sigma_\ep$ as a chain map $K_*^\ep \to K_*^\ep$ given by replacing the restriction $y$ to $N_\ep$ of each field
appearing in an element of  $K_*^\ep$ with $s_\ep(y)$.
Note that $\sigma_\ep(x) \in K'_*$.
\begin{figure}[!ht]
\begin{align*}
y & = \mathfig{0.2}{hochschild/y} &
s_\ep(y) & = \mathfig{0.2}{hochschild/sy}
\end{align*}
\caption{Defining $s_\ep$.}
\label{fig:sy}
\end{figure}

Define a degree 1 chain map $j_\ep : K_*^\ep \to K_*^\ep$ as follows.
Let $x \in K_*^\ep$ be a blob diagram.
If $*$ is not contained in any twig blob, $j_\ep(x)$ is obtained by adding $N_\ep$ to
$x$ as a new twig blob, with label $y - s_\ep(y)$, where $y$ is the restriction of $x$ to $N_\ep$.
If $*$ is contained in a twig blob $B$ with label $u = \sum z_i$, $j_\ep(x)$ is obtained as follows.
Let $y_i$ be the restriction of $z_i$ to $N_\ep$.
Let $x_i$ be equal to $x$ outside of $B$, equal to $z_i$ on $B \setmin N_\ep$,
and have an additional blob $N_\ep$ with label $y_i - s_\ep(y_i)$.
Define $j_\ep(x) = \sum x_i$.
\nn{need to check signs coming from blob complex differential}
Note that if $x \in K'_* \cap K_*^\ep$ then $j_\ep(x) \in K'_*$ also.

The key property of $j_\ep$ is
\eq{
    \bd j_\ep + j_\ep \bd = \id - \sigma_\ep.
}
If $j_\ep$ were defined on all of $K_*(C\otimes C)$, this would show that $\sigma_\ep$
is a homotopy inverse to the inclusion $K'_* \to K_*(C\otimes C)$.
One strategy would be to try to stitch together various $j_\ep$ for progressively smaller
$\ep$ and show that $K'_*$ is homotopy equivalent to $K_*(C\otimes C)$.
Instead, we'll be less ambitious and just show that
$K'_*$ is quasi-isomorphic to $K_*(C\otimes C)$.

If $x$ is a cycle in $K_*(C\otimes C)$, then for sufficiently small $\ep$ we have
$x \in K_*^\ep$.
(This is true for any chain in $K_*(C\otimes C)$, since chains are sums of
finitely many blob diagrams.)
Then $x$ is homologous to $s_\ep(x)$, which is in $K'_*$, so the inclusion map
$K'_* \sub K_*(C\otimes C)$ is surjective on homology.
If $y \in K_*(C\otimes C)$ and $\bd y = x \in K'_*$, then $y \in K_*^\ep$ for some $\ep$
and
\eq{
    \bd y = \bd (\sigma_\ep(y) + j_\ep(x)) .
}
Since $\sigma_\ep(y) + j_\ep(x) \in F'$, it follows that the inclusion map is injective on homology.
This completes the proof that $K'_*$ is quasi-isomorphic to $K_*(C\otimes C)$.

Let $K''_* \sub K'_*$ be the subcomplex of $K'_*$ where $*$ is not contained in any blob.
We will show that the inclusion $i: K''_* \to K'_*$ is a homotopy equivalence.

First, a lemma:  Let $G''_*$ and $G'_*$ be defined the same as $K''_*$ and $K'_*$, except with
$S^1$ replaced some (any) neighborhood of $* \in S^1$.
Then $G''_*$ and $G'_*$ are both contractible
and the inclusion $G''_* \sub G'_*$ is a homotopy equivalence.
For $G'_*$ the proof is the same as in (\ref{bcontract}), except that the splitting
$G'_0 \to H_0(G'_*)$ concentrates the point labels at two points to the right and left of $*$.
For $G''_*$ we note that any cycle is supported \nn{need to establish terminology for this; maybe
in ``basic properties" section above} away from $*$.
Thus any cycle lies in the image of the normal blob complex of a disjoint union
of two intervals, which is contractible by (\ref{bcontract}) and (\ref{disjunion}).
Actually, we need the further (easy) result that the inclusion
$G''_* \to G'_*$ induces an isomorphism on $H_0$.

Next we construct a degree 1 map (homotopy) $h: K'_* \to K'_*$ such that
for all $x \in K'_*$ we have
\eq{
    x - \bd h(x) - h(\bd x) \in K''_* .
}
Since $K'_0 = K''_0$, we can take $h_0 = 0$.
Let $x \in K'_1$, with single blob $B \sub S^1$.
If $* \notin B$, then $x \in K''_1$ and we define $h_1(x) = 0$.
If $* \in B$, then we work in the image of $G'_*$ and $G''_*$ (with respect to $B$).
Choose $x'' \in G''_1$ such that $\bd x'' = \bd x$.
Since $G'_*$ is contractible, there exists $y \in G'_2$ such that $\bd y = x - x''$.
Define $h_1(x) = y$.
The general case is similar, except that we have to take lower order homotopies into account.
Let $x \in K'_k$.
If $*$ is not contained in any of the blobs of $x$, then define $h_k(x) = 0$.
Otherwise, let $B$ be the outermost blob of $x$ containing $*$.
By xxxx above, $x = x' \bullet p$, where $x'$ is supported on $B$ and $p$ is supported away from $B$.
So $x' \in G'_l$ for some $l \le k$.
Choose $x'' \in G''_l$ such that $\bd x'' = \bd (x' - h_{l-1}\bd x')$.
Choose $y \in G'_{l+1}$ such that $\bd y = x' - x'' - h_{l-1}\bd x'$.
Define $h_k(x) = y \bullet p$.
This completes the proof that $i: K''_* \to K'_*$ is a homotopy equivalence.
\nn{need to say above more clearly and settle on notation/terminology}

Finally, we show that $K''_*$ is contractible.
\nn{need to also show that $H_0$ is the right thing; easy, but I won't do it now}
Let $x$ be a cycle in $K''_*$.
The union of the supports of the diagrams in $x$ does not contain $*$, so there exists a
ball $B \subset S^1$ containing the union of the supports and not containing $*$.
Adding $B$ as a blob to $x$ gives a contraction.
\nn{need to say something else in degree zero}
\end{proof}

We can also describe explicitly a map from the standard Hochschild
complex to the blob complex on the circle. \nn{What properties does this
map have?}

\begin{figure}%
$$\mathfig{0.6}{barycentric/barycentric}$$
\caption{The Hochschild chain $a \tensor b \tensor c$ is sent to
the sum of six blob $2$-chains, corresponding to a barycentric subdivision of a $2$-simplex.}
\label{fig:Hochschild-example}%
\end{figure}

As an example, Figure \ref{fig:Hochschild-example} shows the image of the Hochschild chain $a \tensor b \tensor c$. Only the $0$-cells are shown explicitly.
The edges marked $x, y$ and $z$ carry the $1$-chains
\begin{align*}
x & = \mathfig{0.1}{barycentric/ux} & u_x = \mathfig{0.1}{barycentric/ux_ca} - \mathfig{0.1}{barycentric/ux_c-a} \\
y & = \mathfig{0.1}{barycentric/uy} & u_y = \mathfig{0.1}{barycentric/uy_cab} - \mathfig{0.1}{barycentric/uy_ca-b} \\
z & = \mathfig{0.1}{barycentric/uz} & u_z = \mathfig{0.1}{barycentric/uz_c-a-b} - \mathfig{0.1}{barycentric/uz_cab}
\end{align*}
and the $2$-chain labelled $A$ is
\begin{equation*}
A = \mathfig{0.1}{barycentric/Ax}+\mathfig{0.1}{barycentric/Ay}.
\end{equation*}
Note that we then have
\begin{equation*}
\bdy A = x+y+z.
\end{equation*}

In general, the Hochschild chain $\Tensor_{i=1}^n a_i$ is sent to the sum of $n!$ blob $(n-1)$-chains, indexed by permutations,
$$\phi\left(\Tensor_{i=1}^n a_i\right) = \sum_{\pi} \phi^\pi(a_1, \ldots, a_n)$$
with ... (hmmm, problems making this precise; you need to decide where to put the labels, but then it's hard to make an honest chain map!)
