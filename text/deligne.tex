%!TEX root = ../blob1.tex

\section{Higher-dimensional Deligne conjecture}
\label{sec:deligne}
In this section we discuss Property \ref{property:deligne},
\begin{prop}[Higher dimensional Deligne conjecture]
The singular chains of the $n$-dimensional fat graph operad act on blob cochains.
\end{prop}

We will give a more precise statement of the proposition below.

\nn{for now, we just sketch the proof.}

\def\mapinf{\Maps_\infty}

The usual Deligne conjecture \nn{need refs} gives a map
\[
	C_*(LD_k)\otimes \overbrace{Hoch^*(C, C)\otimes\cdots\otimes Hoch^*(C, C)}^{\text{$k$ copies}}
			\to  Hoch^*(C, C) .
\]
Here $LD_k$ is the $k$-th space of the little disks operad, and $Hoch^*(C, C)$ denotes Hochschild
cochains.
The little disks operad is homotopy equivalent to the fat graph operad
\nn{need ref; and need to restrict which fat graphs}, and Hochschild cochains are homotopy equivalent to $A_\infty$ endomorphisms
of the blob complex of the interval.
\nn{need to make sure we prove this above}.
So the 1-dimensional Deligne conjecture can be restated as
\begin{eqnarray*}
	C_*(FG_k)\otimes \mapinf(\bc^C_*(I), \bc^C_*(I))\otimes\cdots
	\otimes \mapinf(\bc^C_*(I), \bc^C_*(I)) & \\
	  & \hspace{-5em} \to  \mapinf(\bc^C_*(I), \bc^C_*(I)) .
\end{eqnarray*}
See Figure \ref{delfig1}.
\begin{figure}[!ht]
$$\mathfig{.9}{tempkw/delfig1}$$
\caption{A fat graph}\label{delfig1}\end{figure}

We can think of a fat graph as encoding a sequence of surgeries, starting at the bottommost interval
of Figure \ref{delfig1} and ending at the topmost interval.
The surgeries correspond to the $k$ bigon-shaped ``holes" in the fat graph.
We remove the bottom interval of the bigon and replace it with the top interval.
To map this topological operation to an algebraic one, we need, for each hole, element of
$\mapinf(\bc^C_*(I_{\text{bottom}}), \bc^C_*(I_{\text{top}}))$.
So for each fixed fat graph we have a map
\[
	 \mapinf(\bc^C_*(I), \bc^C_*(I))\otimes\cdots
	\otimes \mapinf(\bc^C_*(I), \bc^C_*(I))  \to  \mapinf(\bc^C_*(I), \bc^C_*(I)) .
\]
If we deform the fat graph, corresponding to a 1-chain in $C_*(FG_k)$, we get a homotopy
between the maps associated to the endpoints of the 1-chain.
Similarly, higher-dimensional chains in $C_*(FG_k)$ give rise to higher homotopies.

It should now be clear how to generalize this to higher dimensions.
In the sequence-of-surgeries description above, we never used the fact that the manifolds
involved were 1-dimensional.
Thus we can define a $n$-dimensional fat graph to sequence of general surgeries
on an $n$-manifold.
More specifically,
the $n$-dimensional fat graph operad can be thought of as a sequence of general surgeries
$R_i \cup M_i \leadsto R_i \cup N_i$ together with mapping cylinders of diffeomorphisms
$f_i: R_i\cup N_i \to R_{i+1}\cup M_{i+1}$.
(See Figure \ref{delfig2}.)
\begin{figure}[!ht]
$$\mathfig{.9}{tempkw/delfig2}$$
\caption{A fat graph}\label{delfig2}\end{figure}
The components of the $n$-dimensional fat graph operad are indexed by tuples
$(\overline{M}, \overline{N}) = ((M_0,\ldots,M_k), (N_0,\ldots,N_k))$.
Note that the suboperad where $M_i$, $N_i$ and $R_i\cup M_i$ are all diffeomorphic to 
the $n$-ball is equivalent to the little $n{+}1$-disks operad.


If $M$ and $N$ are $n$-manifolds sharing the same boundary, we define
the blob cochains $\bc^*(A, B)$ (analogous to Hochschild cohomology) to be
$A_\infty$ maps from $\bc_*(M)$ to $\bc_*(N)$, where we think of both
collections of complexes as modules over the $A_\infty$ category associated to $\bd A = \bd B$.
The ``holes" in the above 
$n$-dimensional fat graph operad are labeled by $\bc^*(A_i, B_i)$.
\nn{need to make up my mind which notation I'm using for the module maps}

Putting this together we get a collection of maps
\begin{eqnarray*}
	C_*(FG^n_{\overline{M}, \overline{N}})\otimes \mapinf(\bc_*(M_0), \bc_*(N_0))\otimes\cdots\otimes 
\mapinf(\bc_*(M_{k-1}), \bc_*(N_{k-1})) & \\
	& \hspace{-11em}\to  \mapinf(\bc_*(M_k), \bc_*(N_k))
\end{eqnarray*}
which satisfy an operad type compatibility condition.

Note that if $k=0$ then this is just the action of chains of diffeomorphisms from Section \ref{sec:evaluation}.
And indeed, the proof is very similar \nn{...}



\medskip
\hrule\medskip

