%!TEX root = ../blob1.tex

\section{Higher-dimensional Deligne conjecture}
\label{sec:deligne}
In this section we 
sketch
\nn{revisit ``sketch" after proof is done} 
the proof of a higher dimensional version of the Deligne conjecture
about the action of the little disks operad on Hochschild cohomology.
The first several paragraphs lead up to a precise statement of the result
(Proposition \ref{prop:deligne} below).
Then we sketch the proof.

\nn{Does this generalisation encompass Kontsevich's proposed generalisation from \cite[\S2.5]{MR1718044}, 
that (I think...) the Hochschild homology of an $E_n$ algebra is an $E_{n+1}$ algebra? -S}

%from http://www.ams.org/mathscinet-getitem?mr=1805894
%Different versions of the geometric counterpart of Deligne's conjecture have been proven by Tamarkin [``Formality of chain operad of small squares'', preprint, http://arXiv.org/abs/math.QA/9809164], the reviewer [in Conf�rence Mosh� Flato 1999, Vol. II (Dijon), 307--331, Kluwer Acad. Publ., Dordrecht, 2000; MR1805923 (2002d:55009)], and J. E. McClure and J. H. Smith [``A solution of Deligne's conjecture'', preprint, http://arXiv.org/abs/math.QA/9910126] (see also a later simplified version [J. E. McClure and J. H. Smith, ``Multivariable cochain operations and little $n$-cubes'', preprint, http://arXiv.org/abs/math.QA/0106024]). The paper under review gives another proof of Deligne's conjecture, which, as the authors indicate, may be generalized to a proof of a higher-dimensional generalization of Deligne's conjecture, suggested in [M. Kontsevich, Lett. Math. Phys. 48 (1999), no. 1, 35--72; MR1718044 (2000j:53119)]. 


The usual Deligne conjecture (proved variously in \cite{MR1805894, MR2064592, hep-th/9403055, MR1805923} gives a map
\[
	C_*(LD_k)\otimes \overbrace{Hoch^*(C, C)\otimes\cdots\otimes Hoch^*(C, C)}^{\text{$k$ copies}}
			\to  Hoch^*(C, C) .
\]
Here $LD_k$ is the $k$-th space of the little disks operad, and $Hoch^*(C, C)$ denotes Hochschild
cochains.
The little disks operad is homotopy equivalent to the 
(transversely orient) fat graph operad
\nn{need ref, or say more precisely what we mean}, 
and Hochschild cochains are homotopy equivalent to $A_\infty$ endomorphisms
of the blob complex of the interval, thought of as a bimodule for itself.
\nn{need to make sure we prove this above}.
So the 1-dimensional Deligne conjecture can be restated as
\[
	C_*(FG_k)\otimes \hom(\bc^C_*(I), \bc^C_*(I))\otimes\cdots
	\otimes \hom(\bc^C_*(I), \bc^C_*(I))
	  \to  \hom(\bc^C_*(I), \bc^C_*(I)) .
\]
See Figure \ref{delfig1}.
\begin{figure}[t]
$$\mathfig{.9}{deligne/intervals}$$
\caption{A fat graph}\label{delfig1}\end{figure}
We emphasize that in $\hom(\bc^C_*(I), \bc^C_*(I))$ we are thinking of $\bc^C_*(I)$ as a module
for the $A_\infty$ 1-category associated to $\bd I$, and $\hom$ means the 
morphisms of such modules as defined in 
Subsection \ref{ss:module-morphisms}.

We can think of a fat graph as encoding a sequence of surgeries, starting at the bottommost interval
of Figure \ref{delfig1} and ending at the topmost interval.
The surgeries correspond to the $k$ bigon-shaped ``holes" in the fat graph.
We remove the bottom interval of the bigon and replace it with the top interval.
To convert this topological operation to an algebraic one, we need, for each hole, an element of
$\hom(\bc^C_*(I_{\text{bottom}}), \bc^C_*(I_{\text{top}}))$.
So for each fixed fat graph we have a map
\[
	 \hom(\bc^C_*(I), \bc^C_*(I))\otimes\cdots
	\otimes \hom(\bc^C_*(I), \bc^C_*(I))  \to  \hom(\bc^C_*(I), \bc^C_*(I)) .
\]
If we deform the fat graph, corresponding to a 1-chain in $C_*(FG_k)$, we get a homotopy
between the maps associated to the endpoints of the 1-chain.
Similarly, higher-dimensional chains in $C_*(FG_k)$ give rise to higher homotopies.

It should now be clear how to generalize this to higher dimensions.
In the sequence-of-surgeries description above, we never used the fact that the manifolds
involved were 1-dimensional.
Thus we can define an $n$-dimensional fat graph to be a sequence of general surgeries
on an $n$-manifold (Figure \ref{delfig2}).
\begin{figure}[t]
$$\mathfig{.9}{deligne/manifolds}$$
\caption{An $n$-dimensional fat graph}\label{delfig2}
\end{figure}

More specifically, an $n$-dimensional fat graph ($n$-FG for short) consists of:
\begin{itemize}
\item ``Upper" $n$-manifolds $M_0,\ldots,M_k$ and ``lower" $n$-manifolds $N_0,\ldots,N_k$,
with $\bd M_i = \bd N_i = E_i$ for all $i$.
We call $M_0$ and $N_0$ the outer boundary and the remaining $M_i$'s and $N_i$'s the inner
boundaries.
\item Additional manifolds $R_1,\ldots,R_{k}$, with $\bd R_i = E_0\cup \bd M_i = E_0\cup \bd N_i$.
%(By convention, $M_i = N_i = \emptyset$ if $i <1$ or $i>k$.)
\item Homeomorphisms 
\begin{eqnarray*}
	f_0: M_0 &\to& R_1\cup M_1 \\
	f_i: R_i\cup N_i &\to& R_{i+1}\cup M_{i+1}\;\; \mbox{for}\, 1\le i \le k-1 \\
	f_k: R_k\cup N_k &\to& N_0 .
\end{eqnarray*}
Each $f_i$ should be the identity restricted to $E_0$.
\end{itemize}
We can think of the above data as encoding the union of the mapping cylinders $C(f_0),\ldots,C(f_k)$,
with $C(f_i)$ glued to $C(f_{i+1})$ along $R_{i+1}$
(see Figure \ref{xdfig2}).
\begin{figure}[t]
$$\mathfig{.9}{tempkw/dfig2}$$
\caption{$n$-dimensional fat graph from mapping cylinders}\label{xdfig2}
\end{figure}
The $n$-manifolds are the ``$n$-dimensional graph" and the $I$ direction of the mapping cylinders is the ``fat" part.
We regard two such fat graphs as the same if there is a homeomorphism between them which is the 
identity on the boundary and which preserves the 1-dimensional fibers coming from the mapping
cylinders.
More specifically, we impose the following two equivalence relations:
\begin{itemize}
\item If $g: R_i\to R'_i$ is a homeomorphism, we can replace
\begin{eqnarray*}
	(\ldots, R_{i-1}, R_i, R_{i+1}, \ldots) &\to& (\ldots, R_{i-1}, R'_i, R_{i+1}, \ldots) \\
	(\ldots, f_{i-1}, f_i, \ldots) &\to& (\ldots, g\circ f_{i-1}, f_i\circ g^{-1}, \ldots),
\end{eqnarray*}
leaving the $M_i$ and $N_i$ fixed.
(Keep in mind the case $R'_i = R_i$.)
(See Figure \ref{xdfig3}.)
\begin{figure}[t]
$$\mathfig{.9}{tempkw/dfig3}$$
\caption{Conjugating by a homeomorphism}\label{xdfig3}
\end{figure}
\item If $M_i = M'_i \du M''_i$ and $N_i = N'_i \du N''_i$ (and there is a
compatible disjoint union of $\bd M = \bd N$), we can replace
\begin{eqnarray*}
	(\ldots, M_{i-1}, M_i, M_{i+1}, \ldots) &\to& (\ldots, M_{i-1}, M'_i, M''_i, M_{i+1}, \ldots) \\
	(\ldots, N_{i-1}, N_i, N_{i+1}, \ldots) &\to& (\ldots, N_{i-1}, N'_i, N''_i, N_{i+1}, \ldots) \\
	(\ldots, R_{i-1}, R_i, R_{i+1}, \ldots) &\to& 
						(\ldots, R_{i-1}, R_i\cup M''_i, R_i\cup N'_i, R_{i+1}, \ldots) \\
	(\ldots, f_{i-1}, f_i, \ldots) &\to& (\ldots, f_{i-1}, \rm{id}, f_i, \ldots) .
\end{eqnarray*}
(See Figure \ref{xdfig1}.)
\begin{figure}[t]
$$\mathfig{.9}{tempkw/dfig1}$$
\caption{Changing the order of a surgery}\label{xdfig1}
\end{figure}
\end{itemize}

Note that the second equivalence increases the number of holes (or arity) by 1.
We can make a similar identification with the roles of $M'_i$ and $M''_i$ reversed.
In terms of the ``sequence of surgeries" picture, this says that if two successive surgeries
do not overlap, we can perform them in reverse order or simultaneously.

There is an operad structure on $n$-dimensional fat graphs, given by gluing the outer boundary
of one graph into one of the inner boundaries of another graph.
We leave it to the reader to work out a more precise statement in terms of $M_i$'s, $f_i$'s etc.

For fixed $\ol{M} = (M_0,\ldots,M_k)$ and $\ol{N} = (N_0,\ldots,N_k)$, we let
$FG^n_{\ol{M}\ol{N}}$ denote the topological space of all $n$-dimensional fat graphs as above.
(Note that in different parts of $FG^n_{\ol{M}\ol{N}}$ the $M_i$'s and $N_i$'s
are ordered differently.)
The topology comes from the spaces
\[
	\Homeo(M_0\to R_1\cup M_1)\times \Homeo(R_1\cup N_1\to R_2\cup M_2)\times
			\cdots\times \Homeo(R_k\cup N_k\to N_0)
\]
and the above equivalence relations.
We will denote the typical element of $FG^n_{\ol{M}\ol{N}}$ by $\ol{f} = (f_0,\ldots,f_k)$.

\medskip

%The little $n{+}1$-ball operad injects into the $n$-FG operad.
The $n$-FG operad contains the little $n{+}1$-balls operad.
Roughly speaking, given a configuration of $k$ little $n{+}1$-balls in the standard
$n{+}1$-ball, we fiber the complement of the balls by vertical intervals
and let $M_i$ [$N_i$] be the southern [northern] hemisphere of the $i$-th ball.
More precisely, let $x_1,\ldots,x_{n+1}$ be the coordinates of $\r^{n+1}$.
Let $z$ be a point of the $k$-th space of the little $n{+}1$-ball operad, with
little balls $D_1,\ldots,D_k$ inside the standard $n{+}1$-ball.
We assume the $D_i$'s are ordered according to the $x_{n+1}$ coordinate of their centers.
Let $\pi:\r^{n+1}\to \r^n$ be the projection corresponding to $x_{n+1}$.
Let $B\sub\r^n$ be the standard $n$-ball.
Let $M_i$ and $N_i$ be $B$ for all $i$.
Identify $\pi(D_i)$ with $B$ (a.k.a.\ $M_i$ or $N_i$) via translations and dilations (no rotations).
Let $R_i = B\setmin \pi(D_i)$.
Let $f_i = \rm{id}$ for all $i$.
We have now defined a map from the little $n{+}1$-ball operad to the $n$-FG operad,
with contractible fibers.
(The fibers correspond to moving the $D_i$'s in the $x_{n+1}$ 
direction without changing their ordering.)
\nn{issue: we've described this by varying the $R_i$'s, but above we emphasize varying the $f_i$'s.
does this need more explanation?}

Another familiar subspace of the $n$-FG operad is $\Homeo(M\to N)$, which corresponds to 
case $k=0$ (no holes).

\medskip

Let $\ol{f} \in FG^n_{\ol{M}\ol{N}}$.
Let $\hom(\bc_*(M_i), \bc_*(N_i))$ denote the morphisms from $\bc_*(M_i)$ to $\bc_*(N_i)$,
as modules of the $A_\infty$ 1-category $\bc_*(E_i)$.
We define a map
\[
	p(\ol{f}): \hom(\bc_*(M_1), \bc_*(N_1))\ot\cdots\ot\hom(\bc_*(M_k), \bc_*(N_k))
				\to \hom(\bc_*(M_0), \bc_*(N_0)) .
\]
Given $\alpha_i\in\hom(\bc_*(M_i), \bc_*(N_i))$, we define $p(\ol{f}$) to be the composition
\[
	\bc_*(M_0)  \stackrel{f_0}{\to} \bc_*(R_1\cup M_1)
				 \stackrel{\id\ot\alpha_1}{\to} \bc_*(R_1\cup N_1)
				 \stackrel{f_1}{\to} \bc_*(R_2\cup M_2) \stackrel{\id\ot\alpha_2}{\to}
				 \cdots  \stackrel{\id\ot\alpha_k}{\to} \bc_*(R_k\cup N_k)
				 \stackrel{f_k}{\to} \bc_*(N_0)
\]
(Recall that the maps $\id\ot\alpha_i$ were defined in \nn{need ref}.)
\nn{issue: haven't we only defined $\id\ot\alpha_i$ when $\alpha_i$ is closed?}
It is easy to check that the above definition is compatible with the equivalence relations
and also the operad structure.
We can reinterpret the above as a chain map
\[
	p: C_0(FG^n_{\ol{M}\ol{N}})\ot \hom(\bc_*(M_1), \bc_*(N_1))\ot\cdots\ot\hom(\bc_*(M_k), \bc_*(N_k))
				\to \hom(\bc_*(M_0), \bc_*(N_0)) .
\]
The main result of this section is that this chain map extends to the full singular
chain complex $C_*(FG^n_{\ol{M}\ol{N}})$.

\begin{prop}
\label{prop:deligne}
There is a collection of chain maps
\[
	C_*(FG^n_{\overline{M}, \overline{N}})\otimes \hom(\bc_*(M_1), \bc_*(N_1))\otimes\cdots\otimes 
\hom(\bc_*(M_{k}), \bc_*(N_{k})) \to  \hom(\bc_*(M_0), \bc_*(N_0))
\]
which satisfy the operad compatibility conditions.
On $C_0(FG^n_{\ol{M}\ol{N}})$ this agrees with the chain map $p$ defined above.
When $k=0$, this coincides with the $C_*(\Homeo(M_0\to N_0))$ action of Section \ref{sec:evaluation}.
\end{prop}

If, in analogy to Hochschild cochains, we define elements of $\hom(M, N)$
to be ``blob cochains", we can summarize the above proposition by saying that the $n$-FG operad acts on
blob cochains.
As noted above, the $n$-FG operad contains the little $n{+}1$-ball operad, so this constitutes
a higher dimensional version of the Deligne conjecture for Hochschild cochains and the little 2-disk operad.

\begin{proof}
As described above, $FG^n_{\overline{M}, \overline{N}}$ is equal to the disjoint
union of products of homeomorphisms spaces, modulo some relations.
By \ref{CHprop}, 
\nn{...}
\end{proof}

\nn{maybe point out that even for $n=1$ there's something new here.}
