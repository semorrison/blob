%!TEX root = ../blob1.tex

\section{Higher-dimensional Deligne conjecture}
\label{sec:deligne}
In this section we 
sketch
\nn{revisit ``sketch" after proof is done} 
the proof of a higher dimensional version of the Deligne conjecture
about the action of the little disks operad on Hochschild cohomology.
The first several paragraphs lead up to a precise statement of the result
(Proposition \ref{prop:deligne} below).
Then we sketch the proof.

\nn{Does this generalisation encompass Kontsevich's proposed generalisation from \cite{MR1718044}, that (I think...) the Hochschild homology of an $E_n$ algebra is an $E_{n+1}$ algebra? -S}

%from http://www.ams.org/mathscinet-getitem?mr=1805894
%Different versions of the geometric counterpart of Deligne's conjecture have been proven by Tamarkin [``Formality of chain operad of small squares'', preprint, http://arXiv.org/abs/math.QA/9809164], the reviewer [in Conf�rence Mosh� Flato 1999, Vol. II (Dijon), 307--331, Kluwer Acad. Publ., Dordrecht, 2000; MR1805923 (2002d:55009)], and J. E. McClure and J. H. Smith [``A solution of Deligne's conjecture'', preprint, http://arXiv.org/abs/math.QA/9910126] (see also a later simplified version [J. E. McClure and J. H. Smith, ``Multivariable cochain operations and little $n$-cubes'', preprint, http://arXiv.org/abs/math.QA/0106024]). The paper under review gives another proof of Deligne's conjecture, which, as the authors indicate, may be generalized to a proof of a higher-dimensional generalization of Deligne's conjecture, suggested in [M. Kontsevich, Lett. Math. Phys. 48 (1999), no. 1, 35--72; MR1718044 (2000j:53119)]. 


The usual Deligne conjecture (proved variously in \cite{MR1805894, MR2064592, hep-th/9403055, MR1805923} gives a map
\[
	C_*(LD_k)\otimes \overbrace{Hoch^*(C, C)\otimes\cdots\otimes Hoch^*(C, C)}^{\text{$k$ copies}}
			\to  Hoch^*(C, C) .
\]
Here $LD_k$ is the $k$-th space of the little disks operad, and $Hoch^*(C, C)$ denotes Hochschild
cochains.
The little disks operad is homotopy equivalent to the 
(transversely orient) fat graph operad
\nn{need ref, or say more precisely what we mean}, 
and Hochschild cochains are homotopy equivalent to $A_\infty$ endomorphisms
of the blob complex of the interval, thought of as a bimodule for itself.
\nn{need to make sure we prove this above}.
So the 1-dimensional Deligne conjecture can be restated as
\[
	C_*(FG_k)\otimes \hom(\bc^C_*(I), \bc^C_*(I))\otimes\cdots
	\otimes \hom(\bc^C_*(I), \bc^C_*(I))
	  \to  \hom(\bc^C_*(I), \bc^C_*(I)) .
\]
See Figure \ref{delfig1}.
\begin{figure}[!ht]
$$\mathfig{.9}{deligne/intervals}$$
\caption{A fat graph}\label{delfig1}\end{figure}
We emphasize that in $\hom(\bc^C_*(I), \bc^C_*(I))$ we are thinking of $\bc^C_*(I)$ as a module
for the $A_\infty$ 1-category associated to $\bd I$, and $\hom$ means the 
morphisms of such modules as defined in 
Subsection \ref{ss:module-morphisms}.

We can think of a fat graph as encoding a sequence of surgeries, starting at the bottommost interval
of Figure \ref{delfig1} and ending at the topmost interval.
The surgeries correspond to the $k$ bigon-shaped ``holes" in the fat graph.
We remove the bottom interval of the bigon and replace it with the top interval.
To convert this topological operation to an algebraic one, we need, for each hole, an element of
$\hom(\bc^C_*(I_{\text{bottom}}), \bc^C_*(I_{\text{top}}))$.
So for each fixed fat graph we have a map
\[
	 \hom(\bc^C_*(I), \bc^C_*(I))\otimes\cdots
	\otimes \hom(\bc^C_*(I), \bc^C_*(I))  \to  \hom(\bc^C_*(I), \bc^C_*(I)) .
\]
If we deform the fat graph, corresponding to a 1-chain in $C_*(FG_k)$, we get a homotopy
between the maps associated to the endpoints of the 1-chain.
Similarly, higher-dimensional chains in $C_*(FG_k)$ give rise to higher homotopies.

It should now be clear how to generalize this to higher dimensions.
In the sequence-of-surgeries description above, we never used the fact that the manifolds
involved were 1-dimensional.
Thus we can define an $n$-dimensional fat graph to be a sequence of general surgeries
on an $n$-manifold (Figure \ref{delfig2}).
\begin{figure}[!ht]
$$\mathfig{.9}{deligne/manifolds}$$
\caption{An  $n$-dimensional fat graph}\label{delfig2}
\end{figure}

More specifically, an $n$-dimensional fat graph consists of:
\begin{itemize}
\item ``Incoming" $n$-manifolds $M_1,\ldots,M_k$ and ``outgoing" $n$-manifolds $N_1,\ldots,N_k$,
with $\bd M_i = \bd N_i$ for all $i$.
\item An ``outer boundary" $n{-}1$-manifold $E$.
\item Additional manifolds $R_0,\ldots,R_{k+1}$, with $\bd R_i = E\cup \bd M_i = E\cup \bd N_i$.
(By convention, $M_i = N_i = \emptyset$ if $i <1$ or $i>k$.)
We call $R_0$ the outer incoming manifold and $R_{k+1}$ the outer outgoing manifold
\item Homeomorphisms $f_i : R_i\cup N_i\to R_{i+1}\cup M_{i+1}$, $0\le i \le k$.
\end{itemize}
We can think of the above data as encoding the union of the mapping cylinders $C(f_0),\ldots,C(f_k)$,
with $C(f_i)$ glued to $C(f_{i+1})$ along $R_{i+1}$.
\nn{need figure}



\nn{*** resume revising here}



The components of the $n$-dimensional fat graph operad are indexed by tuples
$(\overline{M}, \overline{N}) = ((M_0,\ldots,M_k), (N_0,\ldots,N_k))$.
\nn{not quite true: this is coarser than components}
Note that the suboperad where $M_i$, $N_i$ and $R_i\cup M_i$ are all homeomorphic to 
the $n$-ball is equivalent to the little $n{+}1$-disks operad.
\nn{what about rotating in the horizontal directions?}


If $M$ and $N$ are $n$-manifolds sharing the same boundary, we define
the blob cochains $\bc^*(A, B)$ (analogous to Hochschild cohomology) to be
$A_\infty$ maps from $\bc_*(M)$ to $\bc_*(N)$, where we think of both
collections of complexes as modules over the $A_\infty$ category associated to $\bd A = \bd B$.
The ``holes" in the above 
$n$-dimensional fat graph operad are labeled by $\bc^*(A_i, B_i)$.
\nn{need to make up my mind which notation I'm using for the module maps}

Putting this together we get 
\begin{prop}(Precise statement of Property \ref{property:deligne})
\label{prop:deligne}
There is a collection of maps
\begin{eqnarray*}
	C_*(FG^n_{\overline{M}, \overline{N}})\otimes \hom(\bc_*(M_1), \bc_*(N_1))\otimes\cdots\otimes 
\hom(\bc_*(M_{k}), \bc_*(N_{k})) & \\
	& \hspace{-11em}\to  \hom(\bc_*(M_0), \bc_*(N_0))
\end{eqnarray*}
which satisfy an operad type compatibility condition. \nn{spell this out}
\end{prop}

Note that if $k=0$ then this is just the action of chains of diffeomorphisms from Section \ref{sec:evaluation}.
And indeed, the proof is very similar \nn{...}



\medskip
\hrule\medskip

