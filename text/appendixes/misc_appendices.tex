%!TEX root = ../blob1.tex

\section{Comparing definitions of $A_\infty$ algebras}
\label{sec:comparing-A-infty}
In this section, we make contact between the usual definition of an $A_\infty$ algebra and our definition of a topological $A_\infty$ algebra, from Definition \ref{defn:topological-Ainfty-category}.

We begin be restricting the data of a topological $A_\infty$ algebra to the standard interval $[0,1]$, which we can alternatively characterise as:
\begin{defn}
A \emph{topological $A_\infty$ category on $[0,1]$} $\cC$ has a set of objects $\Obj(\cC)$, and for each $a,b \in \Obj(\cC)$, a chain complex $\cC_{a,b}$, along with
\begin{itemize}
\item an action of the operad of $\Obj(\cC)$-labeled cell decompositions
\item and a compatible action of $\CD{[0,1]}$.
\end{itemize}
\end{defn}
Here the operad of cell decompositions of $[0,1]$ has operations indexed by a finite set of points $0 < x_1< \cdots < x_k < 1$, cutting $[0,1]$ into subintervals. An $X$-labeled cell decomposition labels $\{0, x_1, \ldots, x_k, 1\}$ by $X$. Given two cell decompositions $\cJ^{(1)}$ and $\cJ^{(2)}$, and an index $m$, we can compose them to form a new cell decomposition $\cJ^{(1)} \circ_m \cJ^{(2)}$ by inserting the points of $\cJ^{(2)}$ linearly into the $m$-th interval of $\cJ^{(1)}$. In the $X$-labeled case, we insist that the appropriate labels match up. Saying we have an action of this operad means that for each labeled cell decomposition $0 < x_1< \cdots < x_k < 1$, $a_0, \ldots, a_{k+1} \subset \Obj(\cC)$, there is a chain map $$\cC_{a_0,a_1} \tensor \cdots \tensor \cC_{a_k,a_{k+1}} \to \cC(a_0,a_{k+1})$$ and these chain maps compose exactly as the cell decompositions.
An action of $\CD{[0,1]}$ is compatible with an action of the cell decomposition operad if given a decomposition $\pi$, and a family of diffeomorphisms $f \in \CD{[0,1]}$ which is supported on the subintervals determined by $\pi$, then the two possible operations (glue intervals together, then apply the diffeomorphisms, or apply the diffeormorphisms separately to the subintervals, then glue) commute (as usual, up to a weakly unique homotopy).

Translating between these definitions is straightforward. To restrict to the standard interval, define $\cC_{a,b} = \cC([0,1];a,b)$. Given a cell decomposition $0 < x_1< \cdots < x_k < 1$, we use the map (suppressing labels)
$$\cC([0,1])^{\tensor k+1} \to \cC([0,x_1]) \tensor \cdots \tensor \cC[x_k,1] \to \cC([0,1])$$
where the factors of the first map are induced by the linear isometries $[0,1] \to [x_i, x_{i+1}]$, and the second map is just gluing. The action of $\CD{[0,1]}$ carries across, and is automatically compatible. Going the other way, we just declare $\cC(J;a,b) = \cC_{a,b}$, pick a diffeomorphism $\phi_J : J \isoto [0,1]$ for every interval $J$, define the gluing map $\cC(J_1) \tensor \cC(J_2) \to \cC(J_1 \cup J_2)$ by the first applying the cell decomposition map for $0 < \frac{1}{2} < 1$, then the self-diffeomorphism of $[0,1]$ given by $\frac{1}{2} (\phi_{J_1} \cup (1+ \phi_{J_2})) \circ \phi_{J_1 \cup J_2}^{-1}$. You can readily check that this gluing map is associative on the nose. \todo{really?}

%First recall the \emph{coloured little intervals operad}. Given a set of labels $\cL$, the operations are indexed by \emph{decompositions of the interval}, each of which is a collection of disjoint subintervals $\{(a_i,b_i)\}_{i=1}^k$ of $[0,1]$, along with a labeling of the complementary regions by $\cL$, $\{l_0, \ldots, l_k\}$.  Given two decompositions $\cJ^{(1)}$ and $\cJ^{(2)}$, and an index $m$ such that $l^{(1)}_{m-1} = l^{(2)}_0$ and $l^{(1)}_{m} = l^{(2)}_{k^{(2)}}$, we can form a new decomposition by inserting the intervals of $\cJ^{(2)}$ linearly inside the $m$-th interval of $\cJ^{(1)}$. We call the resulting decomposition $\cJ^{(1)} \circ_m \cJ^{(2)}$.

%\begin{defn}
%A \emph{topological $A_\infty$ category} $\cC$ has a set of objects $\Obj(\cC)$ and for each $a,b \in \Obj(\cC)$ a chain complex $\cC_{a,b}$, along with a compatible `composition map' and an `action of families of diffeomorphisms'.

%A \emph{composition map} $f$ is a family of chain maps, one for each decomposition of the interval, $f_\cJ : A^{\tensor k} \to A$, making $\cC$ into a category over the coloured little intervals operad, with labels $\cL = \Obj(\cC)$. Thus the chain maps satisfy the identity 
%\begin{equation*}
%f_{\cJ^{(1)} \circ_m \cJ^{(2)}} = f_{\cJ^{(1)}} \circ (\id^{\tensor m-1} \tensor f_{\cJ^{(2)}} \tensor \id^{\tensor k^{(1)} - m}).
%\end{equation*}

%An \emph{action of families of diffeomorphisms} is a chain map $ev: \CD{[0,1]} \tensor A \to A$, such that 
%\begin{enumerate}
%\item The diagram 
%\begin{equation*}
%\xymatrix{
%\CD{[0,1]} \tensor \CD{[0,1]} \tensor A \ar[r]^{\id \tensor ev} \ar[d]^{\circ \tensor \id} & \CD{[0,1]} \tensor A \ar[d]^{ev} \\
%\CD{[0,1]} \tensor A \ar[r]^{ev} & A
%}
%\end{equation*}
%commutes up to weakly unique homotopy.
%\item If $\phi \in \Diff([0,1])$ and $\cJ$ is a decomposition of the interval, we obtain a new decomposition $\phi(\cJ)$ and a collection $\phi_m \in \Diff([0,1])$ of diffeomorphisms obtained by taking the restrictions $\restrict{\phi}{[a_m,b_m]} : [a_m,b_m] \to [\phi(a_m),\phi(b_m)]$ and pre- and post-composing these with the linear diffeomorphisms $[0,1] \to [a_m,b_m]$ and $[\phi(a_m),\phi(b_m)] \to [0,1]$. We require that
%\begin{equation*}
%\phi(f_\cJ(a_1, \cdots, a_k)) = f_{\phi(\cJ)}(\phi_1(a_1), \cdots, \phi_k(a_k)).
%\end{equation*}
%\end{enumerate}
%\end{defn}

From a topological $A_\infty$ category on $[0,1]$ $\cC$ we can produce a `conventional' $A_\infty$ category $(A, \{m_k\})$ as defined in, for example, \cite{MR1854636}. We'll just describe the algebra case (that is, a category with only one object), as the modifications required to deal with multiple objects are trivial. Define $A = \cC$ as a chain complex (so $m_1 = d$). Define $m_2 : A\tensor A \to A$ by $f_{\{(0,\frac{1}{2}),(\frac{1}{2},1)\}}$. To define $m_3$, we begin by taking the one parameter family $\phi_3$ of diffeomorphisms of $[0,1]$ that interpolates linearly between the identity and the piecewise linear diffeomorphism taking $\frac{1}{4}$ to $\frac{1}{2}$ and $\frac{1}{2}$ to $\frac{3}{4}$, and then define
\begin{equation*}
m_3(a,b,c) = ev(\phi_3, m_2(m_2(a,b), c)).
\end{equation*}

It's then easy to calculate that
\begin{align*}
d(m_3(a,b,c)) & = ev(d \phi_3, m_2(m_2(a,b),c)) - ev(\phi_3 d m_2(m_2(a,b), c)) \\
 & = ev( \phi_3(1), m_2(m_2(a,b),c)) - ev(\phi_3(0), m_2 (m_2(a,b),c)) - \\ & \qquad - ev(\phi_3, m_2(m_2(da, b), c) + (-1)^{\deg a} m_2(m_2(a, db), c) + \\ & \qquad \quad + (-1)^{\deg a+\deg b} m_2(m_2(a, b), dc) \\
 & = m_2(a , m_2(b,c)) - m_2(m_2(a,b),c) - \\ & \qquad - m_3(da,b,c) + (-1)^{\deg a + 1} m_3(a,db,c) + \\ & \qquad \quad + (-1)^{\deg a + \deg b + 1} m_3(a,b,dc), \\
\intertext{and thus that}
m_1 \circ m_3 & =  m_2 \circ (\id \tensor m_2) - m_2 \circ (m_2 \tensor \id) - \\ & \qquad - m_3 \circ (m_1 \tensor \id \tensor \id) - m_3 \circ (\id \tensor m_1 \tensor \id) - m_3 \circ (\id \tensor \id \tensor m_1)
\end{align*}
as required (c.f. \cite[p. 6]{MR1854636}).
\todo{then the general case.}
We won't describe a reverse construction (producing a topological $A_\infty$ category from a `conventional' $A_\infty$ category), but we presume that this will be easy for the experts.

\section{Morphisms and duals of topological $A_\infty$ modules}
\label{sec:A-infty-hom-and-duals}%

\begin{defn}
If $\cM$ and $\cN$ are topological $A_\infty$ left modules over a topological $A_\infty$ category $\cC$, then a morphism $f: \cM \to \cN$ consists of a chain map $f:\cM(J,p;b) \to \cN(J,p;b)$ for each right marked interval $(J,p)$ with a boundary condition $b$, such that  for each interval $J'$ the diagram
\begin{equation*}
\xymatrix{
\cC(J';a,b) \tensor \cM(J,p;b) \ar[r]^{\text{gl}} \ar[d]^{\id \tensor f} & \cM(J' cup J,a) \ar[d]^f \\
\cC(J';a,b) \tensor \cN(J,p;b) \ar[r]^{\text{gl}}                                & \cN(J' cup J,a) 
}
\end{equation*}
commutes on the nose, and the diagram
\begin{equation*}
\xymatrix{
\CD{(J,p) \to (J',p')} \tensor \cM(J,p;a) \ar[r]^{\text{ev}} \ar[d]^{\id \tensor f} & \cM(J',p';a) \ar[d]^f \\
\CD{(J,p) \to (J',p')} \tensor \cN(J,p;a) \ar[r]^{\text{ev}}  & \cN(J',p';a) \\
}
\end{equation*}
commutes up to a weakly unique homotopy.
\end{defn}

The variations required for right modules and bimodules should be obvious.

\todo{duals}
\todo{ the functors $\hom_{\lmod{\cC}}\left(\cM \to -\right)$ and $\cM^* \tensor_{\cC} -$ from $\lmod{\cC}$ to $\Vect$ are naturally isomorphic}

