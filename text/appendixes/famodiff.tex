%!TEX root = ../../blob1.tex

\section{Adapting families of maps to open covers}  \label{sec:localising}


Let $X$ and $T$ be topological spaces, with $X$ compact.
Let $\cU = \{U_\alpha\}$ be an open cover of $X$ which affords a partition of
unity $\{r_\alpha\}$.
(That is, $r_\alpha : X \to [0,1]$; $r_\alpha(x) = 0$ if $x\notin U_\alpha$;
for fixed $x$, $r_\alpha(x) \ne 0$ for only finitely many $\alpha$; and $\sum_\alpha r_\alpha = 1$.)
Since $X$ is compact, we will further assume that $r_\alpha \ne 0$ (globally) 
for only finitely
many $\alpha$.

Let
\[
	CM_*(X, T) \deq C_*(\Maps(X\to T)) ,
\]
the singular chains on the space of continuous maps from $X$ to $T$.
$CM_k(X, T)$ is generated by continuous maps
\[
	f: P\times X \to T ,
\]
where $P$ is some convex linear polyhedron in $\r^k$.
Recall that $f$ is {\it supported} on $S\sub X$ if $f(p, x)$ does not depend on $p$ when
$x \notin S$, and that $f$ is {\it adapted} to $\cU$ if 
$f$ is supported on the union of at most $k$ of the $U_\alpha$'s.
A chain $c \in CM_*(X, T)$ is adapted to $\cU$ if it is a linear combination of 
generators which are adapted.

\begin{lemma} \label{basic_adaptation_lemma}
The $f: P\times X \to T$, as above.
The there exists
\[
	F: I \times P\times X \to T
\]
such that
\begin{enumerate}
\item $F(0, \cdot, \cdot) = f$ .
\item We can decompose $P = \cup_i D_i$ so that
the restrictions $F(1, \cdot, \cdot) : D_i\times X\to T$ are all adapted to $\cU$.
\item If $f$ restricted to $Q\sub P$ has support $S\sub X$, then the restriction
$F: (I\times Q)\times X\to T$ also has support $S$.
\item If for all $p\in P$ we have $f(p, \cdot):X\to T$ is a 
[submersion, diffeomorphism, PL homeomorphism, bi-Lipschitz homeomorphism]
then the same is true of $F(t, p, \cdot)$ for all $t\in I$ and $p\in P$.
(Of course we must assume that $X$ and $T$ are the appropriate 
sort of manifolds for this to make sense.)
\end{enumerate}
\end{lemma}



\begin{proof}
Our homotopy will have the form
\eqar{
    F: I \times P \times X &\to& X \\
    (t, p, x) &\mapsto& f(u(t, p, x), x)
}
for some function
\eq{
    u : I \times P \times X \to P .
}

First we describe $u$, then we argue that it makes the conclusions of the lemma true.

For each cover index $\alpha$ choose a cell decomposition $K_\alpha$ of $P$
such that the various $K_\alpha$ are in general position with respect to each other.
If we are in one of the cases of item 4 of the lemma, also choose $K_\alpha$
sufficiently fine as described below.

\def\jj{\tilde{L}}
Let $L$ be a common refinement all the $K_\alpha$'s.
Let $\jj$ denote the handle decomposition of $P$ corresponding to $L$.
Each $i$-handle $C$ of $\jj$ has an $i$-dimensional tangential coordinate and,
more importantly for our purposes, a $k{-}i$-dimensional normal coordinate.
We will typically use the same notation for $i$-cells of $L$ and the 
corresponding $i$-handles of $\jj$.

For each (top-dimensional) $k$-cell $C$ of each $K_\alpha$, choose a point $p(C) \in C \sub P$.
Let $D$ be a $k$-handle of $\jj$.
For each $\alpha$ let $C(D, \alpha)$ be the $k$-cell of $K_\alpha$ which contains $D$
and let $p(D, \alpha) = p(C(D, \alpha))$.

For $p \in D$ we define
\eq{
    u(t, p, x) = (1-t)p + t \sum_\alpha r_\alpha(x) p(D, \alpha) .
}
(Recall that $P$ is a convex linear polyhedron, so the weighted average of points of $P$
makes sense.)

Thus far we have defined $u(t, p, x)$ when $p$ lies in a $k$-handle of $\jj$.
We will now extend $u$ inductively to handles of index less than $k$.

Let $E$ be a $k{-}1$-handle.
$E$ is homeomorphic to $B^{k-1}\times [0,1]$, and meets
the $k$-handles at $B^{k-1}\times\{0\}$ and $B^{k-1}\times\{1\}$.
Let $\eta : E \to [0,1]$, $\eta(x, s) = s$ be the normal coordinate
of $E$.
Let $D_0$ and $D_1$ be the two $k$-handles of $\jj$ adjacent to $E$.
There is at most one index $\beta$ such that $C(D_0, \beta) \ne C(D_1, \beta)$.
(If there is no such index $\beta$, choose $\beta$
arbitrarily.)
For $p \in E$, define
\eq{
    u(t, p, x) = (1-t)p + t \left( \sum_{\alpha \ne \beta} r_\alpha(x) p(D_0, \alpha)
            + r_\beta(x) (\eta(p) p(D_0, p) + (1-\eta(p)) p(D_1, p)) \right) .
}

\nn{*** resume revising here ***}

In general, for $E$ a $k{-}j$-handle, there is a normal coordinate
$\eta: E \to R$, where $R$ is some $j$-dimensional polyhedron.
The vertices of $R$ are associated to $k$-cells of the $K_\alpha$, and thence to points of $P$.
If we triangulate $R$ (without introducing new vertices), we can linearly extend
a map from the vertices of $R$ into $P$ to a map of all of $R$ into $P$.
Let $\cN$ be the set of all $\beta$ for which $K_\beta$ has a $k$-cell whose boundary meets
the $k{-}j$-cell corresponding to $E$.
For each $\beta \in \cN$, let $\{q_{\beta i}\}$ be the set of points in $P$ associated to the aforementioned $k$-cells.
Now define, for $p \in E$,
\begin{equation}
\label{eq:u}
    u(t, p, x) = (1-t)p + t \left(
            \sum_{\alpha \notin \cN} r_\alpha(x) p_{c_\alpha}
                + \sum_{\beta \in \cN} r_\beta(x) \left( \sum_i \eta_{\beta i}(p) \cdot q_{\beta i} \right)
             \right) .
\end{equation}
Here $\eta_{\beta i}(p)$ is the weight given to $q_{\beta i}$ by the linear extension
mentioned above.

This completes the definition of $u: I \times P \times X \to P$.

\medskip

Next we verify that $u$ has the desired properties.

Since $u(0, p, x) = p$ for all $p\in P$ and $x\in X$, $F(0, p, x) = f(p, x)$ for all $p$ and $x$.
Therefore $F$ is a homotopy from $f$ to something.

Next we show that if the $K_\alpha$'s are sufficiently fine cell decompositions,
then $F$ is a homotopy through diffeomorphisms.
We must show that the derivative $\pd{F}{x}(t, p, x)$ is non-singular for all $(t, p, x)$.
We have
\eq{
%   \pd{F}{x}(t, p, x) = \pd{f}{x}(u(t, p, x), x) + \pd{f}{p}(u(t, p, x), x) \pd{u}{x}(t, p, x) .
    \pd{F}{x} = \pd{f}{x} + \pd{f}{p} \pd{u}{x} .
}
Since $f$ is a family of diffeomorphisms, $\pd{f}{x}$ is non-singular and
\nn{bounded away from zero, or something like that}.
(Recall that $X$ and $P$ are compact.)
Also, $\pd{f}{p}$ is bounded.
So if we can insure that $\pd{u}{x}$ is sufficiently small, we are done.
It follows from Equation \eqref{eq:u} above that $\pd{u}{x}$ depends on $\pd{r_\alpha}{x}$
(which is bounded)
and the differences amongst the various $p_{c_\alpha}$'s and $q_{\beta i}$'s.
These differences are small if the cell decompositions $K_\alpha$ are sufficiently fine.
This completes the proof that $F$ is a homotopy through diffeomorphisms.

\medskip

Next we show that for each handle $D \sub P$, $F(1, \cdot, \cdot) : D\times X \to X$
is a singular cell adapted to $\cU$.
This will complete the proof of the lemma.
\nn{except for boundary issues and the `$P$ is a cell' assumption}

Let $j$ be the codimension of $D$.
(Or rather, the codimension of its corresponding cell.  From now on we will not make a distinction
between handle and corresponding cell.)
Then $j = j_1 + \cdots + j_m$, $0 \le m \le k$,
where the $j_i$'s are the codimensions of the $K_\alpha$
cells of codimension greater than 0 which intersect to form $D$.
We will show that
if the relevant $U_\alpha$'s are disjoint, then
$F(1, \cdot, \cdot) : D\times X \to X$
is a product of singular cells of dimensions $j_1, \ldots, j_m$.
If some of the relevant $U_\alpha$'s intersect, then we will get a product of singular
cells whose dimensions correspond to a partition of the $j_i$'s.
We will consider some simple special cases first, then do the general case.

First consider the case $j=0$ (and $m=0$).
A quick look at Equation xxxx above shows that $u(1, p, x)$, and hence $F(1, p, x)$,
is independent of $p \in P$.
So the corresponding map $D \to \Diff(X)$ is constant.

Next consider the case $j = 1$ (and $m=1$, $j_1=1$).
Now Equation yyyy applies.
We can write $D = D'\times I$, where the normal coordinate $\eta$ is constant on $D'$.
It follows that the singular cell $D \to \Diff(X)$ can be written as a product
of a constant map $D' \to \Diff(X)$ and a singular 1-cell $I \to \Diff(X)$.
The singular 1-cell is supported on $U_\beta$, since $r_\beta = 0$ outside of this set.

Next case: $j=2$, $m=1$, $j_1 = 2$.
This is similar to the previous case, except that the normal bundle is 2-dimensional instead of
1-dimensional.
We have that $D \to \Diff(X)$ is a product of a constant singular $k{-}2$-cell
and a 2-cell with support $U_\beta$.

Next case: $j=2$, $m=2$, $j_1 = j_2 = 1$.
In this case the codimension 2 cell $D$ is the intersection of two
codimension 1 cells, from $K_\beta$ and $K_\gamma$.
We can write $D = D' \times I \times I$, where the normal coordinates are constant
on $D'$, and the two $I$ factors correspond to $\beta$ and $\gamma$.
If $U_\beta$ and $U_\gamma$ are disjoint, then we can factor $D$ into a constant $k{-}2$-cell and
two 1-cells, supported on $U_\beta$ and $U_\gamma$ respectively.
If $U_\beta$ and $U_\gamma$ intersect, then we can factor $D$ into a constant $k{-}2$-cell and
a 2-cell supported on $U_\beta \cup U_\gamma$.
\nn{need to check that this is true}

\nn{finally, general case...}

\nn{this completes proof}

\end{proof}




\noop{

\nn{move this to later:}

\begin{lemma}  \label{extension_lemma_b}
Let $x \in CM_k(X, T)$ be a singular chain such that $\bd x$ is adapted to $\cU$.
Then $x$ is homotopic (rel boundary) to some $x' \in CM_k(S, T)$ which is adapted to $\cU$.
Furthermore, one can choose the homotopy so that its support is equal to the support of $x$.
If $S$ and $T$ are manifolds, the statement remains true if we replace $CM_*(S, T)$ with
chains of smooth maps or immersions.
\end{lemma}

\medskip
\hrule
\medskip


In this appendix we provide the proof of
\nn{should change this to the more general \ref{extension_lemma_b}}

\begin{lem*}[Restatement of Lemma \ref{extension_lemma}]
Let $x \in CD_k(X)$ be a singular chain such that $\bd x$ is adapted to $\cU$.
Then $x$ is homotopic (rel boundary) to some $x' \in CD_k(X)$ which is adapted to $\cU$.
Furthermore, one can choose the homotopy so that its support is equal to the support of $x$.
\end{lem*}

\nn{for pedagogical reasons, should do $k=1,2$ cases first; probably do this in
later draft}

\nn{not sure what the best way to deal with boundary is; for now just give main argument, worry
about boundary later}

}




\medskip
\hrule
\medskip
\nn{the following was removed from earlier section; it should be reincorporated somewhere
in this section}

Let $\cU = \{U_\alpha\}$ be an open cover of $X$.
A $k$-parameter family of homeomorphisms $f: P \times X \to X$ is
{\it adapted to $\cU$} if there is a factorization
\eq{
    P = P_1 \times \cdots \times P_m
}
(for some $m \le k$)
and families of homeomorphisms
\eq{
    f_i :  P_i \times X \to X
}
such that
\begin{itemize}
\item each $f_i$ is supported on some connected $V_i \sub X$;
\item the sets $V_i$ are mutually disjoint;
\item each $V_i$ is the union of at most $k_i$ of the $U_\alpha$'s,
where $k_i = \dim(P_i)$; and
\item $f(p, \cdot) = g \circ f_1(p_1, \cdot) \circ \cdots \circ f_m(p_m, \cdot)$
for all $p = (p_1, \ldots, p_m)$, for some fixed $g \in \Homeo(X)$.
\end{itemize}
A chain $x \in CH_k(X)$ is (by definition) adapted to $\cU$ if it is the sum
of singular cells, each of which is adapted to $\cU$.
\medskip
\hrule
\medskip





%!TEX root = ../../blob1.tex

Here's an alternative proof of the special case in which $P$, the parameter space for the family of diffeomorphisms, is a cube. It is much more explicit, for better or worse.

\begin{proof}[Alternative, more explicit proof of Lemma \ref{extension_lemma}]


Fix a finite open cover of $X$, say $(U_l)_{l=1}^L$, along with an
associated partition of unity $(r_l)$.

We'll define the homotopy $H:I \times P \times X \to X$ via a function
$u:I \times P \times X \to P$, with
\begin{equation*}
H(t,p,x) = F(u(t,p,x),x).
\end{equation*}

To begin, we'll define a function $u'' : I \times P \times X \to P$, and
a corresponding homotopy $H''$. This homotopy will just be a homotopy of
$F$ through families of maps, not through families of diffeomorphisms. On
the other hand, it will be quite simple to describe, and we'll later
explain how to build the desired function $u$ out of it.

For each $l = 1, \ldots, L$, pick some $C^\infty$ function $f_l : I \to
I$ which is identically $0$ on a neighborhood of the closed interval $[0,\frac{l-1}{L}]$
and identically $1$ on a neighborhood of the closed interval $[\frac{l}{L},1]$. (Monotonic?
Fix a bound for the derivative?) We'll extend it to a function on
$k$-tuples $f_l : I^k \to I^k$ pointwise.

Define $$u''(t,p,x) = \sum_{l=1}^L r_l(x) u_l(t,p),$$ with
$$u_l(t,p) = t f_l(p) + (1-t)p.$$ Notice that the $i$-th component of $u''(t,p,x)$ depends only on the $i$-th component of $p$.

Let's now establish some properties of $u''$ and $H''$. First,
\begin{align*}
H''(0,p,x) & = F(u''(0,p,x),x) \\
           & = F(\sum_{l=1}^L r_l(x) p, x) \\
           & = F(p,x).
\end{align*}
Next, calculate the derivatives
\begin{align*}
\partial_{p_i} H''(1,p,x) & = \partial_{p_i}u''(1,p,x) \partial_1 F(u(1,p,x),x) \\
\intertext{and}
\partial_{p_i}u''(1,p,x) & = \sum_{l=1}^L r_l(x) \partial_{p_i} f_l(p).
\end{align*}
Now $\partial_{p_i} f_l(p) = 0$ unless $\frac{l-1}{L} < p_i < \frac{l}{L}$, and $r_l(x) = 0$ unless $x \in U_l$,
so we conclude that for a fixed $p$, $\partial_p H''(1,p,x) = 0$ for all $x$ outside the union of $k$ open sets from the open cover, namely
$\bigcup_{i=1}^k U_{l_i}$ where for each $i$, we choose $l_i$ so $\frac{l_i -1}{L} \leq p_i \leq \frac{l_i}{L}$. It may be helpful to refer to Figure \ref{fig:supports}.

\begin{figure}[!ht]
\begin{equation*}
\mathfig{0.5}{explicit/supports}
\end{equation*}
\caption{The supports of the derivatives {\color{green}$\partial_p f_1$}, {\color{blue}$\partial_p f_2$} and {\color{red}$\partial_p f_3$}, illustrating the case $k=2$, $L=3$. Notice that any
point $p$ lies in the intersection of at most $k$ supports. The support of $\partial_p u''(1,p,x)$ is contained in the union of these supports.}
\label{fig:supports}
\end{figure}

Unfortunately, $H''$ does not have the desired property that it's a homotopy through diffeomorphisms. To achieve this, we'll paste together several copies
of the map $u''$. First, glue together $2^k$ copies, defining $u':I \times P \times X$ by
\begin{align*}
u'(t,p,x)_i & =
\begin{cases}
\frac{1}{2} u''(t, 2p_i, x)_i & \text{if $0 \leq p_i \leq \frac{1}{2}$} \\
1-\frac{1}{2} u''(t, 2-2p_i, x)_i & \text{if $\frac{1}{2} \leq p_i \leq 1$}.
\end{cases}
\end{align*}
(Note that we're abusing notation somewhat, using the fact that $u''(t,p,x)_i$ depends on $p$ only through $p_i$.)
To see what's going on here, it may be helpful to look at Figure \ref{fig:supports_4}, which shows the support of $\partial_p u'(1,p,x)$.
\begin{figure}[!ht]
\begin{equation*}
\mathfig{0.4}{explicit/supports_4} \qquad \qquad \mathfig{0.4}{explicit/supports_36}
\end{equation*}
\caption{The supports of $\partial_p u'(1,p,x)$ and of $\partial_p u(1,p,x)$ (with $K=3$) are subsets of the indicated region.}
\label{fig:supports_4}
\end{figure}

Second, pick some $K$, and define
\begin{align*}
u(t,p,x) & = \frac{\floor{K p}}{K} + \frac{1}{K} u'\left(t, K \left(p - \frac{\floor{K p}}{K}\right), x\right).
\end{align*}

\todo{Explain that the localisation property survives for $u'$ and $u$.}

We now check that by making $K$ large enough, $H$ becomes a homotopy through diffeomorphisms. We start with
$$\partial_x H(t,p,x) = \partial_x u(t,p,x) \partial_1 F(u(t,p,x), x) + \partial_2 F(u(t,p,x), x)$$
and observe that since $F(p, -)$ is a diffeomorphism, the second term $\partial_2 F(u(t,p,x), x)$ is bounded away from $0$. Thus if we can control the
size of the first term $\partial_x u(t,p,x) \partial_1 F(u(t,p,x), x)$ we're done. The factor $\partial_1 F(u(t,p,x), x)$ is bounded, and we
calculate \todo{err... this is a mess, and probably wrong.}
\begin{align*}
\partial_x u(t,p,x)_i & = \partial_x \frac{1}{K} u'\left(t, K\left(p - \frac{\floor{K p}}{K}\right), x\right)_i \\
                      & = \pm \frac{1}{2 K} \partial_x u''\left(t, (1\mp1)\pm 2K\left(p_i-\frac{\floor{K p}}{K}\right), x\right)_i \\
                      & = \pm \frac{1}{2 K} \sum_{l=1}^L (\partial_x r_l(x)) u_l\left(t, (1\mp1)\pm 2K\left(p_i-\frac{\floor{K p}}{K}\right)\right)_i. \\
\intertext{Since the target of $u_l$ is just the unit cube $I^k$, we can make the estimate}
\norm{\partial_x u(t,p,x)_i} & \leq \frac{1}{2 K} \sum_{l=1}^L \norm{\partial_x r_l(x)}.
\end{align*}
The sum here is bounded, so for large enough $K$ this is small enough that $\partial_x H(t,p,x)$ is never zero.

\end{proof}

