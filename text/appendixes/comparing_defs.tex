%!TEX root = ../../blob1.tex

\section{Comparing $n$-category definitions}
\label{sec:comparing-defs}

In this appendix we relate the ``topological" category definitions of \S\ref{sec:ncats}
to more traditional definitions, for $n=1$ and 2.

\nn{cases to cover: (a) plain $n$-cats for $n=1,2$; (b) $n$-cat modules for $n=1$, also 2?;
(c) $A_\infty$ 1-cat; (b) $A_\infty$ 1-cat module?; (e) tensor products?}

\subsection{$1$-categories over $\Set$ or $\Vect$}
\label{ssec:1-cats}
Given a topological $1$-category $\cX$ we construct a $1$-category in the conventional sense, $c(\cX)$.
This construction is quite straightforward, but we include the details for the sake of completeness, 
because it illustrates the role of structures (e.g. orientations, spin structures, etc) 
on the underlying manifolds, and 
to shed some light on the $n=2$ case, which we describe in \S \ref{ssec:2-cats}.

Let $B^k$ denote the \emph{standard} $k$-ball.
Let the objects of $c(\cX)$ be $c(\cX)^0 = \cX(B^0)$ and the morphisms of $c(\cX)$ be $c(\cX)^1 = \cX(B^1)$.
The boundary and restriction maps of $\cX$ give domain and range maps from $c(\cX)^1$ to $c(\cX)^0$.

Choose a homeomorphism $B^1\cup_{pt}B^1 \to B^1$.
Define composition in $c(\cX)$ to be the induced map $c(\cX)^1\times c(\cX)^1 \to c(\cX)^1$ 
(defined only when range and domain agree).
By isotopy invariance in $\cX$, any other choice of homeomorphism gives the same composition rule.
Also by isotopy invariance, composition is strictly associative.

Given $a\in c(\cX)^0$, define $\id_a \deq a\times B^1$.
By extended isotopy invariance in $\cX$, this has the expected properties of an identity morphism.


If the underlying manifolds for $\cX$ have further geometric structure, then we obtain certain functors.
The base case is for oriented manifolds, where we obtain no extra algebraic data.

For 1-categories based on unoriented manifolds (somewhat confusingly, we're thinking of being 
unoriented as requiring extra data beyond being oriented, namely the identification between the orientations), 
there is a map $*:c(\cX)^1\to c(\cX)^1$
coming from $\cX$ applied to an orientation-reversing homeomorphism (unique up to isotopy) 
from $B^1$ to itself.
Topological properties of this homeomorphism imply that 
$a^{**} = a$ (* is order 2), * reverses domain and range, and $(ab)^* = b^*a^*$
(* is an anti-automorphism).

For 1-categories based on Spin manifolds,
the the nontrivial spin homeomorphism from $B^1$ to itself which covers the identity
gives an order 2 automorphism of $c(\cX)^1$.

For 1-categories based on $\text{Pin}_-$ manifolds,
we have an order 4 antiautomorphism of $c(\cX)^1$.
For 1-categories based on $\text{Pin}_+$ manifolds,
we have an order 2 antiautomorphism and also an order 2 automorphism of $c(\cX)^1$,
and these two maps commute with each other.
\nn{need to also consider automorphisms of $B^0$ / objects}

\medskip

In the other direction, given a $1$-category $C$
(with objects $C^0$ and morphisms $C^1$) we will construct a topological
$1$-category $t(C)$.

If $X$ is a 0-ball (point), let $t(C)(X) \deq C^0$.
If $S$ is a 0-sphere, let $t(C)(S) \deq C^0\times C^0$.
If $X$ is a 1-ball, let $t(C)(X) \deq C^1$.
Homeomorphisms isotopic to the identity act trivially.
If $C$ has extra structure (e.g.\ it's a *-1-category), we use this structure
to define the action of homeomorphisms not isotopic to the identity
(and get, e.g., an unoriented topological 1-category).

The domain and range maps of $C$ determine the boundary and restriction maps of $t(C)$.

Gluing maps for $t(C)$ are determined by composition of morphisms in $C$.

For $X$ a 0-ball, $D$ a 1-ball and $a\in t(C)(X)$, define the product morphism 
$a\times D \deq \id_a$.
It is not hard to verify that this has the desired properties.

\medskip

The compositions of the constructions above, $$\cX\to c(\cX)\to t(c(\cX))$$ 
and $$C\to t(C)\to c(t(C)),$$ give back 
more or less exactly the same thing we started with.  

As we will see below, for $n>1$ the compositions yield a weaker sort of equivalence.

\medskip

Similar arguments show that modules for topological 1-categories are essentially
the same thing as traditional modules for traditional 1-categories.

\subsection{Plain 2-categories}
\label{ssec:2-cats}
Let $\cC$ be a topological 2-category.
We will construct a traditional pivotal 2-category.
(The ``pivotal" corresponds to our assumption of strong duality for $\cC$.)

We will try to describe the construction in such a way the the generalization to $n>2$ is clear,
though this will make the $n=2$ case a little more complicated than necessary.

\nn{Note: We have to decide whether our 2-morphsism are shaped like rectangles or bigons.
Each approach has advantages and disadvantages.
For better or worse, we choose bigons here.}

\nn{maybe we should do both rectangles and bigons?}

Define the $k$-morphisms $C^k$ of $C$ to be $\cC(B^k)_E$, where $B^k$ denotes the standard
$k$-ball, which we also think of as the standard bihedron.
Since we are thinking of $B^k$ as a bihedron, we have a standard decomposition of the $\bd B^k$
into two copies of $B^{k-1}$ which intersect along the ``equator" $E \cong S^{k-2}$.
Recall that the subscript in $\cC(B^k)_E$ means that we consider the subset of $\cC(B^k)$
whose boundary is splittable along $E$.
This allows us to define the domain and range of morphisms of $C$ using
boundary and restriction maps of $\cC$.

Choosing a homeomorphism $B^1\cup B^1 \to B^1$ defines a composition map on $C^1$.
This is not associative, but we will see later that it is weakly associative.

Choosing a homeomorphism $B^2\cup B^2 \to B^2$ defines a ``vertical" composition map 
on $C^2$ (Figure \ref{fzo1}).
Isotopy invariance implies that this is associative.
We will define a ``horizontal" composition later.
\nn{maybe no need to postpone?}

\begin{figure}[t]
\begin{equation*}
\mathfig{.73}{tempkw/zo1}
\end{equation*}
\caption{Vertical composition of 2-morphisms}
\label{fzo1}
\end{figure}

Given $a\in C^1$, define $\id_a = a\times I \in C^1$ (pinched boundary).
Extended isotopy invariance for $\cC$ shows that this morphism is an identity for 
vertical composition.

Given $x\in C^0$, define $\id_x = x\times B^1 \in C^1$.
We will show that this 1-morphism is a weak identity.
This would be easier if our 2-morphisms were shaped like rectangles rather than bigons.
Let $a: y\to x$ be a 1-morphism.
Define maps $a \to a\bullet \id_x$ and $a\bullet \id_x \to a$
as shown in Figure \ref{fzo2}.
\begin{figure}[t]
\begin{equation*}
\mathfig{.73}{tempkw/zo2}
\end{equation*}
\caption{blah blah}
\label{fzo2}
\end{figure}
In that figure, the red cross-hatched areas are the product of $x$ and a smaller bigon,
while the remainder is a half-pinched version of $a\times I$.
\nn{the red region is unnecessary; remove it?  or does it help?
(because it's what you get if you bigonify the natural rectangular picture)}
We must show that the two compositions of these two maps give the identity 2-morphisms
on $a$ and $a\bullet \id_x$, as defined above.
Figure \ref{fzo3} shows one case.
\begin{figure}[t]
\begin{equation*}
\mathfig{.83}{tempkw/zo3}
\end{equation*}
\caption{blah blah}
\label{fzo3}
\end{figure}
In the first step we have inserted a copy of $id(id(x))$ \nn{need better notation for this}.
\nn{also need to talk about (somewhere above) 
how this sort of insertion is allowed by extended isotopy invariance and gluing.
Also: maybe half-pinched and unpinched products can be derived from fully pinched
products after all (?)}
Figure \ref{fzo4} shows the other case.
\begin{figure}[t]
\begin{equation*}
\mathfig{.83}{tempkw/zo4}
\end{equation*}
\caption{blah blah}
\label{fzo4}
\end{figure}
We first collapse the red region, then remove a product morphism from the boundary,

We define horizontal composition of 2-morphisms as shown in Figure \ref{fzo5}.
It is not hard to show that this is independent of the arbitrary (left/right) 
choice made in the definition, and that it is associative.
\begin{figure}[t]
\begin{equation*}
\mathfig{.83}{tempkw/zo5}
\end{equation*}
\caption{Horizontal composition of 2-morphisms}
\label{fzo5}
\end{figure}

\nn{need to find a list of axioms for pivotal 2-cats to check}

\nn{...}

\medskip
\hrule
\medskip

\nn{to be continued...}
\medskip

\subsection{$A_\infty$ $1$-categories}
\label{sec:comparing-A-infty}
In this section, we make contact between the usual definition of an $A_\infty$ category 
and our definition of a topological $A_\infty$ $1$-category, from \S \ref{???}.

That definition associates a chain complex to every interval, and we begin by giving an alternative definition that is entirely in terms of the chain complex associated to the standard interval $[0,1]$. 
\begin{defn}
A \emph{topological $A_\infty$ category on $[0,1]$} $\cC$ has a set of objects $\Obj(\cC)$, 
and for each $a,b \in \Obj(\cC)$, a chain complex $\cC_{a,b}$, along with
\begin{itemize}
\item an action of the operad of $\Obj(\cC)$-labeled cell decompositions
\item and a compatible action of $\CD{[0,1]}$.
\end{itemize}
\end{defn}
Here the operad of cell decompositions of $[0,1]$ has operations indexed by a finite set of 
points $0 < x_1< \cdots < x_k < 1$, cutting $[0,1]$ into subintervals.
An $X$-labeled cell decomposition labels $\{0, x_1, \ldots, x_k, 1\}$ by $X$.
Given two cell decompositions $\cJ^{(1)}$ and $\cJ^{(2)}$, and an index $m$, we can compose 
them to form a new cell decomposition $\cJ^{(1)} \circ_m \cJ^{(2)}$ by inserting the points 
of $\cJ^{(2)}$ linearly into the $m$-th interval of $\cJ^{(1)}$.
In the $X$-labeled case, we insist that the appropriate labels match up.
Saying we have an action of this operad means that for each labeled cell decomposition 
$0 < x_1< \cdots < x_k < 1$, $a_0, \ldots, a_{k+1} \subset \Obj(\cC)$, there is a chain 
map $$\cC_{a_0,a_1} \tensor \cdots \tensor \cC_{a_k,a_{k+1}} \to \cC_{a_0,a_{k+1}}$$ and these 
chain maps compose exactly as the cell decompositions.
An action of $\CD{[0,1]}$ is compatible with an action of the cell decomposition operad 
if given a decomposition $\pi$, and a family of diffeomorphisms $f \in \CD{[0,1]}$ which 
is supported on the subintervals determined by $\pi$, then the two possible operations 
(glue intervals together, then apply the diffeomorphisms, or apply the diffeormorphisms 
separately to the subintervals, then glue) commute (as usual, up to a weakly unique homotopy).

Translating between this notion and the usual definition of an $A_\infty$ category is now straightforward.
To restrict to the standard interval, define $\cC_{a,b} = \cC([0,1];a,b)$.
Given a cell decomposition $0 < x_1< \cdots < x_k < 1$, we use the map (suppressing labels)
$$\cC([0,1])^{\tensor k+1} \to \cC([0,x_1]) \tensor \cdots \tensor \cC[x_k,1] \to \cC([0,1])$$
where the factors of the first map are induced by the linear isometries $[0,1] \to [x_i, x_{i+1}]$, and the second map is just gluing.
The action of $\CD{[0,1]}$ carries across, and is automatically compatible.
Going the other way, we just declare $\cC(J;a,b) = \cC_{a,b}$, pick a diffeomorphism 
$\phi_J : J \isoto [0,1]$ for every interval $J$, define the gluing map 
$\cC(J_1) \tensor \cC(J_2) \to \cC(J_1 \cup J_2)$ by the first applying 
the cell decomposition map for $0 < \frac{1}{2} < 1$, then the self-diffeomorphism of $[0,1]$ 
given by $\frac{1}{2} (\phi_{J_1} \cup (1+ \phi_{J_2})) \circ \phi_{J_1 \cup J_2}^{-1}$.
You can readily check that this gluing map is associative on the nose. \todo{really?}

%First recall the \emph{coloured little intervals operad}. Given a set of labels $\cL$, the operations are indexed by \emph{decompositions of the interval}, each of which is a collection of disjoint subintervals $\{(a_i,b_i)\}_{i=1}^k$ of $[0,1]$, along with a labeling of the complementary regions by $\cL$, $\{l_0, \ldots, l_k\}$.  Given two decompositions $\cJ^{(1)}$ and $\cJ^{(2)}$, and an index $m$ such that $l^{(1)}_{m-1} = l^{(2)}_0$ and $l^{(1)}_{m} = l^{(2)}_{k^{(2)}}$, we can form a new decomposition by inserting the intervals of $\cJ^{(2)}$ linearly inside the $m$-th interval of $\cJ^{(1)}$. We call the resulting decomposition $\cJ^{(1)} \circ_m \cJ^{(2)}$.

%\begin{defn}
%A \emph{topological $A_\infty$ category} $\cC$ has a set of objects $\Obj(\cC)$ and for each $a,b \in \Obj(\cC)$ a chain complex $\cC_{a,b}$, along with a compatible `composition map' and an `action of families of diffeomorphisms'.

%A \emph{composition map} $f$ is a family of chain maps, one for each decomposition of the interval, $f_\cJ : A^{\tensor k} \to A$, making $\cC$ into a category over the coloured little intervals operad, with labels $\cL = \Obj(\cC)$. Thus the chain maps satisfy the identity 
%\begin{equation*}
%f_{\cJ^{(1)} \circ_m \cJ^{(2)}} = f_{\cJ^{(1)}} \circ (\id^{\tensor m-1} \tensor f_{\cJ^{(2)}} \tensor \id^{\tensor k^{(1)} - m}).
%\end{equation*}

%An \emph{action of families of diffeomorphisms} is a chain map $ev: \CD{[0,1]} \tensor A \to A$, such that 
%\begin{enumerate}
%\item The diagram 
%\begin{equation*}
%\xymatrix{
%\CD{[0,1]} \tensor \CD{[0,1]} \tensor A \ar[r]^{\id \tensor ev} \ar[d]^{\circ \tensor \id} & \CD{[0,1]} \tensor A \ar[d]^{ev} \\
%\CD{[0,1]} \tensor A \ar[r]^{ev} & A
%}
%\end{equation*}
%commutes up to weakly unique homotopy.
%\item If $\phi \in \Diff([0,1])$ and $\cJ$ is a decomposition of the interval, we obtain a new decomposition $\phi(\cJ)$ and a collection $\phi_m \in \Diff([0,1])$ of diffeomorphisms obtained by taking the restrictions $\restrict{\phi}{[a_m,b_m]} : [a_m,b_m] \to [\phi(a_m),\phi(b_m)]$ and pre- and post-composing these with the linear diffeomorphisms $[0,1] \to [a_m,b_m]$ and $[\phi(a_m),\phi(b_m)] \to [0,1]$. We require that
%\begin{equation*}
%\phi(f_\cJ(a_1, \cdots, a_k)) = f_{\phi(\cJ)}(\phi_1(a_1), \cdots, \phi_k(a_k)).
%\end{equation*}
%\end{enumerate}
%\end{defn}

From a topological $A_\infty$ category on $[0,1]$ $\cC$ we can produce a `conventional' 
$A_\infty$ category $(A, \{m_k\})$ as defined in, for example, \cite{MR1854636}.
We'll just describe the algebra case (that is, a category with only one object), 
as the modifications required to deal with multiple objects are trivial.
Define $A = \cC$ as a chain complex (so $m_1 = d$).
Define $m_2 : A\tensor A \to A$ by $f_{\{(0,\frac{1}{2}),(\frac{1}{2},1)\}}$.
To define $m_3$, we begin by taking the one parameter family $\phi_3$ of diffeomorphisms 
of $[0,1]$ that interpolates linearly between the identity and the piecewise linear 
diffeomorphism taking $\frac{1}{4}$ to $\frac{1}{2}$ and $\frac{1}{2}$ to $\frac{3}{4}$, and then define
\begin{equation*}
m_3(a,b,c) = ev(\phi_3, m_2(m_2(a,b), c)).
\end{equation*}

It's then easy to calculate that
\begin{align*}
d(m_3(a,b,c)) & = ev(d \phi_3, m_2(m_2(a,b),c)) - ev(\phi_3 d m_2(m_2(a,b), c)) \\
 & = ev( \phi_3(1), m_2(m_2(a,b),c)) - ev(\phi_3(0), m_2 (m_2(a,b),c)) - \\ & \qquad - ev(\phi_3, m_2(m_2(da, b), c) + (-1)^{\deg a} m_2(m_2(a, db), c) + \\ & \qquad \quad + (-1)^{\deg a+\deg b} m_2(m_2(a, b), dc) \\
 & = m_2(a , m_2(b,c)) - m_2(m_2(a,b),c) - \\ & \qquad - m_3(da,b,c) + (-1)^{\deg a + 1} m_3(a,db,c) + \\ & \qquad \quad + (-1)^{\deg a + \deg b + 1} m_3(a,b,dc), \\
\intertext{and thus that}
m_1 \circ m_3 & =  m_2 \circ (\id \tensor m_2) - m_2 \circ (m_2 \tensor \id) - \\ & \qquad - m_3 \circ (m_1 \tensor \id \tensor \id) - m_3 \circ (\id \tensor m_1 \tensor \id) - m_3 \circ (\id \tensor \id \tensor m_1)
\end{align*}
as required (c.f. \cite[p. 6]{MR1854636}).
\todo{then the general case.}
We won't describe a reverse construction (producing a topological $A_\infty$ category 
from a ``conventional" $A_\infty$ category), but we presume that this will be easy for the experts.