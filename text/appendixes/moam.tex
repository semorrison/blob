%!TEX root = ../../blob1.tex

\section{The method of acyclic models}  \label{sec:moam}

In this section we recall the method of acyclic models for the reader's convenience. The material presented here is closely modeled on  \cite[Chapter 4]{MR0210112}.
We use this method throughout the paper (c.f. Theorem \ref{thm:product}, Theorem \ref{thm:gluing} and Theorem \ref{thm:map-recon}), as it provides a very convenient way to show the existence of a chain map with desired properties, even when many non-canonical choices are required in order to construct one, and further to show the up-to-homotopy uniqueness of such maps.

Let $F_*$ and $G_*$ be chain complexes.
Assume $F_k$ has a basis $\{x_{kj}\}$
(that is, $F_*$ is free and we have specified a basis).
(In our applications, $\{x_{kj}\}$ will typically be singular $k$-simplices or 
$k$-blob diagrams.)
For each basis element $x_{kj}$ assume we have specified a ``target" $D^{kj}_*\sub G_*$.

We say that a chain map $f:F_*\to G_*$ is {\it compatible} with the above data (basis and targets)
if $f(x_{kj})\in D^{kj}_*$ for all $k$ and $j$.
Let $\Compat(D^\bullet_*)$ denote the subcomplex of maps from $F_*$ to $G_*$
such that the image of each higher homotopy applied to $x_{kj}$ lies in $D^{kj}_*$.

\begin{thm}[Acyclic models]  \label{moam-thm}
Suppose 
\begin{itemize}
\item $D^{k-1,l}_* \sub D^{kj}_*$ whenever $x_{k-1,l}$ occurs in $\bd x_{kj}$
with non-zero coefficient;
\item $D^{0j}_0$ is non-empty for all $j$; and
\item $D^{kj}_*$ is $(k{-}1)$-acyclic (i.e.\ $H_{k-1}(D^{kj}_*) = 0$) for all $k,j$ .
\end{itemize}
Then $\Compat(D^\bullet_*)$ is non-empty.
If, in addition,
\begin{itemize}
\item $D^{kj}_*$ is $m$-acyclic for $k\le m \le k+i$ and for all $k,j$,
\end{itemize}
then $\Compat(D^\bullet_*)$ is $i$-connected.
\end{thm}

\begin{proof}
(Sketch)
This is a standard result; see, for example, \cite[Chapter 4]{MR0210112}.

We will build a chain map $f\in \Compat(D^\bullet_*)$ inductively.
Choose $f(x_{0j})\in D^{0j}_0$ for all $j$
(possible since $D^{0j}_0$ is non-empty).
Choose $f(x_{1j})\in D^{1j}_1$ such that $\bd f(x_{1j}) = f(\bd x_{1j})$
(possible since $D^{0l}_* \sub D^{1j}_*$ for each $x_{0l}$ in $\bd x_{1j}$
and $D^{1j}_*$ is 0-acyclic).
Continue in this way, choosing $f(x_{kj})\in D^{kj}_k$ such that $\bd f(x_{kj}) = f(\bd x_{kj})$
We have now constructed $f\in \Compat(D^\bullet_*)$, proving the first claim of the theorem.

Now suppose that $D^{kj}_*$ is $k$-acyclic for all $k$ and $j$.
Let $f$ and $f'$ be two chain maps (0-chains) in $\Compat(D^\bullet_*)$.
Using a technique similar to above we can construct a homotopy (1-chain) in $\Compat(D^\bullet_*)$
between $f$ and $f'$.
Thus $\Compat(D^\bullet_*)$ is 0-connected.
Similarly, if $D^{kj}_*$ is $(k{+}i)$-acyclic then we can show that $\Compat(D^\bullet_*)$ is $i$-connected.
\end{proof}

