\section{Homological systems of fields}
\label{sec:homological-fields}

In this section, we extend the definition of blob homology to allow \emph{homological systems of fields}.

We begin with a definition of a \emph{topological $A_\infty$ category}, and then introduce the notion of a homological system of fields. A topological $A_\infty$ category gives a $1$-dimensional homological system of fields. We'll suggest that any good definition of a topological $A_\infty$ $n$-category with duals should allow construction of an $n$-dimensional homological system of fields, but we won't propose any such definition here. Later, we extend the definition of blob homology to allow homological fields as input. These definitions allow us to state and prove a theorem about the blob homology of a product manifold, and an intermediate theorem about gluing, in preparation for the proof of Property \ref{property:gluing}.

\subsection{Topological $A_\infty$ categories}

First recall the \emph{coloured little intervals operad}. Given a set of labels $\cL$, the operations are indexed by \emph{decompositions of the interval}, each of which is a collection of disjoint subintervals $\{(a_i,b_i)\}_{i=1}^k$ of $[0,1]$, along with a labeling of the complementary regions by $\cL$, $\{l_0, \ldots, l_k\}$.  Given two decompositions $\cJ^{(1)}$ and $\cJ^{(2)}$, and an index $m$ such that $l^{(1)}_{m-1} = l^{(2)}_0$ and $l^{(1)}_{m} = l^{(2)}_{k^{(2)}}$, we can form a new decomposition by inserting the intervals of $\cJ^{(2)}$ linearly inside the $m$-th interval of $\cJ^{(1)}$. We call the resulting decomposition $\cJ^{(1)} \circ_m \cJ^{(2)}$.

\begin{defn}
A \emph{topological $A_\infty$ category} $\cC$ has a set of objects $\Obj(\cC)$ and for each $a,b \in \Obj(\cC)$ a chain complex $\cC_{a,b}$, along with a compatible `composition map' and an `action of families of diffeomorphisms'.

A \emph{composition map} $f$ is a family of chain maps, one for each decomposition of the interval, $f_\cJ : A^{\tensor k} \to A$, making $\cC$ into a category over the coloured little intervals operad, with labels $\cL = \Obj(\cC)$. Thus the chain maps satisfy the identity 
\begin{equation*}
f_{\cJ^{(1)} \circ_m \cJ^{(2)}} = f_{\cJ^{(1)}} \circ (\id^{\tensor m-1} \tensor f_{\cJ^{(2)}} \tensor \id^{\tensor k^{(1)} - m}).
\end{equation*}

An \emph{action of families of diffeomorphisms} is a chain map $ev: \CD{[0,1]} \tensor A \to A$, such that \todo{What goes here, if anything?} 
\begin{enumerate}
\item The diagram 
\begin{equation*}
\xymatrix{
\CD{[0,1]} \tensor \CD{[0,1]} \tensor A \ar[r]^{\id \tensor ev} \ar[d]^{\circ \tensor \id} & \CD{[0,1]} \tensor A \ar[d]^{ev} \\
\CD{[0,1]} \tensor A \ar[r]^{ev} & A
}
\end{equation*}
commutes up to weakly unique \todo{???} homotopy.
\item If $\phi \in \Diff([0,1])$ and $\cJ$ is a decomposition of the interval, we obtain a new decomposition $\phi(\cJ)$ and a collection $\phi_m \in \Diff([0,1])$ of diffeomorphisms obtained by taking the restrictions $\restrict{\phi}{[a_m,b_m]} : [a_m,b_m] \to [\phi(a_m),\phi(b_m)]$ and pre- and post-composing these with the linear diffeomorphisms $[0,1] \to [a_m,b_m]$ and $[\phi(a_m),\phi(b_m)] \to [0,1]$. We require that
\begin{equation*}
\phi(f_\cJ(a_1, \cdots, a_k)) = f_{\phi(\cJ)}(\phi_1(a_1), \cdots, \phi_k(a_k)).
\end{equation*}
\end{enumerate}
\end{defn}

From a topological $A_\infty$ category $\cC$ we can produce a `conventional' $A_\infty$ category $(A, \{m_k\})$ as defined in, for example, \cite{MR1854636}. We'll just describe the algebra case (that is, a category with only one object), as the modifications required to deal with multiple objects are trivial. Define $A = \cC$ as a chain complex (so $m_1 = d$). Define $m_2 : A\tensor A \to A$ by $f_{\{(0,\frac{1}{2}),(\frac{1}{2},1)\}}$. To define $m_3$, we begin by taking the one parameter family $\phi_3$ of diffeomorphisms of $[0,1]$ that interpolates linearly between the identity and the piecewise linear diffeomorphism taking $\frac{1}{4}$ to $\frac{1}{2}$ and $\frac{1}{2}$ to $\frac{3}{4}$, and then define
\begin{equation*}
m_3(a,b,c) = ev(\phi_3, m_2(m_2(a,b), c)).
\end{equation*}

It's then easy to calculate that
\begin{align*}
d(m_3(a,b,c)) & = ev(d \phi_3, m_2(m_2(a,b),c)) - ev(\phi_3 d m_2(m_2(a,b), c)) \\
 & = ev( \phi_3(1), m_2(m_2(a,b),c)) - ev(\phi_3(0), m_2 (m_2(a,b),c)) - \\ & \qquad - ev(\phi_3, m_2(m_2(da, b), c) + (-1)^{\deg a} m_2(m_2(a, db), c) + \\ & \qquad \quad + (-1)^{\deg a+\deg b} m_2(m_2(a, b), dc) \\
 & = m_2(a , m_2(b,c)) - m_2(m_2(a,b),c) - \\ & \qquad - m_3(da,b,c) + (-1)^{\deg a + 1} m_3(a,db,c) + \\ & \qquad \quad + (-1)^{\deg a + \deg b + 1} m_3(a,b,dc), \\
\intertext{and thus that}
m_1 \circ m_3 & =  m_2 \circ (\id \tensor m_2) - m_2 \circ (m_2 \tensor \id) - \\ & \qquad - m_3 \circ (m_1 \tensor \id \tensor \id) - m_3 \circ (\id \tensor m_1 \tensor \id) - m_3 \circ (\id \tensor \id \tensor m_1)
\end{align*}
as required (c.f. \cite[p. 6]{MR1854636}).
\todo{then the general case.}
We won't describe a reverse construction (producing a topological $A_\infty$ category from a `conventional' $A_\infty$ category), but we presume that this will be easy for the experts.

\subsection{Homological systems of fields}
A homological system of fields $\cF$ is nothing more than a system of fields in the category $\Kom$ of complexes of vector spaces; that is, the set of top level fields with given boundary conditions is always a complex.

A topological $A_\infty$ category $\cC$ gives rise to a one dimensional homological system of fields. The functor $\cF_0$ simply assigns the set of objects of $\cC$ to a point. 
For a $1$-manifold $X$, define a \emph{decomposition of $X$} with labels in $\cL$ as a (possibly empty) set of disjoint closed intervals $\{J\}$ in $X$, and a labeling of the complementary regions by elements of $\cL$.

The functor $\cF_1$ assigns to a $1$-manifold $X$ the vector space
\begin{equation*}
\cF_1(X) = \DirectSum_{\substack{\cJ \\ \text{a decomposition of $X$}}} \Tensor_{J \in \cJ} \cC_{l(J),r(J)}
\end{equation*}
where $l(J)$ and $r(J)$ denote the labels on the complementary regions on either side of the interval $J$. If $X$ has boundary, and we specify a boundary condition $c$ consisting of a label from $\Obj(\cC)$ at each boundary point, $\cF_1(X;c)$ is just the direct sum over decompositions agreeing with these boundary conditions. For any interval $I$, we define the local relations $\cU(I)$ to be the subcomplex of $\cF_1(I)$
\begin{equation*}
\cU(I) = \DirectSum_{\cJ} \ker\left(f_\cJ : \Tensor_{J \in \cJ} \cC_{l(J),r(J)} \to \cC_{l(I),r(I)} \right),
\end{equation*}
that is, the kernel of the composition map for $\cC$.

\todo{explain why this satisfies the axioms}

We now give two motivating examples, as theorems constructing other homological systems of fields,


\begin{thm}
For a fixed target space $X$, `chains of maps to $X$' is a homological system of fields $\Xi$, defined as
\begin{equation*}
\Xi(M) = \CM{M}{X}.
\end{equation*}
\end{thm}

\begin{thm}
Given an $n$-dimensional system of fields $\cF$, and a $k$-manifold $F$, there is an $n-k$ dimensional homological system of fields $\cF^{\times F}$ defined by
\begin{equation*}
\cF^{\times F}(M) = \cB_*(M \times F, \cF).
\end{equation*}
\end{thm}
We might suggestively write $\cF^{\times F}$ as  $\cB_*(F \times [0,1]^b, \cF)$, interpreting this as an (undefined!) $A_\infty$ $b$-category, and then as the resulting homological system of fields, following a recipe analogous to that given above for $A_\infty$ $1$-categories.


In later sections, we'll prove the following two unsurprising theorems, about the (as-yet-undefined) blob homology of these homological systems of fields.


\begin{thm}
\begin{equation*}
\cB_*(M, \Xi) \iso \Xi(M)
\end{equation*}
\end{thm}

\begin{thm}[Product formula]
Given a $b$-manifold $B$, an $f$-manifold $F$ and a $b+f$ dimensional system of fields,
there is a quasi-isomorphism
\begin{align*}
\cB_*(B \times F, \cF) & \quismto \cB_*(B, \cF^{\times F})
\end{align*}
\end{thm}

\begin{question}
Is it possible to compute the blob homology of a non-trivial bundle in terms of the blob homology of its fiber?
\end{question}

\subsection{Blob homology}
The definition of blob homology for $(\cF, \cU)$ a homological system of fields and local relations is essentially the same as that given before in \S \ref{???}, except now there are some extra terms in the differential accounting for the `internal' differential acting on the fields.

As before
\begin{equation*}
	\cB_*^{\cF,\cU}(M) \deq \bigoplus_{\overline{B}} \bigoplus_{\overline{c}}
		\left( \otimes_j \cU(B_j; c_j)\right) \otimes \cF(M \setmin B^t; c^t)
\end{equation*}
with $\overline{B}$ running over configurations of blobs satisfying the usual conditions, and $\overline{c}$ running over all boundary conditions. This is a doubly-graded vector space, graded by blob degree (the number of blobs) and internal degree (the sum of the homological degrees of the tensor factor fields). It becomes a complex by taking the homological degree to the be the sum of the blob and internal degrees, and defining $d$ by

\begin{equation*}
d f = \sum_{v \in t} \partial_v f + \sum_{v' \in t \cup \{\star\}} d_{v'} f,
\end{equation*}


%We'll write $\cT$ for the set of finite rooted trees. We'll think of each such a rooted tree as a category, with vertices as objects  and each morphism set either empty or a singleton, with $v \to w$ if $w$ is closer to a root of the tree than $v$. We'll write $\hat{v}$ for the `parent' of a vertex $v$ if $v$ is not a root (that is, $\hat{v}$ is the unique vertex such that $v \to \hat{v}$ but there is no $w$ with $v \to w \to \hat{v}$. If $v$ is a root, we'll write $\hat{v}=\star$. Further, for each tree $t$, let's arbitrarily choose an orientation $\lambda_t$, that is, an alternating $\pm1$-valued function on orderings of the vertices.

%Given $v \in t$ there's a functor $\partial_v : t \to t \setminus \{v\}$ which removes the vertex $v$. Notice that removing a vertex naturally produces an orientation on $t \setminus \{v\}$ from the orientation on $t$, by $(\partial_v \lambda_t)(o) = \lambda_t(vo)$. This orientation may or may not agree with the chosen orientation of $t \setminus \{v\}$. We'll define $\sigma(v \in t) = \pm 1$ according to whether or not they agree. Notice that $$\sigma(v \in t) \sigma(w \in \partial_v t) = - \sigma(w \in t) \sigma(v \in \partial_w t).$$

%Let $\operatorname{balls}(M)$ denote the category of open balls in $M$ with inclusions. Given a tree $t \in \cT$ we'll call a functor $b : t \to \operatorname{balls}(M)$ such that if $b(v) \cap b(v') \neq \emptyset$) then either $v \to v'$ or $v' \to v$, \emph{non-intersecting}.\footnote{Equivalently, if $b(v)$ and $b(v')$ are spanned in $\operatorname{balls}(M)$, then $v$ and $v'$ are spanned in $t$. That is, if there exists some ball $B \subset M$ so $B \subset b(v)$ and $B \subset b(v')$, then there must exist some $v'' \in t$ so $v'' \to v$ and $v'' \to v'$. Because $t$ is a tree, this implies either $v \to v'$ or $v' \to v$} For each non-intersecting functor $b$ define  
%\begin{equation*}
%\cF(t,b) = \cF\left(M \setminus b(t)\right) \tensor \left(\Tensor_{\substack{v \in t \\ \text{$v$ not a leaf}}} \cF\left(b(v) \setminus b(v' \to v)\right)\right) \tensor \left(\Tensor_{\substack{v \in t \\ \text{$v$ a leaf}}} \cU\left(b(v)\right)\right)
%\end{equation*}
%and then the vector space
%\begin{equation*}
%\cB_*^{\cF,\cU}(M) = \DirectSum_{t \in \cT} \DirectSum_{\substack{\text{non-intersecting}\\\text{functors} \\ b: t \to \operatorname{balls}(M)}} \cF(t,b)
%\end{equation*}

The blob degree of an element of $\cF(t,b)$ is the number of vertices in $t$, and the internal degree is the sum of the homological degrees in the tensor factors.
The vector space $\cB_*^{\cF,\cU}(M)$ becomes a chain complex by taking the homological degree to be the sum of the blob and internal degrees, and defining $d$ on $\cF(t,b)$ by
\begin{equation*}
d f = \sum_{v \in t} \partial_v f + \sum_{v' \in t \cup \{\star\}} d_{v'} f,
\end{equation*}
where if $f \in \cF(t,b)$ is an elementary tensor of the form $f = f_\star \tensor \Tensor_{v \in t} f_v$ with
\begin{align*}
f_\star & \in \cF(M \setminus b(t)) && \\
f_v       & \in \cF(b(v) \setminus b(v' \to v)) && \text{if $v$ is not a leaf} \\
f_v       & \in \cU(b(v)) && \text{if $v$ is a leaf}
\end{align*}
the terms $\partial_v f$ are elementary tensors in $\cF(\partial_v t, \restrict{b}{\partial_v t})$ defined by
\begin{equation*}
(\partial_v f)_{v'} = \begin{cases} \sigma(v \in t) f_{\hat{v}} \circ f_v & \text{if $v' = \hat{v}$} \\ f_{v'} & \text{otherwise} \end{cases}
\end{equation*}
and the terms $d_v f$ are also elementary tensors in $\cF(t, b)$ defined by
\begin{equation*}
(d_v f)_{v'} = \begin{cases} (-1)^{\sum_{v \to v'} \deg f(v')} & \text{if $v'=v$} \\ f_v & \text{otherwise.} \end{cases}
\end{equation*}

We remark that if $\cF$ takes values in vector spaces, not chain complexes, then the $d_v$ terms vanish, and this coincides with our earlier definition of blob homomology for (non-homological) systems of fields.

\todo{We'll quickly check $d^2=0$.}


\subsection{An intermediate gluing theorem}

\begin{thm}[Gluing, intermediate form]
Suppose $M = M_1 \cup_Y M_2$ is the union of two submanifolds $M_1$ and $M_2$ along a codimension $1$ manifold $Y$. The blob homology of $M$ can be computed as
\begin{equation*}
\cB_*(M, \cF) = \cB_*(([0,1], \{0\}, \{1\}), (\cB_*(Y, \cF), \cB_*(M_1, \cF), \cB_*(M_2, \cF))).
\end{equation*}
The right hand side is the blob homology of the interval, using ...
\end{thm}