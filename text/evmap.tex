%!TEX root = ../blob1.tex

Let $CD_*(X, Y)$ denote $C_*(\Diff(X \to Y))$, the singular chain complex of
the space of diffeomorphisms
\nn{or homeomorphisms}
between the $n$-manifolds $X$ and $Y$ (extending a fixed diffeomorphism $\bd X \to \bd Y$).
For convenience, we will permit the singular cells generating $CD_*(X, Y)$ to be more general
than simplices --- they can be based on any linear polyhedron.
\nn{be more restrictive here?  does more need to be said?}
We also will use the abbreviated notation $CD_*(X) \deq CD_*(X, X)$.

\begin{prop}  \label{CDprop}
For $n$-manifolds $X$ and $Y$ there is a chain map
\eq{
    e_{XY} : CD_*(X, Y) \otimes \bc_*(X) \to \bc_*(Y) .
}
On $CD_0(X, Y) \otimes \bc_*(X)$ it agrees with the obvious action of $\Diff(X, Y)$ on $\bc_*(X)$
(Proposition (\ref{diff0prop})).
For any splittings $X = X_1 \cup X_2$ and $Y = Y_1 \cup Y_2$, 
the following diagram commutes up to homotopy
\eq{ \xymatrix{
     CD_*(X, Y) \otimes \bc_*(X) \ar[r]^{e_{XY}}    & \bc_*(Y) \\
     CD_*(X_1, Y_1) \otimes CD_*(X_2, Y_2) \otimes \bc_*(X_1) \otimes \bc_*(X_2)
        \ar@/_4ex/[r]_{e_{X_1Y_1} \otimes e_{X_2Y_2}}  \ar[u]^{\gl \otimes \gl}  &
            \bc_*(Y_1) \otimes \bc_*(Y_2) \ar[u]_{\gl}
} }
Any other map satisfying the above two properties is homotopic to $e_X$.
\end{prop}

\nn{need to rewrite for self-gluing instead of gluing two pieces together}

\nn{Should say something stronger about uniqueness.
Something like: there is
a contractible subcomplex of the complex of chain maps
$CD_*(X) \otimes \bc_*(X) \to \bc_*(X)$ (0-cells are the maps, 1-cells are homotopies, etc.),
and all choices in the construction lie in the 0-cells of this
contractible subcomplex.
Or maybe better to say any two choices are homotopic, and
any two homotopies and second order homotopic, and so on.}

\nn{Also need to say something about associativity.
Put it in the above prop or make it a separate prop?
I lean toward the latter.}
\medskip

The proof will occupy the remainder of this section.
\nn{unless we put associativity prop at end}

Without loss of generality, we will assume $X = Y$.

\medskip

Let $f: P \times X \to X$ be a family of diffeomorphisms and $S \sub X$.
We say that {\it $f$ is supported on $S$} if $f(p, x) = f(q, x)$ for all
$x \notin S$ and $p, q \in P$. Equivalently, $f$ is supported on $S$ if there is a family of diffeomorphisms $f' : P \times S \to S$ and a `background'
diffeomorphism $f_0 : X \to X$ so that
\begin{align}
	f(p,s) & = f_0(f'(p,s)) \;\;\;\; \mbox{for}\; (p, s) \in P\times S \\
\intertext{and}
	f(p,x) & = f_0(x) \;\;\;\; \mbox{for}\; (p, x) \in {P \times (X \setmin S)}.
\end{align}
Note that if $f$ is supported on $S$ then it is also supported on any $R \sup S$.

Let $\cU = \{U_\alpha\}$ be an open cover of $X$.
A $k$-parameter family of diffeomorphisms $f: P \times X \to X$ is
{\it adapted to $\cU$} if there is a factorization
\eq{
    P = P_1 \times \cdots \times P_m
}
(for some $m \le k$)
and families of diffeomorphisms
\eq{
    f_i :  P_i \times X \to X
}
such that
\begin{itemize}
\item each $f_i$ is supported on some connected $V_i \sub X$;
\item the sets $V_i$ are mutually disjoint;
\item each $V_i$ is the union of at most $k_i$ of the $U_\alpha$'s,
where $k_i = \dim(P_i)$; and
\item $f(p, \cdot) = g \circ f_1(p_1, \cdot) \circ \cdots \circ f_m(p_m, \cdot)$
for all $p = (p_1, \ldots, p_m)$, for some fixed $g \in \Diff(X)$.
\end{itemize}
A chain $x \in CD_k(X)$ is (by definition) adapted to $\cU$ if it is the sum
of singular cells, each of which is adapted to $\cU$.

(Actually, in this section we will only need families of diffeomorphisms to be 
{\it weakly adapted} to $\cU$, meaning that the support of $f$ is contained in the union
of at most $k$ of the $U_\alpha$'s.)

\begin{lemma}  \label{extension_lemma}
Let $x \in CD_k(X)$ be a singular chain such that $\bd x$ is adapted to $\cU$.
Then $x$ is homotopic (rel boundary) to some $x' \in CD_k(X)$ which is adapted to $\cU$.
Furthermore, one can choose the homotopy so that its support is equal to the support of $x$.
\end{lemma}

The proof will be given in Section \ref{sec:localising}.

\medskip

Before diving into the details, we outline our strategy for the proof of Proposition \ref{CDprop}.

%Suppose for the moment that evaluation maps with the advertised properties exist.
Let $p$ be a singular cell in $CD_k(X)$ and $b$ be a blob diagram in $\bc_*(X)$.
Suppose that there exists $V \sub X$ such that
\begin{enumerate}
\item $V$ is homeomorphic to a disjoint union of balls, and
\item $\supp(p) \cup \supp(b) \sub V$.
\end{enumerate}
Let $W = X \setmin V$, $W' = p(W)$ and $V' = X\setmin W'$.
We then have a factorization 
\[
	p = \gl(q, r),
\]
where $q \in CD_k(V, V')$ and $r \in CD_0(W, W')$.
We can also factorize $b = \gl(b_V, b_W)$, where $b_V\in \bc_*(V)$ and $b_W\in\bc_0(W)$.
According to the commutative diagram of the proposition, we must have
\[
	e_X(p\otimes b) = e_X(\gl(q\otimes b_V, r\otimes b_W)) = 
				gl(e_{VV'}(q\otimes b_V), e_{WW'}(r\otimes b_W)) .
\]
Since $r$ is a plain, 0-parameter family of diffeomorphisms, we must have
\[
	e_{WW'}(r\otimes b_W) = r(b_W),
\]
where $r(b_W)$ denotes the obvious action of diffeomorphisms on blob diagrams (in
this case a 0-blob diagram).
Since $V'$ is a disjoint union of balls, $\bc_*(V')$ is acyclic in degrees $>0$ 
(by \ref{disjunion} and \ref{bcontract}).
Assuming inductively that we have already defined $e_{VV'}(\bd(q\otimes b_V))$,
there is, up to (iterated) homotopy, a unique choice for $e_{VV'}(q\otimes b_V)$
such that 
\[
	\bd(e_{VV'}(q\otimes b_V)) = e_{VV'}(\bd(q\otimes b_V)) .
\]

Thus the conditions of the proposition determine (up to homotopy) the evaluation
map for generators $p\otimes b$ such that $\supp(p) \cup \supp(b)$ is contained in a disjoint
union of balls.
On the other hand, Lemma \ref{extension_lemma} allows us to homotope 
\nn{is this commonly used as a verb?} arbitrary generators to sums of generators with this property.
\nn{should give a name to this property; also forward reference}
This (roughly) establishes the uniqueness part of the proposition.
To show existence, we must show that the various choices involved in constructing
evaluation maps in this way affect the final answer only by a homotopy.

\nn{maybe put a little more into the outline before diving into the details.}

\nn{Note: At the moment this section is very inconsistent with respect to PL versus smooth,
homeomorphism versus diffeomorphism, etc.
We expect that everything is true in the PL category, but at the moment our proof
avails itself to smooth techniques.
Furthermore, it traditional in the literature to speak of $C_*(\Diff(X))$
rather than $C_*(\Homeo(X))$.}

\medskip

Now for the details.

Notation: Let $|b| = \supp(b)$, $|p| = \supp(p)$.

Choose a metric on $X$.
Choose a monotone decreasing sequence of positive real numbers $\ep_i$ converging to zero
(e.g.\ $\ep_i = 2^{-i}$).
Choose another sequence of positive real numbers $\delta_i$ such that $\delta_i/\ep_i$
converges monotonically to zero (e.g.\ $\delta_i = \ep_i^2$).
Let $\phi_l$ be an increasing sequence of positive numbers
satisfying the inequalities of Lemma \ref{xx2phi}.
Given a generator $p\otimes b$ of $CD_*(X)\otimes \bc_*(X)$ and non-negative integers $i$ and $l$
define
\[
	N_{i,l}(p\ot b) \deq \Nbd_{l\ep_i}(|b|) \cup \Nbd_{\phi_l\delta_i}(|p|).
\]
In other words, we use the metric to choose nested neighborhoods of $|b|\cup |p|$ (parameterized
by $l$), with $\ep_i$ controlling the size of the buffers around $|b|$ and $\delta_i$ controlling
the size of the buffers around $|p|$.

Next we define subcomplexes $G_*^{i,m} \sub CD_*(X)\otimes \bc_*(X)$.
Let $p\ot b$ be a generator of $CD_*(X)\otimes \bc_*(X)$ and let $k = \deg(p\ot b)
= \deg(p) + \deg(b)$.
$p\ot b$ is (by definition) in $G_*^{i,m}$ if either (a) $\deg(p) = 0$ or (b)
there exist codimension-zero submanifolds $V_0,\ldots,V_m \sub X$ such that each $V_j$
is homeomorphic to a disjoint union of balls and
\[
	N_{i,k}(p\ot b) \subeq V_0 \subeq N_{i,k+1}(p\ot b)
			\subeq V_1 \subeq \cdots \subeq V_m \subeq N_{i,k+m+1}(p\ot b) .
\]
Further, we require (inductively) that $\bd(p\ot b) \in G_*^{i,m}$.
We also require that $b$ is splitable (transverse) along the boundary of each $V_l$.

Note that $G_*^{i,m+1} \subeq G_*^{i,m}$.

As sketched above and explained in detail below, 
$G_*^{i,m}$ is a subcomplex where it is easy to define
the evaluation map.
The parameter $m$ controls the number of iterated homotopies we are able to construct
(see Lemma \ref{m_order_hty}).
The larger $i$ is (i.e.\ the smaller $\ep_i$ is), the better $G_*^{i,m}$ approximates all of
$CD_*(X)\ot \bc_*(X)$ (see Lemma \ref{Gim_approx}).

Next we define a chain map (dependent on some choices) $e: G_*^{i,m} \to \bc_*(X)$.
Let $p\ot b \in G_*^{i,m}$.
If $\deg(p) = 0$, define
\[
	e(p\ot b) = p(b) ,
\]
where $p(b)$ denotes the obvious action of the diffeomorphism(s) $p$ on the blob diagram $b$.
For general $p\ot b$ ($\deg(p) \ge 1$) assume inductively that we have already defined
$e(p'\ot b')$ when $\deg(p') + \deg(b') < k = \deg(p) + \deg(b)$.
Choose $V = V_0$ as above so that 
\[
	N_{i,k}(p\ot b) \subeq V \subeq N_{i,k+1}(p\ot b) .
\]
Let $\bd(p\ot b) = \sum_j p_j\ot b_j$, and let $V^j$ be the choice of neighborhood
of $|p_j|\cup |b_j|$ made at the preceding stage of the induction.
For all $j$, 
\[
	V^j \subeq N_{i,k}(p_j\ot b_j) \subeq N_{i,k}(p\ot b) \subeq V .
\]
(The second inclusion uses the facts that $|p_j| \subeq |p|$ and $|b_j| \subeq |b|$.)
We therefore have splittings
\[
	p = p'\bullet p'' , \;\; b = b'\bullet b'' , \;\; e(\bd(p\ot b)) = f'\bullet b'' ,
\]
where $p' \in CD_*(V)$, $p'' \in CD_*(X\setmin V)$, 
$b' \in \bc_*(V)$, $b'' \in \bc_*(X\setmin V)$, 
$f' \in \bc_*(p(V))$, and $f'' \in \bc_*(p(X\setmin V))$.
(Note that since the family of diffeomorphisms $p$ is constant (independent of parameters)
near $\bd V$, the expressions $p(V) \sub X$ and $p(X\setmin V) \sub X$ are
unambiguous.)
We have $\deg(p'') = 0$ and, inductively, $f'' = p''(b'')$.
%We also have that $\deg(b'') = 0 = \deg(p'')$.
Choose $x' \in \bc_*(p(V))$ such that $\bd x' = f'$.
This is possible by \ref{bcontract}, \ref{disjunion} and \nn{property 2 of local relations (isotopy)}.
Finally, define
\[
	e(p\ot b) \deq x' \bullet p''(b'') .
\]

Note that above we are essentially using the method of acyclic models.
For each generator $p\ot b$ we specify the acyclic (in positive degrees) 
target complex $\bc_*(p(V)) \bullet p''(b'')$.

The definition of $e: G_*^{i,m} \to \bc_*(X)$ depends on two sets of choices:
The choice of neighborhoods $V$ and the choice of inverse boundaries $x'$.
The next lemma shows that up to (iterated) homotopy $e$ is independent
of these choices.
(Note that independence of choices of $x'$ (for fixed choices of $V$)
is a standard result in the method of acyclic models.)

%\begin{lemma}
%Let $\tilde{e} :  G_*^{i,m} \to \bc_*(X)$ be a chain map constructed like $e$ above, but with
%different choices of $x'$ at each step.
%(Same choice of $V$ at each step.)
%Then $e$ and $\tilde{e}$ are homotopic via a homotopy in $\bc_*(p(V)) \bullet p''(b'')$.
%Any two choices of such a first-order homotopy are second-order homotopic, and so on, 
%to arbitrary order.
%\end{lemma}

%\begin{proof}
%This is a standard result in the method of acyclic models.
%\nn{should we say more here?}
%\nn{maybe this lemma should be subsumed into the next lemma.  probably it should.}
%\end{proof}

\begin{lemma} \label{m_order_hty}
Let $\tilde{e} :  G_*^{i,m} \to \bc_*(X)$ be a chain map constructed like $e$ above, but with
different choices of $V$ (and hence also different choices of $x'$) at each step.
If $m \ge 1$ then $e$ and $\tilde{e}$ are homotopic.
If $m \ge 2$ then any two choices of this first-order homotopy are second-order homotopic.
And so on.
In other words,  $e :  G_*^{i,m} \to \bc_*(X)$ is well-defined up to $m$-th order homotopy.
\end{lemma}

\begin{proof}
We construct $h: G_*^{i,m} \to \bc_*(X)$ such that $\bd h + h\bd = e - \tilde{e}$.
$e$ and $\tilde{e}$ coincide on bidegrees $(0, j)$, so define $h$
to be zero there.
Assume inductively that $h$ has been defined for degrees less than $k$.
Let $p\ot b$ be a generator of degree $k$.
Choose $V_1$ as in the definition of $G_*^{i,m}$ so that
\[
	N_{i,k+1}(p\ot b) \subeq V_1 \subeq N_{i,k+2}(p\ot b) .
\]
There are splittings
\[
	p = p'_1\bullet p''_1 , \;\; b = b'_1\bullet b''_1 , 
			\;\; e(p\ot b) - \tilde{e}(p\ot b) - h(\bd(p\ot b)) = f'_1\bullet f''_1 ,
\]
where $p'_1 \in CD_*(V_1)$, $p''_1 \in CD_*(X\setmin V_1)$, 
$b'_1 \in \bc_*(V_1)$, $b''_1 \in \bc_*(X\setmin V_1)$, 
$f'_1 \in \bc_*(p(V_1))$, and $f''_1 \in \bc_*(p(X\setmin V_1))$.
Inductively, $\bd f'_1 = 0$ and $f_1'' = p_1''(b_1'')$.
Choose $x'_1 \in \bc_*(p(V_1))$ so that $\bd x'_1 = f'_1$.
Define 
\[
	h(p\ot b) \deq x'_1 \bullet p''_1(b''_1) .
\]
This completes the construction of the first-order homotopy when $m \ge 1$.

The $j$-th order homotopy is constructed similarly, with $V_j$ replacing $V_1$ above.
\end{proof}

Note that on $G_*^{i,m+1} \subeq G_*^{i,m}$, we have defined two maps,
call them $e_{i,m}$ and $e_{i,m+1}$.
An easy variation on the above lemma shows that $e_{i,m}$ and $e_{i,m+1}$ are $m$-th 
order homotopic.

Next we show how to homotope chains in $CD_*(X)\ot \bc_*(X)$ to one of the 
$G_*^{i,m}$.
Choose a monotone decreasing sequence of real numbers $\gamma_j$ converging to zero.
Let $\cU_j$ denote the open cover of $X$ by balls of radius $\gamma_j$.
Let $h_j: CD_*(X)\to CD_*(X)$ be a chain map homotopic to the identity whose image is spanned by diffeomorphisms with support compatible with $\cU_j$, as described in Lemma \ref{xxxxx}.
Recall that $h_j$ and also the homotopy connecting it to the identity do not increase
supports.
Define
\[
	g_j \deq h_j\circ h_{j-1} \circ \cdots \circ h_1 .
\]
The next lemma says that for all generators $p\ot b$ we can choose $j$ large enough so that
$g_j(p)\ot b$ lies in $G_*^{i,m}$, for arbitrary $m$ and sufficiently large $i$ 
(depending on $b$, $n = \deg(p)$ and $m$).
(Note: Don't confuse this $n$ with the top dimension $n$ used elsewhere in this paper.)

\begin{lemma} \label{Gim_approx}
Fix a blob diagram $b$, a homotopy order $m$ and a degree $n$ for $CD_*(X)$.
Then there exists a constant $k_{bmn}$ such that for all $i \ge k_{bmn}$
there exists another constant $j_i$ such that for all $j \ge j_i$ and all $p\in CD_n(X)$ 
we have $g_j(p)\ot b \in G_*^{i,m}$.
\end{lemma}

\begin{proof}
Let $c$ be a subset of the blobs of $b$.
There exists $l > 0$ such that $\Nbd_u(c)$ is homeomorphic to $|c|$ for all $u < l$ 
and all such $c$.
(Here we are using a piecewise smoothness assumption for $\bd c$, and also
the fact that $\bd c$ is collared.
We need to consider all such $c$ because all generators appearing in
iterated boundaries of must be in $G_*^{i,m}$.)

Let $r = \deg(b)$ and 
\[
	t = r+n+m+1 = \deg(p\ot b) + m + 1.
\]

Choose $k = k_{bmn}$ such that
\[
	t\ep_k < l
\]
and
\[
	n\cdot (2 (\phi_t + 1) \delta_k) < \ep_k .
\]
Let $i \ge k_{bmn}$.
Choose $j = j_i$ so that
\[
	\gamma_j < \delta_i
\]
and also so that $\phi_t \gamma_j$ is less than the constant $\rho(M)$ of Lemma \ref{xxzz11}.

Let $j \ge j_i$ and $p\in CD_n(X)$.
Let $q$ be a generator appearing in $g_j(p)$.
Note that $|q|$ is contained in a union of $n$ elements of the cover $\cU_j$,
which implies that $|q|$ is contained in a union of $n$ metric balls of radius $\delta_i$.
We must show that $q\ot b \in G_*^{i,m}$, which means finding neighborhoods
$V_0,\ldots,V_m \sub X$ of $|q|\cup |b|$ such that each $V_j$
is homeomorphic to a disjoint union of balls and
\[
	N_{i,n}(q\ot b) \subeq V_0 \subeq N_{i,n+1}(q\ot b)
			\subeq V_1 \subeq \cdots \subeq V_m \subeq N_{i,t}(q\ot b) .
\]
By repeated applications of Lemma \ref{xx2phi} we can find neighborhoods $U_0,\ldots,U_m$
of $|q|$, each homeomorphic to a disjoint union of balls, with
\[
	\Nbd_{\phi_{n+l} \delta_i}(|q|) \subeq U_l \subeq \Nbd_{\phi_{n+l+1} \delta_i}(|q|) .
\]
The inequalities above \nn{give ref} guarantee that we can find $u_l$ with 
\[
	(n+l)\ep_i \le u_l \le (n+l+1)\ep_i
\]
such that each component of $U_l$ is either disjoint from $\Nbd_{u_l}(|b|)$ or contained in 
$\Nbd_{u_l}(|b|)$.
This is because there are at most $n$ components of $U_l$, and each component
has radius $\le (\phi_t + 1) \delta_i$.
It follows that
\[
	V_l \deq \Nbd_{u_l}(|b|) \cup U_l
\]
is homeomorphic to a disjoint union of balls and satisfies
\[
	N_{i,n+l}(q\ot b) \subeq V_l \subeq N_{i,n+l+1}(q\ot b) .
\]

The same argument shows that each generator involved in iterated boundaries of $q\ot b$
is in $G_*^{i,m}$.
\end{proof}

In the next few lemmas we have made no effort to optimize the various bounds.
(The bounds are, however, optimal in the sense of minimizing the amount of work
we do.  Equivalently, they are the first bounds we thought of.)

We say that a subset $S$ of a metric space has radius $\le r$ if $S$ is contained in
some metric ball of radius $r$.

\begin{lemma}
Let $S \sub \ebb^n$ (Euclidean $n$-space) have radius $\le r$.  
Then $\Nbd_a(S)$ is homeomorphic to a ball for $a \ge 2r$.
\end{lemma}

\begin{proof} \label{xxyy2}
Let $S$ be contained in $B_r(y)$, $y \in \ebb^n$.
Note that if $a \ge 2r$ then $\Nbd_a(S) \sup B_r(y)$.
Let $z\in \Nbd_a(S) \setmin B_r(y)$.
Consider the triangle
\nn{give figure?} with vertices $z$, $y$ and $s$ with $s\in S$.
The length of the edge $yz$ is greater than $r$ which is greater
than the length of the edge $ys$.
It follows that the angle at $z$ is less than $\pi/2$ (less than $\pi/3$, in fact),
which means that points on the edge $yz$ near $z$ are closer to $s$ than $z$ is,
which implies that these points are also in $\Nbd_a(S)$.
Hence $\Nbd_a(S)$ is star-shaped with respect to $y$.
\end{proof}

If we replace $\ebb^n$ above with an arbitrary compact Riemannian manifold $M$,
the same result holds, so long as $a$ is not too large:

\begin{lemma} \label{xxzz11}
Let $M$ be a compact Riemannian manifold.
Then there is a constant $\rho(M)$ such that for all
subsets $S\sub M$ of radius $\le r$ and all $a$ such that $2r \le a \le \rho(M)$,
$\Nbd_a(S)$ is homeomorphic to a ball.
\end{lemma}

\begin{proof}
Choose $\rho = \rho(M)$ such that $3\rho/2$ is less than the radius of injectivity of $M$,
and also so that for any point $y\in M$ the geodesic coordinates of radius $3\rho/2$ around
$y$ distort angles by only a small amount.
Now the argument of the previous lemma works.
\end{proof}



\begin{lemma} \label{xx2phi}
Let $S \sub M$ be contained in a union (not necessarily disjoint)
of $k$ metric balls of radius $r$.
Let $\phi_1, \phi_2, \ldots$ be an increasing sequence of real numbers satisfying
$\phi_1 \ge 2$ and $\phi_{i+1} \ge \phi_i(2\phi_i + 2) + \phi_i$.
For convenience, let $\phi_0 = 0$.
Assume also that $\phi_k r \le \rho(M)$.
Then there exists a neighborhood $U$ of $S$,
homeomorphic to a disjoint union of balls, such that
\[
	\Nbd_{\phi_{k-1} r}(S) \subeq U \subeq \Nbd_{\phi_k r}(S) .
\]
\end{lemma}

\begin{proof}
For $k=1$ this follows from Lemma \ref{xxzz11}.
Assume inductively that it holds for $k-1$.
Partition $S$ into $k$ disjoint subsets $S_1,\ldots,S_k$, each of radius $\le r$.
By Lemma \ref{xxzz11}, each $\Nbd_{\phi_{k-1} r}(S_i)$ is homeomorphic to a ball.
If these balls are disjoint, let $U$ be their union.
Otherwise, assume WLOG that $S_{k-1}$ and $S_k$ are distance less than $2\phi_{k-1}r$ apart.
Let $R_i = \Nbd_{\phi_{k-1} r}(S_i)$ for $i = 1,\ldots,k-2$ 
and $R_{k-1} = \Nbd_{\phi_{k-1} r}(S_{k-1})\cup \Nbd_{\phi_{k-1} r}(S_k)$.
Each $R_i$ is contained in a metric ball of radius $r' \deq (2\phi_{k-1}+2)r$.
Note that the defining inequality of the $\phi_i$ guarantees that
\[
	\phi_{k-1}r' = \phi_{k-1}(2\phi_{k-1}+2)r \le \phi_k r \le \rho(M) .
\]
By induction, there is a neighborhood $U$ of $R \deq \bigcup_i R_i$, 
homeomorphic to a disjoint union
of balls, and such that
\[
	U \subeq \Nbd_{\phi_{k-1}r'}(R) = \Nbd_{t}(S) \subeq \Nbd_{\phi_k r}(S) ,
\]
where $t = \phi_{k-1}(2\phi_{k-1}+2)r + \phi_{k-1} r$.
\end{proof}


\medskip


\hrule\medskip\hrule\medskip

\nn{outline of what remains to be done:}

\begin{itemize}
\item We need to assemble the maps for the various $G^{i,m}$ into
a map for all of $CD_*\ot \bc_*$.
One idea: Think of the $g_j$ as a sort of homotopy (from $CD_*\ot \bc_*$ to itself) 
parameterized by $[0,\infty)$.  For each $p\ot b$ in $CD_*\ot \bc_*$ choose a sufficiently
large $j'$.  Use these choices to reparameterize $g_\bullet$ so that each
$p\ot b$ gets pushed as far as the corresponding $j'$.
\item Independence of metric, $\ep_i$, $\delta_i$:
For a different metric etc. let $\hat{G}^{i,m}$ denote the alternate subcomplexes
and $\hat{N}_{i,l}$ the alternate neighborhoods.
Main idea is that for all $i$ there exists sufficiently large $k$ such that
$\hat{N}_{k,l} \sub N_{i,l}$, and similarly with the roles of $N$ and $\hat{N}$ reversed.
\item Also need to prove associativity.
\end{itemize}


\nn{to be continued....}

\noop{

\begin{lemma}

\end{lemma}

\begin{proof}

\end{proof}

}




%\nn{say something about associativity here}





