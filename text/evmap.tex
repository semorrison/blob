%!TEX root = ../blob1.tex

Let $CD_*(X, Y)$ denote $C_*(\Diff(X \to Y))$, the singular chain complex of
the space of diffeomorphisms
\nn{or homeomorphisms}
between the $n$-manifolds $X$ and $Y$ (extending a fixed diffeomorphism $\bd X \to \bd Y$).
For convenience, we will permit the singular cells generating $CD_*(X, Y)$ to be more general
than simplices --- they can be based on any linear polyhedron.
\nn{be more restrictive here?  does more need to be said?}
We also will use the abbreviated notation $CD_*(X) \deq CD_*(X, X)$.

\begin{prop}  \label{CDprop}
For $n$-manifolds $X$ and $Y$ there is a chain map
\eq{
    e_{XY} : CD_*(X, Y) \otimes \bc_*(X) \to \bc_*(Y) .
}
On $CD_0(X, Y) \otimes \bc_*(X)$ it agrees with the obvious action of $\Diff(X, Y)$ on $\bc_*(X)$
(Proposition (\ref{diff0prop})).
For any splittings $X = X_1 \cup X_2$ and $Y = Y_1 \cup Y_2$, 
the following diagram commutes up to homotopy
\eq{ \xymatrix{
     CD_*(X, Y) \otimes \bc_*(X) \ar[r]^{e_{XY}}    & \bc_*(Y) \\
     CD_*(X_1, Y_1) \otimes CD_*(X_2, Y_2) \otimes \bc_*(X_1) \otimes \bc_*(X_2)
        \ar@/_4ex/[r]_{e_{X_1Y_1} \otimes e_{X_2Y_2}}  \ar[u]^{\gl \otimes \gl}  &
            \bc_*(Y_1) \otimes \bc_*(Y_2) \ar[u]_{\gl}
} }
Any other map satisfying the above two properties is homotopic to $e_X$.
\end{prop}

\nn{need to rewrite for self-gluing instead of gluing two pieces together}

\nn{Should say something stronger about uniqueness.
Something like: there is
a contractible subcomplex of the complex of chain maps
$CD_*(X) \otimes \bc_*(X) \to \bc_*(X)$ (0-cells are the maps, 1-cells are homotopies, etc.),
and all choices in the construction lie in the 0-cells of this
contractible subcomplex.
Or maybe better to say any two choices are homotopic, and
any two homotopies and second order homotopic, and so on.}

\nn{Also need to say something about associativity.
Put it in the above prop or make it a separate prop?
I lean toward the latter.}
\medskip

The proof will occupy the remainder of this section.
\nn{unless we put associativity prop at end}

Without loss of generality, we will assume $X = Y$.

\medskip

Let $f: P \times X \to X$ be a family of diffeomorphisms and $S \sub X$.
We say that {\it $f$ is supported on $S$} if $f(p, x) = f(q, x)$ for all
$x \notin S$ and $p, q \in P$. Equivalently, $f$ is supported on $S$ if there is a family of diffeomorphisms $f' : P \times S \to S$ and a `background'
diffeomorphism $f_0 : X \to X$ so that
\begin{align}
	f(p,s) & = f_0(f'(p,s)) \;\;\;\; \mbox{for}\; (p, s) \in P\times S \\
\intertext{and}
	f(p,x) & = f_0(x) \;\;\;\; \mbox{for}\; (p, x) \in {P \times (X \setmin S)}.
\end{align}
Note that if $f$ is supported on $S$ then it is also supported on any $R \sup S$.

Let $\cU = \{U_\alpha\}$ be an open cover of $X$.
A $k$-parameter family of diffeomorphisms $f: P \times X \to X$ is
{\it adapted to $\cU$} if there is a factorization
\eq{
    P = P_1 \times \cdots \times P_m
}
(for some $m \le k$)
and families of diffeomorphisms
\eq{
    f_i :  P_i \times X \to X
}
such that
\begin{itemize}
\item each $f_i$ is supported on some connected $V_i \sub X$;
\item the sets $V_i$ are mutually disjoint;
\item each $V_i$ is the union of at most $k_i$ of the $U_\alpha$'s,
where $k_i = \dim(P_i)$; and
\item $f(p, \cdot) = g \circ f_1(p_1, \cdot) \circ \cdots \circ f_m(p_m, \cdot)$
for all $p = (p_1, \ldots, p_m)$, for some fixed $g \in \Diff(X)$.
\end{itemize}
A chain $x \in CD_k(X)$ is (by definition) adapted to $\cU$ if it is the sum
of singular cells, each of which is adapted to $\cU$.

(Actually, in this section we will only need families of diffeomorphisms to be 
{\it weakly adapted} to $\cU$, meaning that the support of $f$ is contained in the union
of at most $k$ of the $U_\alpha$'s.)

\begin{lemma}  \label{extension_lemma}
Let $x \in CD_k(X)$ be a singular chain such that $\bd x$ is adapted to $\cU$.
Then $x$ is homotopic (rel boundary) to some $x' \in CD_k(X)$ which is adapted to $\cU$.
Furthermore, one can choose the homotopy so that its support is equal to the support of $x$.
\end{lemma}

The proof will be given in Section \ref{sec:localising}.

\medskip

Before diving into the details, we outline our strategy for the proof of Proposition \ref{CDprop}.

Let $p$ be a singular cell in $CD_k(X)$ and $b$ be a blob diagram in $\bc_*(X)$.
Suppose that there exists $V \sub X$ such that
\begin{enumerate}
\item $V$ is homeomorphic to a disjoint union of balls, and
\item $\supp(p) \cup \supp(b) \sub V$.
\end{enumerate}
Let $W = X \setmin V$, and let $V' = p(V)$ and $W' = p(W)$.
We then have a factorization 
\[
	p = \gl(q, r),
\]
where $q \in CD_k(V, V')$ and $r' \in CD_0(W, W')$.
According to the commutative diagram of the proposition, we must have
\[
	e_X(p) = e_X(\gl(q, r)) = gl(e_{VV'}(q), e_{WW'}(r)) .
\]
\nn{need to add blob parts to above}
Since $r$ is a plain, 0-parameter family of diffeomorphisms, 
\medskip

\nn{to be continued....}


%\nn{say something about associativity here}





