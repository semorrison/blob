%!TEX root = ../blob1.tex

Let $CD_*(X, Y)$ denote $C_*(\Diff(X \to Y))$, the singular chain complex of
the space of diffeomorphisms
\nn{or homeomorphisms}
between the $n$-manifolds $X$ and $Y$ (extending a fixed diffeomorphism $\bd X \to \bd Y$).
For convenience, we will permit the singular cells generating $CD_*(X, Y)$ to be more general
than simplices --- they can be based on any linear polyhedron.
\nn{be more restrictive here?  does more need to be said?}
We also will use the abbreviated notation $CD_*(X) \deq CD_*(X, X)$.

\begin{prop}  \label{CDprop}
For $n$-manifolds $X$ and $Y$ there is a chain map
\eq{
    e_{XY} : CD_*(X, Y) \otimes \bc_*(X) \to \bc_*(Y) .
}
On $CD_0(X, Y) \otimes \bc_*(X)$ it agrees with the obvious action of $\Diff(X, Y)$ on $\bc_*(X)$
(Proposition (\ref{diff0prop})).
For any splittings $X = X_1 \cup X_2$ and $Y = Y_1 \cup Y_2$, 
the following diagram commutes up to homotopy
\eq{ \xymatrix{
     CD_*(X, Y) \otimes \bc_*(X) \ar[r]^{e_{XY}}    & \bc_*(Y) \\
     CD_*(X_1, Y_1) \otimes CD_*(X_2, Y_2) \otimes \bc_*(X_1) \otimes \bc_*(X_2)
        \ar@/_4ex/[r]_{e_{X_1Y_1} \otimes e_{X_2Y_2}}  \ar[u]^{\gl \otimes \gl}  &
            \bc_*(Y_1) \otimes \bc_*(Y_2) \ar[u]_{\gl}
} }
Any other map satisfying the above two properties is homotopic to $e_X$.
\end{prop}

\nn{need to rewrite for self-gluing instead of gluing two pieces together}

\nn{Should say something stronger about uniqueness.
Something like: there is
a contractible subcomplex of the complex of chain maps
$CD_*(X) \otimes \bc_*(X) \to \bc_*(X)$ (0-cells are the maps, 1-cells are homotopies, etc.),
and all choices in the construction lie in the 0-cells of this
contractible subcomplex.
Or maybe better to say any two choices are homotopic, and
any two homotopies and second order homotopic, and so on.}

\nn{Also need to say something about associativity.
Put it in the above prop or make it a separate prop?
I lean toward the latter.}
\medskip

The proof will occupy the remainder of this section.
\nn{unless we put associativity prop at end}

Without loss of generality, we will assume $X = Y$.

\medskip

Let $f: P \times X \to X$ be a family of diffeomorphisms and $S \sub X$.
We say that {\it $f$ is supported on $S$} if $f(p, x) = f(q, x)$ for all
$x \notin S$ and $p, q \in P$. Equivalently, $f$ is supported on $S$ if there is a family of diffeomorphisms $f' : P \times S \to S$ and a `background'
diffeomorphism $f_0 : X \to X$ so that
\begin{align}
	f(p,s) & = f_0(f'(p,s)) \;\;\;\; \mbox{for}\; (p, s) \in P\times S \\
\intertext{and}
	f(p,x) & = f_0(x) \;\;\;\; \mbox{for}\; (p, x) \in {P \times (X \setmin S)}.
\end{align}
Note that if $f$ is supported on $S$ then it is also supported on any $R \sup S$.

Let $\cU = \{U_\alpha\}$ be an open cover of $X$.
A $k$-parameter family of diffeomorphisms $f: P \times X \to X$ is
{\it adapted to $\cU$} if there is a factorization
\eq{
    P = P_1 \times \cdots \times P_m
}
(for some $m \le k$)
and families of diffeomorphisms
\eq{
    f_i :  P_i \times X \to X
}
such that
\begin{itemize}
\item each $f_i$ is supported on some connected $V_i \sub X$;
\item the sets $V_i$ are mutually disjoint;
\item each $V_i$ is the union of at most $k_i$ of the $U_\alpha$'s,
where $k_i = \dim(P_i)$; and
\item $f(p, \cdot) = g \circ f_1(p_1, \cdot) \circ \cdots \circ f_m(p_m, \cdot)$
for all $p = (p_1, \ldots, p_m)$, for some fixed $g \in \Diff(X)$.
\end{itemize}
A chain $x \in CD_k(X)$ is (by definition) adapted to $\cU$ if it is the sum
of singular cells, each of which is adapted to $\cU$.

(Actually, in this section we will only need families of diffeomorphisms to be 
{\it weakly adapted} to $\cU$, meaning that the support of $f$ is contained in the union
of at most $k$ of the $U_\alpha$'s.)

\begin{lemma}  \label{extension_lemma}
Let $x \in CD_k(X)$ be a singular chain such that $\bd x$ is adapted to $\cU$.
Then $x$ is homotopic (rel boundary) to some $x' \in CD_k(X)$ which is adapted to $\cU$.
Furthermore, one can choose the homotopy so that its support is equal to the support of $x$.
\end{lemma}

The proof will be given in Section \ref{sec:localising}.

\medskip

Before diving into the details, we outline our strategy for the proof of Proposition \ref{CDprop}.

%Suppose for the moment that evaluation maps with the advertised properties exist.
Let $p$ be a singular cell in $CD_k(X)$ and $b$ be a blob diagram in $\bc_*(X)$.
Suppose that there exists $V \sub X$ such that
\begin{enumerate}
\item $V$ is homeomorphic to a disjoint union of balls, and
\item $\supp(p) \cup \supp(b) \sub V$.
\end{enumerate}
Let $W = X \setmin V$, and let $V' = p(V)$ and $W' = p(W)$.
We then have a factorization 
\[
	p = \gl(q, r),
\]
where $q \in CD_k(V, V')$ and $r' \in CD_0(W, W')$.
We can also factorize $b = \gl(b_V, b_W)$, where $b_V\in \bc_*(V)$ and $b_W\in\bc_0(W)$.
According to the commutative diagram of the proposition, we must have
\[
	e_X(p\otimes b) = e_X(\gl(q\otimes b_V, r\otimes b_W)) = 
				gl(e_{VV'}(q\otimes b_V), e_{WW'}(r\otimes b_W)) .
\]
Since $r$ is a plain, 0-parameter family of diffeomorphisms, we must have
\[
	e_{WW'}(r\otimes b_W) = r(b_W),
\]
where $r(b_W)$ denotes the obvious action of diffeomorphisms on blob diagrams (in
this case a 0-blob diagram).
Since $V'$ is a disjoint union of balls, $\bc_*(V')$ is acyclic in degrees $>0$ 
(by \ref{disjunion} and \ref{bcontract}).
Assuming inductively that we have already defined $e_{VV'}(\bd(q\otimes b_V))$,
there is, up to (iterated) homotopy, a unique choice for $e_{VV'}(q\otimes b_V)$
such that 
\[
	\bd(e_{VV'}(q\otimes b_V)) = e_{VV'}(\bd(q\otimes b_V)) .
\]

Thus the conditions of the proposition determine (up to homotopy) the evaluation
map for generators $p\otimes b$ such that $\supp(p) \cup \supp(b)$ is contained in a disjoint
union of balls.
On the other hand, Lemma \ref{extension_lemma} allows us to homotope 
\nn{is this commonly used as a verb?} arbitrary generators to sums of generators with this property.
\nn{should give a name to this property; also forward reference}
This (roughly) establishes the uniqueness part of the proposition.
To show existence, we must show that the various choices involved in constructing
evaluation maps in this way affect the final answer only by a homotopy.

\nn{maybe put a little more into the outline before diving into the details.}

\nn{Note: At the moment this section is very inconsistent with respect to PL versus smooth,
homeomorphism versus diffeomorphism, etc.
We expect that everything is true in the PL category, but at the moment our proof
avails itself to smooth techniques.
Furthermore, it traditional in the literature to speak of $C_*(\Diff(X))$
rather than $C_*(\Homeo(X))$.}

\medskip

Now for the details.

Notation: Let $|b| = \supp(b)$, $|p| = \supp(p)$.

Choose a metric on $X$.
Choose a monotone decreasing sequence of positive real numbers $\ep_i$ converging to zero
(e.g.\ $\ep_i = 2^{-i}$).
Choose another sequence of positive real numbers $\delta_i$ such that $\delta_i/\ep_i$
converges monotonically to zero (e.g.\ $\delta_i = \ep_i^2$).
Given a generator $p\otimes b$ of $CD_*(X)\otimes \bc_*(X)$ and non-negative integers $i$ and $k$
define
\[
	N_{i,k}(p\ot b) \deq \Nbd_{k\ep_i}(|b|) \cup \Nbd_{k\delta_i}(|p|).
\]
In other words, we use the metric to choose nested neighborhoods of $|b|\cup |p|$ (parameterized
by $k$), with $\ep_i$ controlling the size of the buffer around $|b|$ and $\delta_i$ controlling
the size of the buffer around $|p|$.

Next we define subcomplexes $G_*^{i,m} \sub CD_*(X)\otimes \bc_*(X)$.
Let $p\ot b$ be a generator of $CD_*(X)\otimes \bc_*(X)$ and let $k = \deg(p\ot b)
= \deg(p) + \deg(b)$.
$p\ot b$ is (by definition) in $G_*^{i,m}$ if either (a) $\deg(p) = 0$ or (b)
there exist codimension-zero submanifolds $V_1,\ldots,V_m \sub X$ such that each $V_j$
is homeomorphic to a disjoint union of balls and
\[
	N_{i,k}(p\ot b) \subeq V_1 \subeq N_{i,k+1}(p\ot b)
			\subeq V_2 \subeq \cdots \subeq V_m \subeq N_{i,k+m}(p\ot b) .
\]
Further, we require (inductively) that $\bd(p\ot b) \in G_*^{i,m}$.
We also require that $b$ is splitable (transverse) along the boundary of each $V_l$.

Note that $G_*^{i,m+1} \subeq G_*^{i,m}$.

As sketched above and explained in detail below, 
$G_*^{i,m}$ is a subcomplex where it is easy to define
the evaluation map.
The parameter $m$ controls the number of iterated homotopies we are able to construct.
The larger $i$ is (i.e.\ the smaller $\ep_i$ is), the better $G_*^{i,m}$ approximates all of
$CD_*(X)\ot \bc_*(X)$.

Next we define a chain map (dependent on some choices) $e: G_*^{i,m} \to \bc_*(X)$.
Let $p\ot b \in G_*^{i,m}$.
If $\deg(p) = 0$, define
\[
	e(p\ot b) = p(b) ,
\]
where $p(b)$ denotes the obvious action of the diffeomorphism(s) $p$ on the blob diagram $b$.
For general $p\ot b$ ($\deg(p) \ge 1$) assume inductively that we have already defined
$e(p'\ot b')$ when $\deg(p') + \deg(b') < k = \deg(p) + \deg(b)$.
Choose $V_1$ as above so that 
\[
	N_{i,k}(p\ot b) \subeq V_1 \subeq N_{i,k+1}(p\ot b) .
\]
Let $\bd(p\ot b) = \sum_j p_j\ot b_j$, and let $V_1^j$ be the choice of neighborhood
of $|p_j|\cup |b_j|$ made at the preceding stage of the induction.
For all $j$, 
\[
	V_1^j \subeq N_{i,(k-1)+1}(p_j\ot b_j) \subeq N_{i,k}(p\ot b) \subeq V_1 .
\]
(The second inclusion uses the facts that $|p_j| \subeq |p|$ and $|b_j| \subeq |b|$.)
We therefore have splittings
\[
	p = p'\bullet p'' , \;\; b = b'\bullet b'' , \;\; e(\bd(p\ot b)) = f'\bullet f'' ,
\]
where $p' \in CD_*(V_1)$, $p'' \in CD_*(X\setmin V_1)$, 
$b' \in \bc_*(V_1)$, $b'' \in \bc_*(X\setmin V_1)$, 
$e' \in \bc_*(p(V_1))$, and $e'' \in \bc_*(p(X\setmin V_1))$.
(Note that since the family of diffeomorphisms $p$ is constant (independent of parameters)
near $\bd V_1)$, the expressions $p(V_1) \sub X$ and $p(X\setmin V_1) \sub X$ are
unambiguous.)
We also have that $\deg(b'') = 0 = \deg(p'')$.
Choose $x' \in \bc_*(p(V_1))$ such that $\bd x' = f'$.
This is possible by \nn{...}.
Finally, define
\[
	e(p\ot b) \deq x' \bullet p''(b'') .
\]


\medskip

\nn{to be continued....}


%\nn{say something about associativity here}





