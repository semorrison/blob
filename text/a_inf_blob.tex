%!TEX root = ../blob1.tex

\section{The blob complex for \texorpdfstring{$A_\infty$}{A-infinity} \texorpdfstring{$n$}{n}-categories}
\label{sec:ainfblob}
Given an $A_\infty$ $n$-category $\cC$ and an $n$-manifold $M$, we make the following 
anticlimactically tautological definition of the blob
complex.
\begin{defn}
The blob complex
 $\bc_*(M;\cC)$ of an $n$-manifold $n$ with coefficients in an $A_\infty$ $n$-category is the homotopy colimit $\cl{\cC}(M)$ of \S\ref{ss:ncat_fields}.
\end{defn}

We will show below 
in Corollary \ref{cor:new-old}
that when $\cC$ is obtained from a system of fields $\cE$ 
as the blob complex of an $n$-ball (see Example \ref{ex:blob-complexes-of-balls}), 
$\cl{\cC}(M)$ is homotopy equivalent to
our original definition of the blob complex $\bc_*(M;\cE)$.

%\medskip

%An important technical tool in the proofs of this section is provided by the idea of ``small blobs".
%Fix $\cU$, an open cover of $M$.
%Define the ``small blob complex" $\bc^{\cU}_*(M)$ to be the subcomplex of $\bc_*(M)$ 
%of all blob diagrams in which every blob is contained in some open set of $\cU$, 
%and moreover each field labeling a region cut out by the blobs is splittable 
%into fields on smaller regions, each of which is contained in some open set of $\cU$.
%
%\begin{thm}[Small blobs] \label{thm:small-blobs}
%The inclusion $i: \bc^{\cU}_*(M) \into \bc_*(M)$ is a homotopy equivalence.
%\end{thm}
%The proof appears in \S \ref{appendix:small-blobs}.

\subsection{A product formula}
\label{ss:product-formula}


Given an $n$-dimensional system of fields $\cE$ and a $n{-}k$-manifold $F$, recall from 
Example \ref{ex:blob-complexes-of-balls} that there is an  $A_\infty$ $k$-category $\cC_F$ 
defined by $\cC_F(X) = \cE(X\times F)$ if $\dim(X) < k$ and
$\cC_F(X) = \bc_*(X\times F;\cE)$ if $\dim(X) = k$.


\begin{thm} \label{thm:product}
Let $Y$ be a $k$-manifold which admits a ball decomposition
(e.g.\ any triangulable manifold).
Then there is a homotopy equivalence between ``old-fashioned" (blob diagrams) 
and ``new-fangled" (hocolimit) blob complexes
\[
	\cB_*(Y \times F) \htpy \cl{\cC_F}(Y) .
\]\end{thm}

\begin{proof}
We will use the concrete description of the homotopy colimit from \S\ref{ss:ncat_fields}.

First we define a map 
\[
	\psi: \cl{\cC_F}(Y) \to \bc_*(Y\times F;\cE) .
\]
On 0-simplices of the hocolimit 
we just glue together the various blob diagrams on $X_i\times F$
(where $X_i$ is a component of a permissible decomposition of $Y$) to get a blob diagram on
$Y\times F$.
For simplices of dimension 1 and higher we define the map to be zero.
It is easy to check that this is a chain map.

In the other direction, we will define (in the next few paragraphs) 
a subcomplex $G_*\sub \bc_*(Y\times F;\cE)$ and a map
\[
	\phi: G_* \to \cl{\cC_F}(Y) .
\]

Given a decomposition $K$ of $Y$ into $k$-balls $X_i$, let $K\times F$ denote the corresponding
decomposition of $Y\times F$ into the pieces $X_i\times F$.

Let $G_*\sub \bc_*(Y\times F;\cE)$ be the subcomplex generated by blob diagrams $a$ such that there
exists a decomposition $K$ of $Y$ such that $a$ splits along $K\times F$.
It follows from Lemma \ref{thm:small-blobs} that $\bc_*(Y\times F; \cE)$ 
is homotopic to a subcomplex of $G_*$.
(If the blobs of $a$ are small with respect to a sufficiently fine cover then their
projections to $Y$ are contained in some disjoint union of balls.)
Note that the image of $\psi$ is equal to $G_*$.

We will define $\phi: G_* \to \cl{\cC_F}(Y)$ using the method of acyclic models.
Let $a$ be a generator of $G_*$.
Let $D(a)$ denote the subcomplex of $\cl{\cC_F}(Y)$ generated by all $(b, \ol{K})$
where $b$ is a generator appearing
in an iterated boundary of $a$ (this includes $a$ itself)
and $b$ splits along $K_0\times F$.
(Recall that $\ol{K} = (K_0,\ldots,K_l)$ denotes a chain of decompositions;
see \S\ref{ss:ncat_fields}.)
By $(b, \ol{K})$ we really mean $(b^\sharp, \ol{K})$, where $b^\sharp$ is 
$b$ split according to $K_0\times F$.
To simplify notation we will just write plain $b$ instead of $b^\sharp$.
Roughly speaking, $D(a)$ consists of 0-simplices which glue up to give
$a$ (or one of its iterated boundaries), 1-simplices which connect all the 0-simplices, 
2-simplices which kill the homology created by the 
1-simplices, and so on.
More formally,
 
\begin{lemma} \label{lem:d-a-acyclic}
$D(a)$ is acyclic in positive degrees.
\end{lemma}

\begin{proof}
Let $P(a)$ denote the finite cone-product polyhedron composed of $a$ and its iterated boundaries.
(See Remark \ref{blobsset-remark}.)
We can think of $D(a)$ as a cell complex equipped with an obvious
map $p: D(a) \to P(a)$ which forgets the second factor.
For each cell $b$ of $P(a)$, let $I(b) = p\inv(b)$.
It suffices to show that each $I(b)$ is acyclic and more generally that
each intersection $I(b)\cap I(b')$ is acyclic.

If $I(b)\cap I(b')$ is nonempty then then as a cell complex it is isomorphic to
$(b\cap b') \times E(b, b')$, where $E(b, b')$ consists of those simplices
$\ol{K} = (K_0,\ldots,K_l)$ such that both $b$ and $b'$ split along $K_0\times F$.
(Here we are thinking of $b$ and $b'$ as both blob diagrams and also faces of $P(a)$.)
So it suffices to show that $E(b, b')$ is acyclic.

Let $K$ and $K'$ be two decompositions of $Y$ (i.e.\ 0-simplices) in $E(b, b')$.
We want to find 1-simplices which connect $K$ and $K'$.
We might hope that $K$ and $K'$ have a common refinement, but this is not necessarily
the case.
(Consider the $x$-axis and the graph of $y = e^{-1/x^2} \sin(1/x)$ in $\r^2$.)
However, we {\it can} find another decomposition $L$ such that $L$ shares common
refinements with both $K$ and $K'$. (For instance, in the example above, $L$ can be the graph of $y=x^2-1$.)
This follows from Axiom \ref{axiom:splittings}, which in turn follows from the
splitting axiom for the system of fields $\cE$.
Let $KL$ and $K'L$ denote these two refinements.
Then 1-simplices associated to the four anti-refinements
$KL\to K$, $KL\to L$, $K'L\to L$ and $K'L\to K'$
give the desired chain connecting $(a, K)$ and $(a, K')$
(see Figure \ref{zzz4}).
(In the language of Lemma \ref{lemma:vcones}, this is $\vcone(K \du K')$.)

\begin{figure}[t] \centering
\begin{tikzpicture}
\foreach \x/\label in {-3/K, 0/L, 3/K'} {
	\node(\label) at (\x,0) {$\label$};
}
\foreach \x/\la/\lb in {-1.5/K/L, 1.5/K'/L} {
	\node(\la \lb) at (\x,-1.5) {$\la \lb$};
	\draw[->] (\la \lb) -- (\la);
	\draw[->] (\la \lb) -- (\lb); 
}
\end{tikzpicture}
\caption{Connecting $K$ and $K'$ via $L$}
\label{zzz4}
\end{figure}

Consider next a 1-cycle in $E(b, b')$, such as one arising from
a different choice of decomposition $L'$ in place of $L$ above.
%We want to find 2-simplices which fill in this cycle.
By Lemma \ref{lemma:vcones} we can fill in this 1-cycle with 2-simplices.
Choose a decomposition $M$ which has common refinements with each of 
$K$, $KL$, $L$, $K'L$, $K'$, $K'L'$, $L'$ and $KL'$.
(We also require that $KLM$ antirefines to $KM$, etc.)
Then we have 2-simplices, as shown in Figure \ref{zzz5}, which do the trick.
(Each small triangle in Figure \ref{zzz5} can be filled with a 2-simplex.)

\begin{figure}[t] \centering
\begin{tikzpicture}
\node(M) at (0,0) {$M$};
\foreach \angle/\label in {0/K', 45/K'L, 90/L, 135/KL, 180/K, 225/KL', 270/L', 315/K'L'} {
	\node(\label) at (\angle:4) {$\label$};
}
\foreach \label in {K', L, K, L'} {
	\node(\label M) at ($(M)!0.6!(\label)$) {$\label M$};
	\draw[->] (\label M)--(M);
	\draw[->] (\label M)--(\label);
}
\foreach \k in {K, K'} {
	\foreach \l in {L, L'} {
		\node(\k \l M) at (intersection cs: first line={(\k M)--(\l)}, second line={(\l M)--(\k)}) {$\k \l M$};
		\draw[->] (\k \l M)--(M);
		\draw[->] (\k \l M)--(\k \l );
		\draw[->] (\k \l M)--(\k M);
		\draw[->] (\k \l M)--(\l);
		\draw[->] (\k \l M)--(\l M);
		\draw[->] (\k \l M)--(\k);
	}
}
\draw[->] (K'L') to[bend right=10] (K');
\draw[->] (K'L') to[bend left=10] (L');
\draw[->] (KL') to[bend left=10] (K);
\draw[->] (KL') to[bend right=10] (L');
\draw[->] (K'L) to[bend left=10] (K');
\draw[->] (K'L) to[bend right=10] (L);
\draw[->] (KL) to[bend right=10] (K);
\draw[->] (KL) to[bend left=10] (L);
\end{tikzpicture}
\caption{Filling in $K$-$KL$-$L$-$K'L$-$K'$-$K'L'$-$L'$-$KL'$-$K$}
\label{zzz5}
\end{figure}

Continuing in this way we see that $D(a)$ is acyclic.
By Lemma \ref{lemma:vcones} we can fill in any cycle with a V-Cone.
\end{proof}

We are now in a position to apply the method of acyclic models to get a map
$\phi:G_* \to \cl{\cC_F}(Y)$.
We may assume that $\phi(a)$ has the form $(a, K) + r$, where $(a, K)$ is a 0-simplex
and $r$ is a sum of simplices of dimension 1 or higher.

We now show that $\phi\circ\psi$ and $\psi\circ\phi$ are homotopic to the identity.

First, $\psi\circ\phi$ is the identity on the nose:
\[
	\psi(\phi(a)) = \psi((a,K)) + \psi(r) = a + 0.
\]
Roughly speaking, $(a, K)$ is just $a$ chopped up into little pieces, and 
$\psi$ glues those pieces back together, yielding $a$.
We have $\psi(r) = 0$ since $\psi$ is zero on $(\ge 1)$-simplices.
 
Second, $\phi\circ\psi$ is the identity up to homotopy by another argument based on the method of acyclic models.
To each generator $(b, \ol{K})$ of $\cl{\cC_F}(Y)$ we associate the acyclic subcomplex $D(b)$ defined above.
Both the identity map and $\phi\circ\psi$ are compatible with this
collection of acyclic subcomplexes, so by the usual method of acyclic models argument these two maps
are homotopic.

This concludes the proof of Theorem \ref{thm:product}.
\end{proof}

%\nn{need to prove a version where $E$ above has dimension $m<n$; result is an $n{-}m$-category}

If $Y$ has dimension $k-m$, then we have an $m$-category $\cC_{Y\times F}$ whose value at
a $j$-ball $X$ is either $\cE(X\times Y\times F)$ (if $j<m$) or $\bc_*(X\times Y\times F)$
(if $j=m$).
(See Example \ref{ex:blob-complexes-of-balls}.)
Similarly we have an $m$-category whose value at $X$ is $\cl{\cC_F}(X\times Y)$.
These two categories are equivalent, but since we do not define functors between
disk-like $n$-categories in this paper we are unable to say precisely
what ``equivalent" means in this context.
We hope to include this stronger result in a future paper.

\medskip

Taking $F$ in Theorem \ref{thm:product} to be a point, we obtain the following corollary.

\begin{cor}
\label{cor:new-old}
Let $\cE$ be a system of fields (with local relations) and let $\cC_\cE$ be the $A_\infty$
$n$-category obtained from $\cE$ by taking the blob complex of balls.
Then for all $n$-manifolds $Y$ the old-fashioned and new-fangled blob complexes are
homotopy equivalent:
\[
	\bc^\cE_*(Y) \htpy \cl{\cC_\cE}(Y) .
\]
\end{cor}

\medskip

Theorem \ref{thm:product} extends to the case of general fiber bundles
\[
	F \to E \to Y ,
\]
and indeed even to the case of general maps
\[
	M\to Y .
\]
We outline two approaches to these generalizations.
The first is somewhat tautological, while the second is more amenable to
calculation.

We can generalize the definition of a $k$-category by replacing the categories
of $j$-balls ($j\le k$) with categories of $j$-balls $D$ equipped with a map $p:D\to Y$
(c.f. \cite{MR2079378}).
Call this a {\it $k$-category over $Y$}.
A fiber bundle $F\to E\to Y$ gives an example of a $k$-category over $Y$:
assign to $p:D\to Y$ the blob complex $\bc_*(p^*(E))$, when $\dim(D) = k$,
or the fields $\cE(p^*(E))$, when $\dim(D) < k$.
(Here $p^*(E)$ denotes the pull-back bundle over $D$.)
Let $\cF_E$ denote this $k$-category over $Y$.
We can adapt the homotopy colimit construction (based decompositions of $Y$ into balls) to
get a chain complex $\cl{\cF_E}(Y)$.

\begin{thm}
Let $F \to E \to Y$ be a fiber bundle and let $\cF_E$ be the $k$-category over $Y$ defined above.
Then
\[
	\bc_*(E) \simeq \cl{\cF_E}(Y) .
\]
\qed
\end{thm}

\begin{proof}
The proof is nearly identical to the proof of Theorem \ref{thm:product}, so we will only give a sketch which 
emphasizes the few minor changes that need to be made.

As before, we define a map
\[
	\psi: \cl{\cF_E}(Y) \to \bc_*(E) .
\]
0-simplices of the homotopy colimit $\cl{\cF_E}(Y)$ are glued up to give an element of $\bc_*(E)$.
Simplices of positive degree are sent to zero.

Let $G_* \sub \bc_*(E)$ be the image of $\psi$.
By Lemma \ref{thm:small-blobs}, $\bc_*(Y\times F; \cE)$ 
is homotopic to a subcomplex of $G_*$.
We will define a homotopy inverse of $\psi$ on $G_*$, using acyclic models.
To each generator $a$ of $G_*$ we assign an acyclic subcomplex $D(a) \sub \cl{\cF_E}(Y)$ which consists of
0-simplices which map via $\psi$ to $a$, plus higher simplices (as described in the proof of Theorem \ref{thm:product})
which insure that $D(a)$ is acyclic.
\end{proof}

We can generalize this result still further by noting that it is not really necessary
for the definition of $\cF_E$ that $E\to Y$ be a fiber bundle.
Let $M\to Y$ be a map, with $\dim(M) = n$ and $\dim(Y) = k$.
Call a map $D^j\to Y$ ``good" with respect to $M$ if the fibered product
$D\widetilde{\times} M$ is a manifold of dimension $n-k+j$ with a collar structure along the boundary of $D$.
(If $D\to Y$ is an embedding then $D\widetilde{\times} M$ is just the part of $M$
lying above $D$.)
We can define a $k$-category $\cF_M$ based on maps of balls into $Y$ which are good with respect to $M$.
We can again adapt the homotopy colimit construction to
get a chain complex $\cl{\cF_M}(Y)$.
The proof of Theorem \ref{thm:product} again goes through essentially unchanged 
to show that
\begin{thm}
Let $M \to Y$ be a map of manifolds and let $\cF_M$ be the $k$-category over $Y$ defined above.
Then
\[
	\bc_*(M) \simeq \cl{\cF_M}(Y) .
\]
\qed
\end{thm}


\medskip

In the second approach we use a decorated colimit (as in \S \ref{ssec:spherecat}) 
and various sphere modules based on $F \to E \to Y$
or $M\to Y$, instead of an undecorated colimit with fancier $k$-categories over $Y$.
Information about the specific map to $Y$ has been taken out of the categories
and put into sphere modules and decorations.

Let $F \to E \to Y$ be a fiber bundle as above.
Choose a decomposition $Y = \cup X_i$
such that the restriction of $E$ to $X_i$ is homeomorphic to a product $F\times X_i$,
and choose trivializations of these products as well.

Let $\cF$ be the $k$-category associated to $F$.
To each codimension-1 face $X_i\cap X_j$ we have a bimodule ($S^0$-module) for $\cF$.
More generally, to each codimension-$m$ face we have an $S^{m-1}$-module for a $(k{-}m{+}1)$-category
associated to the (decorated) link of that face.
We can decorate the strata of the decomposition of $Y$ with these sphere modules and form a 
colimit as in \S \ref{ssec:spherecat}.
This colimit computes $\bc_*(E)$.

There is a similar construction for general maps $M\to Y$.

%Note that Theorem \ref{thm:gluing} can be viewed as a special case of this one.
%Let $X_1$ and $X_2$ be $n$-manifolds
%\nn{...}




\subsection{A gluing theorem}
\label{sec:gluing}

Next we prove a gluing theorem. Throughout this section fix a particular $n$-dimensional system of fields $\cE$ and local relations. Each blob complex below is  with respect to this $\cE$.
Let $X$ be a closed $k$-manifold with a splitting $X = X'_1\cup_Y X'_2$.
We will need an explicit collar on $Y$, so rewrite this as
$X = X_1\cup (Y\times J) \cup X_2$.
Given this data we have:
\begin{itemize}
\item An $A_\infty$ $n{-}k$-category $\bc(X)$, which assigns to an $m$-ball
$D$ fields on $D\times X$ (for $m+k < n$) or the blob complex $\bc_*(D\times X; c)$
(for $m+k = n$).
(See Example \ref{ex:blob-complexes-of-balls}.)
%\nn{need to explain $c$}.
\item An $A_\infty$ $n{-}k{+}1$-category $\bc(Y)$, defined similarly.
\item Two $\bc(Y)$ modules $\bc(X_1)$ and $\bc(X_2)$, which assign to a marked
$m$-ball $(D, H)$ either fields on $(D\times Y) \cup (H\times X_i)$ (if $m+k < n$)
or the blob complex $\bc_*((D\times Y) \cup (H\times X_i))$ (if $m+k = n$).
(See Example \ref{bc-module-example}.)
\item The tensor product $\bc(X_1) \otimes_{\bc(Y), J} \bc(X_2)$, which is
an $A_\infty$ $n{-}k$-category.
(See \S \ref{moddecss}.)
\end{itemize}

It is the case that the $n{-}k$-categories $\bc(X)$ and $\bc(X_1) \otimes_{\bc(Y), J} \bc(X_2)$
are equivalent for all $k$, but since we do not develop a definition of functor between $n$-categories
in this paper, we cannot state this precisely.
(It will appear in a future paper.)
So we content ourselves with

\begin{thm}
\label{thm:gluing}
Suppose $X$ is an $n$-manifold, and $X = X_1\cup (Y\times J) \cup X_2$ (i.e. just as with  $k=n$ above). Then $\bc(X)$ is homotopy equivalent to the $A_\infty$ tensor product $\bc(X_1) \otimes_{\bc(Y), J} \bc(X_2)$.
\end{thm}

\begin{proof}
%We will assume $k=n$; the other cases are similar.
The proof is similar to that of Theorem \ref{thm:product}.
We give a short sketch with emphasis on the differences from 
the proof of Theorem \ref{thm:product}.

Let $\cT$ denote the chain complex $\bc(X_1) \otimes_{\bc(Y), J} \bc(X_2)$.
Recall that this is a homotopy colimit based on decompositions of the interval $J$.

We define a map $\psi:\cT\to \bc_*(X)$.
On 0-simplices it is given
by gluing the pieces together to get a blob diagram on $X$.
On simplices of dimension 1 and greater $\psi$ is zero.

The image of $\psi$ is the subcomplex $G_*\sub \bc(X)$ generated by blob diagrams which split
over some decomposition of $J$.
It follows from Lemma \ref{thm:small-blobs} that $\bc_*(X)$ is homotopic to 
a subcomplex of $G_*$. 

Next we define a map $\phi:G_*\to \cT$ using the method of acyclic models.
As in the proof of Theorem \ref{thm:product}, we assign to a generator $a$ of $G_*$
an acyclic subcomplex which is (roughly) $\psi\inv(a)$.
The proof of acyclicity is easier in this case since any pair of decompositions of $J$ have
a common refinement.

The proof that these two maps are inverse to each other is the same as in
Theorem \ref{thm:product}.
\end{proof}

\medskip

\subsection{Reconstructing mapping spaces}
\label{sec:map-recon}

The next theorem shows how to reconstruct a mapping space from local data.
Let $T$ be a topological space, let $M$ be an $n$-manifold, 
and recall the $A_\infty$ $n$-category $\pi^\infty_{\leq n}(T)$ 
of Example \ref{ex:chains-of-maps-to-a-space}.
Think of $\pi^\infty_{\leq n}(T)$ as encoding everything you would ever
want to know about spaces of maps of $k$-balls into $T$ ($k\le n$).
To simplify notation, let $\cT = \pi^\infty_{\leq n}(T)$.

\begin{thm}
\label{thm:map-recon}
The blob complex for $M$ with coefficients in the fundamental $A_\infty$ $n$-category for $T$ 
is quasi-isomorphic to singular chains on maps from $M$ to $T$.
$$\cB^\cT(M) \simeq C_*(\Maps(M\to T)).$$
\end{thm}
\begin{rem}
Lurie has shown in \cite[Theorem 3.8.6]{0911.0018} that the topological chiral homology 
of an $n$-manifold $M$ with coefficients in a certain $E_n$ algebra constructed from $T$ recovers 
the same space of singular chains on maps from $M$ to $T$, with the additional hypothesis that $T$ is $n{-}1$-connected.
This extra hypothesis is not surprising, in view of the idea described in Example \ref{ex:e-n-alg} 
that an $E_n$ algebra is roughly equivalent data to an $A_\infty$ $n$-category which 
is trivial at levels 0 through $n-1$.
Ricardo Andrade also told us about a similar result.

Specializing still further, Theorem \ref{thm:map-recon} is related to the classical result that for connected spaces $T$
we have $HH_*(C_*(\Omega T)) \cong H_*(LT)$, that is, the Hochschild homology of based loops in $T$ is isomorphic
to the homology of the free loop space of $T$ (see \cite{MR793184} and \cite{MR842427}).
Theorem \ref{thm:map-recon} says that for any space $T$ (connected or not) we have
$\bc_*(S^1; C_*(\pi^\infty_{\le 1}(T))) \simeq C_*(LT)$.
Here $C_*(\pi^\infty_{\le 1}(T))$ denotes the singular chain version of the fundamental infinity-groupoid of $T$, 
whose objects are points in $T$ and morphism chain complexes are $C_*(\paths(t_1 \to t_2))$ for $t_1, t_2 \in T$.
If $T$ is connected then the $A_\infty$ 1-category $C_*(\pi^\infty_{\le 1}(T))$ is Morita equivalent to the
$A_\infty$ algebra $C_*(\Omega T)$; 
the bimodule for the equivalence is the singular chains of the space of paths which start at the base point of $T$.
Theorem \ref{thm:hochschild} holds for $A_\infty$ 1-categories (though we do not prove that in this paper),
which then implies that
\[
	Hoch_*(C_*(\Omega T)) \simeq Hoch_*(C_*(\pi^\infty_{\le 1}(T)))
			\simeq \bc_*(S^1; C_*(\pi^\infty_{\le 1}(T))) \simeq C_*(LT) .
\]
\end{rem}

\begin{proof}[Proof of Theorem \ref{thm:map-recon}]
The proof is again similar to that of Theorem \ref{thm:product}.

We begin by constructing a chain map $\psi: \cB^\cT(M) \to C_*(\Maps(M\to T))$.

Recall that 
the 0-simplices of the homotopy colimit $\cB^\cT(M)$ 
are a direct sum of chain complexes with the summands indexed by
decompositions of $M$ which have their $n{-}1$-skeletons labeled by $n{-}1$-morphisms
of $\cT$.
Since $\cT = \pi^\infty_{\leq n}(T)$, this means that the summands are indexed by pairs
$(K, \vphi)$, where $K$ is a decomposition of $M$ and $\vphi$ is a continuous
map from the $n{-}1$-skeleton of $K$ to $T$.
The summand indexed by $(K, \vphi)$ is
\[
	\bigotimes_b D_*(b, \vphi),
\]
where $b$ runs through the $n$-cells of $K$ and $D_*(b, \vphi)$ denotes
chains of maps from $b$ to $T$ compatible with $\vphi$.
We can take the product of these chains of maps to get chains of maps from
all of $M$ to $K$.
This defines $\psi$ on 0-simplices.

We define $\psi$ to be zero on $(\ge1)$-simplices.
It is not hard to see that this defines a chain map from 
$\cB^\cT(M)$ to $C_*(\Maps(M\to T))$.

The image of $\psi$ is the subcomplex $G_*\sub C_*(\Maps(M\to T))$ generated by 
families of maps whose support is contained in a disjoint union of balls.
It follows from Lemma \ref{extension_lemma_c} 
that $C_*(\Maps(M\to T))$ is homotopic to a subcomplex of $G_*$.

We will define a map $\phi:G_*\to \cB^\cT(M)$ via acyclic models.
Let $a$ be a generator of $G_*$.
Define $D(a)$ to be the subcomplex of $\cB^\cT(M)$ generated by all 
pairs $(b, \ol{K})$, where $b$ is a generator appearing in an iterated boundary of $a$
and $\ol{K}$ is an index of the homotopy colimit $\cB^\cT(M)$.
(See the proof of Theorem \ref{thm:product} for more details.)
The same proof as of Lemma \ref{lem:d-a-acyclic} shows that $D(a)$ is acyclic.
By the usual acyclic models nonsense, there is a (unique up to homotopy)
map $\phi:G_*\to \cB^\cT(M)$ such that $\phi(a)\in D(a)$.
Furthermore, we may choose $\phi$ such that for all $a$ 
\[
	\phi(a) = (a, K) + r
\]
where $(a, K)$ is a 0-simplex and $r$ is a sum of simplices of dimension 1 and greater.

It is now easy to see that $\psi\circ\phi$ is the identity on the nose.
Another acyclic models argument shows that $\phi\circ\psi$ is homotopic to the identity.
(See the proof of Theorem \ref{thm:product} for more details.)
\end{proof}