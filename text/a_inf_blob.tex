%!TEX root = ../blob1.tex

\section{The blob complex for $A_\infty$ $n$-categories}
\label{sec:ainfblob}

Given an $A_\infty$ $n$-category $\cC$ and an $n$-manifold $M$, we define the blob
complex $\bc_*(M)$ to the be the colimit $\cC(M)$ of Section \ref{sec:ncats}.
\nn{say something about this being anticlimatically tautological?}
We will show below 
\nn{give ref}
that this agrees (up to homotopy) with our original definition of the blob complex
in the case of plain $n$-categories.
When we need to distinguish between the new and old definitions, we will refer to the 
new-fangled and old-fashioned blob complex.

\medskip

Let $M^n = Y^k\times F^{n-k}$.  
Let $C$ be a plain $n$-category.
Let $\cF$ be the $A_\infty$ $k$-category which assigns to a $k$-ball
$X$ the old-fashioned blob complex $\bc_*(X\times F)$.

\begin{thm}
The old-fashioned blob complex $\bc_*^C(Y\times F)$ is homotopy equivalent to the
new-fangled blob complex $\bc_*^\cF(Y)$.
\end{thm}

\begin{proof}
We will use the concrete description of the colimit from Subsection \ref{ss:ncat_fields}.

First we define a map from $\bc_*^\cF(Y)$ to $\bc_*^C(Y\times F)$.
In filtration degree 0 we just glue together the various blob diagrams on $X\times F$
(where $X$ is a component of a permissible decomposition of $Y$) to get a blob diagram on
$Y\times F$.
In filtration degrees 1 and higher we define the map to be zero.
It is easy to check that this is a chain map.

Next we define a map from $\phi: \bc_*^C(Y\times F) \to \bc_*^\cF(Y)$.
Actually, we will define it on the homotopy equivalent subcomplex
$\cS_* \sub \bc_*^C(Y\times F)$ generated by blob diagrams which are small with 
respect to some open cover
of $Y\times F$.
\nn{need reference to small blob lemma}
We will have to show eventually that this is independent (up to homotopy) of the choice of cover.
Also, for a fixed choice of cover we will only be able to define the map for blob degree less than
some bound, but this bound goes to infinity as the cover become finer.

Given a decomposition $K$ of $Y$ into $k$-balls $X_i$, let $K\times F$ denote the corresponding
decomposition of $Y\times F$ into the pieces $X_i\times F$.

%We will define $\phi$ inductively, starting at blob degree 0.
%Given a 0-blob diagram $x$ on $Y\times F$, we can choose a decomposition $K$ of $Y$
%such that $x$ is splittable with respect to $K\times F$.
%This defines a filtration degree 0 element of $\bc_*^\cF(Y)$

We will define $\phi$ using a variant of the method of acyclic models.
Let $a\in S_m$ be a blob diagram on $Y\times F$.
For $m$ sufficiently small there exist decompositions of $K$ of $Y$ into $k$-balls such that the
codimension 1 cells of $K\times F$ miss the blobs of $a$, and more generally such that $a$ is splittable along $K\times F$.
Let $D(a)$ denote the subcomplex of $\bc_*^\cF(Y)$ generated by all $(a, \bar{K})$
such that each $K_i$ has the aforementioned splittable property
(see Subsection \ref{ss:ncat_fields}).
\nn{need to define $D(a)$ more clearly; also includes $(b_j, \bar{K})$ where
$\bd(a) = \sum b_j$.}
(By $(a, \bar{K})$ we really mean $(a^\sharp, \bar{K})$, where $a^\sharp$ is 
$a$ split according to $K_0\times F$.
To simplify notation we will just write plain $a$ instead of $a^\sharp$.)
Roughly speaking, $D(a)$ consists of filtration degree 0 stuff which glues up to give
$a$, filtration degree 1 stuff which makes all of the filtration degree 0 stuff homologous, 
filtration degree 2 stuff which kills the homology created by the 
filtration degree 1 stuff, and so on.
More formally,
 
\begin{lemma}
$D(a)$ is acyclic.
\end{lemma}

\begin{proof}
We will prove acyclicity in the first couple of degrees, and \nn{in this draft, at least}
leave the general case to the reader.

Let $K$ and $K'$ be two decompositions of $Y$ compatible with $a$.
We want to show that $(a, K)$ and $(a, K')$ are homologous via filtration degree 1 stuff.
\nn{need to say this better; these two chains don't have the same boundary.}
We might hope that $K$ and $K'$ have a common refinement, but this is not necessarily
the case.
(Consider the $x$-axis and the graph of $y = x^2\sin(1/x)$ in $\r^2$.)
However, we {\it can} find another decomposition $L$ such that $L$ shares common
refinements with both $K$ and $K'$.
Let $KL$ and $K'L$ denote these two refinements.
Then filtration degree 1 chains associated to the four anti-refinemnts
$KL\to K$, $KL\to L$, $K'L\to L$ and $K'L\to K'$
give the desired chain connecting $(a, K)$ and $(a, K')$
(see Figure xxxx).

Consider a different choice of decomposition $L'$ in place of $L$ above.
This leads to a cycle consisting of filtration degree 1 stuff.
We want to show that this cycle bounds a chain of filtration degree 2 stuff.
Choose a decomposition $M$ which has common refinements with each of 
$K$, $KL$, $L$, $K'L$, $K'$, $K'L'$, $L'$ and $KL'$.
\nn{need to also require that $KLM$ antirefines to $KM$, etc.}
Then we have a filtration degree 2 chain, as shown in Figure yyyy, which does the trick.
For example, ....


\end{proof}


\nn{....}
\end{proof}


\nn{need to say something about dim $< n$ above}



\medskip
\hrule
\medskip

\nn{to be continued...}
\medskip

