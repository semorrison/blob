%!TEX root = ../blob1.tex

\section{The blob complex for $A_\infty$ $n$-categories}
\label{sec:ainfblob}
\label{sec:gluing}

Given an $A_\infty$ $n$-category $\cC$ and an $n$-manifold $M$, we define the blob
complex $\bc_*(M)$ to the be the homotopy colimit $\cC(M)$ of Section \ref{sec:ncats}.
\nn{say something about this being anticlimatically tautological?}
We will show below 
\nn{give ref}
that this agrees (up to homotopy) with our original definition of the blob complex
in the case of plain $n$-categories.
When we need to distinguish between the new and old definitions, we will refer to the 
new-fangled and old-fashioned blob complex.

\medskip

Let $M^n = Y^k\times F^{n-k}$.  
Let $C$ be a plain $n$-category.
Let $\cF$ be the $A_\infty$ $k$-category which assigns to a $k$-ball
$X$ the old-fashioned blob complex $\bc_*(X\times F)$.

\begin{thm} \label{product_thm}
The old-fashioned blob complex $\bc_*^C(Y\times F)$ is homotopy equivalent to the
new-fangled blob complex $\bc_*^\cF(Y)$.
\end{thm}

%!TEX root = ../blob1.tex
\nn{Not sure where this goes yet: small blobs, unfinished:}

Fix $\cU$, an open cover of $M$. Define the `small blob complex' $\bc^{\cU}_*(M)$ to be the subcomplex of $\bc_*(M)$ of all blob diagrams in which every blob is contained in some open set of $\cU$.

\begin{thm}[Small blobs]
The inclusion $i: \bc^{\cU}_*(M) \into \bc_*(M)$ is a homotopy equivalence.
\end{thm}
\begin{proof}
We begin by describing the homotopy inverse in small degrees, to illustrate the general technique.
We will construct a chain map $s:  \bc_*(M) \to \bc^{\cU}_*(M)$ and a homotopy $h:\bc_*(M) \to \bc_{*+1}(M)$ so that $\bdy h+h \bdy=\id - i\circ s$. The composition $s \circ i$ will just be the identity.

On $0$-blobs, $s$ is just the identity; a blob diagram without any blobs is compatible with any open cover. Nevertheless, we'll begin introducing nomenclature at this point: for configuration $\beta$ of disjoint embedded balls in $M$ we'll associate a one parameter family of homeomorphisms $\phi_\beta : \Delta^1 \to \Homeo(M)$ (here $\Delta^m$ is the standard simplex $\setc{\mathbf{x} \in \Real^{m+1}}{\sum_{i=0}^m x_i = 1}$). For $0$-blobs, where $\beta = \eset$, all these homeomorphisms are just the identity.

On a $1$-blob $b$, with ball $\beta$, $s$ is defined as the sum of two terms. Essentially, the first term `makes $\beta$ small', while the other term `gets the boundary right'. First, pick a one-parameter family $\phi_\beta : \Delta^1 \to \Homeo(M)$ of homeomorphisms, so $\phi_\beta(1,0)$ is the identity and $\phi_\beta(0,1)$ makes the ball $\beta$ small. Next, pick a two-parameter family $\phi_{\eset \prec \beta} : \Delta^2 \to \Homeo(M)$ so that $\phi_{\eset \prec \beta}(0,x_1,x_2)$ makes the ball $\beta$ small for all $x_1+x_2=1$, while $\phi_{\eset \prec \beta}(x_0,0,x_2) = \phi_\eset(x_0,x_2)$ and $\phi_{\eset \prec \beta}(x_0,x_1,0) = \phi_\beta(x_0,x_1)$. (It's perhaps not obvious that this is even possible --- see Lemma \ref{lem:extend-small-homeomorphisms} below.) We now define $s$ by
$$s(b) = \restrict{\phi_\beta}{x_0=0}(b) + \restrict{\phi_{\eset \prec \beta}}{x_0=0}(\bdy b).$$
Here, $\restrict{\phi_\beta}{x_0=0} = \phi_\beta(0,1)$ is just a homeomorphism, which we apply to $b$, while $\restrict{\phi_{\eset \prec \beta}}{x_0=0}$ is a one parameter family of homeomorphisms which acts on the $0$-blob $\bdy b$ to give a $1$-blob. We now check that $s$, as defined so far, is a chain map, calculating
\begin{align*}
\bdy (s(b)) & = \restrict{\phi_\beta}{x_0=0}(\bdy b) + (\bdy \restrict{\phi_{\eset \prec \beta}}{x_0=0})(\bdy b) \\
		 & = \restrict{\phi_\beta}{x_0=0}(\bdy b) + \restrict{\phi_\eset}{x_0=0}(\bdy b) - \restrict{\phi_\beta}{x_0=0}(\bdy b) \\
		 & = \restrict{\phi_\eset}{x_0=0}(\bdy b) \\
		 & = s(\bdy b)
\end{align*}
Next, we compute the compositions $s \circ i$ and $i \circ s$. If we start with a small $1$-blob diagram $b$, first include it up to the full blob complex then apply $s$, we get exactly back to $b$, at least assuming we adopt the convention that for any ball $\beta$ which is already small, we choose the families of homeomorphisms $\phi_\beta$ and $\phi_{\eset \prec \beta}$ to always be the identity. In the other direction, $i \circ s$, we will need to construct a homotopy $h:\bc_*(M) \to \bc_{*+1}(M)$ for $*=0$ or $1$. This is defined by $h(b) = \phi_\eset(b)$ when $b$ is a $0$-blob (here $\phi_\eset$ is a one parameter family of homeomorphisms, so this is a $1$-blob), and $h(b) = \phi_\beta(b) + \phi_{\eset \prec \beta}(\bdy b)$ when $b$ is a $1$-blob (here $\beta$ is the ball in $b$, and the first term is the action of a one parameter family of homeomorphisms on a $1$-blob, and the second term is the action of a two parameter family of homeomorphisms on a $0$-blob, so both are $2$-blobs).
\begin{align*}
(\bdy h+h \bdy)(b) & = \bdy (\phi_{\beta}(b) + \phi_{\eset \prec \beta}{\bdy b}) + \phi_\eset(\bdy b)  \\
	& =  \restrict{\phi_\beta}{x_0=0}(b) - \restrict{\phi_\beta}{x_1=0}(b) - \phi_\beta(\bdy b) + (\bdy \phi_{\eset \prec \beta})(\bdy b) + \phi_\eset(\bdy b) \\
	& =  \restrict{\phi_\beta}{x_0=0}(b) - b - \phi_\beta(\bdy b) + \restrict{\phi_{\eset \prec \beta}}{x_0=0}(\bdy b) -  \phi_\eset(\bdy b) + \phi_\beta(\bdy b) + \phi_\eset(\bdy b) \\
	& = \restrict{\phi_\beta}{x_0=0}(b) - b + \restrict{\phi_{\eset \prec \beta}}{x_0=0}(\bdy b) \\
	& = (i \circ s - \id)(b)
\end{align*}


Given a blob diagram $b \in \bc_k(M)$, denote by $b_\cS$ for $\cS \subset \{1, \ldots, k\}$ the blob diagram obtained by erasing the corresponding blobs. In particular, $b_\eset = b$, $b_{\{1,\ldots,k\}} \in \bc_0(M)$, and $d b_\cS = \sum_{\cS' = \cS'\sqcup\{i\}} \pm  b_{\cS'}$.
Similarly, for a disjoint embedding of $k$ balls $\beta$ (that is, a blob diagram but without the labels on regions), $\beta_\cS$ denotes the result of erasing a subset of blobs. We'll write $\beta' \prec \beta$ if $\beta' = \beta_\cS$ for some $\cS$. Finally, for finite sequences, we'll write $i \prec i'$ if $i$ is subsequence of $i'$, and $i \prec_1 i$ if the lengths differ by exactly 1.

Next, we'll choose a `shrinking system' for $\cU$, namely for each increasing sequence of blob configurations
$\beta_0 \prec \beta_1 \prec \cdots \prec \beta_m$, an $m$ parameter family of diffeomorphisms
$\phi_{\beta_0 \prec \cdots \prec \beta_m} : \Delta^m \to \Diff{M}$ (here $\Delta^m$ is the standard simplex $\setc{\mathbf{x} \in \Real^{m+1}}{\sum_i x_i = 1}$), such that
\begin{itemize}
\item if $\beta$ is the empty configuration, $\phi_{\beta}(1) = \id_M$,
\item if $\beta$ is a single configuration of blobs, then $\phi_{\beta}(1)(\beta)$ (which is another configuration of blobs: $\phi_{\beta}(1)$ is a diffeomorphism of $M$) is subordinate to $\cU$,
\item (more generally) for any $x$ with $x_0 = 0$, $\phi_{\beta_0 \prec \cdots \prec \beta_m}(x)(\beta)$ is subordinate to $\cU$, and
\item for each $i = 1, \ldots, m$,
\begin{align*}
\phi_{\beta_0 \prec \cdots \prec \beta_m}(x_0, \ldots, x_{i-1},0,x_{i+1},\ldots,x_m) & = \phi_{\beta_0 \prec \cdots \beta_{i-1} \prec \beta_{i+1} \prec \beta_m}(x_0,\ldots, x_{i-1},x_{i+1},\ldots,x_m).
\end{align*}
\end{itemize}
It's not immediately obvious that it's possible to make such choices, but it follows readily from the following Lemma.

When $\beta$ is a collection of disjoint embedded balls in $M$, we say that a homeomorphism of $M$ `makes $\beta$ small' if the image of each ball in $\beta$ under the homeomorphism is contained in some open set of $\cU$.

\begin{lem}
\label{lem:extend-small-homeomorphisms}
Fix a collection of disjoint embedded balls $\beta$ in $M$. Suppose we have a map $f :  X \to \Homeo(M)$ on some compact $X$ such that for each $x \in \bdy X$, $f(x)$ makes $\beta$ small. Then we can extend $f$ to a map $\tilde{f} : X \times [0,1] \to \Homeo(M)$ so that $\tilde{f}(x,0) = f(x)$ and for every $x \in \bdy X \times [0,1] \cup X \times \{1\}$, $\tilde{f}(x)$ makes $\beta$ small.
\end{lem}
\begin{proof}
Fix a metric on $M$, and pick $\epsilon > 0$ so every $\epsilon$ ball in $M$ is contained in some open set of $\cU$. First construct a family of homeomorphisms $g_s : M \to M$, $s \in [1,\infty)$ so $g_1$ is the identity, and $g_s(\beta_i) \subset \beta_i$ and $\rad g_s(\beta_i) \leq \frac{1}{s} \rad \beta_i$ for each ball $\beta_i$. 
There is some $K$ which uniformly bounds the expansion factors of all the homeomorphisms $f(x)$, that is $d(f(x)(a), f(x)(b)) < K d(a,b)$ for all $x \in X, a,b \in M$. Write $S=\epsilon^{-1} K \max_i \{\rad \beta_i\}$ (note that is $S<1$, we can just take $S=1$, as already $f(x)$ makes $\beta$ small for all $x$). Now define $\tilde{f}(t, x) = f(x) \compose g_{(S-1)t+1}$.

If $x \in \bdy X$, then $g_{(S-1)t+1}(\beta_i) \subset \beta_i$, and by hypothesis $f(x)$ makes $\beta_i$ small, so $\tilde{f}(t, x)$ makes $\beta$ small for all $t \in [0,1]$. Alternatively, $\rad g_S(\beta_i) \leq \frac{1}{S} \rad \beta_i \leq \frac{\epsilon}{K}$, so $\rad \tilde{f}(1,x)(\beta_i) \leq \epsilon$, and so $\tilde{f}(1,x)$ makes $\beta$ small for all $x \in X$.
\end{proof}

We'll need a stronger version of Property \ref{property:evaluation}; while the evaluation map $ev: \CD{M} \tensor \bc_*(M) \to \bc_*(M)$ is not unique, it has an up-to-homotopy representative (satisfying the usual conditions) which restricts to become a chain map $ev: \CD{M} \tensor \bc^{\cU}_*(M) \to \bc^{\cU}_*(M)$. The proof is straightforward: when deforming the family of diffeomorphisms to shrink its supports to a union of open sets, do so such that those open sets are subordinate to the cover.

Now define a map $s: \bc_*(M) \to \bc^{\cU}_*(M)$, and then a homotopy $h:\bc_*(M) \to \bc_{*+1}(M)$ so that $dh+hd=i\circ s$. The map $s: \bc_0(M) \to \bc^{\cU}_0(M)$ is just the identity; blob diagrams without blobs are automatically compatible with any cover. Given a blob diagram $b$, we'll abuse notation and write $\phi_b$ to mean $\phi_\beta$ for the blob configuration $\beta$ underlying $b$. We have
$$s(b) = \sum_{i} ev(\restrict{\phi_{i(b)}}{x_0 = 0} \tensor b_i)$$
where the sum is over sequences $i=(i_1,\ldots,i_m)$ in $\{1,\ldots,k\}$, with $0\leq m < k$, $i(b)$ denotes the increasing sequence of blob configurations
$$\beta_{(i_1,\ldots,i_m)} \prec \beta_{(i_2,\ldots,i_m)} \prec \cdots \prec \beta_{()},$$
and, as usual, $i(b)$ denotes $b$ with blobs $i_1, \ldots i_m$ erased. We'll also write
$$s(b) = \sum_{m=0}^{k-1} \sum_{\norm{i}=m} ev(\restrict{\phi_{i(b)}}{x_0 = 0} \tensor b_i),$$
arranging the sum according to the length $\norm{i}$ of $i$.


We need to check that $s$ is a chain map, and that the image of $s$ in fact lies in $\bc^{\cU}_*(M)$. \todo{} Calculate
\begin{align*}
\bdy(s(b)) & = \sum_{m=0}^{k-1} \sum_{\norm{i}=m} \ev\left(\bdy(\restrict{\phi_{i(b)}}{x_0 = 0})\tensor b_i\right) + (-1)^m \ev\left(\restrict{\phi_{i(b)}}{x_0 = 0} \tensor \bdy b_i\right) \\
                & = \sum_{m=0}^{k-1} \sum_{\norm{i}=m} \ev\left(\sum_{i' \prec_1 i} \pm \restrict{\phi_{i'(b)}}{x_0 = 0})\tensor b_i\right) + (-1)^m \ev\left(\restrict{\phi_{i(b)}}{x_0 = 0}\tensor \sum_{i \prec_1 i'} \pm b_{i'}\right) \\
\intertext{and telescoping the sum}
		& = \sum_{m=0}^{k-2} \left(\sum_{\norm{i}=m}  (-1)^m \ev\left(\restrict{\phi_{i(b)}}{x_0 = 0} \tensor \sum_{i \prec_1 i'} \pm b_{i'}\right) \right) + \left(\sum_{\norm{i}=m+1} \ev\left(\sum_{i' \prec_1 i} \pm \restrict{\phi_{i'(b)}}{x_0 = 0} \tensor b_i\right) \right) + \\
		& \qquad + (-1)^{k-1} \sum_{\norm{i}=k-1} \ev\left(\restrict{\phi_{i(b)}}{x_0 = 0} \tensor \sum_{i \prec_1 i'} \pm b_{i'}\right) \\
		& = (-1)^{k-1} \sum_{\norm{i}=k-1} \ev\left(\restrict{\phi_{i(b)}}{x_0 = 0} \tensor \sum_{i \prec_1 i'} \pm b_{i'}\right)
\end{align*}

Next, we define the homotopy $h:\bc_*(M) \to \bc_{*+1}(M)$ by
$$h(b) = \sum_{i} ev(\phi_{i(b)}, b_i).$$
\todo{and check that it's the right one...}
\end{proof}


\begin{proof}[Proof of Theorem \ref{product_thm}]
We will use the concrete description of the colimit from Subsection \ref{ss:ncat_fields}.

First we define a map 
\[
	\psi: \bc_*^\cF(Y) \to \bc_*^C(Y\times F) .
\]
In filtration degree 0 we just glue together the various blob diagrams on $X\times F$
(where $X$ is a component of a permissible decomposition of $Y$) to get a blob diagram on
$Y\times F$.
In filtration degrees 1 and higher we define the map to be zero.
It is easy to check that this is a chain map.

Next we define a map 
\[
	\phi: \bc_*^C(Y\times F) \to \bc_*^\cF(Y) .
\]
Actually, we will define it on the homotopy equivalent subcomplex
$\cS_* \sub \bc_*^C(Y\times F)$ generated by blob diagrams which are small with 
respect to some open cover
of $Y\times F$.
\nn{need reference to small blob lemma}
We will have to show eventually that this is independent (up to homotopy) of the choice of cover.
Also, for a fixed choice of cover we will only be able to define the map for blob degree less than
some bound, but this bound goes to infinity as the cover become finer.

Given a decomposition $K$ of $Y$ into $k$-balls $X_i$, let $K\times F$ denote the corresponding
decomposition of $Y\times F$ into the pieces $X_i\times F$.

%We will define $\phi$ inductively, starting at blob degree 0.
%Given a 0-blob diagram $x$ on $Y\times F$, we can choose a decomposition $K$ of $Y$
%such that $x$ is splittable with respect to $K\times F$.
%This defines a filtration degree 0 element of $\bc_*^\cF(Y)$

We will define $\phi$ using a variant of the method of acyclic models.
Let $a\in \cS_m$ be a blob diagram on $Y\times F$.
For $m$ sufficiently small there exists a decomposition $K$ of $Y$ into $k$-balls such that the
codimension 1 cells of $K\times F$ miss the blobs of $a$, and more generally such that $a$ is splittable along (the codimension-1 part of) $K\times F$.
Let $D(a)$ denote the subcomplex of $\bc_*^\cF(Y)$ generated by all $(a, \bar{K})$
such that each $K_i$ has the aforementioned splittable property
(see Subsection \ref{ss:ncat_fields}).
\nn{need to define $D(a)$ more clearly; also includes $(b_j, \bar{K})$ where
$\bd(a) = \sum b_j$.}
(By $(a, \bar{K})$ we really mean $(a^\sharp, \bar{K})$, where $a^\sharp$ is 
$a$ split according to $K_0\times F$.
To simplify notation we will just write plain $a$ instead of $a^\sharp$.)
Roughly speaking, $D(a)$ consists of filtration degree 0 stuff which glues up to give
$a$, filtration degree 1 stuff which makes all of the filtration degree 0 stuff homologous, 
filtration degree 2 stuff which kills the homology created by the 
filtration degree 1 stuff, and so on.
More formally,
 
\begin{lemma}
$D(a)$ is acyclic.
\end{lemma}

\begin{proof}
We will prove acyclicity in the first couple of degrees, and \nn{in this draft, at least}
leave the general case to the reader.

Let $K$ and $K'$ be two decompositions of $Y$ compatible with $a$.
We want to show that $(a, K)$ and $(a, K')$ are homologous via filtration degree 1 stuff.
\nn{need to say this better; these two chains don't have the same boundary.}
We might hope that $K$ and $K'$ have a common refinement, but this is not necessarily
the case.
(Consider the $x$-axis and the graph of $y = x^2\sin(1/x)$ in $\r^2$.)
However, we {\it can} find another decomposition $L$ such that $L$ shares common
refinements with both $K$ and $K'$.
Let $KL$ and $K'L$ denote these two refinements.
Then filtration degree 1 chains associated to the four anti-refinemnts
$KL\to K$, $KL\to L$, $K'L\to L$ and $K'L\to K'$
give the desired chain connecting $(a, K)$ and $(a, K')$
(see Figure \ref{zzz4}).

\begin{figure}[!ht]
\begin{equation*}
\mathfig{.63}{tempkw/zz4}
\end{equation*}
\caption{Connecting $K$ and $K'$ via $L$}
\label{zzz4}
\end{figure}

Consider a different choice of decomposition $L'$ in place of $L$ above.
This leads to a cycle consisting of filtration degree 1 stuff.
We want to show that this cycle bounds a chain of filtration degree 2 stuff.
Choose a decomposition $M$ which has common refinements with each of 
$K$, $KL$, $L$, $K'L$, $K'$, $K'L'$, $L'$ and $KL'$.
\nn{need to also require that $KLM$ antirefines to $KM$, etc.}
Then we have a filtration degree 2 chain, as shown in Figure \ref{zzz5}, which does the trick.
(Each small triangle in Figure \ref{zzz5} can be filled with a filtration degree 2 chain.)

\begin{figure}[!ht]
\begin{equation*}
\mathfig{1.0}{tempkw/zz5}
\end{equation*}
\caption{Filling in $K$-$KL$-$L$-$K'L$-$K'$-$K'L'$-$L'$-$KL'$-$K$}
\label{zzz5}
\end{figure}

Continuing in this way we see that $D(a)$ is acyclic.
\end{proof}

We are now in a position to apply the method of acyclic models to get a map
$\phi:\cS_* \to \bc_*^\cF(Y)$.
This map is defined in sufficiently low degrees, sends a blob diagram $a$ to $D(a)$, 
and is well-defined up to (iterated) homotopy.

The subcomplex $\cS_* \subset \bc_*^C(Y\times F)$ depends on choice of cover of $Y\times F$.
If we refine that cover, we get a complex $\cS'_* \subset \cS_*$
and a map $\phi':\cS'_* \to \bc_*^\cF(Y)$.
$\phi'$ is defined only on homological degrees below some bound, but this bound is higher than 
the corresponding bound for $\phi$.
We must show that $\phi$ and $\phi'$ agree, up to homotopy,
on the intersection of the subcomplexes on which they are defined.
This is clear, since the acyclic subcomplexes $D(a)$ above used in the definition of 
$\phi$ and $\phi'$ do not depend on the choice of cover.

\nn{need to say (and justify) that we now have a map $\phi$ indep of choice of cover}

We now show that $\phi\circ\psi$ and $\psi\circ\phi$ are homotopic to the identity.

$\psi\circ\phi$ is the identity.  $\phi$ takes a blob diagram $a$ and chops it into pieces 
according to some decomposition $K$ of $Y$.
$\psi$ glues those pieces back together, yielding the same $a$ we started with.

$\phi\circ\psi$ is the identity up to homotopy by another MoAM argument...

This concludes the proof of Theorem \ref{product_thm}.
\nn{at least I think it does; it's pretty rough at this point.}
\end{proof}

\nn{need to say something about dim $< n$ above}

\medskip

\begin{cor}
The new-fangled and old-fashioned blob complexes are homotopic.
\end{cor}
\begin{proof}
Apply Theorem \ref{product_thm} with the fiber $F$ equal to a point.
\end{proof}

\medskip

Next we prove a gluing theorem.
Let $X$ be a closed $k$-manifold with a splitting $X = X'_1\cup_Y X'_2$.
We will need an explicit collar on $Y$, so rewrite this as
$X = X_1\cup (Y\times J) \cup X_2$.
\nn{need figure}
Given this data we have: \nn{need refs to above for these}
\begin{itemize}
\item An $A_\infty$ $n{-}k$-category $\bc(X)$, which assigns to an $m$-ball
$D$ fields on $D\times X$ (for $m+k < n$) or the blob complex $\bc_*(D\times X; c)$
(for $m+k = n$). \nn{need to explain $c$}.
\item An $A_\infty$ $n{-}k{+}1$-category $\bc(Y)$, defined similarly.
\item Two $\bc(Y)$ modules $\bc(X_1)$ and $\bc(X_2)$, which assign to a marked
$m$-ball $(D, H)$ either fields on $(D\times Y) \cup (H\times X_i)$ (if $m+k < n$)
or the blob complex $\bc_*((D\times Y) \cup (H\times X_i))$ (if $m+k = n$).
\end{itemize}

\begin{thm}
$\bc(X) \cong \bc(X_1) \otimes_{\bc(Y), J} \bc(X_2)$.
\end{thm}

\begin{proof}
The proof is similar to that of Theorem \ref{product_thm}.
\nn{need to say something about dimensions less than $n$, 
but for now concentrate on top dimension.}

Let $\cT$ denote the $n{-}k$-category $\bc(X_1) \otimes_{\bc(Y), J} \bc(X_2)$.
Let $D$ be an $n{-}k$-ball.
There is an obvious map from $\cT(D)$ to $\bc_*(D\times X)$.
To get a map in the other direction, we replace $\bc_*(D\times X)$ with a subcomplex
$\cS_*$ which is adapted to a fine open cover of $D\times X$.
For sufficiently small $j$ (depending on the cover), we can find, for each $j$-blob diagram $b$
on $D\times X$, a decomposition of $J$ such that $b$ splits on the corresponding
decomposition of $D\times X$.
The proof that these two maps are inverse to each other is the same as in
Theorem \ref{product_thm}.
\end{proof}


\medskip
\hrule
\medskip

\nn{to be continued...}
\medskip
\nn{still to do: fiber bundles, general maps}

\todo{}
Various citations we might want to make:
\begin{itemize}
\item \cite{MR2061854} McClure and Smith's review article
\item \cite{MR0420610} May, (inter alia, definition of $E_\infty$ operad)
\item \cite{MR0236922,MR0420609} Boardman and Vogt
\item \cite{MR1256989} definition of framed little-discs operad
\end{itemize}

We now turn to establishing the gluing formula for blob homology, restated from Property \ref{property:gluing} in the Introduction
\begin{itemize}
%\mbox{}% <-- gets the indenting right
\item For any $(n-1)$-manifold $Y$, the blob homology of $Y \times I$ is
naturally an $A_\infty$ category. % We'll write $\bc_*(Y)$ for $\bc_*(Y \times I)$ below.

\item For any $n$-manifold $X$, with $Y$ a codimension $0$-submanifold of its boundary, the blob homology of $X$ is naturally an
$A_\infty$ module for $\bc_*(Y \times I)$.

\item For any $n$-manifold $X$, with $Y \cup Y^{\text{op}}$ a codimension
$0$-submanifold of its boundary, the blob homology of $X'$, obtained from
$X$ by gluing along $Y$, is the $A_\infty$ self-tensor product of
$\bc_*(X)$ as an $\bc_*(Y \times I)$-bimodule.
\begin{equation*}
\bc_*(X') \iso \bc_*(X) \Tensor^{A_\infty}_{\mathclap{\bc_*(Y \times I)}} \!\!\!\!\!\!\xymatrix{ \ar@(ru,rd)@<-1ex>[]}
\end{equation*}
\end{itemize}

